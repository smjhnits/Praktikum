\begin{exercise}{Schräger Wurf}{0}
  Betrachten Sie den zweidimensionalen schrägen Wurf.
  Zum Zeitpunkt $t_0 = \SI{0}{\second}$ befindet sich
  ein Teilchen in den Koordinaten $x_0 = \SI{5}{\meter}$ und $y_0 = h = \SI{5}{\meter}$.

    \begin{enumerate}
        \item Wie sieht die Bewegung für $v_x\neq 0$ und $v_y\neq 0$ aus?\\
        Berechnen Sie die Bahnkurve $\vec{r}$ des Teilchens in Abhängigkeit von $x$. Skizzieren Sie die beschriebene Situation.
        \item Wo befindet sich das Teilchen für $x = \SI{10}{\meter}$, wenn $v_x = \num{5}\frac{\si{\meter}}{\si{\second}}$ und $v_y = \num{-7}\frac{\si{\meter}}{\si{\second}}$ betragen?
        \item Diskutieren Sie die Fälle:
        \begin{itemize}
          \item[(i)] $v_x > 0$ und $v_y > 0$
          \item[(ii)] $v_x > 0$ und $v_y = 0$
          \item[(iii)] $v_x = v_y = 0$
        \end{itemize}
        Welchen Bewegungen entsprechen die Fälle (i) - (iii)?
    \end{enumerate}
\end{exercise}

\begin{exercise}{Wegintegrale}{0}
  Gegeben sei das folgende Vektorfeld:
  \begin{equation*}
    \vec{F}(x,y) = \left( \begin{array}{c} y\\\ x^2\  \end{array}\right)
  \end{equation*}
  Berechnen sie das Linienintegral von (0,1) zu (2,0) entlang der folgenden Wege.
  \begin{enumerate}
    \item $y = \frac{1}{4}(x-2)^2$\\
    \item $y = \frac{1}{1-e^{-2}}(e^{-x}-e^{-2})$\\
    \item $y = 1 - \frac{1}{2}x$\\
    \item $y = 1 - \frac{1}{4}x^2$
  \end{enumerate}
\end{exercise}
