\begin{exercise}{Perle auf Draht}{10}
  Betrachten Sie eine Perle auf einem Draht. Die Perle besitzt die Masse $m$
  und gleitet reibungsfrei auf einem Draht, welcher durch die Funktion $z = h(x)$
  beschrieben wird.

  \begin{figure}[h]
    \centering
    \includegraphics[width=\textwidth]{Blatt_06_PerleaufDraht.png}
    \label{fig:Höhenprofil}
  \end{figure}

    \begin{enumerate}
        \item Bestimmen Sie für einen beliebig geformten Draht $h(x)$ die Lagrange-Funktion
        in der generalisierten Koordinate $x$.
        \item Nutzen Sie die Euler-Lagrange-Gleichung, um die Bewegungsgleichung
        des Massepunktes aufzustellen.
        \item Setzen Sie nun die folgenden Funktionen $h(x)$ in die Bewegungsgleichung
        ein:
        \begin{itemize}
          \item[(i)] Welche Bewegung wird für $h(x) = h_0$ angenommen?
          \item[(ii)] $h(x) = ax$. Zeigen Sie anhand der Bewegungsgleichung, dass
          auf die Perle nur die konstante Hangabtriebskraft
          \begin{equation}
            \label{eqn:Hangabtrieb}
            \left|\vec{F}_{H}\right| = mg\sin\left(\alpha\right)
          \end{equation}
          wirkt, wobei $\alpha$ den Steigungswinkel der Funktion, d.h. den Winkel
          zwischen Funktionsgraph und der $x$-Achse, bezeichnet.
          \item[(iii)] Sei nun $h(x) = \frac{b}{2}x^2$. Die Bewegungsgleichung enthält
           neben der Hangabtriebskraft einen weiteren Term. Welche Kraft beschreibt
           dieser?\\
           
           Welche Form nimmt die Bewegungsgleichung für $b > 0$ an, wenn die
           Auslenkungen $x$
           und die Geschwindigkeiten $\dot{x}$ so klein sind, dass nur die linearen Terme
           berücksichtigt werden müssen?\\
         \end{itemize}
           \item Berechnen Sie aus der Lagrange-Funktion, die Sie in Aufgabenteil (a)
           bestimmt haben, die Erhaltungsgrö\ss e

    \begin{equation}
      \label{eqn:Hamilton}
      H = \dot{x}\frac{\partial L}{\partial\dot{x}} - L.
    \end{equation}
    Um welche Grö\ss e handelt es sich hierbei?
    \end{enumerate}
\end{exercise}
\end{document}
