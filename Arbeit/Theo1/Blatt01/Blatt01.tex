% document class
\documentclass{scrartcl}

% encoding
\usepackage[T1]{fontenc}
\usepackage[utf8]{inputenc}%utf8

% language
\usepackage[ngerman]{babel}
\usepackage{scrhack} % nach \documentclass
\usepackage[aux]{rerunfilecheck}
\usepackage{polyglossia}
\setmainlanguage{german}

% math
\usepackage{amsmath}
\usepackage{amsfonts}
\usepackage{amssymb}
\usepackage{amstext}
\usepackage{isomath}
\usepackage{mathtools}

% physics
\usepackage{tensor}
\usepackage{slashed}
\usepackage{braket}
\usepackage[strict, separate-uncertainty, sticky-per]{siunitx}

% fonts
\usepackage{microtype}
\usepackage{fourier}
\usepackage{tgheros}
\usepackage{tgcursor}
\usepackage{tgpagella}

% tikz
\usepackage{tikz}

% margins
\usepackage[top=2.5cm]{geometry}

% variables
\newcommand{\thehandover}{Freitag, den 20.\,Oktober 2017 12:00 Uhr}
\newcommand{\thesheet}{1}
\newcommand{\thesemester}{WS 17/18}
\newcommand{\theprofessor}{Priv.-Doz.~U.~Löw}

% enumeration
\renewcommand{\labelenumi}{(\alph{enumi})} % alphabetic tasks
\renewcommand{\theenumi}{(\alph{enumi})} % alphabetic tasks
\usepackage{enumitem} % allows to continue lists with [resume]

% clickable links and '\url' command
\usepackage[hidelinks]{hyperref}

\usepackage{placeins}



\newcommand{\ua}[1]{_\symup{#1}}
\newcommand{\su}[1]{\symup{#1}}

\newcounter{exercise}
\newenvironment{exercise}
[2]
{\addtocounter{exercise}{1}{\bfseries{Aufgabe \arabic{exercise}:~#1}\hfill(#2 Punkte)}\newline}
{\medskip}

\setlength{\parindent}{0mm}
\begin{document}
{\large\bfseries 1. Übungsblatt zur Vorlesung\hfill\thesemester}\\ %/thesheet
{\large\bfseries Theoretische Physik I\hfill\theprofessor}\\
%{\large\bfseries Abgabe: bis \thehandover}\\
{\large\bfseries Abgabe bis: 20.10.}\\
\textbf{Webseite zur Vorlesung: \\}
\url{https://moodle.tu-dortmund.de/course/view.php?id=9519} \\
\rule{\columnwidth}{0.1ex}
\medskip


\begin{exercise}{Raumwinkel}{3}
  \begin{enumerate}
    \item Berechnen Sie den Winkel, den der Mond am Himmel einnimmt und vergleichen
          Sie den Wert mit der Periheldrehung des Merkur in einem Jahrhundert.
    \item Welchen Winkel überdeckt Ihr Daumen am Himmel, wenn Sie den Arm ausstrecken?
  \end{enumerate}
\end{exercise}

\begin{exercise}{Teilchen in 3D}{7}\\
  Die Trajektorie eines Teilchens mit Masse m im dreidimensionalen Raum sei in
  sphärischen Polarkoordinaten gegeben durch:

  \begin{align}
    \vec{r} = \left( \begin{array}{c} R(t)\sin{\vartheta(t)}\cos{\varphi(t)} \\\
    R(t)\sin{\vartheta(t)}\sin{\varphi(t)} \\\ R(t)\cos{\vartheta(t)} \ \end{array}\right)
  \end{align}

  \begin{enumerate}
    \item Bestimmen Sie die Geschwindigkeit $\dot{\vec{r}}$ und die Beschleunigung
          $\ddot{\vec{r}}$ des Teilchens.
    \item Unter welcher Bedingung gilt $\vec{r} \, \perp \, \dot{\vec{r}}$ ?
    \item Berechnen Sie die kinetische Energie des Teilchens:
          \begin{equation}
            E = \frac{1}{2} m \dot{\vec{r}}^2
          \end{equation}
  \end{enumerate}
\end{exercise}


  \textbf{Präsenzaufgabe:} \\
  \\
  Betrachten Sie 2 Massen $m_1$ und $m_2$, die sich mit den Geschwindigkeiten
  $\vec{v_1}$ und $\vec{v_2}$ bewegen. $m_1$ stellt die bekannte Referenzmasse dar. \\
  Betrachten Sie einen elastischen zentralen Sto\ss\,\! der beiden Massen.
  Drücken Sie $m_2$ durch $m_1, \vec{v_1}, \vec{v_2}$ und die Geschwindigkeiten
  der Massen nach dem Sto\ss\,\! aus.
\end{document}
