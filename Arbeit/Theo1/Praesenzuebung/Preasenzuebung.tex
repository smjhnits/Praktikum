% document class
\documentclass{scrartcl}

% encoding
\usepackage[T1]{fontenc}
\usepackage[utf8]{inputenc}%utf8

% language
\usepackage[ngerman]{babel}
\usepackage{scrhack} % nach \documentclass
\usepackage[aux]{rerunfilecheck}
\usepackage{polyglossia}
\setmainlanguage{german}

% math
\usepackage{amsmath}
\usepackage{amsfonts}
\usepackage{amssymb}
\usepackage{amstext}
\usepackage{isomath}
\usepackage{mathtools}

% physics
\usepackage{tensor}
\usepackage{slashed}
\usepackage{braket}
\usepackage[strict, separate-uncertainty, sticky-per]{siunitx}

% fonts
\usepackage{microtype}
\usepackage{fourier}
\usepackage{tgheros}
\usepackage{tgcursor}
\usepackage{tgpagella}

% tikz
\usepackage{tikz}

% margins
\usepackage[top=2.5cm]{geometry}

% variables
\newcommand{\thehandover}{Freitag, den 20.\,Oktober 2017 12:00 Uhr}
\newcommand{\thesheet}{1}
\newcommand{\thesemester}{WS 17/18}
\newcommand{\theprofessor}{Priv.-Doz.~U.~Löw}

% enumeration
\renewcommand{\labelenumi}{(\alph{enumi})} % alphabetic tasks
\renewcommand{\theenumi}{(\alph{enumi})} % alphabetic tasks
\usepackage{enumitem} % allows to continue lists with [resume]

% clickable links and '\url' command
\usepackage[hidelinks]{hyperref}

\usepackage{placeins}



\newcommand{\ua}[1]{_\symup{#1}}
\newcommand{\su}[1]{\symup{#1}}

\newcounter{exercise}
\newenvironment{exercise}
[2]
{\addtocounter{exercise}{1}{\bfseries{Aufgabe \arabic{exercise}:~#1}\hfill(#2 Punkte)}\newline}
{\medskip}

\setlength{\parindent}{0mm}
\begin{document}
{\large\bfseries Präsenzblatt zur Vorlesung\hfill\thesemester}\\ %/thesheet
{\large\bfseries Theoretische Physik I\hfill\theprofessor}\\
%{\large\bfseries Abgabe: bis \thehandover}\\
\textbf{Webseite zur Vorlesung: \\}
\url{https://moodle.tu-dortmund.de/course/view.php?id=9519} \\
\rule{\columnwidth}{0.1ex}
\medskip

\begin{exercise}{Schräger Wurf}{0}
  Betrachten Sie den zweidimensionalen schrägen Wurf.
  Zum Zeitpunkt $t_0 = \SI{0}{\second}$ befindet sich
  ein Teilchen in den Koordinaten $x_0 = \SI{5}{\meter}$ und $y_0 = h = \SI{5}{\meter}$.

    \begin{enumerate}
        \item Wie sieht die Bewegung für $v_x\neq 0$ und $v_y\neq 0$ aus?\\
        Berechnen Sie die Bahnkurve $\vec{r}$ des Teilchens in Abhängigkeit von $x$. Skizzieren Sie die beschriebene Situation.
        \item Wo befindet sich das Teilchen für $x = \SI{10}{\meter}$, wenn $v_x = \num{5}\frac{\si{\meter}}{\si{\second}}$ und $v_y = \num{-7}\frac{\si{\meter}}{\si{\second}}$ betragen?
        \item Diskutieren Sie die Fälle:
        \begin{itemize}
          \item[(i)] $v_x > 0$ und $v_y > 0$
          \item[(ii)] $v_x > 0$ und $v_y = 0$
          \item[(iii)] $v_x = v_y = 0$
        \end{itemize}
        Welchen Bewegungen entsprechen die Fälle (i) - (iii)?
    \end{enumerate}
\end{exercise}

\begin{exercise}{Wegintegrale}{0}
  Gegeben sei das folgende Vektorfeld:
  \begin{equation*}
    \vec{F}(x,y) = \left( \begin{array}{c} y\\\ x^2\  \end{array}\right)
  \end{equation*}
  Berechnen sie das Linienintegral von (0,1) zu (2,0) entlang der folgenden Wege.
  \begin{enumerate}
    \item $y = \frac{1}{4}(x-2)^2$\\
    \item $y = \frac{1}{1-e^{-2}}(e^{-x}-e^{-2})$\\
    \item $y = 1 - \frac{1}{2}x$\\
    \item $y = 1 - \frac{1}{4}x^2$
  \end{enumerate}
\end{exercise}

\end{document}
