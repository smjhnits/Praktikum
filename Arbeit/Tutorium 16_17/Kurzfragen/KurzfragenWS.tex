% bei Standalone in documentclass noch:
% \RequirePackage{luatex85}

\documentclass[captions=tableheading, ]{scrartcl}
%paper = a5 für andere optinen
% titlepage= firstiscover
% bibliography=totoc für bibdateien
% parskip=half  Veränderung um Absätze zu verbessern

\usepackage{scrhack} % nach \documentclass
\usepackage[aux]{rerunfilecheck}
\usepackage{polyglossia}
\usepackage[style=alphabetic]{biblatex} % mit [style = alphabetic oder numeric]
% nach polyglossia
\setmainlanguage{german}

\usepackage{scrpage2}   %Kopf und Fußzeilen
\pagestyle{scrheadings}  %\ihead{Kopfzeile innen} \chead{Kopfzeile Mitte} \ohead{Kopfzeile außen}
                         %\ifoot{Fußzeile innen} \cfoot{Fußzeile Mitte} \ofoot{Fußzeile außen}
\clearscrheadfoot

\usepackage[autostyle]{csquotes}
\usepackage{amsmath} % unverzichtbare Mathe-Befehle
\usepackage{amssymb} % viele Mathe-Symbole
\usepackage{mathtools} % Erweiterungen für amsmath
\usepackage{fontspec} % nach amssymb
% muss ins document: \usefonttheme{professionalfonts} % für Beamer Präsentationen

\usepackage[
math-style=ISO,    % \
bold-style=ISO,    % |
sans-style=italic, % | ISO-Standard folgen
nabla=upright,     % |
partial=upright,   % /
]{unicode-math} % "Does exactly what it says on the tin."
\setmathfont{Latin Modern Math}
% \setmathfont{Tex Gyre Pagella Math} % alternativ

\usepackage[
% die folgenden 3 nur einschalten bei documenten
locale=DE,
separate-uncertainty=true, % Immer Fehler mit ±
per-mode=symbol-or-fraction, % m/s im Text, sonst \frac
]{siunitx}

% alternativ:
% per-mode=reciprocal, % m s^{-1}
% output-decimal-marker=., % . statt , für Dezimalzahlen

\usepackage[
version=4,
math-greek=default,
text-greek=default,
]{mhchem}

\usepackage[section, below]{placeins}
\usepackage{caption} % Captions schöner machen
\usepackage{graphicx}
\usepackage{grffile}
\usepackage{subcaption}

% \usepackage{showframe} Wenn man die Ramen sehen will

\usepackage{float}
\floatplacement{figure}{htbp}
\floatplacement{table}{htbp}

\usepackage{booktabs}
 \addbibresource{lit.bib}

 \usepackage{microtype}
 \usepackage{xfrac}

 \usepackage{expl3}
 \usepackage{xparse}

 % \ExplSyntaxOn
 % \NewDocumentComman \I {}  %Befehl\I definieren, keine Argumente
 % {
 %    \symup{i}              %Ergebnis von \I
 % }
 % \ExplSyntaxOff

 \usepackage{pdflscape}
 \usepackage{mleftright}

 % Mit dem mathtools-Befehl \DeclarePairedDelimiter können Befehle erzeugen werden,
 % die Symbole um Ausdrücke setzen.
 % \DeclarePairedDelimiter{\abs}{\lvert}{\rvert}
 % \DeclarePairedDelimiter{\norm}{\lVert}{\rVert}
 % in Mathe:
 %\abs{x} \abs*{\frac{1}{x}}
 %\norm{\symbf{y}}

 % Für Physik IV und Quantenmechanik
 \DeclarePairedDelimiter{\bra}{\langle}{\rvert}
 \DeclarePairedDelimiter{\ket}{\lvert}{\rangle}
 % <name> <#arguments> <left> <right> <body>
 \DeclarePairedDelimiterX{\braket}[2]{\langle}{\rangle}{
 #1 \delimsize| #2
 }

 \usepackage{tikz}
 \usepackage{tikz-feynman}


% \multicolumn{#Spalten}{Ausrichtung}{Inhalt}

\usepackage[unicode]{hyperref}
\usepackage{bookmark}


\ihead{Tu Dortmund}
\chead{Kurzfragen}
\ohead{Tutorium Pape/Nitschke}

\begin{document}

\begin{enumerate}
  \item Wie lauten die drei Newtonschen Axiome?
  \item Was besagt das Superpositionsprinzip?
  \item Wie lautet die allgemeine Weg-Zeit Funktion für eine unbeschleunigte
  und eine beschleunigte Bewegung?
  \item Wann ist ein Kraftfeld $\vec{F(\vec{r})}$ konservativ? Gib mindestens 3 Kriterien
  an. Ist das Kraftfeld $\vec{F(\vec{r})} = \vec{r}/|\vec{r}|^5$ konservativ?
  \item Welche Beziehung besteht zwischen einem konservativem Kraftfeld und seinem
  Potential? Berechne das Potential $V(\vec{r})$ für das Kraftfeld $\vec{F(\vec{r})}
  = y\vec{e_x} + x\vec{e_y}$ an.
  \item Welche Koordinatensysteme, abgesehen vom Karthesischem kennst du?
  Gib jeweils die dazugehörige Parametrisierung an.
  \item Welche Scheinkräfte können in einem gleichförmig beschleunigten Bezugssystem
  auftreten? Gib die Formeln mit an.
  \item Du kaufst eine neue Wäscheschleuder. Solltest du lieber die mit 1.5-fachem
  Radius oder die mit 1.5-facher Umdrehungszahl nehmen? Begründe deine Entscheidung.
  \item Wie ist die Standardabweichung einer Stichprobe definiert? Wie ist die
  Standardabweichung des Mittelwertes definiert?
  \item Bei welchem Stoß bleibt die Energie erhalten und bei welchem nicht?
  \item Was bedeutet Energieerhaltung und wann gilt diese? Gib eine Situation an,
  bei der die Energieerhaltung \underline{nicht} gilt.
  \item In welche Richtung wirkt die Corioliskraft auf einen Zug, der sich entlang
  des Äquators bewegt?
  \item Du willst dein Raumschiff fotografieren. Nun schweben du aber mit der
  schweren Kamera in der Hand 50 Meter neben dem Raumschiff und weißt nicht, wie
  du zurückkommen sollst. Oder doch?
  \item Wie lautet das Newtonsche Gesetz ($\vec{\dot{p}} = \vec{F}$) für Drehbewegungen?
  \item Wie lauten die verschiedenen Lösungen eines gedämpften Oszillators?
  Skizziere alle Lösungen.
  \item Wie lautet die Kraftgleichung einer Feder mit Federkonstante k,
  und wie lautet die dazugehörige Formel der potentiellen Energie?
  \item Wie lautet die DGL eines harmonischen Oszillators? Gib eine allgemeine
  Lösung mit an.
  \item Was sind Inertialsysteme? Welche Transformation überführt Inertialsysteme
  ineinander?
\end{enumerate}

\newpage

\begin{enumerate}
  \item Wie lauten die drei Newtonschen Axiome?
  \item Was besagt das Superpositionsprinzip?
  \item Wie lautet die allgemeine Weg-Zeit Funktion für eine unbeschleunigte
  und eine beschleunigte Bewegung?
  \item Wann ist ein Kraftfeld $\vec{F(\vec{r})}$ konservativ? Gib mindestens 3 Kriterien
  an. Ist das Kraftfeld $\vec{F(\vec{r})} = \vec{r}/|\vec{r}|^5$ konservativ?
  \item Welche Beziehung besteht zwischen einem konservativem Kraftfeld und seinem
  Potential? Berechne das Potential $V(\vec{r})$ für das Kraftfeld $\vec{F(\vec{r})}
  = y\vec{e_x} + x\vec{e_y}$ an.
  \item Welche Koordinatensysteme, abgesehen vom Karthesischem kennst du?
  Gib jeweils die dazugehörige Parametrisierung an.
  \item Welche Scheinkräfte können in einem gleichförmig beschleunigten Bezugssystem
  auftreten? Gib die Formeln mit an.
  \item Du kaufst eine neue Wäscheschleuder. Solltest du lieber die mit 1.5-fachem
  Radius oder die mit 1.5-facher Umdrehungszahl nehmen? Begründe deine Entscheidung.
  \item Wie ist die Standardabweichung einer Stichprobe definiert? Wie ist die
  Standardabweichung des Mittelwertes definiert?
  \item Bei welchem Stoß bleibt die Energie erhalten und bei welchem nicht?
  \item Was bedeutet Energieerhaltung und wann gilt diese? Gib eine Situation an,
  bei der die Energieerhaltung \underline{nicht} gilt.
  \item In welche Richtung wirkt die Corioliskraft auf einen Zug, der sich entlang
  des Äquators bewegt?
  \item Du willst dein Raumschiff fotografieren. Nun schweben du aber mit der
  schweren Kamera in der Hand 50 Meter neben dem Raumschiff und weißt nicht, wie
  du zurückkommen sollst. Oder doch?
  \item Wie lautet das Newtonsche Gesetz ($\vec{\dot{p}} = \vec{F}$) für Drehbewegungen?
  \item Wie lauten die verschiedenen Lösungen eines gedämpften Oszillators?
  Skizziere alle Lösungen.
  \item Wie lautet die Kraftgleichung einer Feder mit Federkonstante k,
  und wie lautet die dazugehörige Formel der potentiellen Energie?
  \item Wie lautet die DGL eines harmonischen Oszillators? Gib eine allgemeine
  Lösung mit an.
  \item Was sind Inertialsysteme? Welche Transformation überführt Inertialsysteme
  ineinander?
\end{enumerate}

\end{document}
