\input{"../Praeambel_doc.tex"}

\ihead{Tu Dortmund - 21.07.2017}
\chead{Kurzfragen Physik II}
\ohead{Tutorium Pape/Nitschke}

\begin{document}

\begin{enumerate}
  \item Wie ist der Fluß durch eine geschlossene Oberfäche in einem quellenfreien Feld? (Begründung) \\
  % $\emph{0, Satz von Gauß}$ \\

  \item Wie ist die Coulombkraft definiert? \\
  % $F_{1,2} = \frac{Q_1Q_2}{4\pi\epsilon_0r^2}$ \\

  \item Wie ist das Potential zu einem elektrischen Feld definiert? \\
  % $vec{E} = - \nabla \Phi$  \\

  \item Leite die Poisson-Gleichung her. \\
  %siehe vorherige kurzfrage: $\nabla \vec{E} = - \nabla \nabla \Phi = - \frac{\rho_0}{\epsilon_0}$
  % Sebastian

  \item Berechne $\int_{0}^{\infty} \delta (x^2 - 6x - 16) \cdot f(x) \symup{dx}$. \\
  % Ergebnis: f(8)/10, da man innere funktion beachten muss: Ableitung bei 8

  \item Wie lautet die Formel für die Fouriertransformation einer Funktion f(x)?\\
  % $(F f)(y) = \frac{1}{\sqrt{2\pi}} \cdot \int_{-\infty}^{\infty} f(x) e^{-iyx} \symup{dx} $

  \item Welche Funktion ist unter der Fouriertransformation invariant?\\
  % blabla

  \item Welche Form des Magnetismus kommt bei allen Molekülen mit ungepaarten
  Elektronenpaaren vor? Bei welchen Atomen ist der Diamagnetismus
  am besten messbar? (Begründung) \\
  % Paramagnetismus.  bei Edelgasen Diamagnetismus

  \item Skiziere und Beschrifte die Hystereseschleife. \\
  % Sebastian

  \item Skizziere die Freqeunzabhängigkeit der Impedanz eines Widerstandes, einer Spule und eines Kondensators
  bei einer anliegenen Wechselspannung.\\

  \item Ist die Phase in einer Schaltung mit ohmschem Widerstand und Kondensator positiv oder
  negativ bezogen auf die Spannung ? (Schreibe die zugehörige Merkregel hin) \\
  %Beim Kondensator geht der Strom vor

  \item Sind das Vektorpotential und das Potential des Elektrischen Feldes eindeutig bestimmt? (Begründung)\\
  % Nein, Eichfreiheit : \\
  % $\Phi' = \Phi - \partial_t \Lambda$ \\
  % $\vec{A}' = vec{A} + \nabla \Lambda$ \\

  \item Wie sieht die Coulomb-Eichung aus ? \\
  % $\vec{E} = - \nabla \Phi \partial_t \vec{A}(\vec{r},t)$

  \item Wie bewegt sich ein magnetischer Dipol im homogenen elektrischen und magnetischen Feld? \\
  % nichts im elektrischen, ausrichtung im magnetischen mit anfänglichem drehmoment

  \item Berechne das Magnetfeld einer Torroidspule und fertige eine Skizze an. \\
  % $B = \frac{\mu_0NI}{2\pi r}$

  \item Was wird unter welcher Annahme bei der Multipolentwicklung genähert? (Elektrostatik) \\
  % Potential, Abstand viel größer als Ladungsausdehnung (weit weg)

  \item Was zeichnet ein lineares Medium aus? \\
  % linearer Abhang von E und D

  \item Welche Komponenten einer EM-Welle sind an Grenzflächen stetig ? \\
  % E-Feld: tangential, B-Feld: normal

  \item Gebe die Wellengleichung und eine Lösung dieser an. \\
  % $ \frac{a}{c^2} \partial_t^2 \Psi - \increment \Psi = 0 $ \\
  % $\Psi = f_+(kx+\omega t) + f_-(kx-\omega t) $ \\

  \item Welchen geometrischen Zusammenhang haben B, E und k bei einer EM-Welle? \\
  % $E \perp B \perp k $

  \item Du stehst unter einem vertikal ausgerichteten Dipolsender. Empfängst du ein Signal? (Begründe) \\
  % Winkelabhängigkeit sin

  \item Welche Ladungen strahlen? \\
  % Beschleunigte

  \item Skizziere den Lichtkegel im Minkowski-Raum und beschrifte ihn. \\

  \item Wie sieht die Minkowski Metrik aus und wie lautet die Beziehung zwischen kontra- und kovarianten Vektoren? \\

  \item Wie lautet das Faraday-Paradoxon? Wie wird das Paradoxon erklärt? \\
  % kreisförmiger Permanentmagnet: bewegte Metallplatte : Spannung \\
  % bewegter Magnet: keine Spannung; beides bewegend: Spannung \\
  % Erklärung: Magnetfeld rotationsunabhängig

\end{enumerate}

\end{document}
