\section{Auswertung}

Für die Berechnung
der Wellenlängen und Energien werden die Formeln \eqref{eqn:Wellenlänge}
und \eqref{eqn:Winkel} verwendet. Dabei werden sie in
folgende Form umgestellt, wobei n aus Formeln \eqref{eqn:Lambda} das betrachtete
Beugungsmaximum beschreibt (welches im Folgenden immer das 1. ist).

\begin{align}
  \label{eqn:Lambda}
  \lambda\ua{min} &= 2 \cdot d \cdot \sin{\theta} \cdot \frac{1}{n} \\
  \label{eqn:Energie}
  E\ua{max} &= \frac{h \cdot c}{\lambda} \\
\end{align}

\subsection{Überprüfung der Bragg-Bedingung}

Bei der Überprüfung der Bragg-Bedingung werden in dem verwendeten Messprogramm
folgende Einstellungen für den Kristallwinkel $\vartheta$, das Zählrohrwinkelintervall
$\alpha\ua{GM}$, den Winkelzuwachs $\increment\alpha$ sowie die Integrationszeit
$\increment t$ gewählt:

\begin{align*}
  \vartheta        &= 14 ° \\
  \increment\alpha &= 0.1 ° \\
  \increment t     &= \SI{10}{s} \\
  \alpha\ua{GM}    &\in [26 °, 30 °] .
\end{align*}

Mithilfe von Python und der Daten aus Abbildung \ref{fig:MessungA} wird dann das Maximum der
Impulsrate bestimmt. Der Winkel wird dann zuerst in die minimale Wellenlänge (Formel
\eqref{eqn:Lambda}) sowie
anschließend in die maximale Energie (Formel \eqref{eqn:Energie}) umgerechnet,
so dass sich folgende Werte ergeben:

\begin{align*}
  \vartheta\ua{3.1,max} &= 28.6 ° \\
  \lambda\ua{3.1,min} &= \SI{1.93e-10}{m} \\
  E\ua{3.1, max} &= \SI{6.43}{eV} .
\end{align*}

\begin{figure}
  \centering
  \includegraphics[width = 0.8\textwidth]{Python/MessungA.pdf}
  \caption{Gemessene Impulsrate bei Messung 3.1 .}
  \label{fig:MessungA}
\end{figure}

Im Vergleich mit dem Erwartungswert von $\vartheta$ = 14 ° ergibt sich eine
Abweichung von $\increment\vartheta$ = 0.6 ° sowie prozentual eine Abweichung
von $\increment\vartheta$ = 2.1 $\%$.

\subsection{Das Emissionsspektrum einer Cu-Röntgenröhre}

Für die Untersuchung des Emissionsspektrums einer Cu-Röntgenröhre wurden in dem
2:1 Koppelmodus die folgenden Einstellungen gewählt:

\begin{align*}
  \vartheta        &\in [4 °, 26 °]  \\
  \increment\vartheta &= 0.2 ° \\
  \increment t     &= \SI{5}{s} .
\end{align*}

Die erhaltenen Daten sind in Abbildung \ref{fig:MessungB} dargestellt. Zur Bestimmung der
$K\ua{\alpha}$- und $K\ua{\beta}$-Linie wird zuerst der Untergrund aus den Daten
herausgerechnet. Dafür wird an die Werte in Tabelle \ref{tab:Untergrund} ein Polynom 3.Ordnung
gefittet, für das sich die folgenden Parameter ergeben:

\begin{align*}
  U(x) &= A\cdot x^3 + B\cdot x^2 + C\cdot x + D \\
  A &=  (0.27 \pm 0.02) \, \si{\frac{Imp}{s\cdot \alpha^3}} \\
  B &= (-12.2 \pm -0.7) \, \si{\frac{Imp}{s\cdot \alpha^2}} \\
  C &= (170 \pm 9) \, \si{\frac{Imp}{s\cdot \alpha}} \\
  D &= (-525 \pm 31) \, \si{\frac{Imp}{s}} .
\end{align*}

Des weiteren wird an die Daten aus Tabelle \ref{Gauß1} und \ref{Gauß2} jeweils eine Gauß-Funktion
der Form \eqref{eqn:GaußFit} gefittet (Formel \eqref{eqn:GaußFit}). Dabei werden mithilfe der
Werte des Untergrunds noch die Daten der beiden Tabellen vor dem Fitten angepasst.

\begin{equation}
  G(x) = A \cdot \exp{ \frac{-(\vartheta-\vartheta_0)^2}{2\cdot s^2}}
  \label{eqn:GaußFit}
\end{equation}

$\vartheta_0$ und $s$ geben dabei die Lage der jeweiligen Linie sowie die
Halbwertsbreite an. Damit ergeben sich für die $K\ua{\alpha}$-Linie die folgenden
Parameter:

\begin{align*}
  A\ua{K\ua{\alpha}} &= (4.7 \pm 0.3) \cdot 10^3 \, \si{\frac{Imp}{s}} \\
  \vartheta\ua{K\ua{\alpha}} &= \vartheta_0 =  (22.57 \pm 0.01) ° \\
  \lambda\ua{K\ua{\alpha}} &= (1.5458 \pm 0.0008) \cdot 10^{-10} \, \si{m} \\
  E\ua{K\ua{\alpha}} &= (8020 \pm 4) \, \si{eV} \\
  s\ua{K\ua{\alpha}} &= (0.19 \pm 0.1) ° .
\end{align*}

Analog ergeben sich für die $K\ua{\beta}$-Linie die folgenden Werte :

\begin{align*}
  A\ua{K\ua{\beta}} &= (1.62 \pm 0.04) \cdot 10^3 \, \si{\frac{Imp}{s}} \\
  \vartheta\ua{K\ua{\beta}} &= \vartheta_0 =  (20.289 \pm 0.003) ° \\
  \lambda\ua{K\ua{\beta}} &= (1.3967 \pm 0.0002) \cdot 10^{-10} \, \si{m} \\
  E\ua{K\ua{\beta}} &= (8876 \pm 1) \, \si{eV} \\
  s\ua{K\ua{\beta}} &= (0.136 \pm 0.005) ° .
\end{align*}

Mithilfe der Formeln \ref{} und \ref{} können dann die Abschirmkonstanten der K-Kante
und der L-Kante bestimmt werden, für die sich die folgenden Werte ergeben:

\begin{align*}
\sigma\ua{K} &= (3.452 \pm 0.002) \\
\sigma\ua{L} &= (17.77 \pm 0.03)
\end{align*}

Die beiden Halbwertsbreiten können nun ebenfalls in Energien umgerechnet werden.
Theoretisch sollten die beiden Energien dann exakt gleich sein. Da dies in der
Praxis jedoch nicht der Fall ist, kann mithilfe des Quotienten der beiden Energien
eine Güte des Versuches angegeben werden, dessen Optimalwert bei 1 liegt. Für
den Versuch ergibt sich damit das folgende Auflösungsvermögen:

\begin{align*}
  E\ua{s\ua{K\ua{\alpha}}} &= (9.2 \pm 0.7) \cdot 10^2 \, \si{eV} \\
  E\ua{s\ua{K\ua{\beta}}} &= (1.3 \pm 0.04) \cdot 10^3 \, \si{eV} \\
  \phi &= \frac{E\ua{s\ua{K\ua{\beta}}}}{E\ua{s\ua{K\ua{\alpha}}}} = 1.41 \pm 0.11 .
\end{align*}

\begin{figure}
  \centering
  \includegraphics[width = 0.8\textwidth]{Python/MessungB.pdf}
  \caption{Gemessene Impulsrate einer Cu-Röntgenröhre}
  \label{fig:MessungB}
\end{figure}

\begin{table}
  \centering
  \caption{Die für den Untergrund verwendeten Messdaten. }
  \label{tab:Untergrund}
  \begin{tabular}{c c | c c | c c}
    \toprule
    2$\vartheta$ in deg & N in $\frac{\su{Imp}}{\su{s}}$ & 2$\vartheta$ in deg &
    N in $\frac{\su{Imp}}{\su{s}}$ & 2$\vartheta$ in deg & N in $\frac{\su{Imp}}{\su{s}}$ \\
    \midrule
     8.0	& 45.0  & 21.2 & 227.0 & 34.4	& 153.0 \\
     8.4	& 38.0  & 21.6 & 245.0 & 34.8	& 153.0 \\
     8.8	& 31.0  & 22.0 & 254.0 & 35.2 & 147.0 \\
     9.2	& 31.0  & 22.4 & 247.0 & 35.5 & 153.0 \\
     9.6	& 33.0  & 22.8 & 253.0 & 36.0 & 155.0 \\
     10.0	& 34.0  & 23.2 & 240.0 & 36.4 & 152.0 \\
     10.4	& 50.0  & 23.6 & 232.0 & 36.8 & 130.0 \\
     10.8	& 64.0  & 24.0 & 233.0 & 37.2 & 134.0 \\
     11.2	& 73.0  & 24.4 & 239.0 & 37.5 & 137.0 \\
     11.6	& 85.0  & 24.8 & 241.0 & 38.0 & 123.0 \\
     12.0	& 105.0 & 25.2 & 239.0 & 38.4 & 124.0 \\
     12.4	& 114.0 & 25.6 & 234.0 & 38.8 & 119.0 \\
     12.8	& 110.0 & 26.0 & 250.0 & 39.2 & 126.0 \\
     13.2	& 123.0 & 26.4 & 224.0 & 39.5 & 136.0 \\
     13.6	& 121.0 & 26.8 & 213.0 & 42.4 & 152.0 \\
     14.0	& 136.0 & 27.2 & 207.0 & 42.8 & 153.0 \\
     14.4	& 149.0 & 27.6 & 197.0 & 43.2 & 147.0 \\
     14.8	& 154.0 & 28.0 & 176.0 & 43.6 & 150.0 \\
     15.2	& 163.0 & 28.4 & 181.0 & 44.0 & 167.0 \\
     15.6	& 172.0 & 28.8 & 197.0 & 46.8 & 124.0 \\
     16.0	& 181.0 & 29.2 & 182.0 & 47.2 & 108.0 \\
     16.4	& 187.0 & 29.6 & 177.0 & 47.6 & 100.0 \\
     16.7	& 204.0 & 30.0 & 176.0 & 48.0 & 86.0  \\
     17.2	& 200.0 & 30.4 & 174.0 & 48.4 & 92.0  \\
     17.6	& 188.0 & 30.8 & 180.0 & 48.8 & 86.0  \\
     18.0	& 197.0 & 31.2 & 176.0 & 49.2 & 86.0  \\
     18.4	& 202.0 & 31.6 & 181.0 & 49.6 & 77.0  \\
     18.8	& 215.0 & 32.0 & 173.0 & 50.0 & 72.0  \\
     19.2	& 216.0 & 32.4 & 159.0 & 50.4 & 75.0  \\
     19.6	& 227.0 & 32.8 & 169.0 & 50.8 & 61.0  \\
     20.0	& 236.0 & 33.2 & 158.0 & 51.2 & 71.0  \\
     20.4	& 235.0 & 33.6 & 167.0 & 51.6 & 64.0  \\
     20.8	& 235.0 & 34.0 & 160.0 & 52.0 & 65.0  \\
    \bottomrule
  \end{tabular}
\end{table}


\begin{table}
  \centering
  \caption{Die für den Gauß-Fit vorgesehenen Messdaten. }
  \label{tab:Gauß1}
  \begin{tabular}{c c | c c | c c}
    \toprule
    2$\vartheta$ in deg & N in $\frac{\su{Imp}}{\su{s}}$ & 2$\vartheta$ in deg &
    N in $\frac{\su{Imp}}{\su{s}}$ & 2$\vartheta$ in deg & N in $\frac{\su{Imp}}{\su{s}}$ \\
    \midrule
    32.0 & 173.0 & 38.8 & 119.0  & 45.5 & 3634.0 \\
    32.4 & 159.0 & 39.2 & 126.0  & 46.0 & 219.0  \\
    32.8 & 169.0 & 39.5 & 136.0  & 46.4 & 138.0  \\
    33.2 & 158.0 & 40.0 & 255.0  & 46.8 & 124.0  \\
    33.6 & 167.0 & 40.4 & 1483.0 & 47.2 & 108.0  \\
    34.0 & 160.0 & 40.8 & 1282.0 & 47.6 & 100.0  \\
    34.4 & 153.0 & 41.2 & 349.0  & 48.0 & 86.0   \\
    34.8 & 153.0 & 41.5 & 199.0  & 48.4 & 92.0   \\
    35.2 & 147.0 & 42.0 & 163.0  & 48.8 & 86.0   \\
    35.5 & 153.0 & 42.4 & 152.0  & 49.2 & 86.0   \\
    36.0 & 155.0 & 42.8 & 153.0  & 49.6 & 77.0   \\
    36.4 & 152.0 & 43.2 & 147.0  & 50.0 & 72.0   \\
    36.8 & 130.0 & 43.6 & 150.0  & 50.4 & 75.0   \\
    37.2 & 134.0 & 44.0 & 167.0  & 50.8 & 61.0   \\
    37.5 & 137.0 & 44.4 & 267.0  & 51.2 & 71.0   \\
    38.0 & 123.0 & 44.8 & 3928.0 & 51.6 & 64.0   \\
    38.4 & 124.0 & 45.2 & 4280.0 & 52.0 & 65.0   \\
    \bottomrule
  \end{tabular}
\end{table}


\begin{table}
  \centering
  \caption{Die für den rechten Gauß-Fit vorgesehenen Messdaten. }
  \label{tab:Gauß2}
  \begin{tabular}{c c | c c | c c}
    \toprule
    $\vartheta$ in deg & N in $\frac{\su{Imp}}{\su{s}}$ & $\vartheta$ in deg &
    N in $\frac{\su{Imp}}{\su{s}}$ & $\vartheta$ in deg & N in $\frac{\su{Imp}}{\su{s}}$ \\
    \midrule
    41.5 & 199.0  & 45.2 & 4280.0 & 48.8 & 86.0 \\
    42.0 & 163.0  & 45.5 & 3634.0 & 49.2 & 86.0 \\
    42.4 & 152.0  & 46.0 & 219.0  & 49.6 & 77.0 \\
    42.8 & 153.0  & 46.4 & 138.0  & 50.0 & 72.0 \\
    43.2 & 147.0  & 46.8 & 124.0  & 50.4 & 75.0 \\
    43.6 & 150.0  & 47.2 & 108.0  & 50.8 & 61.0 \\
    44.0 & 167.0  & 47.6 & 100.0  & 51.2 & 71.0 \\
    44.4 & 267.0  & 48.0 & 86.0   & 51.6 & 64.0 \\
    44.8 & 3928.0 & 48.4 & 92.0   & 52.0 & 65.0 \\
    \bottomrule
  \end{tabular}
\end{table}


\newpage

\subsection{Das Absorptionsspektrum}

Für die Untersuchung des Absorptionsspektrums verschiedener Materialien wurden in dem
2:1 Koppelmodus die folgenden Einstellungen gewählt:

\begin{align*}
  \increment\vartheta &= 0.1 ° \\
  \increment t     &= \SI{20}{s} .
\end{align*}

Das Messintervall wurde dabei bei jedem Material unabhängig gewählt. Für die
Berechnung der Abschirmkonstante wird Formel \eqref{eqn:Bindungsenergien} umgestellt,
so dass sich die folgende Relation ergibt.  $\alpha$
beschreibt dabei die Sommerfeldsche Feinstrukturkonstante.

\begin{equation}
  \sigma\ua{K} = Z - \sqrt{ \frac{E\ua{max}}{R\ua{\infty}} - \frac{\alpha^4 Z^4}{4}}
  \label{eqn:Abschirmkonst}
\end{equation}

\subsubsection{Germanium}

Bei Germanium wurde ein Messintervall von $\vartheta$ $\in$ [14 °, 18 °] gewählt.
Mithilfe der Daten aus Tabelle \ref{tab:Germanium} (grafisch siehe Abbildung
\ref{fig:Germanium})
kann für Germanium nun die Absorptionsenergie (Formel \eqref{eqn:Energie})
der K-Kante bestimmt werden sowie
die daraus resultierende Abschirmkonstante nach Formel \eqref{eqn:Abschirmkonst}.
Die für die Berechnung
der Werte verwendete Ordnungszahl ist 32.

\begin{align*}
  \vartheta\ua{GE} &= 16.15 ° \\
  \lambda\ua{GE,min} &= 1.12 \cdot 10^{-10} \, \si{m} \\
  E\ua{GE,max} &= 11066 \, \si{eV} \\
  \sigma\ua{GE} &= 3.72
\end{align*}

\begin{figure}
  \centering
  \includegraphics[width = 0.8\textwidth]{Python/Germanium.pdf}
  \caption{Gemessenes Absorptionsspektrum bei Germanium.}
  \label{fig:Germanium}
\end{figure}


\input{"Tabellen/Germanium.tex"}

Da die Berechnung der Werte für alle Materialien gleich erfolgt, werden im Folgenden
lediglich die Tabellen, Graphen sowie Ergebniss für Zink, Zirkonium, Strontium und
Brom vorgestellt.

\newpage

\subsubsection{Brom}

Für Brom wird ein Messintervall von $\vartheta$ $\in$ [11 °, 15 °] gewählt.
Die Ordnungszahl ist 35. Die gemessenen Werte sind in Tabelle \ref{tab:Brom}
sowie grafisch in Abbildung \ref{fig:Brom} dargestellt.

\begin{align*}
  \vartheta\ua{BR} &= 13.2 ° \\
  \lambda\ua{BR,min} &= 9.20 \cdot 10^{-10} \, \si{m} \\
  E\ua{BR,max} &= 13480 \, \si{eV} \\
  \sigma\ua{BR} &= 3.84
\end{align*}

\begin{figure}
  \centering
  \includegraphics[width = 0.8\textwidth]{Python/Brom.pdf}
  \caption{Gemessenes Absorptionsspektrum bei Brom.}
  \label{fig:Brom}
\end{figure}

\input{"Tabellen/Brom.tex"}

\newpage

\subsubsection{Zirkonium}

Das gewählte Messintervall von Zirkonium ist $\vartheta$ $\in$ [8 °, 12 °]. Die
Ordnungszahl von Zirkonium ist 40. Alle Daten sind in Tabelle \ref{tab:Zirkonium}
sowie Abbildung \ref{fig:Zirkonium} dargestellt.

\begin{align*}
  \vartheta\ua{ZR} &= 9.85 ° \\
  \lambda\ua{ZR,min} &= 6.89 \cdot 10^{-11} \, \si{m} \\
  E\ua{ZR,max} &= 17993 \, \si{eV} \\
  \sigma\ua{ZR} &= 4.10
\end{align*}

\begin{figure}
  \centering
  \includegraphics[width = 0.8\textwidth]{Python/Zirkonium.pdf}
  \caption{Gemessenes Absorptionsspektrum bei Zirkonium.}
  \label{fig:Zirkonium}
\end{figure}

\begin{table}
  \centering
  \caption{Gemessene Impulsrate N bei Zirkonium.}
  \label{tab:Zirkonium}
  \begin{tabular}{c c | c c | c c}
    \toprule
    2$\vartheta$ in deg & N in $\frac{\su{Imp}}{\su{s}}$ & 2$\vartheta$ in deg &
    N in $\frac{\su{Imp}}{\su{s}}$ & 2$\vartheta$ in deg & N in $\frac{\su{Imp}}{\su{s}}$ \\
    \midrule
    16.0 & 83.0 & 18.8 & 76.0  & 21.6 & 188.0 \\
    16.2 & 82.0 & 19.0 & 69.0  & 21.7 & 190.0 \\
    16.4 & 87.0 & 19.2 & 76.0  & 22.0 & 194.0 \\
    16.6 & 84.0 & 19.4 & 96.0  & 22.2 & 188.0 \\
    16.7 & 85.0 & 19.6 & 132.0 & 22.4 & 180.0 \\
    17.0 & 84.0 & 19.8 & 155.0 & 22.6 & 186.0 \\
    17.2 & 85.0 & 20.0 & 178.0 & 22.8 & 187.0 \\
    17.4 & 83.0 & 20.2 & 185.0 & 23.0 & 180.0 \\
    17.6 & 83.0 & 20.4 & 184.0 & 23.2 & 174.0 \\
    17.7 & 79.0 & 20.6 & 186.0 & 23.4 & 175.0 \\
    18.0 & 82.0 & 20.8 & 192.0 & 23.6 & 169.0 \\
    18.2 & 76.0 & 21.0 & 181.0 & 23.8 & 166.0 \\
    18.4 & 76.0 & 21.2 & 192.0 & 24.0 & 161.0 \\
    18.6 & 75.0 & 21.4 & 187.0 &      &       \\
    \bottomrule
  \end{tabular}
\end{table}


\newpage

\subsubsection{Strontium}

Strontium besitzt die Ordnungszahl 38. Für die Messung wird ein Intervalle von
$\vartheta$ $\in$ [9 °, 13 °] betrachtet. Die Ergebnisse sind in Tabelle \ref{tab:Strontium}
und Abbildung \ref{fig:Strontium} dargestellt.

\begin{align*}
  \vartheta\ua{SR} &= 11.0 ° \\
  \lambda\ua{SR,min} &= 7.69 \cdot 10^{-11} \, \si{m} \\
  E\ua{SR,max} &= 16132 \, \si{eV} \\
  \sigma\ua{SR} &= 3.96
\end{align*}

\begin{figure}
  \centering
  \includegraphics[width = 0.8\textwidth]{Python/Strontium.pdf}
  \caption{Gemessenes Absorptionsspektrum bei Strontium.}
  \label{fig:Strontium}
\end{figure}

\begin{table}
  \centering
  \caption{Gemessene Impulsrate N bei Strontium.}
  \label{tab:Strontium}
  \begin{tabular}{c c | c c | c c}
    \toprule
    $\vartheta$ in deg & N in $\frac{\su{Imp}}{\su{s}}$ & $\vartheta$ in deg &
    N in $\frac{\su{Imp}}{\su{s}}$ & $\vartheta$ in deg & N in $\frac{\su{Imp}}{\su{s}}$ \\
    \midrule
    18.0 & 42.0 & 20.8 & 29.0  & 23.6 & 117.0 \\
    18.2 & 40.0 & 21.0 & 32.0  & 23.8 & 113.0 \\
    18.4 & 39.0 & 21.2 & 30.0  & 24.0 & 113.0 \\
    18.6 & 39.0 & 21.4 & 32.0  & 24.2 & 113.0 \\
    18.8 & 39.0 & 21.6 & 35.0  & 24.4 & 110.0 \\
    19.0 & 37.0 & 21.7 & 46.0  & 24.6 & 105.0 \\
    19.2 & 38.0 & 22.0 & 91.0  & 24.8 & 105.0 \\
    19.4 & 36.0 & 22.2 & 120.0 & 25.0 & 103.0 \\
    19.6 & 34.0 & 22.4 & 129.0 & 25.2 & 101.0 \\
    19.8 & 34.0 & 22.6 & 126.0 & 25.4 & 100.0 \\
    20.0 & 35.0 & 22.8 & 124.0 & 25.6 & 97.0  \\
    20.2 & 33.0 & 23.0 & 123.0 & 25.8 & 96.0  \\
    20.4 & 30.0 & 23.2 & 124.0 & 26.0 & 94.0  \\
    20.6 & 29.0 & 23.4 & 117.0 &      &       \\      
    \bottomrule
  \end{tabular}
\end{table}


\newpage

\subsubsection{Zink}

Für Zink wird ein Messintervall von $\vartheta$ $\in$ [17 °, 21 °] gewählt. Die
Ordnungszahl von Zink ist 30. Alle verwerteten Daten sind in Tabelle \ref{tab:Zink}
sowie grafisch in Abbildung \ref{fig:Zink} dargestellt.

\begin{align*}
  \vartheta\ua{SR} &= 18.58 ° \\
  \lambda\ua{SR,min} &= 1.28 \cdot 10^{-10} \, \si{m} \\
  E\ua{SR,max} &= 9663 \, \si{eV} \\
  \sigma\ua{SR} &= 3.55
\end{align*}

\begin{figure}
  \centering
  \includegraphics[width = 0.8\textwidth]{Python/Zink.pdf}
  \caption{Gemessenes Absorptionsspektrum bei Zink.}
  \label{fig:Zink}
\end{figure}

\begin{table}
  \centering
  \caption{Gemessene Impulsrate N bei Zink.}
  \label{tab:Zink}
  \begin{tabular}{c c | c c | c c}
    \toprule
    $\vartheta$ in deg & N in $\frac{\su{Imp}}{\su{s}}$ & $\vartheta$ in deg &
    N in $\frac{\su{Imp}}{\su{s}}$ & $\vartheta$ in deg & N in $\frac{\su{Imp}}{\su{s}}$ \\
    \midrule
    34.0 & 58.0 & 36.8 & 52.0 & 39.5 & 91.0   \\
    34.2 & 55.0 & 37.0 & 60.0 & 39.8 & 117.0  \\
    34.4 & 56.0 & 37.2 & 74.0 & 40.0 & 228.0  \\
    34.5 & 58.0 & 37.4 & 83.0 & 40.2 & 879.0  \\
    34.8 & 55.0 & 37.5 & 93.0 & 40.4 & 1033.0 \\
    35.0 & 53.0 & 37.8 & 93.0 & 40.5 & 960.0  \\
    35.2 & 55.0 & 38.0 & 95.0 & 40.8 & 895.0  \\
    35.4 & 53.0 & 38.2 & 97.0 & 41.0 & 639.0  \\
    35.5 & 53.0 & 38.4 & 91.0 & 41.2 & 207.0  \\
    35.8 & 51.0 & 38.6 & 87.0 & 41.4 & 136.0  \\
    36.0 & 51.0 & 38.8 & 85.0 & 41.5 & 127.0  \\
    36.2 & 52.0 & 39.0 & 84.0 & 41.8 & 114.0  \\
    36.4 & 48.0 & 39.2 & 86.0 & 42.0 & 113.0  \\
    36.5 & 51.0 & 39.4 & 86.0 &      &        \\
    \bottomrule
  \end{tabular}
\end{table}


\subsubsection{Rydbergenergie}
