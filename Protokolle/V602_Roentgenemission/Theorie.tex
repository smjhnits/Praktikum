\section{Theorie}

Der Versuch V602 untersucht das Emissionsspektrum einer $\ce{Cu}-$Röntgenröhre
und Absorptionsspektren verschiedener Absorber.

Röntgenstrahlen werden in einer Röntgenröhre erzeugt. Diese ist
evakuiert und beinhaltet eine Glühkathode und eine Anode.
Aufgrund des Glühelektrischeneffektes werden Elektronen aus der Kathode gelöst und
in Richtung der positiv geladenen Anode beschleunigt. Treffen die beschleunigten
Elektronen auf das Anodenmaterial auf entsteht Röntgenstrahlung, welche sich
aus dem kontinuierlichen Bremsspektrum und der charakteristischen Röntgenstrahlung
zusammensetzt. Die hier auftretende charakteristische Röntgenstrahlung ist
dem Anodenmaterial Kupfer zugeordnet. Das Bremsspektrum wird wegen des
Coulombfeldes des Atomkernes hervorgerufen. Beim Abbremsprozess wird ein Photon
emittiert. Die Energie des Photons entspricht dem Energieverlust des abgebremsten
Elektrons. Die Wellenlänge $\lambda$ des emittierten Photons lässt sich aus der Energie $E\ua{kin}$ über
den folgenden Zusammenhang ermitteln.

\begin{equation}
  \label{eqn:Wellenlänge}
  \lambda = \frac{h\cdot c}{E\ua{kin}}
\end{equation}

Dabei ist $c$ die Lichtgeschwindigkeit und $h$ das \emph{Plank'sche Wirkungsquantum}.
Die Wellenlänge wird minimal für den maximalen Wert der Energie.
Die knetische Energie ist maximal für $e_0U$, wodurch

\begin{equation}
  \label{eqn:Wellenlänge_minimal}
  \lambda\ua{min} = \frac{h\cdot c}{e_0U}.
\end{equation}

In \eqref{eqn:Wellenlänge_minimal} wird die gesamte kinetische Energie in
Strahlungsenergie umgewandelt.

Das charakteristische Spektrum entsteht aufgrund der Ionisierung des Anodenmaterials.
Dabei wird ein Elektron aus einer inneren Schale herausgelöst.
Die entstandene Leerstelle wird durch ein Elektron der äußeren Schale gefüllt.
Aufgrund der Energiedifferenz der Elektronenschalen wird beim Wechsel eines
äußeren Elektrons auf eine inner liegende Schale ein Röntgenquant mit der
Energie $h\nu = E\ua{m} - E\ua{n}$ emittiert. Die auftretenden Energien $E\ua{m}$
und $E\ua{n}$ stellen die Energien der verschiedenen Elektronenschalen dar.
