% bei Standalone in documentclass noch:
% \RequirePackage{luatex85}

\documentclass[captions=tableheading, titlepage= firstiscover, parskip = half , bibliography=totoc]{scrartcl}
%paper = a5 für andere optinen
% titlepage= firstiscover
% bibliography=totoc für bibdateien
% parskip=half  Veränderung um Absätze zu verbessern

\usepackage{scrhack} % nach \documentclass
\usepackage[aux]{rerunfilecheck}
\usepackage{polyglossia}
\usepackage[style=numeric, backend=biber]{biblatex} % mit [style = alphabetic oder numeric] nach polyglossia
\addbibresource{lit.bib}
\setmainlanguage{german}

\usepackage[autostyle]{csquotes}
\usepackage{amsmath} % unverzichtbare Mathe-Befehle
\usepackage{amssymb} % viele Mathe-Symbole
\usepackage{mathtools} % Erweiterungen für amsmath
\usepackage{fontspec} % nach amssymb
% muss ins document: \usefonttheme{professionalfonts} % für Beamer Präsentationen
\usepackage{longtable}

\usepackage[
math-style=ISO,    % \
bold-style=ISO,    % |
sans-style=italic, % | ISO-Standard folgen
nabla=upright,     % |
partial=upright,   % /
]{unicode-math} % "Does exactly what it says on the tin."
\setmathfont{Latin Modern Math}
% \setmathfont{Tex Gyre Pagella Math} % alternativ

\usepackage[
% die folgenden 3 nur einschalten bei documenten
locale=DE,
separate-uncertainty=true, % Immer Fehler mit ±
per-mode=symbol-or-fraction, % m/s im Text, sonst \frac
]{siunitx}

% alternativ:
% per-mode=reciprocal, % m s^{-1}
% output-decimal-marker=., % . statt , für Dezimalzahlen

\usepackage[
version=4,
math-greek=default,
text-greek=default,
]{mhchem}

\usepackage[section, below]{placeins}
\usepackage{caption} % Captions schöner machen
\usepackage{graphicx}
\usepackage{grffile}
\usepackage{subcaption}

% \usepackage{showframe} Wenn man die Ramen sehen will

\usepackage{float}
\floatplacement{figure}{htbp}
\floatplacement{table}{htbp}

\usepackage{mhchem} %chemische Symbole Beispiel: \ce{^{227}_{90}Th+}


\usepackage{booktabs}

 \usepackage{microtype}
 \usepackage{xfrac}

 \usepackage{expl3}
 \usepackage{xparse}

 % \ExplSyntaxOn
 % \NewDocumentComman \I {}  %Befehl\I definieren, keine Argumente
 % {
 %    \symup{i}              %Ergebnis von \I
 % }
 % \ExplSyntaxOff

 \usepackage{pdflscape}
 \usepackage{mleftright}

 % Mit dem mathtools-Befehl \DeclarePairedDelimiter können Befehle erzeugen werden,
 % die Symbole um Ausdrücke setzen.
 % \DeclarePairedDelimiter{\abs}{\lvert}{\rvert}
 % \DeclarePairedDelimiter{\norm}{\lVert}{\rVert}
 % in Mathe:
 %\abs{x} \abs*{\frac{1}{x}}
 %\norm{\symbf{y}}

 % Für Physik IV und Quantenmechanik
 \DeclarePairedDelimiter{\bra}{\langle}{\rvert}
 \DeclarePairedDelimiter{\ket}{\lvert}{\rangle}
 % <name> <#arguments> <left> <right> <body>
 \DeclarePairedDelimiterX{\braket}[2]{\langle}{\rangle}{
 #1 \delimsize| #2
 }

\setlength{\delimitershortfall}{-1sp}

 \usepackage{tikz}
 \usepackage{tikz-feynman}

 \usepackage{csvsimple}
 % Tabellen mit \csvautobooktabular{"file"}
 % muss in table umgebung gesetzt werden


% \multicolumn{#Spalten}{Ausrichtung}{Inhalt}

\usepackage{hyperref}
\usepackage{bookmark}
\usepackage[shortcuts]{extdash} %nach hyperref, bookmark

\newcommand{\ua}[1]{_\symup{#1}}
\newcommand{\su}[1]{\symup{#1}}


\title{Versuch 356}
\subtitle{Kettenschaltung mit LC-Gliedern}
\author{Sebastian Pape\\
        sepa@gmx.de \and
        Jonah Nitschke\\
        lejonah@web.de}
\date{Durchführung: 17.01.2017\\
      Abgabe: 24.01.2017}

\begin{document}
\maketitle

\section{Einleitung}

In dem folgenden Versuch geht es um die Betrachtung von Wellen mithilfe eines
Schwingkreises als Analogon zum verketteten harmonischen Oszillators in der
Mechanik. Dabei werden verschiedene charakteristische Größen der entstehenden
stehenden Wellen betrachtet, wie zum Beispiel Dispersionsrelation sowie Phasenverschub
zwischen der eingehenden und ausgehenden Spannung.

\section{Theorie}

\subsection{Die Dispersionsrelationen}

Im allgemeinen gibt die Dispersionsrelation an, in wie weit verschiedene physikalische
Größen Abhängigkeiten von der Frequenz aufweisen. Um diese Abhängigkeiten für
eine LC-Kettenschaltung zu bestimmen, wird zunächst mithilfe der Kirchoffschen
Regeln die Schwingungsgleichung hergeleitet und auf n Kettenglieder verallgemeinert.
Mit der Annahme, dass alle Kettenglieder mit der gleichen Frequenz $\omega$ schwingen
und lediglich beim Durchlaufen einen Phasenverschub $\theta$ erfahren, lässt sich
für die Kette folgende Dispersionsrelation herleiten:

\begin{equation}
  \omega^2 = \frac{2}{LC}(1-\cos{\theta}) .
  \label{eqn:RelationC}
\end{equation}

Aus dieser Formel lässt sich also die Phasenänderung pro Kettenglied in Abhängigkeit
der angelegten Frequenz bestimmen.

Die bei Formel \eqref{eqn:RelationC} bestimmte Dispersionsrelation lässt
sich etwas verallgemeinern, indem eine Kettenschaltung  mit alternierend eingebauten
Kondensatoren $C\ua{1}$ und $C\ua{2}$ verwendet wird. Bei der Lösung der analog
zu der obigen Schaltung bestimmten Schwingungsgleichung wird hierbei die Annahme
getroffen, dass alle Kettenglieder immernoch mit der gleichen Frequenz schwingen,
allerdings jedes zweite Kettenglied eine andere Amplitude aufweist. Somit ergibt
sich für die $LC\ua{1}C\ua{2}$-Kette folgende Dispersionsrelation:

\begin{equation}
  \omega\ua{1,2}^2 = \frac{1}{L} \left\{ \frac{1}{C\ua{1}} + \frac{1}{C\ua{1}}
  \right\} \, \pm \, \frac{1}{L} \sqrt{ \left\{ \frac{1}{C\ua{1}} + \frac{1}{C\ua{1}}
  \right\}^2  - \frac{4\cos(\theta)^2}{C\ua{1}C\ua{2}}} .
  \label{eqn:RelationC1C2}
\end{equation}

Wie der Formel \eqref{eqn:RelationC1C2} entnommen werden kann, bilden sich bei
dieser Dispersionsrelation zwei Äste, auch akustischer und optische Ast genannt
(siehe Abbildung ??). Zwischen den beiden Ästen existiert ein Bereich, welcher
bei der Schwingung nicht auftretenden Frequenzen beinhält. Während $\omega\ua{1}$
bei $\theta = \frac{\pi}{2}$ ein Maximum besitzt, tritt dort für $\omega\ua{2}$
ein Minimum auf (siehe Abbildung ??).

\begin{align}
  \omega\ua{1} &= \sqrt{ \frac{1}{L} \frac{2(C\ua{1} + C\ua{2})}{C\ua{1}C\ua{2}}} \,
  \left\{ 1 - \theta^2 \frac{C\ua{1}C\ua{2}}{(C\ua{1} + C\ua{2})^2} \right\} \\
  \omega\ua{2} &= \sqrt{ \frac{2}{L(C\ua{1} + C\ua{2})}} \, \theta
\end{align}

Aus den bestimmten Dispersionsrelationen lassen sich nun ebenfalls die Phasen- und
die Gruppengeschwindigkeit herleiten. Da die Gruppenheschwindigkeit in dem Versuch
jedoch nicht betrachtet wird, wird im folgenden nur auf die Phasengeschwindigkeit
eingegangen:

\begin{equation}
  v\ua{Ph} = \frac{\omega}{\theta} = \frac{\omega}{\arccos(-\frac{1}{2}\omega^2LC)}
\end{equation}

Die Dispersion verschwindet in dem Bereich kleiner Frequenzen, also bei $\omega$
<< 1/LC. Beim Erreichen der Grenzfrequenz $\omega\ua{G}$ = 2/$\sqrt{\su{LC}}$ erreicht
die Phasengeschwindigkeit einen Minimalwert:

\begin{equation}
  v\ua{Ph\ua{min}} =  \frac{2}{\pi} \frac{1}{\sqrt{\su{LC}}}
\end{equation}


\end{document}
