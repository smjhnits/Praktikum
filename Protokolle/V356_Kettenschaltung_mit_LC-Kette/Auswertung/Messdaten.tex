\section{Messdaten}

Die Messungen der Durchlasskurve ergaben die folgenden Werte.

\FloatBarrier
\floatplacement{table}{htbp}
\begin{table}
 \centering
 \begin{tabular}[width=\textwidth]{S| S[table-format=4.0] S[table-format=5.0] S[table-format=5.0] S[table-format=5.0] S[table-format=5.0] S[table-format=5.0] S[table-format=5.0]}
    \midrule
    $\nu_C$ \text{\;in\;} $\si{\hertz}$ & 1338 & 2055 & 2843 & 3730 & 5023 & 6471 & 8624 \\
    $\nu_{C_1C_2}$ \text{\;in\;} $\si{hertz}$ & 7345 & 10478 & 21072 & 30336 & 50353 & 79169 \\
    \bottomrule
\end{tabular}
  \caption{Messdaten der Durchlasskurve}
  \label{tab:Durchlasskurve}
\end{table}
\FloatBarrier

Die Messungen der Dispersionsrelation ergaben die folgenden Werte.


\floatplacement{table}{htbp}
\begin{table}
 \centering
 \sisetup{table-format=5.0}
 \begin{tabular}[width=\textwidth]{S S S}
     \toprule
      {Verschiebung} & {$\nu_C$} & {$\nu_{C_1C_2}$}\\
     \midrule
      0 & 0 & 0 \\
      \pi & 7927 & 7158 \\
      2\pi & 15610 & 14188 \\
      3\pi & 23372 & 21078 \\
      4\pi & 30703 & 27714 \\
      5\pi & 38072 & 34188 \\
      6\pi & 43171 & 40094 \\
      7\pi & 49000 & 45378 \\
      8\pi & \text{--} & 50298 \\
      9\pi & \text{--} & 54295 \\
      10\pi & \text{--} & 57976 \\
      11\pi & \text{--} & 60550 \\
      12\pi & \text{--} & 62625 \\
      \bottomrule
\end{tabular}
  \caption{Messdaten der Dispersionsrelation}
  \label{tab:Dispersionsrelation}
\end{table}

Die Messungen der Spannungsamplituden der offenen $LC$-Kettenschaltung, $LC_1C_2$-Kettenschaltung
und der abgeschlossenen $LC$-Kettenschaltung ergab die folgenden Werte. Die Angaben sind
in \si{\milli\volt}.

\FloatBarrier
\floatplacement{table}{htbp}
\begin{table}
 \centering
 \sisetup{table-format=1.3}
 \begin{tabular}[width=\textwidth]{S[table-format=1.0] S S S[table-format=2.0]}
     \toprule
      {Kettenglied} & {$\nu_C=\SI{7133}{\hertz}$} & {$\nu_{C_1C_2} = \SI{14307}{\hertz}$} & {$\nu_{abge}=\SI{7337}{\hertz}$}\\
     \midrule
      1 & 1,55 & 0,925 & 25 \\
      2 & 1,425 & 0,65 & 25 \\
      3 & 1,2 & 0,25 & 25 \\
      4 & 0,95 & 0,2 & 25 \\
      5 & 0,66 & 0,61 & 25 \\
      6 & 0,3 & 0,9  & 25 \\
      7 & 0,027 & 1  & 25 \\
      8 & 0,3 & 0,98 & 25  \\
      9 & 0,95& 0,71  & 25 \\
      10 & 1,2 & 0,28 & 25 \\
      11 & 1,4 & 0,17 & 25 \\
      12 & 1,55 & 0,65 & 24,5 \\
      13 & 1,575 & 1 & 24,5 \\
      14 & \text{--} & 1,05 & 24,5 \\
      \bottomrule
\end{tabular}
  \caption{Messdaten der Dispersionsrelation}
  \label{tab:Dispersionsrelation}
\end{table}
