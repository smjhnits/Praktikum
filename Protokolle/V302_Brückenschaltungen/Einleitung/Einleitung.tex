\input{"../../Praeambel_prak.tex"}

\title{Versuch 302}
\subtitle{Elektrische Brückenschaltunge}
\author{Sebastian Pape\\
        sepa@gmx.de \and
        Jonah Nitschke\\
        lejonah@web.de}
\date{Durchführung: 13.12.2016\\
      Abgabe: 20.12.2016}

\begin{document}
\maketitle
\setcounter{page}{1}
%\section{Zielsetzung}
%In diesem Versuch sollte mit Hilfe von Brückenschaltungen die physikalische Größe
%von verschieden Bauteilen bestimmt werden.
\section{Theorie}
\subsection{Elektrische Brückenschaltungen}
Bei Brückenschaltungen handelt es sich um elektrische Schaltungen, mit dessen
Hifle die Widerstände von Bauteilen sehr genau bestimmt werden können.
Hierbei sind auch komplexe Widerstände erlaubt, sodass auch die Kapazität eines
Kondestors und die Induktivität einer Spulen gemessen werden können.
Die grundlegende Struktur einer Brückenschaltung ist in Abb. \ref{fig:Brückenschaltung}
dargestellt. Es wird ausgenutzt, dass zwischen zwei getrennten stromdurchflossenen
Leitern eine Potentialdifferenz besteht, die durch die \emph{Kirchhoffschen Gesetze}
bestimmt werden kann.\\
1.\emph{Kirchhoffsches Gesetz}(Knotenregel):\\
Die Summe aller in ein Knoten eingehenden Ströme ist gleich der Summe, der aus
einem Knoten herausfließenden Ströme. Diese Gleichung ergibt sich aus der
Ladungserhaltung.
\begin{equation}
  \label{eqn:Kirchhoff1}
  \sum\ua{k} I\ua{k} = 0
\end{equation}
2.\emph{Kirchhoffsche Gesetz}(Maschenregel):\\
Die Summe aller Spannungen in einer Masche ist gleich Null.
Dieses Gesetzt entstammt aus der Energieerhaltung.
\begin{equation}
  \sum\ua{k}U\ua{k} = \sum\ua{k}I\ua{k}\R\ua{k}
\end{equation}
\end{document}
