\input{../../Praeambel_prak.tex}

\begin{document}

\section{Messergebnisse}

\begin{table}
  \centering
  \label{tab:MessungA}
  \caption{Gemessene Werte für R = Wert 10}
  \begin{tabular}{ c c c c }
    \toprule
     & Messung 1 & Messung 2 & Messung 3 \\
    \midrule
    $R_{2} \, in \, \Omega $ \vline & 1000 & 664 & 332 \\
    $R_{3} \, in \, \Omega $ \vline & 196 & 268 & 422  \\
    $R_{4} \, in \, \Omega $ \vline & 804 & 732 & 578  \\
    \bottomrule
  \end{tabular}
\end{table}

\begin{table}
  \centering
  \label{tab:MessungA}
  \caption{Gemessene Werte für R = Wert 10}
  \begin{tabular}{ c c c c }
    \toprule
     & Messung 1 & Messung 2 & Messung 3 \\
    \midrule
    $R_{2}$  in  $\su{\Omega} $ \vline & 1000 & 664 & 332 \\
    $R_{3}$  in  $\su{\Omega} $ \vline & 284 & 373 & 543  \\
    $R_{4}$  in  $\su{\Omega} $ \vline & 716 & 627 & 457  \\
    \bottomrule
  \end{tabular}
\end{table}

\end{document}
