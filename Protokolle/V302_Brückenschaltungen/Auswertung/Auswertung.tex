\input{../../Praeambel_prak.tex}

\begin{document}

\section{Messergebnisse}

\begin{table}
  \centering
  \caption{Gemessene Werte bei der Wheatstoneschen Brücke}
  \label{tab:Wheatstone}
  \begin{tabular}{ c c c c c }
    \toprule
      & & Messung 1 & Messung 2 & Messung 3 \\
    \midrule
    Wert 10 & \multicolumn{1}{c|}{$\su{R_2}$  in  $\su{\Omega} $ } & 1000 & 664 & 332 \\
            & \multicolumn{1}{c|}{$\su{R_3}$  in  $\su{\Omega} $ } & 196 & 268 & 422  \\
            & \multicolumn{1}{c|}{$\su{R_4}$  in  $\su{\Omega} $ } & 804 & 732 & 578  \\
    \midrule
    Wert 12 & \multicolumn{1}{c|}{$\su{R_2}$  in  $\su{\Omega} $ } & 1000 & 664 & 332 \\
            & \multicolumn{1}{c|}{$\su{R_3}$  in  $\su{\Omega} $ } & 284 & 373 & 543  \\
            & \multicolumn{1}{c|}{$\su{R_4}$  in  $\su{\Omega} $ } & 716 & 627 & 457  \\
    \bottomrule
  \end{tabular}
\end{table}

\begin{table}
  \centering
  \caption{Gemessene Werte für die Kapazitätsmessbrücke ohne Wiederstände}
  \label{tab:Kapazitätohne}
  \begin{tabular}{ c c c c c }
    \toprule
    & & Messung 1 & Messung 2 & Messung 3 \\
    \midrule
    Wert 3 & \multicolumn{1}{c|}{$\su{C_2}$  in  $\su{nF}    $}  & 450 & 399 & 597  \\
           & \multicolumn{1}{c|}{$\su{R_3}$  in  $\su{\Omega}$}  & 519 & 490 & 590  \\
           & \multicolumn{1}{c|}{$\su{R_4}$  in  $\su{\Omega}$}  & 481 & 510 & 410  \\
    \midrule
    Wert 1 & \multicolumn{1}{c|}{$\su{C_2}$  in  $\su{nF}    $}  & 450 & 399 & 597 \\
           & \multicolumn{1}{c|}{$\su{R_3}$  in  $\su{\Omega}$}  & 407 & 380 & 478 \\
           & \multicolumn{1}{c|}{$\su{R_4}$  in  $\su{\Omega}$}  & 593 & 520 & 522 \\
    \bottomrule
  \end{tabular}
\end{table}


\begin{table}
  \centering
  \caption{Gemessene Werte für die Kapazitätsmessbrücke mit Wiederständen}
  \label{tab:Kapazitätmit}
  \begin{tabular}{ c c c c c}
    \toprule
    & & Messung 1 & Messung 2 & Messung 3 \\
    \midrule
    Wert 8 & \multicolumn{1}{c|}{$\su{C_2}$  in  $\su{nF}     $ } & 450 & 399 & 597 \\
           & \multicolumn{1}{c|}{$\su{R_2}$  in  $\su{\Omega} $ } & 371 & 418 & 278 \\
           & \multicolumn{1}{c|}{$\su{R_3}$  in  $\su{\Omega} $ } & 606 & 578 & 673 \\
           & \multicolumn{1}{c|}{$\su{R_4}$  in  $\su{\Omega} $ } & 394 & 422 & 327 \\
    \midrule
    Wert 9 & \multicolumn{1}{c|}{$\su{C_2}$  in  $\su{nF}     $ } & 450 & 399 & 597 \\
           & \multicolumn{1}{c|}{$\su{R_2}$  in  $\su{\Omega} $ } & 466 & 524 & 352 \\
           & \multicolumn{1}{c|}{$\su{R_3}$  in  $\su{\Omega} $ } & 511 & 481 & 581 \\
           & \multicolumn{1}{c|}{$\su{R_4}$  in  $\su{\Omega} $ } & 489 & 519 & 419 \\
    \bottomrule
  \end{tabular}
\end{table}


\begin{table}
  \centering
  \caption{Gemessene Werte für die Induktivitätmessbrücke}
  \label{tab:Induktivitätsmessbrücke}
  \begin{tabular}{ c c c c c}
    \toprule
    & & Messung 1 & Messung 2 & Messung 3 \\
    \midrule
    Wert 10 & \multicolumn{1}{c|}{$\su{L_2}$  in  $\su{mH}     $ } & 14.6 & 20.1 & 27.5 \\
            & \multicolumn{1}{c|}{$\su{R_2}$  in  $\su{\Omega} $ } & 45  & 57  & 85  \\
            & \multicolumn{1}{c|}{$\su{R_3}$  in  $\su{\Omega} $ } & 907 & 875 & 837 \\
            & \multicolumn{1}{c|}{$\su{R_4}$  in  $\su{\Omega} $ } & 83  & 125 & 163 \\
    \midrule
    Wert 18 & \multicolumn{1}{c|}{$\su{L_2}$  in  $\su{mH}     $ } & 14.6 & 20.1 & 27.5 \\
            & \multicolumn{1}{c|}{$\su{R_2}$  in  $\su{\Omega} $ } & 108 & 143 & 197 \\
            & \multicolumn{1}{c|}{$\su{R_3}$  in  $\su{\Omega} $ } & 775 & 715 & 648 \\
            & \multicolumn{1}{c|}{$\su{R_4}$  in  $\su{\Omega} $ } & 225 & 285 & 352 \\
    \bottomrule
  \end{tabular}
\end{table}


\begin{table}
  \centering
  \caption{Gemessene Werte für R-L-Glieder mit der Maxwell-Brücke}
  \label{tab:Maxwell}
  \begin{tabular}{ c c c c c }
    \toprule
    & & Messung 1 & Messung 2 & Messung 3 \\
    \midrule
    Wert 10 & \multicolumn{1}{c|}{$\su{R_2}$  in  $\su{\Omega} $ } & 100 & 664 & 332 \\
            & \multicolumn{1}{c|}{$\su{R_3}$  in  $\su{\Omega} $ } & 347 & 523 & 1036 \\
            & \multicolumn{1}{c|}{$\su{R_4}$  in  $\su{\Omega} $ } & 829 & 829 & 829 \\
    \midrule
    Wert 18 & \multicolumn{1}{c|}{$\su{R_2}$  in  $\su{\Omega} $ } & 100 & 664 & 332 \\
            & \multicolumn{1}{c|}{$\su{R_3}$  in  $\su{\Omega} $ } & 128 & 193 & 382 \\
            & \multicolumn{1}{c|}{$\su{R_4}$  in  $\su{\Omega} $ } & 347 & 349 & 348 \\
    \bottomrule
  \end{tabular}
\end{table}

\begin{table}
  \centering
  \caption{Gemessene Werte bei der Wien-Robinson-Brücke}
  \label{tab:MessungE}
  \begin{tabular}{ c c c}
    \toprule
    $\su{\nu}$ in $\su{\frac{1}{s}}$ & $\su{U\ua{Br}}$ in V & $\su{U\ua{Sp}}$ in V \\
    \midrule
    20   & 2.480  & 2.700 \\
    100  & 1.960  & 2.550 \\
    200  & 1.000  & 2.200 \\
    270  & 0.524  & 2.000 \\
    320  & 0.260  & 1.800 \\
    340  & 0.172  & 1.800 \\
    360  & 0.078  & 1.700 \\
    370  & 0.044  & 1.700 \\
    390   & 0.053 & 1.650 \\
    395   & 0.084 & 1.650 \\
    400   & 0.094 & 1.650 \\
    405   & 0.136 & 1.650 \\
    410   & 0.140 & 1.600 \\
    415   & 0.166 & 1.600 \\
    420   & 0.184 & 1.600 \\
    430   & 0.210 & 1.600 \\
    440   & 0.254 & 1.550 \\
    469   & 0.328 & 1.500 \\
    480   & 0.384 & 1.450 \\
    500   & 0.468 & 1.400 \\
    550   & 0.590 & 1.250 \\
    600   & 0.740 & 1.200 \\
    700   & 1.020 & 1.100 \\
    800   & 1.150 & 0.950 \\
    1000  & 1.450 & 0.750 \\
    1200  & 1.660 & 0.680 \\
    1500  & 1.860 & 0.540 \\
    2000  & 2.000 & 0.410 \\
    3000  & 2.140 & 0.285 \\
    5000  & 2.180 & 0.175 \\
    10000 & 2.040 & 0.090 \\
    20000 & 1.660 & 0.060 \\
    30000 & 1.180 & 0.055 \\
    \bottomrule
  \end{tabular}
\end{table}

\section{Auswertung}

Die Mittelwerte und Fehler der errechneten Größen in den folgenden Abschnitten
errechnen sich mit folgenden Formeln:

\begin{align}
  \label{eqn:barx}
  \bar{x}            &= \frac{1}{N} \sum_{i=1}^N x_i \\
  \increment \bar{x} &= \sqrt{ \frac{1}{N(N-1)} \sum_{i=1}^N \left(x_i - \bar{x}\right)^2}
  \label{eqn:incrementx}
\end{align}

\subsection{Wheatstonesche Brückenschaltung}
Mit den gemessenen Werte aus Tabelle \ref{tab:Wheatstone} und der Formel ??
lassen sich nun die gesuchten
Widerstände berechnen. Dabei ergeben sich folgende Werte:

\begin{align}
  \su{R\ua{x,10}} &= (243.1 \pm 0.4) \, \su{\Omega} \\
  \su{R\ua{x,12}} &= (395.4 \pm 0.7) \, \su{\Omega}
\end{align}

\subsection{Kapazitätsmessbrücke}

Mit den Werten aus Tabelle \ref{tab:Kapazitätohne} für die Kapazitätsmessbrücke
ohne zwischengeschaltete Widerstände und der Formel ?? ergeben sich für die
gesuchten Kapazitäten die folgenden Werte:

\begin{align}
  \su{C\ua{x,1}} &= (652.9 \pm 1.4) \, \su{nF} \\
  \su{C\ua{x,3}} &= (415.7 \pm 0.7) \, \su{nF}
\end{align}

Mit den Werten aus Tabelle \ref{tab:Kapazitätmit} für die Kapazitätsmessbrücke
mit Widerständen und den Formen ?? und ?? lassen sich ebenfalls die gesuchten
Werte der R-C-Glieder bestimmen:

\begin{align}
  \su{R\ua{x,8}} &= (571.8 \pm 0.6) \, \su{\Omega} \\
  \su{C\ua{x,8}} &= (291.3 \pm 0.7) \, \su{nF}
\end{align}

\begin{align}
  \su{R\ua{x,9}} &= (487.6 \pm 0.3) \, \su{\Omega} \\
  \su{C\ua{x,9}} &= (430.0 \pm 0.6) \, \su{nF}
\end{align}

\subsection{Induktivitätsmessbrücke}

Mit den Formeln ?? und ?? und den Werten aus der Tabelle \ref{tab:Induktivitätsmessbrücke}
ergeben sich für die gesuchten R-L-Glieder folgende Werte:

\begin{align}
  \su{R\ua{x,16}} &= (442 \pm 27) \, \su{\Omega} \\
  \su{L\ua{x,16}} &= (147 \pm 6) \, \su{mH}
\end{align}

\begin{align}
  \su{R\ua{x,18}} &= (364 \pm 4) \, \su{\Omega}  \\
  \su{L\ua{x,18}} &= (50.5 \pm 0.1) \, \su{mH}
\end{align}

\subsection{Maxwell-Brücke}
Die gleichen R-L-Glieder wie bei der Induktivitätsmessbrücke wurden nochmal mithilfe
der Maxwell-Brücke gemessen, bei der ein Kondensator mit der Kapazität $\su{C_4} = 399 \, nF$
verwendet wurde. Mit den Werte aus der Tabelle \ref{tab:Maxwell}
sowie den Formeln ?? und ?? ergeben sich für die beiden R-L-Glieder somit folgende
Werte:

\begin{align}
  \su{R\ua{x,16}} &= (417.5 \pm 1.3) \, \su{\Omega} \\
  \su{L\ua{x,16}} &= (138.1 \pm 0.4) \, \su{mH}
\end{align}

\begin{align}
  \su{R\ua{x,18}} &= (366.8 \pm 1.3) \, \su{\Omega}  \\
  \su{L\ua{x,18}} &= (50.94 \pm 0.17) \, \su{mH}
\end{align}

\subsection{Robinson-Wien-Brücke}

Bei der Messung die für Robinson-Wien-Schaltung wurden die folgenden Komponenten
verwendet:

\begin{align}
  \su{C}   &=  (415.7 \pm 0.7) \, \su{nF}  \\
  \su{R'}  &= 332 \, \su{\Omega}          \\
  \su{2R'} &= 664 \, \su{\Omega}         \\
  \su{R}   &= 1000 \, \su{\Omega}
\end{align}

In dieser Messreihe wurde die Frequenzabhängigkeit der Brückenspannung untersucht.
Dazu wird in dem Graphen ?? das Verhältnis der effektiven Brückenspannung $\su{U\ua{Br,eff}}$
zur Speisespannung $\su{U\ua{Sp}}$ gegen $\su{\Omega}$ = $\frac{\su{\nu}}{\su{\nu_0}}$
aufgetragen. Für $\su{U\ua{Br,eff}}$ gilt dabei:

\begin{equation}
  \su{U\ua{Br,eff}} = \frac{\su{U\ua{Br}}}{2\sqrt{2}} .
\end{equation}

Als Frequenz, bei der die Brückenspannung verschwinden sollte, ergibt sich folgender
Wert:

\begin{align}
  \su{\omega_0} &= \frac{1}{\u{RC}} = \frac{1}{1000 \, \su{\Omega} \cdot (415.7 \pm 0.7) \cdot 10^{-9} \, \su{F}} = (2406 \pm 4) \, Hz \\
  \su{\nu_0}    &= \frac{\su{\omega_0}}{2\pi} = (382.9 \pm 0.6) \, Hz
\end{align}

\end{document}
