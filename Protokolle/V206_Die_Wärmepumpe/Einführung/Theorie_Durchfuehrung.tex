% bei Standalone in documentclass noch:
% \RequirePackage{luatex85}

\documentclass[captions=tableheading, titlepage= firstiscover, parskip = half , bibliography=totoc]{scrartcl}
%paper = a5 für andere optinen
% titlepage= firstiscover
% bibliography=totoc für bibdateien
% parskip=half  Veränderung um Absätze zu verbessern

\usepackage{scrhack} % nach \documentclass
\usepackage[aux]{rerunfilecheck}
\usepackage{polyglossia}
\usepackage[style=numeric, backend=biber]{biblatex} % mit [style = alphabetic oder numeric] nach polyglossia
\addbibresource{lit.bib}
\setmainlanguage{german}

\usepackage[autostyle]{csquotes}
\usepackage{amsmath} % unverzichtbare Mathe-Befehle
\usepackage{amssymb} % viele Mathe-Symbole
\usepackage{mathtools} % Erweiterungen für amsmath
\usepackage{fontspec} % nach amssymb
% muss ins document: \usefonttheme{professionalfonts} % für Beamer Präsentationen
\usepackage{longtable}

\usepackage[
math-style=ISO,    % \
bold-style=ISO,    % |
sans-style=italic, % | ISO-Standard folgen
nabla=upright,     % |
partial=upright,   % /
]{unicode-math} % "Does exactly what it says on the tin."
\setmathfont{Latin Modern Math}
% \setmathfont{Tex Gyre Pagella Math} % alternativ

\usepackage[
% die folgenden 3 nur einschalten bei documenten
locale=DE,
separate-uncertainty=true, % Immer Fehler mit ±
per-mode=symbol-or-fraction, % m/s im Text, sonst \frac
]{siunitx}

% alternativ:
% per-mode=reciprocal, % m s^{-1}
% output-decimal-marker=., % . statt , für Dezimalzahlen

\usepackage[
version=4,
math-greek=default,
text-greek=default,
]{mhchem}

\usepackage[section, below]{placeins}
\usepackage{caption} % Captions schöner machen
\usepackage{graphicx}
\usepackage{grffile}
\usepackage{subcaption}

% \usepackage{showframe} Wenn man die Ramen sehen will

\usepackage{float}
\floatplacement{figure}{htbp}
\floatplacement{table}{htbp}

\usepackage{mhchem} %chemische Symbole Beispiel: \ce{^{227}_{90}Th+}


\usepackage{booktabs}

 \usepackage{microtype}
 \usepackage{xfrac}

 \usepackage{expl3}
 \usepackage{xparse}

 % \ExplSyntaxOn
 % \NewDocumentComman \I {}  %Befehl\I definieren, keine Argumente
 % {
 %    \symup{i}              %Ergebnis von \I
 % }
 % \ExplSyntaxOff

 \usepackage{pdflscape}
 \usepackage{mleftright}

 % Mit dem mathtools-Befehl \DeclarePairedDelimiter können Befehle erzeugen werden,
 % die Symbole um Ausdrücke setzen.
 % \DeclarePairedDelimiter{\abs}{\lvert}{\rvert}
 % \DeclarePairedDelimiter{\norm}{\lVert}{\rVert}
 % in Mathe:
 %\abs{x} \abs*{\frac{1}{x}}
 %\norm{\symbf{y}}

 % Für Physik IV und Quantenmechanik
 \DeclarePairedDelimiter{\bra}{\langle}{\rvert}
 \DeclarePairedDelimiter{\ket}{\lvert}{\rangle}
 % <name> <#arguments> <left> <right> <body>
 \DeclarePairedDelimiterX{\braket}[2]{\langle}{\rangle}{
 #1 \delimsize| #2
 }

\setlength{\delimitershortfall}{-1sp}

 \usepackage{tikz}
 \usepackage{tikz-feynman}

 \usepackage{csvsimple}
 % Tabellen mit \csvautobooktabular{"file"}
 % muss in table umgebung gesetzt werden


% \multicolumn{#Spalten}{Ausrichtung}{Inhalt}

\usepackage{hyperref}
\usepackage{bookmark}
\usepackage[shortcuts]{extdash} %nach hyperref, bookmark

\newcommand{\ua}[1]{_\symup{#1}}
\newcommand{\su}[1]{\symup{#1}}


\title{Versuch 206}
\subtitle{Die Wärmepumpe}
\author{Jonah Nitschke\\
        lejonah@web.de \and
        Sebastian Pape\\
        sepa@gmx.de}
\date{Durchführung: 15.11.2016\\
      Abgabe: 22.11.2016}

\begin{document}

\maketitle

\section{Einführung}

Im folgenden Versuch geht es um den transport von Wärmeenergie zwischen zwei Wärmereservoiren.
Imm Gegensatz zu der allgemein gültigen Regel wir hier nun mithilfe einer Wärmepumpe
Wärmeenergie von einem Reservoir mit kaltem Wasser in ein Reservoir mit warmen Wasser transponiert.
Während des Versuchs werden verschiedene Messwerte aufgenommen um hinterher das Verhältniss von Temperatur,
Druck sowie aufgewandter Arbeit zu beurteilen.

\section{Theorie}

% Einfügen von Kommentaren zur Thermodynamik aus blauer Mappe

Um nun in dem folgenden Versuch einen Fluss der Wärmeenergie von dem kälteren reservoir zu dem
wärmeren reservoir zu realisieren, muss zusätzliche Arbeit aufgewandt werden. Für diesen Prozess wird
im folgenden eine Wärmepumpe benutzt, deren Aufbau später noch in Kapitel 3 erläutet wird und deren Bedingungen
zur Vereinfachung der Berechnungen als idealisiert betrachtet werden.

Das Verhältniss von transponierter Wärmemenge zu aufgewandter Arbeit anzugeben, wird die Güteziffer
$\nu$ eingeführt. Nach dem ersten Hauptsatz der Thermodynamik \eqref{eqn:1HSa} gilt für den Wärmenergietransport
zwischen zwei Medien:

\begin{align}
  \increment U &= \increment Q + \increment W \label{eqn:1HSa} \\
  Q_1          &= Q_2 + A   \label{eqn:1HSb}
\end{align}

Die in unserem Fall geltende 2. Formel \eqref{eqn:1HSb} sagt, dass die vom Transportmedium an Reservoir 2 abgegebene
Wärmeenergie $Q_1$ der Summe der aus Reservoir 1 entnommenen Wärmeenergie $Q_2$ und der aufgewandten Arbeit $A$
entsprechen muss. Die Güteziffer der Wärmepumpe kann somit über folgende Formel errechnet werden:

\begin{equation}
  \nu = \frac{Q1}{A}
\end{equation}

Nach dem 2.HS der Thermodynamik lässt sich zudem die Beziehung zwischend den Wärmemengen und Temperaturen der
beiden Reservoiren durch folgende Formel ausdrücken:

\begin{equation}
  \label{eqn:2HS}
  \frac{Q_1}{T_1} - \frac{Q_2}{T_2} = 0
\end{equation}

Für die Gültigkeit dieser Formel muss jedoch gelten, dass der stattfindende Übertragungsprozess reversibel sein.
Somit müsste die aufgewandte mechanische Energie jederzeit vollständig zurückgewonnen werden können. Da es
sich dabei um eine idealisierte Annahme handelt, die in der Realität nie zutrifft, muss \eqref{eqn:2HS} etwas
umformuliert werden:

\begin{equation}
  \label{eqn:2HS1}
  \frac{Q_1}{T_1} - \frac{Q_2}{T_2} > 0
\end{equation}

Aus den Gleichungen (1) bis (4) folgt somit:

\begin{align}
  Q1         &= A + \frac{T_2}{T_1}Q_1 \label{eqn:A1.1} \\
  \nu_{id}   &= \frac{Q_1}{A} = \frac{T_1}{T_1 - T_2} \label{eqn:A1.2} \\
  \nu_{real} &< \frac{Q_1}{A} = \frac{T_1}{T_1 - T_2} \label{eqn:A1.3}
\end{align}

Die Gleichungen \eqref{eqn:A1.2} und \eqref{eqn:A1.3} zeigen, dass eine Wärmepumpe umso effektiver eingestuft werden
kann, je kleiner die Differenz zwischen $T_1$ und $T_2$ ist.

\subsection{Bestimmung der realen Güteziffer \texorpdfstring{$\nu$}{z}}

Mit dem Werten von $T_1$ kann nun die pro Zeiteinheit gewonnene Wärmemenge berechnet werden:

\begin{equation}
  \label{eqn:gWM}
  \frac{\increment Q_1}{\increment t} = (m_1c_W + m_kc_k) \frac{\increment T_1}{\increment t}
\end{equation}

$m_1c_w$ und $m_kc_k$ entsprechen dabei den Wärmekapazitäten der Kupferschlange und des Eimers.



\end{document}
