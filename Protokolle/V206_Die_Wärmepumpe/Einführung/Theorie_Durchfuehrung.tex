\input{"../../Praeambel_prak.tex"}

\title{Versuch 206}
\subtitle{Die Wärmepumpe}
\author{Jonah Nitschke\\
        lejonah@web.de \and
        Sebastian Pape\\
        sepa@gmx.de}
\date{Durchführung: 15.11.2016\\
      Abgabe: 22.11.2016}

\begin{document}

\maketitle

\section{Einführung}

Im folgenden Versuch geht es um den transport von Wärmeenergie zwischen zwei Wärmereservoiren.
Imm Gegensatz zu der allgemein gültigen Regel wir hier nun mithilfe einer Wärmepumpe
Wärmeenergie von einem Reservoir mit kaltem Wasser in ein Reservoir mit warmen Wasser transponiert.
Während des Versuchs werden verschiedene Messwerte aufgenommen um hinterher das Verhältniss von Temperatur,
Druck sowie aufgewandter Arbeit zu beurteilen.

\section{Theorie}

% Einfügen von Kommentaren zur Thermodynamik aus blauer Mappe

Um nun in dem folgenden Versuch einen Fluss der Wärmeenergie von dem kälteren reservoir zu dem
wärmeren reservoir zu realisieren, muss zusätzliche Arbeit aufgewandt werden. Für diesen Prozess wird
im folgenden eine Wärmepumpe benutzt, deren Aufbau später noch in Kapitel 3 erläutet wird und deren Bedingungen
zur Vereinfachung der Berechnungen als idealisiert betrachtet werden.

Das Verhältniss von transponierter Wärmemenge zu aufgewandter Arbeit anzugeben, wird die Güteziffer
$\nu$ eingeführt. Nach dem ersten Hauptsatz der Thermodynamik \eqref{eqn:1HSa} gilt für den Wärmenergietransport
zwischen zwei Medien:

\begin{align}
  \increment U &= \increment Q + \increment W \label{eqn:1HSa} \\
  Q_1          &= Q_2 + A   \label{eqn:1HSb}
\end{align}

Die in unserem Fall geltende 2. Formel \eqref{eqn:1HSb} sagt, dass die vom Transportmedium an Reservoir 2 abgegebene
Wärmeenergie $Q_1$ der Summe der aus Reservoir 1 entnommenen Wärmeenergie $Q_2$ und der aufgewandten Arbeit $A$
entsprechen muss. Die Güteziffer der Wärmepumpe kann somit über folgende Formel errechnet werden:

\begin{equation}
  \nu = \frac{Q1}{A}
\end{equation}

Nach dem 2.HS der Thermodynamik lässt sich zudem die Beziehung zwischend den Wärmemengen und Temperaturen der
beiden Reservoiren durch folgende Formel ausdrücken:

\begin{equation}
  \label{eqn:2HS}
  \frac{Q_1}{T_1} - \frac{Q_2}{T_2} = 0
\end{equation}

Für die Gültigkeit dieser Formel muss jedoch gelten, dass der stattfindende Übertragungsprozess reversibel sein.
Somit müsste die aufgewandte mechanische Energie jederzeit vollständig zurückgewonnen werden können. Da es
sich dabei um eine idealisierte Annahme handelt, die in der Realität nie zutrifft, muss \eqref{eqn:2HS} etwas
umformuliert werden:

\begin{equation}
  \label{eqn:2HS1}
  \frac{Q_1}{T_1} - \frac{Q_2}{T_2} > 0
\end{equation}

Aus den Gleichungen (1) bis (4) folgt somit:

\begin{align}
  Q1         &= A + \frac{T_2}{T_1}Q_1 \label{eqn:A1.1} \\
  \nu_{id}   &= \frac{Q_1}{A} = \frac{T_1}{T_1 - T_2} \label{eqn:A1.2} \\
  \nu_{real} &< \frac{Q_1}{A} = \frac{T_1}{T_1 - T_2} \label{eqn:A1.3}
\end{align}

Die Gleichungen \eqref{eqn:A1.2} und \eqref{eqn:A1.3} zeigen, dass eine Wärmepumpe umso effektiver eingestuft werden
kann, je kleiner die Differenz zwischen $T_1$ und $T_2$ ist.

\subsection{Bestimmung der realen Güteziffer \texorpdfstring{$\nu$}{z}}

Mit dem Werten von $T_1$ kann nun die pro Zeiteinheit gewonnene Wärmemenge berechnet werden:

\begin{align}
  \label{eqn:Gueteziffer}
  \frac{\increment Q_1}{\increment t} &= (m_1c_W + m_kc_k) \frac{\increment T_1}{\increment t} \\
  \nu                                 &= \frac{\increment Q_1}{\increment t N}
\end{align}

$m_1c_w$ und $m_kc_k$ entsprechen dabei den Wärmekapazitäten der Kupferschlange und des Eimers.
Für die Güteziffer wird noch N als die zeitlich gemittelte Leistung benötigt.

\subsection{Bestimmung des Massendurchsatzes}

Mit den Werten von $T_2$ und der Verdampfungswärme $L$ kann nun der Massendurchsatz $\increment m $
berechnet werden:

\begin{align}
  \label{eqn:Massendurchsatz}
  \frac{\increment Q_2}{\increment t} &= (m_2 c_W + m_k c_k) \frac{\increment T_2}{\increment t} \\
  \frac{\increment Q_2}{\increment t} &= L \frac{\increment m}{\increment t}
\end{align}

\subsection{Bestimmung der mechanischen Kompressorleistung \texorpdfstring{$N_{mech}$}{t} }

Um die mechnanische Kompresorleistung $N_{mech}$ zu bestimmen muss vorher die vom Kompressor aufgebrachte
Arbeit zur Komprimierung des Volumens $V_a$ auf das Volumen $V_b$ berechnet werden:

\begin{equation}
  \label{eqn:Am}
  A_m = \frac{1}{\kappa - 1} \left( p_b \sqrt[\kappa]{\frac{p_a}{p_b}} - p_a \right) V_a
\end{equation}

Mit der Dichte $\rho$ im gasförmigen Zustand, also beim Druck $p_a$, kann nun $N_{mech}$ berechnet werden:

\begin{equation}
  \label{eqn:Nmech}
  N_{mech} = \frac{\increment A_m}{\increment t} =  \frac{1}{\kappa - 1} \left( p_b \sqrt[\kappa]{\frac{p_a}{p_b}} - p_a \right) \frac{1}{\rho} \frac{\increment m}{\increment t}
\end{equation}

\section{Aufbau und Durchführung}

\subsection{Aufbau}

Die verwendte Wärmepumpe besteht aus mehreren Komponenten. Die Apparatur besteht aus zwei thermisch Isolierten Reservoiren
mit Wasser. Durch beide Reservoire läuft eine Kupferrohr, in dem sich ein Gas befindet. Der Kompressor erzeugt in beiden Hälften
unterschiedliche Drücke, indem er das Gas komprimiert. Das erst flüssige Gas durchströmt nun das Kupferrohr in Reservoir 2 und
verdampft unter dort herrschenden Druck. Dabei nimmt entzieht es dem Reservoir die Verdampfungswärme und wird danach im Kompressor
adiabatisch komprimiert, bis der Druck hoch genug ist, um  das Gas im Reservoir 1 wieder zu verflüssigen. Die entstehende
Kondensationswärme wird dort somit ans Wasser abgegeben. In einem nachgeschalteten Reiniger wird die Flüssigkeit von Gasresten
getrennt und dann durch das Drosselventil geleitet, damit der Zyklus im Reservoir 2 von vorne beginnt. Damit in den Kompressor keine
Flüssigkeitsreste gelangen, wird eine Steuervorrichtung angebracht. Dieser misst die Temperaturdifferenz zwischen Ausgang und Eingang
und verringert die Flüssigkeitszufuhr, sollte die Verdampfungrate im Reservoir 2 zu stark abfallen. So wird nach und nach das
Reservoir 2 abgekühlt und das Reservoir 2 erwärmt und die Wärmeenergie vom 1. in das 2. Reservoir transportiert.

\subsection{Durchführung}

Am Anfang des Experimentes werden die beiden Reservoire mit 4 Litern Wasser befüllt. Dann werden die beiden Reservoire an der Apparatur
befestigt, sodass das beide Behälter optimal abgedichtet sind. Anschließend werden die beiden Rührstäbe und der Kompressor
ein geschaltet. Nun werden im Abstand von einer Minute die verschiedenen Werte [$p_a, p_b, T_1, T_2, L$] gemessen und notiert. Sobald
die Temperatur im Reservoir 2 die 50 °C Marke erreicht, werden Kompressor und Rührstäbe ausgeschaltet und der Versuch ist beendet.






\end{document}
