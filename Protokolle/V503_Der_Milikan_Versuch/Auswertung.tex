\section{Auswertung}

Anhand der genommenen Messgrößen kann der Radius der Öltröpfchen
und die Ladung dieser ermittelt werden. Die Messdaten sind in Tabelle \ref{tab: Messdaten}
aufgeführt.

\subsection{Bestimmung der Radien}

Der Radius eines vermessenen Öltröpfchens wird über Formel \eqref{eqn:} bestimmt.
Für die Bestimmung der Radien wird die Viskorität von Luft, die Fallgeschwindigkeit
und die Dichte des verwedeten Öles benötigt.
Die Viskosität von Luft wurde mithilfe des Diagrammes aus \cite{anleitung01} bestimmt.
Dafür wurden zwei Punkte aus dem Diagramm abgelesen und die dazugehörige Geradengleichung ermittelt.
Die zu den gemessenen Temperaturen zugehörige Viskosität der Luft ist mittels der
Geradengleichung zu bestimmen.
Für die Dichte des verwendeten Öles wurde der Literaturwert
$\rho\ua{Oel} = \SI{886}{\kilogram\per\meter^3}$\cite{anleitung01} verwendet.
Die Fallzeit der Öltropfchen wurde über eine Strecke von $\SI{0,5}{\milli\meter}$
gemessen. Die Fallgeschwindigkeiten ergeben sich durch das teilen der Strecke durch
die gemessene Fallzeit.

Die berechneten Radien sind in Tab. \ref{tab: Messdaten} dargestellt. Alle Werte
sind als fehlerfrei angenommen worden, weshlab die Radien auch als fehlerfrei
betrachtet werden.

\subsection{Bestimmung der Ladung}

Die Ladung eines vermessenen Öltröpfchens wir über Formel \eqref{eqn:} ermittelt.
Darüberhinaus muss der Korrekturterm \eqref{eqn:} berücksichtigt werden.
In die Berechnung der Ladung fließt die herrschende E-Feldstärke, der Radius des Öltröpfchens,
und der Druck mit ein. Die Konstante $B$ wurde in der Literatur \cite{anleitung01}
mit $\SI{6,17e-3}{\centi\meter}\su{Torr}$ angegeben. Als Druck wurde der
Atmosphärendruck mit $\SI{1.01325}{bar}$ verwendet.
Die $E$-Feldstärke ist durch die gemessene Spannung über den folgenden Zusammenhang
festgelegt.

\begin{equation}
  E = \frac{U}{d}
\end{equation}

Die Spannungen $U$ sind in Tabelle \ref{tab: Messdaten} einzusehen. Der Abstand
zwischen den Kondensatorplatten ist mit $\SI{7.6250 (51)}{\milli\meter}$ \cite{anleitung01}
angegeben.

Damit sind alle Größen zur Berechnung der Ladung eines Öltröpfchens gegeben. Die
berechneten Ladungen sind in Tabelle \ref{tab: Messdaten} einzusehen.

\begin{figure}
  \centering
  \includegraphics[width=\textwidth]{Pics/Ladungen_E_0.pdf}
  \caption{Ermittelte Ladungen. Das obere Diagramm zeigt die tatsächliche Messreihenfolge
  und das untere zeigt die bestimmten Ladungen der Größe nach sortiert.}
  \label{fig:Ladungen}
\end{figure}

Das Diagramm \ref{fig:Ladungen} zeigt die gemessenen Ladungen im Verhältnis zu der
Elementarladung. Die Elementarladung wurde aus dem \emph{SciPy}-Packet \emph{constants}
benutzt. Diese enspricht ungefähr einem Wert von $\SI{1,6022}{\coulomb}$.

Die Elementarladung entspricht der Ladung, die den größten gemeinsamen Teiler
aller gemessenen Ladungen bildet.
Zuerst wurde dafür die Messdaten sortiert und alle Ladungen mit ähnlicher
Ladung als gleich angenommen.
Die folgeden Formel stellt dar, wie verfahren wurde.

\begin{equation}
  \su{min}\left\{\left|\su{rd}\left(\frac{Q}{Q\ua{Test}}\right) - \frac{Q}{Q\ua{Test}}\right|\right\}
\end{equation}

Als Testladungen wurde das Intervall $\left\{e_0 - 10^{-19}|e_0 + 10^{-19}\right\}$
verwendet. Das Intervall wurde so gewählt, da anzunehmen ist, dass die gemessene Elementarladung
in der Umgebung der tatsächlichen Elementarladung $e_0$ liegt.

Als gemessene Elementarladung ergibt sich der folgende Wert.

\begin{align}
  \label{eqn:Elementarladung}
  e\ua{gemessen} &= \SI{1,568e-19}{\coulomb} \\
  \left|\Delta_{e_0}\right| &= \SI{3,448e-21}{\coulomb}\\
  \delta_{e_0} &= \SI{0,022}{\coulomb}
\end{align}

$\Delta_{e_0}$ ist dabei der absolute und $\delta_{e_0}$ der relative Fehler.
Auf eine Fehlerrechnung wurde weitestgehend verzichtet, da auftretende Fehler lediglich aus
dem Abstand der Kondensatorplatten resultieren.

\section{Diskussion}

Möglich Fehlerquellen des Versuches sind in erster Linie Ablesefehler, da die
Zeitmessung mit einer Stoppuhr realisiert wurde. Zudem ist nicht auszuschließen,
dass die Fallbewegung der Öltropfchen vollständig horizontal und unbeeinflusst von
äußeren Einflüssen, wie Luftstöße war. Darüberhinaus ist die Viskositätsbestimmung der
Luft zu hinterfragen, da die bestimmte Geradengleichung aus abgelesenen Punkten
konstruiert wurde. Ablsesefehler wurden dabei vernachlässigt. Außerdem sind die Werte des
Thermistors in \cite{anleitung01} mit drei Nachkommastellen angegeben. Die
Widerstandsmessung ermöglichte jedoch nur Messungen auf bis zu zwei Nachkommastellen.


Das untere Diagramm aus Abb. \ref{fig:Ladungen} wurde der Übersichtlichkeit eingebunden.
Der gequantelte Charakter der Ladung ist deutlich zu erkennen.

Die gemessene Elementarladung weicht um $\approx 2,2\%$ von dem Literaturwert ab.
Im Rahmen der Messgenauigkeit ist dieses Ergebnis erstaunlich präzise.



\section{Messdaten}

In diesem Kapitel sind die gemessenen und berechneten Größen tabellarisch
dargestellt.
In der Tabelle gelten die folgenden Bezeichnungen. $\Omega$ ist der gemessene
Widerstand des Thermistors, $t_0$ ist
die Zeit die ein Öltröpfchen ohne Feld für 0,5 mm benötigt, $\su{U}\ua{g}$
ist die Gleichgewichtsspannung, $r$ der bestimmte Radius des Öltröpfchens und
$q$ die berechnete Ladung dieses Öltröpfchens.

\begin{table} 
\centering 
\caption{Messdaten von V503.} 
\label{tab: Messdaten} 
\begin{tabular}{S S S S S S } 
\toprule  
{$\Omega$ in $\si{\mega\ohm}$} & {$t_0$ in $\si{\second}$} & {$\symup{U}_{\symup{g}}$ in $\si{\volt}$} & {$r$ in $\si{\nano\meter}$} & {$q$ in $10^{-20}\si{\coulomb}$} & {$\Delta_q$}  \\ 
\midrule  
 1.96  & 17.78  & 269  & 481.68  & 11.53  & 0.01\\ 
1.92  & 29.26  & 96  & 367.85  & 14.39  & 0.01\\ 
1.87  & 35.03  & 11  & 333.67  & 93.72  & 0.06\\ 
1.81  & 15.76  & 58  & 516.72  & 66.01  & 0.04\\ 
1.78  & 34.00  & 35  & 340.40  & 31.27  & 0.02\\ 
1.71  & 23.20  & 147  & 420.96  & 14.08  & 0.01\\ 
1.75  & 15.41  & 77  & 524.17  & 51.91  & 0.03\\ 
1.75  & 6.83  & 187  & 806.51  & 77.85  & 0.05\\ 
1.75  & 11.40  & 200  & 615.59  & 32.37  & 0.02\\ 
1.75  & 6.61  & 112  & 820.46  & 136.85  & 0.09\\ 
1.74  & 9.93  & 118  & 662.57  & 68.40  & 0.05\\ 
1.74  & 18.00  & 53  & 482.38  & 58.77  & 0.04\\ 
1.73  & 19.26  & 61  & 465.27  & 45.82  & 0.03\\ 
1.73  & 13.78  & 41  & 556.91  & 116.91  & 0.08\\ 
1.73  & 9.40  & 134  & 682.31  & 65.79  & 0.04\\ 
1.72  & 20.30  & 92  & 452.41  & 27.93  & 0.02\\ 
1.73  & 11.56  & 122  & 611.52  & 52.02  & 0.03\\ 
1.72  & 6.84  & 113  & 806.81  & 128.98  & 0.09\\ 
1.72  & 13.58  & 50  & 561.49  & 98.25  & 0.07\\ 
1.72  & 15.49  & 40  & 523.33  & 99.44  & 0.07\\ 
1.71  & 8.49  & 76  & 720.51  & 136.58  & 0.09\\ 
1.71  & 9.55  & 76  & 677.16  & 113.38  & 0.08\\ 
1.71  & 8.13  & 175  & 737.13  & 63.52  & 0.04\\ 
1.71  & 15.55  & 140  & 522.45  & 28.27  & 0.02\\ 
1.71  & 9.03  & 192  & 697.47  & 49.04  & 0.03\\ 
1.70  & 12.13  & 140  & 596.78  & 42.13  & 0.03\\ 
1.70  & 16.18  & 113  & 511.64  & 32.89  & 0.02\\ 
1.71  & 12.67  & 118  & 582.87  & 46.57  & 0.03\\ 
1.70  & 6.95  & 155  & 800.71  & 91.91  & 0.06\\ 
1.70  & 14.95  & 90  & 533.79  & 46.90  & 0.03\\ 
\bottomrule 
\end{tabular} 
\end{table}
