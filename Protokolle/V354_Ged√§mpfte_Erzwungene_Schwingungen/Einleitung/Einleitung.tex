% bei Standalone in documentclass noch:
% \RequirePackage{luatex85}

\documentclass[captions=tableheading, titlepage= firstiscover, parskip = half , bibliography=totoc]{scrartcl}
%paper = a5 für andere optinen
% titlepage= firstiscover
% bibliography=totoc für bibdateien
% parskip=half  Veränderung um Absätze zu verbessern

\usepackage{scrhack} % nach \documentclass
\usepackage[aux]{rerunfilecheck}
\usepackage{polyglossia}
\usepackage[style=numeric, backend=biber]{biblatex} % mit [style = alphabetic oder numeric] nach polyglossia
\addbibresource{lit.bib}
\setmainlanguage{german}

\usepackage[autostyle]{csquotes}
\usepackage{amsmath} % unverzichtbare Mathe-Befehle
\usepackage{amssymb} % viele Mathe-Symbole
\usepackage{mathtools} % Erweiterungen für amsmath
\usepackage{fontspec} % nach amssymb
% muss ins document: \usefonttheme{professionalfonts} % für Beamer Präsentationen
\usepackage{longtable}

\usepackage[
math-style=ISO,    % \
bold-style=ISO,    % |
sans-style=italic, % | ISO-Standard folgen
nabla=upright,     % |
partial=upright,   % /
]{unicode-math} % "Does exactly what it says on the tin."
\setmathfont{Latin Modern Math}
% \setmathfont{Tex Gyre Pagella Math} % alternativ

\usepackage[
% die folgenden 3 nur einschalten bei documenten
locale=DE,
separate-uncertainty=true, % Immer Fehler mit ±
per-mode=symbol-or-fraction, % m/s im Text, sonst \frac
]{siunitx}

% alternativ:
% per-mode=reciprocal, % m s^{-1}
% output-decimal-marker=., % . statt , für Dezimalzahlen

\usepackage[
version=4,
math-greek=default,
text-greek=default,
]{mhchem}

\usepackage[section, below]{placeins}
\usepackage{caption} % Captions schöner machen
\usepackage{graphicx}
\usepackage{grffile}
\usepackage{subcaption}

% \usepackage{showframe} Wenn man die Ramen sehen will

\usepackage{float}
\floatplacement{figure}{htbp}
\floatplacement{table}{htbp}

\usepackage{mhchem} %chemische Symbole Beispiel: \ce{^{227}_{90}Th+}


\usepackage{booktabs}

 \usepackage{microtype}
 \usepackage{xfrac}

 \usepackage{expl3}
 \usepackage{xparse}

 % \ExplSyntaxOn
 % \NewDocumentComman \I {}  %Befehl\I definieren, keine Argumente
 % {
 %    \symup{i}              %Ergebnis von \I
 % }
 % \ExplSyntaxOff

 \usepackage{pdflscape}
 \usepackage{mleftright}

 % Mit dem mathtools-Befehl \DeclarePairedDelimiter können Befehle erzeugen werden,
 % die Symbole um Ausdrücke setzen.
 % \DeclarePairedDelimiter{\abs}{\lvert}{\rvert}
 % \DeclarePairedDelimiter{\norm}{\lVert}{\rVert}
 % in Mathe:
 %\abs{x} \abs*{\frac{1}{x}}
 %\norm{\symbf{y}}

 % Für Physik IV und Quantenmechanik
 \DeclarePairedDelimiter{\bra}{\langle}{\rvert}
 \DeclarePairedDelimiter{\ket}{\lvert}{\rangle}
 % <name> <#arguments> <left> <right> <body>
 \DeclarePairedDelimiterX{\braket}[2]{\langle}{\rangle}{
 #1 \delimsize| #2
 }

\setlength{\delimitershortfall}{-1sp}

 \usepackage{tikz}
 \usepackage{tikz-feynman}

 \usepackage{csvsimple}
 % Tabellen mit \csvautobooktabular{"file"}
 % muss in table umgebung gesetzt werden


% \multicolumn{#Spalten}{Ausrichtung}{Inhalt}

\usepackage{hyperref}
\usepackage{bookmark}
\usepackage[shortcuts]{extdash} %nach hyperref, bookmark

\newcommand{\ua}[1]{_\symup{#1}}
\newcommand{\su}[1]{\symup{#1}}


\title{Versuch 354}
\subtitle{Gedämpfte und erzwungene Schwingungen}
\author{Sebastian Pape\\
        sepa@gmx.de \and
        Jonah Nitschke\\
        lejonah@web.de}
\date{Durchführung: 01.02.2017\\
      Abgabe: 08.02.2017}

\begin{document}

\maketitle
\newpage

\section{Theorie}

In dem folgenden Versuch wird ein elektronischer Kreis, im Kern bestehend aus einem
Widerstand, einer Spule und einem Kondensator. Über diese Schaltung kann eine
gedämpfte Schwingung betrachtet werden, sowie durch Anschluss eines Generators
aus eine erzwungene Schwingung. Im Laufe des Versuches sollen der Dämpfungswiderstand,
die Frequenzabhängigkeit der Kondensatorspannung sowie die Phase zwischen
Erreger- und Kondensatorspannung betrachtet werden.

\subsection{Der gedämpfte Schwingkreis}

Der in diesem Experiment betrachtete gedämpfte Schwingkreis ist in Im Grunde nur
eine Erweiterung des RC-Kreise mit einer Spule. Somit kommen in der Schaltung
zwei Energiespeicher vor, zwischen denen die Energie hin und her pendelt. Durch
den eingebauten Widerstand geht bei der Schwingung Energie in Form von Wärme verloren
und somit wird das ganze System gedämpft. Betrachtet man solch eine Schaltung, kann
mithilfe der Kirchhoffschen Gesetze eine Differentialgleichung für den Strom
aufgestellt werden:

\begin{equation}
  \frac{\su{d}^2I}{\su{d}t^2} + \frac{R}{L}\frac{\su{d}I}{\su{d}t} + \frac{1}{LC} I = 0 .
\end{equation}

Durch Lösen der Differentialgleichung ergibt sich mit einem geeigneten Ansatz für
die Frequenz ein Term, der von der Stärke der Dämpfung abhängig ist:

\begin{equation}
  \tilde{\omega}_{1,2} = j \frac{R}{2L} \pm \sqrt{ \frac{1}{LC} - \frac{R^2}{4L^2}} .
\end{equation}

Je nachdem wie sich der unter der Wurzel stehende Term verhält, können für die
gedämpfte Schwingung verschiedene Fälle betrachtet werden, von denen zwei im folgenden
Erläutert werden.

\textbf{1.Fall :} $\frac{1}{LC} > \frac{R^2}{4L^2}$

In diesem Fall ist der Term mit der Wurzel rein reel und  es entsteht eine
harmonische Schwingung, deren Amplitude mit zunehmender
Zeit gegen Null geht. Die Einhüllende wird dabei durch eine Exponentialfunktion
beschrieben. Mithilfe der Schwingungsdauer lässt sich dann errechnen, nach welcher
Zeit die Amplitude auf den $e$-ten Teil ihrer Ursprungsamplitude abgesenkt ist:

\begin{equation}
  T\ua{ex} := \frac{2L}{R} \su{s} .
\end{equation}

\newpage

\textbf{2.Fall :} $\frac{1}{LC} < \frac{R^2}{4L^2}$

Bei diesem Fall handelt es sich um eine aperiodische Dämpfung, bei der die Lösung
keinen oszillatorischen Teil besitzt. In diesem Versuch wird dabei nur der als
aperiodischer Grenzfall bezeichneter Spezialfall betrachtet, bei dem
$\frac{1}{LC} = \frac{R^2}{4L^2}$ gilt und der Strom ohne Überschwinger am schnellsten
gegen Null geht.

\subsection{Die erzwungene Schwingung}

Wird bei dem vorher betrachteten Schaltkreis noch ein Generator mit eingebaut,
handelt es sich um eine erzwungene Schwingung, für die sich mithilfe der
Kirchhoffschen Gesetze für die Kondensatorspannung $U\ua{C}$
folgende Differentialgleichung ergibt:

\begin{equation}
  LC\frac{\su{d}^2U\ua{C}}{\su{d}t^2} + RC\frac{\su{d}U\ua{C}}{\su{d}t} + U\ua{C} = U\ua{0} \exp^{j\omega t} .
\end{equation}

Aus dieser Differentialgleichung können nun mit einem geeigneten Ansatz eine Funktion
für die Kondensatorspannung $U\ua{c}$ sowie eine Gleichung für die Phasenverschiebung
$\varphi$ zwischen Erreger- und Kondensatorspannung bestimmt werden:

\begin{align}
  \varphi(\omega) &= \arctan \left( \frac{-\omega RC}{1 - LC\omega^2} \right) \\
  U\ua{C}(\omega) &= \frac{U\ua{0}}{ \sqrt{ \left(1 - LC\omega^2 \right)^2 + \omega^2R^2C^2}} .
\end{align}

Die Kondensatorspannung kann bei der sogenannten Resonanzfrequenz $\omega\ua{res}$
auch einen Wert größer als $U\ua{0}$ annehmen:

\begin{equation}
  \omega\ua{res} = \sqrt{\frac{1}{LC} - \frac{R^2}{2L^2}} .
\end{equation}

Wird bei der Schaltung eine schwache Dämpfung betrachtet, für die
$\frac{R^2}{2L^2} \, << \, \frac{1}{LC}$ gilt, nähert sich $\omega\ua{res}$
der Kreisfrequenz $\omega\ua{0}$ der ungedämpften Schwingung an. In diesem
Fall übertrifft $U\ua{C}$ die Erregerspannung $U\ua{0}$ um den Faktor
$\frac{1}{\omega\ua{0}RC}$, welcher auch als Güte des Schwingkreises bezeichnet wird.



\end{document}
