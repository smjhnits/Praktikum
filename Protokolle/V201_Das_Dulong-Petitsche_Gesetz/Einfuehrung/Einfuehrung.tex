% bei Standalone in documentclass noch:
% \RequirePackage{luatex85}

\documentclass[captions=tableheading, titlepage= firstiscover, parskip = half , bibliography=totoc]{scrartcl}
%paper = a5 für andere optinen
% titlepage= firstiscover
% bibliography=totoc für bibdateien
% parskip=half  Veränderung um Absätze zu verbessern

\usepackage{scrhack} % nach \documentclass
\usepackage[aux]{rerunfilecheck}
\usepackage{polyglossia}
\usepackage[style=numeric, backend=biber]{biblatex} % mit [style = alphabetic oder numeric] nach polyglossia
\addbibresource{lit.bib}
\setmainlanguage{german}

\usepackage[autostyle]{csquotes}
\usepackage{amsmath} % unverzichtbare Mathe-Befehle
\usepackage{amssymb} % viele Mathe-Symbole
\usepackage{mathtools} % Erweiterungen für amsmath
\usepackage{fontspec} % nach amssymb
% muss ins document: \usefonttheme{professionalfonts} % für Beamer Präsentationen
\usepackage{longtable}

\usepackage[
math-style=ISO,    % \
bold-style=ISO,    % |
sans-style=italic, % | ISO-Standard folgen
nabla=upright,     % |
partial=upright,   % /
]{unicode-math} % "Does exactly what it says on the tin."
\setmathfont{Latin Modern Math}
% \setmathfont{Tex Gyre Pagella Math} % alternativ

\usepackage[
% die folgenden 3 nur einschalten bei documenten
locale=DE,
separate-uncertainty=true, % Immer Fehler mit ±
per-mode=symbol-or-fraction, % m/s im Text, sonst \frac
]{siunitx}

% alternativ:
% per-mode=reciprocal, % m s^{-1}
% output-decimal-marker=., % . statt , für Dezimalzahlen

\usepackage[
version=4,
math-greek=default,
text-greek=default,
]{mhchem}

\usepackage[section, below]{placeins}
\usepackage{caption} % Captions schöner machen
\usepackage{graphicx}
\usepackage{grffile}
\usepackage{subcaption}

% \usepackage{showframe} Wenn man die Ramen sehen will

\usepackage{float}
\floatplacement{figure}{htbp}
\floatplacement{table}{htbp}

\usepackage{mhchem} %chemische Symbole Beispiel: \ce{^{227}_{90}Th+}


\usepackage{booktabs}

 \usepackage{microtype}
 \usepackage{xfrac}

 \usepackage{expl3}
 \usepackage{xparse}

 % \ExplSyntaxOn
 % \NewDocumentComman \I {}  %Befehl\I definieren, keine Argumente
 % {
 %    \symup{i}              %Ergebnis von \I
 % }
 % \ExplSyntaxOff

 \usepackage{pdflscape}
 \usepackage{mleftright}

 % Mit dem mathtools-Befehl \DeclarePairedDelimiter können Befehle erzeugen werden,
 % die Symbole um Ausdrücke setzen.
 % \DeclarePairedDelimiter{\abs}{\lvert}{\rvert}
 % \DeclarePairedDelimiter{\norm}{\lVert}{\rVert}
 % in Mathe:
 %\abs{x} \abs*{\frac{1}{x}}
 %\norm{\symbf{y}}

 % Für Physik IV und Quantenmechanik
 \DeclarePairedDelimiter{\bra}{\langle}{\rvert}
 \DeclarePairedDelimiter{\ket}{\lvert}{\rangle}
 % <name> <#arguments> <left> <right> <body>
 \DeclarePairedDelimiterX{\braket}[2]{\langle}{\rangle}{
 #1 \delimsize| #2
 }

\setlength{\delimitershortfall}{-1sp}

 \usepackage{tikz}
 \usepackage{tikz-feynman}

 \usepackage{csvsimple}
 % Tabellen mit \csvautobooktabular{"file"}
 % muss in table umgebung gesetzt werden


% \multicolumn{#Spalten}{Ausrichtung}{Inhalt}

\usepackage{hyperref}
\usepackage{bookmark}
\usepackage[shortcuts]{extdash} %nach hyperref, bookmark

\newcommand{\ua}[1]{_\symup{#1}}
\newcommand{\su}[1]{\symup{#1}}


\title{Versuch 201}
\subtitle{Das Dulong-Petitsche Gesetz}
\author{Jonah Nitschke\\
        lejonah@web.de \and
        Sebastian Pape\\
        sepa@gmx.de}
\date{Durchführung: 29.11.2016\\
      Abgabe: 06.12.2016}

\begin{document}

\maketitle
\tableofcontents
\newpage

\section{Einfuehrung}
Bei dem folgenden Versuch geht es darum, die Aussage des Dulong-Petitschen Gesetzes
über die Gleichmäßigkeit der Molwärme von verschiedenen Stoffen zu überprüfen und
darauf basierend zu Entscheiden, ob die klassische Mechanik zur Beschreibung
der oszillatorischen Bewegung von Atomen ausreicht oder ob dies nur auf der
Grundlage der Quantenmechanik geschehen kann.

\section{Theorie}

\subsection{\texorpdfstring{$\emph{spezifische Wärmekapazität}$}{z}}

Erhöht sich die Temperatur eines Körpers um $\increment T$ so ergibt sich
für aufgenommene Wärmeenergie und Temperaturdifferenz entsprechend
des 1. Thermodynamischen Hauptsatzes folgende Beziehung:

\begin{equation}
  \increment Q = m c \increment T
\end{equation}

Bei c handelt es sich dabei um die Wärmekapazität bzw. im Bezug auf den
vorliegenden Stoff um die \emph{spezifische Wärmekapazität}. Von
Bedeutung für das vorliegende Experiment ist zudem die \emph{Molwärme} C.
Sie beschreibt die benötigte Wärmemenge, um ein Mol eines Stoffes um dT zu
erwärmen. Dabei wird noch unter $C_V$ für konstante Volumen und $C_p$ für
konstanten Druck unterschieden.

\subsection{Dulong-Petit}

Das Dulong-Petitsche Gesetz trifft die Aussage, dass die Atomwärme bei konstanten
Volumen $C_V$ im festen Aggregatzustand unabhängig von dem Charakter des Elements
ist, sondern konstant den Wert 3R annimmt (R = Allgemeine Gaskonstante). Die
Herleitung dieses Zusammenhanges basiert dabei auf der Annahme, dass Atome in
einem Festkörpergitter um feste Punkte schwingen, und ihre potentielle und
kinetische Energie dabei gleich sind. Gleichzeitig besagt das
\emph{Äquipartitionstheorem}, dass ein Atom dabei die kinetische Energie
$ < E_{kin} > = 1/2 k T$ besitzt. Aus beiden folgt dann für die mittlere Energie
des Festkörpers der Wert 3RT und aus diesem der oben geschriebene Wert für die
\emph{Molwärme} $C_V$.

Die Molwärme $C_V$ von allen festen chemischen Elementen nimmt bei hoher
Temperatur tatsächlich etwa den Wert 3R an, bei vielen Stoffen auch schon bei
Zimmertemperatur. Die kinetische Theorie kann allerdings nicht beschreiben, warum
die Molwärme aller chemischen Elementen bei hinreichend tiefen Temperaturen
beliebig klein wird.

Das liegt an der Annahme, dass die Energie der atomaren Oszillatoren sich
beliebig ändern kann. Dies steht im Widerspruch zu Quantentheorie, die
Energieänderungen nur in bestimmten Beträgen erlaubt. Die nun auf
komplizierte Weise von T abhängige mittlere Energie kann somit nur über
eine Aufsummierung der verschiedenen Energiezustände multipliziert mit der
jeweiligen Wahrscheinlichkeit ihres Auftretens geschehen. Mit der
Boltzmann-Verteilung ergibt sich für die innere Energie damit folgender Ausdruck:

\begin{equation}
  < U_{qu} > = \frac{ 3 N_l \hbar \omega}{\textbf{exp} (\hbar \omega / kT) - 1}
\end{equation}

Für den Grenzfall von $ T \to \infty$ ergibt sich jedoch auch hier wieder der
bekannte Zusammenhang von $< U_{qu} > = 3RT$.

\subsection{Messung der spez. Wärmekapazität fester Körper mit dem Mischungskalorimeter}

Da eine Messung mit konstantem Volumen schwer zu realisieren ist, wird bei dem
Experiment auf ein konstanten Druck zurückgegriffen. Dafür ist es notwendig, den
Zusammenhang zwischen $C_V$ und $C_p$ zu kennen:

\begin{equation}
  C_p - C_V = 9 \alpha^2 \kappa V_0 T
\end{equation}

Bei $\alpha$, $\kappa$ und $V_=0$ handelt es sich um Konstanten. Für die abgegebene
Wärmeenergie des Körpers $(Q_1)$ und die aufgenommene Wärmemenge der Kalorimeterwände
ergeben sich folgende Formel:

\begin{align}
  Q_1 &= c_k m_k (T_k - T_m) \\
  Q_2 &= (c_W m_W + c_g m_g) (T_k - T_m)
\end{align}

Bei $T_m$ handelt es sich um die sich einstellende Mischtemperatur und alle Werte
mit Index k und W beziehen sich auf den Körper bzw. das Wasser.

Bei dem Experiment wird ein Wärmeverlust des Systems vernachlässigt, so dass
für die Bestimmung der spez. Wärmekapazität des Körpers die obigen Formeln
lediglich gleichgesetzt werden müssen $(Q_1=Q_2)$:

\begin{equation}
  c_k = \frac{(c_W m_W + c_g m_g) (T_k - T_m)}{m_k (T_k - T_m)}
\end{equation}

Die Wärmekapazität $c_g m_g$ des Kalorimeters muss am Anfang des Experiments
noch einmal extra bestimmt werden und kann mit folgender Formel errechnet werden:

\begin{equation}
  \label{eqn:cgmg}
  c_g m_g = \frac{c_W m_y \left(T_y - T_m' \right) - c_W m_x \left(T_x - T_m' \right)}{\left( T_m' - T_x \right)}
\end{equation}


\section{Messtechnische Hinweise}

Für die Messung der Temperaturen bei diesem Versuch wird ein Thermoelement benutzt.


%
%
%

\section{Durchführung}

Bevor die Messungen für die \emph{spezifische Wärmekapazität} begonnen werden,
muss zuerst eine Referenzwert ermittelt werden, um hinterher mithilfe der gemessenen
Spannungen die vorliegenden Temperaturen zu bestimmen. Dafür wird ein Dewar-Gefäß
mit Eiswasser gefüllt und in einem Becherglas zum Kochen gebracht. Erst werden
beide Enden des Kabels in das Eiswasser gelegt, um die Spannungsdifferent bei
einem Temperaturunterschied von 0 K zu überprüfen. Anschließend wird ein Ende in
das kochende Wasser gelegt um die Spannungsdifferenz bei einem Unterschied von
100 K zu überprüfen. Mit beiden Spannungen kann ein lineare Regression ausgeführt
werden um die Paramter zur Bestimmmung der Temperatur in Abhängigkeit der
gemessenen Spannungen zu bestimmen.

Danach wird noch die die \emph{spezifische Wärmekapazität} des Kalorimeters bestimmt.
In das Dewar-Gefäß werden zwei Mengen an Wasser der Masse $m_x$ und $m_y$ und der
Temperaturen $T_x$ und $T_y$ gegeben. Anschließend wird die Mischtemperatur
$T_m$ gemessen und mit \eqref{eqn:cgmg} kann die \emph{spezifische Wärmekapazität}
bestimmt werden.

Nach den Vorbereitungen kann mit den eigentlichen Messungen gestartet werden.
Zuerst wird das Dewar-Gefäß mit der gleichen Flüssigkeitsmenge wie bei der
Bestimmung von $c_g m_g$ gefüllt. Dann wird das Blei in dem Becherglas mit kochendem
Wasser aufgeheizt. Dabei wird mit dem Thermoelement die Temperatur des Bleis
gemessen. Vor dem Eintauchen des Körpers in das Diwar-Gefäß wird die
Temperatur des Wassers noch einmal bestimmt. Nach einer kurzen Wartezeit wird
dann die Mischtemperatur gemessen.
Für die Metalle Blei 2 und Graphit wird die Messung drei mal durchgeführt.
Anschließend wird die Messung noch einmal für Kupfer durchgeführt.

\end{document}
