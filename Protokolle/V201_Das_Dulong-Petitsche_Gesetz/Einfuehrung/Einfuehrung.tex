\input{"../../Praeambel_prak.tex"}

\title{Versuch 201}
\subtitle{Das Dulong-Petitsche Gesetz}
\author{Jonah Nitschke\\
        lejonah@web.de \and
        Sebastian Pape\\
        sepa@gmx.de}
\date{Durchführung: 29.11.2016\\
      Abgabe: 06.12.2016}

\begin{document}

\maketitle
\tableofcontents
\newpage

\section{Einfuehrung}
Bei dem folgenden Versuch geht es darum, die Aussage des Dulong-Petitschen Gesetzes
über die Gleichmäßigkeit der Molwärme von verschiedenen Stoffen zu überprüfen und
darauf basierend zu Entscheiden, ob die klassische Mechanik zur Beschreibung
der oszillatorischen Bewegung von Atomen ausreicht oder ob dies nur auf der
Grundlage der Quantenmechanik geschehen kann.

\section{Theorie}

\subsection{spezifische Wärmekapazität}

Erhöht sich die Temperatur eines Körpers um $\increment T$ so ergibt sich
für aufgenommene Wärmeenergie und Temperaturdifferenz entsprechend
des 1. Thermodynamischen Hauptsatzes folgende Beziehung:

\begin{equation}
  \increment Q = m c \increment T
\end{equation}

Bei c handelt es sich dabei um die Wärmekapazität bzw. im Bezug auf den
vorliegenden Stoff um die \emph{spezifische Wärmekapazität}. Von
Bedeutung für das vorliegende Experiment ist zudem die \emph{Molwärme} C.
Sie beschreibt die benötigte Wärmemenge, um ein Mol eines Stoffes um dT zu
erwärmen. Dabei wird noch unter $C_V$ für konstante Volumen und $C_p$ für
konstanten Druck unterschieden.

\subsection{Dulong-Petit}

Das Dulong-Petitsche Gesetz trifft die Aussage, dass die Atomwärme bei konstanten
Volumen $C_V$ im festen Aggregatzustand unabhängig von dem Charakter des Elements
ist, sondern konstant den Wert 3R annimmt (R = Allgemeine Gaskonstante). Die
Herleitung dieses Zusammenhanges basiert dabei auf der Annahme, dass Atome in
einem Festkörpergitter um feste Punkte schwingen, und ihre potentielle und
kinetische Energie dabei gleich sind. Gleichzeitig besagt das
\emph{Äquipartitionstheorem}, dass ein Atom dabei die kinetische Energie
$ < E_{kin} > = 1/2 k T$ besitzt. Aus beiden folgt dann für die mittlere Energie
des Festkörpers der Wert 3RT und aus diesem der oben geschriebene Wert für die
\emph{Molwärme} $C_V$.

Die Molwärme $C_V$ von allen festen chemischen Elementen nimmt bei hoher
Temperatur tatsächlich etwa den Wert 3R an, bei vielen Stoffen auch schon bei
Zimmertemperatur. Die kinetische Theorie kann allerdings nicht beschreiben, warum
die Molwärme aller chemischen Elementen bei hinreichend tiefen Temperaturen
beliebig klein wird.

Das liegt an der Annahme, dass die Energie der atomaren Oszillatoren sich
beliebig ändern kann. Dies steht im Widerspruch zu Quantentheorie, die
Energieänderungen nur in bestimmten Beträgen erlaubt. Die nun auf
komplizierte Weise von T abhängige mittlere Energie kann somit nur über
eine Aufsummierung der verschiedenen Energiezustände multipliziert mit der
jeweiligen Wahrscheinlichkeit ihres Auftretens geschehen. Mit der
Boltzmann-Verteilung ergibt sich für die innere Energie damit folgender Ausdruck:

\begin{equation}
  < U_{qu} > = \frac{ 3 N_l \hbar \omega}{\textbf{exp} (\hbar \omega / kT) - 1}
\end{equation}

Für den Grenzfall von $ T \to \infty$ ergibt sich jedoch auch hier wieder der
bekannte Zusammenhang von $< U_{qu} > = 3RT$.


\end{document}
