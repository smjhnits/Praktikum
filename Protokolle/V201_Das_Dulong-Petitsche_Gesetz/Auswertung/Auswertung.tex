\input{../../Praeambel_prak.tex}
\begin{document}

\section{Auswertung}

\subsection{Justieren des Thermoelementes}
Zu Beginn des Veruches musste das Thermoelement justiert werden. Für die
Ausgleichgerade ergibt sich die folgende Funktion.

\begin{align*}
  T(U) &\approx 25.77 \cdot U + 273,15 & [T(U)] = \si{\kelvin}
\end{align*}

Die Funktion ist abhängig von den gemessenen Spannungen $U$ ($[U] = \si{\milli\volt}$).

\subsection{Bestimmen der spezifischen Wärmekapazität des Kalorimeters}

Zu Beginn des Versuches wurde die spezifische Wärmekapazität des Kalorimeters
$c_gm_g$ bestimmt, da diese Größe für die Berechnung der spezifischen
Wärmekapazität der Stoffe $c_k$ notwendig ist. Mittels Formel (??) wurde
$c_gm_c$ ermittelt. Mit den folgenden Werten wurde $c_gm_g$ berechnet.

\begin{description}
  \item[$T_x$]$ = \SI{294,28}{\kelvin}$
  \item[$T_y$]$ = \SI{354,59}{\kelvin}$
  \item[$T_m$]$ = \SI{322,38}{\kelvin}$
  \item[$m_x$]$ = \SI{278,97}{\gram}$
  \item[$m_y$]$ = \SI{298,98}{\gram}$
\end{description}

Für die spezifische Wärmekapazität von Wasser wurde der Wert $c_w =
\SI{4,18}{\joule\per\gram\per\kelvin}$ verwendet.
Es ergibt sich ein Wert von $c_gm_g = \SI{267,09}{\joule\per\kelvin}$.

\section{Bestimmen der spezifischen Wärmekapazität von verschiedenen Stoffen}

Es wurden in dem Versuch die spezifische Wärmekapazität der Stoffe Graphit,
Blei und Kupfer bestimmt, wobei für Blei die Probe Blei 2 verwendet wurde.
Für Graphit und Blei wurden jeweils drei Messungen und für Kupfer lediglich eine
Messung durchgeführt. Die spezifische Wärmekapazität $c_k$ eines Körpers
wird über Formel (??) ermittelt. In der beiliegenden Tabelle sind die gemessenen
Größen des jeweiligen Stoffes eingetragen.

\floatplacement{table}{htpb}
\begin{table}
 \centering
 \caption{Messdaten der verwendeten Stoffe}
 \label{tab:Messdaten1}
 \begin{tabular}[width=0.4\textwidth]{S S[table-format=3.2] S[table-format=3.2]
   S[table-format=3.2] S[table-format=3.2]}
     \toprule
     {}  & {$T_w$ in $\si{\kelvin}$} & {$T_k$ in $\si{\kelvin}$} &
     {$T_m$ in $\si{\kelvin}$} & {$m_w$ in $\si{\gram}$} \\
     \midrule
     \text{Graphit} & & & & \\
     \text{Messung 1} & 293,77 & 377,27 & 296,09 & 772,50 \\
     \text{Messung 2} & 297,38 & 374,44 & 299,70 & 772,50 \\
     \text{Messung 3} & 299,95 & 375,45 & 302,53 & 772,50 \\
     & & & & \\
     \text{Blei} & & & & \\
     \text{Messung 1} & 295,31 & 371,60 & 296,86 & 765,89 \\
     \text{Messung 2} & 296,86 & 369,28 & 298,41 & 765,89 \\
     \text{Messung 3} & 298,41 & 370,57 & 299,95 & 765,89 \\
     & & & & \\
     \text{Kupfer} & & & & \\
     \text{Messung 1} & 293,77 & 377,79 & 294,80 & 769,56 \\
     \bottomrule
\end{tabular}
\end{table}
\FloatBarrier

Für die untersuchten Proben ergibt sich somit:

\begin{align*}
  c_{Graphit} &= (\num{1,02 +- 0,064})\si{\joule\per\gram\per\kelvin} \\
  c_{Blei} &= (\num{0,19 +- 0,004})\si{\joule\per\gram\per\kelvin} \\
  c_{Kupfer} &= \num{0,18}\si{\joule\per\gram\per\kelvin} \\
\end{align*}

Die Fehler für Graphit und Blei wurden über die folgende Formel bestimmt.

\begin{equation}
  \label{eqn:Fehler}
  \increment\bar{x} = \sqrt{\frac{1}{N(N - 1)}
  \cdot\sum_{i = 1}^N(x_i-\bar{x})^2}
\end{equation}

Dabei ist $\bar{x}$ der Mittelwert der gemessenen Größe.

\subsection{Bestimmen der Atomwärme}

Damit die Atomwärme $C_p$ eines Stoffes bestimmt werden kann, muss die
spezifische Wärmekapazität dieses mit seiner Molarenmasse multipliziert werden.

\begin{equation}
  C_p = c_k\cdot M
\end{equation}

Für den jewiligen Stoff ergibt sich somit:

\begin{align*}
  C_{pG} &= (\num{12,29 +- 0,768})\si{\joule\per\mol\per\kelvin} \\
  C_{pB} &= (\num{39,99 +- 0,727})\si{\joule\per\mol\per\kelvin} \\
  C_{pK} &= \num{11,50}\si{\joule\per\mol\per\kelvin}
\end{align*}

\subsection{Vergleich mit Dulong-Petit}

Aus den Überlegungen der klassischen Mechanik, die bereits in der Theorie
erwähnt wurden ergibt sich, dass die Atomwärme $C_V$ materialunabhängig
und konstant den Wert $3\cdot R \approx \SI{24,94}{\joule\per\mol\per\kelvin}$
beträgt. Nun gilt es diese Aussage zuüberprüfen.
Der Zusammenhang zwischen $C_p$ und $C_V$ ist nach Formel (??) bekannt.
Für die geprüften Stoffe ergit sich:

\begin{align*}
  C_{VG} &= (\num{12.26}\pm\num{0,0002})\si{\joule\per\kelvin} \\
  C_{VB} &= (\num{38,26}\pm\num{0,0052})\si{\joule\per\kelvin} \\
  C_{VK} &= \num{10,78}\si{\joule\per\kelvin}
\end{align*}



\end{document}
