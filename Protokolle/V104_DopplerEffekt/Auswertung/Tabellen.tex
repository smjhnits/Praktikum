% bei Standalone in documentclass noch:
% \RequirePackage{luatex85}

\documentclass[captions=tableheading, ]{scrartcl}
%paper = a5 für andere optinen
% titlepage= firstiscover
% bibliography=totoc für bibdateien
% parskip=half  Veränderung um Absätze zu verbessern

\usepackage{scrhack} % nach \documentclass
\usepackage[aux]{rerunfilecheck}
\usepackage{polyglossia}
\usepackage[style=alphabetic]{biblatex} % mit [style = alphabetic oder numeric]
% nach polyglossia
\setmainlanguage{german}

\usepackage{scrpage2}   %Kopf und Fußzeilen
\pagestyle{scrheadings}  %\ihead{Kopfzeile innen} \chead{Kopfzeile Mitte} \ohead{Kopfzeile außen}
                         %\ifoot{Fußzeile innen} \cfoot{Fußzeile Mitte} \ofoot{Fußzeile außen}
\clearscrheadfoot

\usepackage[autostyle]{csquotes}
\usepackage{amsmath} % unverzichtbare Mathe-Befehle
\usepackage{amssymb} % viele Mathe-Symbole
\usepackage{mathtools} % Erweiterungen für amsmath
\usepackage{fontspec} % nach amssymb
% muss ins document: \usefonttheme{professionalfonts} % für Beamer Präsentationen

\usepackage[
math-style=ISO,    % \
bold-style=ISO,    % |
sans-style=italic, % | ISO-Standard folgen
nabla=upright,     % |
partial=upright,   % /
]{unicode-math} % "Does exactly what it says on the tin."
\setmathfont{Latin Modern Math}
% \setmathfont{Tex Gyre Pagella Math} % alternativ

\usepackage[
% die folgenden 3 nur einschalten bei documenten
locale=DE,
separate-uncertainty=true, % Immer Fehler mit ±
per-mode=symbol-or-fraction, % m/s im Text, sonst \frac
]{siunitx}

% alternativ:
% per-mode=reciprocal, % m s^{-1}
% output-decimal-marker=., % . statt , für Dezimalzahlen

\usepackage[
version=4,
math-greek=default,
text-greek=default,
]{mhchem}

\usepackage[section, below]{placeins}
\usepackage{caption} % Captions schöner machen
\usepackage{graphicx}
\usepackage{grffile}
\usepackage{subcaption}

% \usepackage{showframe} Wenn man die Ramen sehen will

\usepackage{float}
\floatplacement{figure}{htbp}
\floatplacement{table}{htbp}

\usepackage{booktabs}
 \addbibresource{lit.bib}

 \usepackage{microtype}
 \usepackage{xfrac}

 \usepackage{expl3}
 \usepackage{xparse}

 % \ExplSyntaxOn
 % \NewDocumentComman \I {}  %Befehl\I definieren, keine Argumente
 % {
 %    \symup{i}              %Ergebnis von \I
 % }
 % \ExplSyntaxOff

 \usepackage{pdflscape}
 \usepackage{mleftright}

 % Mit dem mathtools-Befehl \DeclarePairedDelimiter können Befehle erzeugen werden,
 % die Symbole um Ausdrücke setzen.
 % \DeclarePairedDelimiter{\abs}{\lvert}{\rvert}
 % \DeclarePairedDelimiter{\norm}{\lVert}{\rVert}
 % in Mathe:
 %\abs{x} \abs*{\frac{1}{x}}
 %\norm{\symbf{y}}

 % Für Physik IV und Quantenmechanik
 \DeclarePairedDelimiter{\bra}{\langle}{\rvert}
 \DeclarePairedDelimiter{\ket}{\lvert}{\rangle}
 % <name> <#arguments> <left> <right> <body>
 \DeclarePairedDelimiterX{\braket}[2]{\langle}{\rangle}{
 #1 \delimsize| #2
 }

 \usepackage{tikz}
 \usepackage{tikz-feynman}


% \multicolumn{#Spalten}{Ausrichtung}{Inhalt}

\usepackage[unicode]{hyperref}
\usepackage{bookmark}


\usepackage{csvsimple}

\csvset{
  autotabularcenter/.style={
    file=#1,
    after head=\csv@pretable\begin{tabular}{|*{\csv@columncount}{c|}}\csv@tablehead,
    table head=\hline\csvlinetotablerow\\\hline,
    late after line=\\,
    table foot=\\\hline,
    late after last line=\csv@tablefoot\end{tabular}\csv@posttable,
    command=\csvlinetotablerow},
  autobooktabularcenter/.style={
    file=#1,
    after head=\csv@pretable\begin{tabular}{*{\csv@columncount}{c}}\csv@tablehead,
    table head=\toprule\csvlinetotablerow\\\midrule,
    late after line=\\,
    table foot=\\\bottomrule,
    late after last line=\csv@tablefoot\end{tabular}\csv@posttable,
    command=\csvlinetotablerow},
}

\begin{document}

Messung a) :

\begin{table}
 \centering
 \csvautobooktabular{MessungA.csv}
\end{table}

Messung b) :

\begin{table}
 \centering
  \begin{tabular}{c c}
    \toprule
    {Strecke[cm]} & {Phasenverschiebung} \\
    \midrule
    0.0  & 0     \\
    0.9  & $\pi$ \\
    1.8  & 0     \\
    2.7  & $\pi$ \\
    3.5  & 0     \\
    4.4  & $\pi$ \\
    \bottomrule
  \end{tabular}
\end{table}

\begin{table}
 \centering
  \begin{tabular}{c c c}
    \toprule
    {Differenz der Positionen} & {$\frac{\lambda}{s}$ [cm]} & {$\lambda$ [cm]} \\
    \midrule
    2-1  & 0.09 & 0.18  \\
    3-2  & 0.09 & 0.18  \\
    4-3  & 0.09 & 0.18  \\
    5-4  & 0.08 & 0.16  \\
    6-5  & 0.09 & 0.18  \\
    \bottomrule
  \end{tabular}
\end{table}

\begin{landscape}
  Messung c) :

  1) Entgegen der Ausbreitungsrichtung
  \begin{table}
    \csvautobooktabular{MessungC.csv}
  \end{table}

  2) Mit der Ausbreitungsrichtung
  \begin{table}
    \csvautobooktabular{MessungC2.csv}
  \end{table}p
\end{landscape}


Nach dem Gang Nr. 42 mussten wir die Messreihe beenden, das die Strecke zu kurz wurde
und immer entsprechend bei den Gängen 42, 48, 52 und 60 die gleichen Ergebnisse
entstanden wie bei Gang 36 ( Vergleiche Gänge 36 und 42 in Tabelle c)).

Schwebungsmethode:
\\
\\
Für den Gang 30 haben wir 10 Werte gemesse:

\begin{table}
  \centering
  \begin{tabular}{c c}
    \toprule
    {Messung} & {Impulsveränderung} \\
    \midrule
    1.  & 30 \\
    2.  & 17 \\
    3.  & 8  \\
    4.  & 30 \\
    5.  & 10 \\
    6.  & 30 \\
    7.  & 30 \\
    8.  & 30 \\
    9.  & 31 \\
    10. & 30 \\
    \bottomrule
  \end{tabular}
\end{table}

Die restlichen Werte:

\begin{table}
 \centering
 \csvautobooktabular{MessungCS.csv}
\end{table}

Die Geschwindigkeiten:

\begin{table}
  \centering
  \csvautobooktabular{Speed.csv}
\end{table}

\begin{table}
  \centering
  \csvautobooktabular{Delta1.csv}
\end{table}

\begin{table}
  \centering
  \csvautobooktabular{Delta2.csv}
\end{table}

\end{document}
