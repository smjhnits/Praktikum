\input{"../../Praeambel_doc.tex"}

\usepackage{csvsimple}

\csvset{
  autotabularcenter/.style={
    file=#1,
    after head=\csv@pretable\begin{tabular}{|*{\csv@columncount}{c|}}\csv@tablehead,
    table head=\hline\csvlinetotablerow\\\hline,
    late after line=\\,
    table foot=\\\hline,
    late after last line=\csv@tablefoot\end{tabular}\csv@posttable,
    command=\csvlinetotablerow},
  autobooktabularcenter/.style={
    file=#1,
    after head=\csv@pretable\begin{tabular}{*{\csv@columncount}{c}}\csv@tablehead,
    table head=\toprule\csvlinetotablerow\\\midrule,
    late after line=\\,
    table foot=\\\bottomrule,
    late after last line=\csv@tablefoot\end{tabular}\csv@posttable,
    command=\csvlinetotablerow},
}

\begin{document}

Messung a) :

\begin{table}
 \centering
 \csvautobooktabular{MessungA.csv}
\end{table}

Messung b) :

\begin{table}
 \centering
  \begin{tabular}{c c}
    \toprule
    {Strecke[cm]} & {Phasenverschiebung} \\
    \midrule
    0.0  & 0     \\
    0.9  & $\pi$ \\
    1.8  & 0     \\
    2.7  & $\pi$ \\
    3.5  & 0     \\
    4.4  & $\pi$ \\
    \bottomrule
  \end{tabular}
\end{table}

\begin{table}
 \centering
  \begin{tabular}{c c c}
    \toprule
    {Differenz der Positionen} & {$\frac{\lambda}{s}$ [cm]} & {$\lambda$ [cm]} \\
    \midrule
    2-1  & 0.09 & 0.18  \\
    3-2  & 0.09 & 0.18  \\
    4-3  & 0.09 & 0.18  \\
    5-4  & 0.08 & 0.16  \\
    6-5  & 0.09 & 0.18  \\
    \bottomrule
  \end{tabular}
\end{table}

\begin{landscape}
  Messung c) :

  1) Entgegen der Ausbreitungsrichtung
  \begin{table}
    \csvautobooktabular{MessungC.csv}
  \end{table}

  2) Mit der Ausbreitungsrichtung
  \begin{table}
    \csvautobooktabular{MessungC2.csv}
  \end{table}p
\end{landscape}


Nach dem Gang Nr. 42 mussten wir die Messreihe beenden, das die Strecke zu kurz wurde
und immer entsprechend bei den Gängen 42, 48, 52 und 60 die gleichen Ergebnisse
entstanden wie bei Gang 36 ( Vergleiche Gänge 36 und 42 in Tabelle c)).

Schwebungsmethode:
\\
\\
Für den Gang 30 haben wir 10 Werte gemesse:

\begin{table}
  \centering
  \begin{tabular}{c c}
    \toprule
    {Messung} & {Impulsveränderung} \\
    \midrule
    1.  & 30 \\
    2.  & 17 \\
    3.  & 8  \\
    4.  & 30 \\
    5.  & 10 \\
    6.  & 30 \\
    7.  & 30 \\
    8.  & 30 \\
    9.  & 31 \\
    10. & 30 \\
    \bottomrule
  \end{tabular}
\end{table}

Die restlichen Werte:

\begin{table}
 \centering
 \csvautobooktabular{MessungCS.csv}
\end{table}

Die Geschwindigkeiten:

\begin{table}
  \centering
  \csvautobooktabular{Speed.csv}
\end{table}

\begin{table}
  \centering
  \csvautobooktabular{Delta1.csv}
\end{table}

\begin{table}
  \centering
  \csvautobooktabular{Delta2.csv}
\end{table}

\end{document}
