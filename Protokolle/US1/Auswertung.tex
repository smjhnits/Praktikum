\section{Auswertung}

Die verwendeten Zylinder wurden mithilfe einer Schieblehre vermessen und
hatten die folgenden Längen. Dabei werden die Werte als fehlerfrei angenommen.

\begin{description}
  \item[Zylinder 1] \SI{40,35}{\milli\meter}
  \item[Zylinder 2] \SI{80,55}{\milli\meter}
  \item[Zylinder 3] \SI{80,45}{\milli\meter}
  \item[Zylinder 4] \SI{102,1}{\milli\meter}
  \item[Zylinder 5] \SI{31,1}{\milli\meter}
  \item[Zylinder 6] \SI{39,7}{\milli\meter}
  \item[Zylinder 7] \SI{61,5}{\milli\meter}
\end{description}

Für die Messung mit Ultraschall wurde an dem Gerät der Literaturwert
der Schallgeschwindigkeit in Acryl ($c\ua{Acyl, lit} = \SI{2730}{\meter\per\second}$\cite{lit})
eingegeben. Damit wurde die Zeitachse auf diesen Wert kalibriert.

\subsection{Bestimmung der Dämpfungskonstante und der Schallgeschwindigkeit mit dem Impuls--Echo--Verfahren}

Die Dämpfungskonstante $\alpha$ aus \eqref{eqn:Intensität}
lässt sich aus den genommenen Daten berechnen. Dafür werden die Messdaten
in die Formel eingesetzt und nach es wird $\alpha$ aufgelöst.

\begin{equation}
  \label{eqn:dämpfung}
  \alpha = - \frac{1}{x_1} \ln{\left(\frac{I_0^\text{'}}{I_0}\right)}
\end{equation}

Dabei ist $I_0^\text{'}$ die Amplitude an der Stelle $x_1 > 0$ und $I_0$
die Amplitude an der Stelle $x  = 0$.

Die Messdaten sind in Tabelle \ref{tab:} dargestellt.
Die Werte für Zylinder 4 und den zusammengesetzten Zylinder mit dem
achtem Messwert wurden für die Berechnung der Dämpfungskonstante nicht
verwendet, weil die Peaks nur bei Verstärkug ausgemessen werden konnten.
Auf das Herausrechnen des Verstärkunsfaktors wurde der Einfachheit halber verzichtet.
Im Mittel ergibt sich die Dämpfungskontante zu:

\begin{equation}
  \alpha = \SI{21.257(301)}{\per\meter}
\end{equation}

Der Fehler ist die Standardabweichung des Mittelwertes.
Eine Darstellung der Dämpfung in Acryl ist in dem folgende Diagramm
eizusehen. Dabei sind die Messdaten der Anfangs- und Endamplituden
ebenfalls eingetragen.


\begin{figure}
  \centering
  \includegraphics[width=\textwidth]{Pics/Dämpfung.pdf}
  \caption{Darstellung der Dämpfung der Amplitude mit Zunahme der Strecke.}
  \label{fig:Dämpfung}
\end{figure}

Die Schallgeschwindigkeit ist aus den Laufzeiten zwischen den
gemessenen Peaks und den vermessenen Zylinderlägen mithilfe von
\eqref{eqn:Fehlstellen} zu berechnen. Dafür wurde mit dem
\emph{Python}-Packet \emph{curve\_fit} eine lineare Ausgleichsrechung
durchgeführt. Der systematische Fehler der Sonde ist der Ordinaten-Abschnitt
des Ausgleichgeraden und die Schallgeschwindigkeit ist in der Steigung
wiederzufinden.

Die Werte ergeben sich zu:

\begin{align}
  \label{eqn:schallgesch_echo}
  c\ua{Acryl, echo} &= \SI{2880.943}{\meter\per\second} \\
  \Delta\ua{Sonde,echo} &= \SI{-3.862}{\meter\per\second}
\end{align}

Das zugehörige Diagramm der Messung ist im Folgendem dargestellt.

\begin{figure}
  \centering
  \includegraphics[width=\textwidth]{Pics/schallgesch_echo.pdf}
  \caption{Schallgeschwindigkeit in Acryl, bestimmt über das Impuls-Echo-Verfahren.}
  \label{fig:schallgesch_echo}
\end{figure}

\subsection{Bestimmung der Schallgeschwindigkeit mit dem Durchschallungsverfahren}

Die Messdaten zum Durchschallungsverfahren sind in Tabelle \ref{tab:}
dargestellt.
Es wurde gleich verfahren wie bei dem Impuls-Echo-Verfahren.

Die Werte ergeben sich zu:

\begin{align}
  \label{eqn:schallgesch_durch}
  c\ua{Acryl, durch} &= \SI{2878.377}{\meter\per\second} \\
  \Delta\ua{Sonde, durch} &= \SI{-2.946}{\meter\per\second}
\end{align}

Das zugehörige Diagramm der Messung ist im Folgendem dargestellt.

\begin{figure}
  \centering
  \includegraphics[width=\textwidth]{Pics/schallgesch_durch.pdf}
  \caption{Schallgeschwindigkeit in Acryl, bestimmt über das Durchschallungsverfahren.}
  \label{fig:schallgesch_durch}
\end{figure}


\subsection{Spektrale Analyse und Cepstrum}

Die verwendeten Acrylplatten wurde mit einer Schieblehre vermessen.
Die Dicken wurden ebenfalls als fehlerfrei angenommen.

\begin{description}
  \item[Platte 1] $d_1 = \SI{6}{\milli\meter}$
  \item[Platte 2] $d_2 = \SI{9.9}{\milli\meter}$
\end{description}

Das Spektrum und das Cepstrum wurde aufgenommen. Die genommenen
Diagramme sind im Folgendem dargestellt.

\begin
{figure}
\centering
\begin{subfigure}{0.48\textwidth}
\centering
\includegraphics[height=3cm]{Pics/Z6_M3_S.bmp}
\caption{Aufgenommenes Spektrum.}
\label{fig:Spektrum}
\end{subfigure}
\begin
{subfigure}{0.48\textwidth}
\centering
\includegraphics
[height=3cm]{Pics/Z6_M3_C.bmp}
\caption{Aufgenommnes Cepstrum.}
\label{fig:Cepstrum}
\end{subfigure}
\end{figure}

Anhand des Spektrums ...

\subsection{Biometrische Untersuchung eines Augenmodells}

Es wurde eine Auge wie aus Abb. \ref{fig:Auge} untersucht.
Insgesamt wurden fünf Peaks aufgenommen und dem Auge sind
die folgenden Bestandteile zuzuordnen.

\begin{enumerate}
  \item Hornhaut
  \item Iris
  \item Linseneingang
  \item Linsenausgang
  \item Retina
\end{enumerate}

Die Peaks sind chronologisch den Bestandteilen zuzuordnen.
Für die Bereiche zwischen Hornhaut und Linseneingang, sowie
Linsenausgang und Retina wurde die Schallgeschwindigkeit für
Glaskörper verwendet ($c\ua{Glaskörper} = \SI{1410}{\meter\per\second}$
\cite{anleitung01}).
Für dem Bereich zwischen Linseneingang und Linsenausgang
wurde eine Schallgeschwindigkeit von $c\ua{Linse} = \SI{2500}{\meter\per\second}$
angenommen.

Die Bestandteile des Auges haben die folgenden Tiefen.

\begin{description}
  \item[Hornhaut] $\SI{0}{\meter\per\second}$
  \item[Iris]
\end{description}
