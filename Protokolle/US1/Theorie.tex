\section{Theorie}

Die Ultraschalltechnik ist heutzutage ein wichtiges Element im Alltag, das
vor allem in der Medizin einer grundlegende Methode sowohl in der Therapie als
auch in der Diagnose ist. In dem folgenden Versuch sollen deshalb grundlegende
Eigenschaften und Definitionen der Ultraschallechographie kennengelernt sowie
angewandt werden.

\subsection{Theoretische Grundlagen}

Der Ultraschall besitzt eine für den Menschen nicht mehr wahrnehmbare Frequenz
zwischen $\SI{20}{kHz}$ und $\SI{1}{GHz}$. Im allgemeinen ist Schall eine
longitudinale Welle, die mittels Druckschwankungen weitergeleitet wird und durch
folgende Formel beschrieben werden kann:

\begin{equation}
  p(x,t) = p\ua{0} + v\ua{0}Z\cos{\omega t - kx} .
  \label{eqn:Druckwelle}
\end{equation}

Bei dem Faktor Z handelt es sich um die akustische Impedanz, welche mithilfe der
Dichte und der Schallgeschwindigkeit des vorliegenden Stoffes beschrieben wird:

\begin{equation}
  Z = c \cdot \rho .
  \label{eqn:Impedanz}
\end{equation}

Schalwellen zeigen im Allgemeinen das selbe Verhalten wie elektromagnetische Wellen,
jedoch ist bei Schwalwellen aufgrund der Änderung von Druck und Dichte die
Phasengeschwindigkeit materialabhängig.

Bei Schwalwellen muss aufgrund von unterschiedlichen Eigenschaften noch zwischen
gasförmigen oder flüssigen Medien und Festkörpern unterschieden. In Gasen und
Flüssigkeiten treten lediglich Longitudinalwellen auf, sodass sich für die
Schallgeschwindigkeit folgende Relation ergibt:

\begin{equation}
  c\ua{Fl} = \sqrt{ \frac{1}{\kappa \cdot \rho} }.
  \label{eqn:C_Fl}
\end{equation}

$\kappa$ beschreibt dabei die Kompressibilität des Mediums.

Im Gegnsatz zu Gasen und Flüssigkeiten treten bei Festkörpen zusätzlich zu den
Longitudinalwellen auch noch Transversalwellen auf, sodass die Schallgeschwindigkeit
in Festkörpern allgemein richtungsabhängig ist. Für die Schallgeschwindigkeit ergibt
sich mit dem Elastizitätmodul $E$ damit folgender Zusammenhang:

\begin{equation}
  c\ua{Fe} = \sqrt{ \frac{E}{\rho}}.
  \label{eqn:C_Fk}
\end{equation}

Allerdings muss beachtet werden, dass $c\ua{Fe}$ für Transversal- und
Logitudinalwellen unterschiedlich aussieht.

Bei der Betrachtung von Schallwellen in einem Median muss beachtet werden, dass
im Allgemeinen immer ein Teil der Energie durch Absorption verloren geht, weshalb
sich die Intensität der Welle als eine Funktion des Ortes und des
absorptionskoeffizienten $\alpha$ beschreiben lässt:

\begin{equation}
  I(\su{x}) = I\ua{0} \cdot e^{-\alpha \su{x}} .
  \label{eqn:Intensiät}
\end{equation}

Da Luft einen sehr großen Absorptionskoeffizienten besitzt, wird bei dem folgenden
Versuch immer ein Kontaktmittel verwendet. Bei dem Kontaktmittel handelt es sich
um bidestilliertes Wasser oder Hydrogel.

Während des Experimentes werden zwei verschiedenen Messmethoden verwendet, wovon
eine auf der reflektierten Welle und ein auf der transmittierten Welle basiert.
Der Reflexionskoeffizient R beschreibt ein Verhältnis der einfallenden und
reflektierten Intensität und ist von den akustischen Impedanzen der beiden
Medien ab, die die Grenzfälche bilden.

\begin{equation}
  R = \left( \frac{Z\ua{1} - Z\ua{2}}{Z\ua{1} + Z\ua{2}}).
  \label{eqn:Reflexionskoeffizient}
\end{equation}

Der Transmittierende Anteil kann dann entsprechend aus $T = 1-R$ berechnet werden.

Für die Erzeugung von Ultraschall wird in diesem Versuch der piezo-elektrische
Effekt genutzt. Piezo-elektrische Kristalle können durch entsprechende Anregung
durch ein äußeres elektrisches Feld in Schwingung versetzt werden. Die Amplitude
der entstehenden Wellen und die daraus relutierende Energiedichte können dabei
maximiert werden, wenn zwischen Anregerfrequenz und Eigenfrequenz des Kristalls
Resonanz entsteht.
Des Weiteren können Piezokristalle auch durch Schallwellen angeregt werden,
weshalb sie auch als Empfänger genutzt werden. Bei Quarz handelt es sich um den
am meisten verwendeten Piezokristall, da dieser gleichbleibende physikalische
Eigenschaften besitzt. Jedoch ist der Piezo-Effekt dafür relativ gering.

Wie bereits vorher erwähnt kann mithilfe von Ultraschall Information über
den Aufbau eines Stoffes gewonnen werden. Dies geschieht meistens mit einer
Laufzeitmessung, welche hierbei mithilfe zwei verschiedener Methoden durchgeführt
wird. Im Allgemeinen basiert es jedoch darauf, dass ein Ultraschallsignal
losgesendet wird und die Laufzeit durch eine definierte Messstrecke mittels eines
Empfängers gemessen wird.

Bei der ersten Methode handelt es sich um das Durchschallungs-Verfahren. Hier
wird von einem Ultraschallsender ein Signal durch einen Stoff geschickt und auf
der anderen Seite von einem Empfänger aufgefangen. Werden nun die Amplituden der
losgesendeten und der empfangenen Welle verglichen, kann eine Aussage darüber
getroffen werden, ob sich in der Probe eine Fehlstelle befindet. Über den Ort der
Fehlstelle lässt sich jedoch keine Aussage tätigen.

Die zweite Methode ist das Impuls-Echo-Verfahren. Hier wird lediglich ein Schallkopf
verwendet, welcher sowohl Schalwellen lossendet als auch die reflektierten Signale
wahrnimmt. Mithilfe der Amplitude des Echos kann hier die Größe des Echos bestimmt
werden. Mithilfe der Schallgeschwindigkeit und der Laufzeit kann hier nun auch
die Lage der Fehlstelle bestimmt werden:

\begin{equation}
  s = \frac{1}{2} c t.
  \label{eqn:Fehlstelle}
\end{equation}

Für beide Verfahren gibt es verschiedenen Darstellungsmöglichkeiten, den A-Scan,
den B-Scan und den TM-Scan. Bei dem A-Scan (Amplitude-Scan) wird auf einem
Bildschirm die Amplitude der registrierten Welle dargestellt. Bei dem B-Scan
(Brightness-Scan) wird hingegen die augenommene Amplitude in eine Helligkeitsstufe
umgewandelt, sodass durch viele punktuelle Messungen ein 2-dimensionales Bild
eines Mediums aufgenommen werden kann. Beim TM-Scan (Time-Scan) wird eine hohe
Impulswiederholungsfrequenz verwendet, sodass zum Beispiel Bewegungen des Gewebes
durch unterschiedliche Impulsechos registriert werden und zeitlich dargestellt
werden können. Diese Scan-Methode wird häufig in der Medizin kombiniert mit dem
B-Scan angewandt.

\subsection{Versuchsaufbau}

Die grundlegenden Bestandteile des Aufbaus sind ein Echoskop, zwei Ultraschallsonden,
ein PC zur Auswertung der Daten sowie verschiedene Acrylplatten und -Zylinder.
Das Echoskop kann ausschließlich im Impuls-Betrieb genutzt werden. Zwischen den
beiden Scan-Einstellungen (siehe Abschnitt 1.1) kann mittels einem Kipp-Schalter
gewechselt werden, indem der Schalter auf \textbf{REFLEC.} oder \textbf{TRANS.}
umgelegt wird. Für den Versuch werden zwei Ultraschallsonden mit 2 MHz verwendet,
deren Empfangsleistung zwischen 0 und 35 dB liegen.

Bei der verwendeten Messsoftware handelt es sich um das Programm Echo-View, mit
dem sowohl A-Scan, FFt Spektrum, Cepstrum und das Verstärkung TGC betrachtet
werden kann. Mithilfe der beiden Cursor können Differenzen zwischen zwei Positionen
auf der X-Achse bestimmt werden. Zudem geben diese beiden Cursor das Intervall
an, welches für das Frequenzspektrum und das Cepstrum verwendet werden.
Bei dem A-Scan kann das Signal sowohl als Funktion der Zeit (in $\si{\micro\second}$)
als auch als Funktion der Eindringtiefe (in mm) betrachtet werden. Mithilfe der
Freeze- und Start-Taste kann die durchgehende Aktualisierung des A-Scan-Bildes
zudem angehalten und wieder gestartet werden.

Mit der Time Gain Control kann die Verstärkung über die Parameter Treshhold, Wide,
Slope und Start eingestellt weren. Die erstellte Grafik kann mit Export abgespeichert
werden.
