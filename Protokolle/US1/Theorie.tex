\section{Theorie}

Die Ultraschalltechnik ist heutzutage ein wichtiges Element im Alltag, das
vor allem in der Medizin einer grundlegende Methode sowohl in der Therapie als
auch in der Diagnose ist. In dem folgenden Versuch sollen deshalb grundlegende
Eigenschaften und Definitionen der Ultraschallechographie kennengelernt sowie
angewandt werden.

\subsection{Theoretische Grundlagen}

Der Ultraschall besitzt eine für den Menschen nicht mehr wahrnehmbare Frequenz
zwischen $\SI{20}{kHz}$ und $\SI{1}{GHz}$. Im allgemeinen ist Schall eine
longitudinale Welle, die mittels Druckschwankungen weitergeleitet wird und durch
folgende Formel beschrieben werden kann:

\begin{equation}
  p(x,t) = p\ua{0} + v\ua{0}Z\cos{\omega t - kx} .
  \label{eqn:Druckwelle}
\end{equation}

Bei dem Faktor Z handelt es sich um die akustische Impedanz, welche mithilfe der
Dichte und der Schallgeschwindigkeit des vorliegenden Stoffes beschrieben wird:

\begin{equation}
  Z = c \cdot \rho .
  \label{eqn:Impedanz}
\end{equation}

Schalwellen zeigen im Allgemeinen das selbe Verhalten wie elektromagnetische Wellen,
jedoch ist bei Schwalwellen aufgrund der Änderung von Druck und Dichte die
Phasengeschwindigkeit materialabhängig.

Bei Schwalwellen muss aufgrund von unterschiedlichen Eigenschaften noch zwischen
gasförmigen oder flüssigen Medien und Festkörpern unterschieden. In Gasen und
Flüssigkeiten treten lediglich Longitudinalwellen auf, sodass sich für die
Schallgeschwindigkeit folgende Relation ergibt:

\begin{equation}
  c\ua{Fl} = \sqrt{ \frac{1}{\kappa \cdot \rho} }.
  \label{eqn:C_Fl}
\end{equation}

$\kappa$ beschreibt dabei die Kompressibilität des Mediums.

Im Gegnsatz zu Gasen und Flüssigkeiten treten bei Festkörpen zusätzlich zu den
Longitudinalwellen auch noch Transversalwellen auf, sodass die Schallgeschwindigkeit
in Festkörpern allgemein richtungsabhängig ist. Für die Schallgeschwindigkeit ergibt
sich mit dem Elastizitätmodul $E$ damit folgender Zusammenhang:

\begin{equation}
  c\ua{Fe} = \sqrt{ \frac{E}{\rho}}.
  \label{eqn:C_Fk}
\end{equation}

Allerdings muss beachtet werden, dass $c\ua{Fe}$ für Transversal- und
Logitudinalwellen unterschiedlich aussieht.

Bei der Betrachtung von Schallwellen in einem Median muss beachtet werden, dass
im Allgemeinen immer ein Teil der Energie durch Absorption verloren geht, weshalb
sich die Intensität der Welle als eine Funktion des Ortes und des
absorptionskoeffizienten $\alpha$ beschreiben lässt:

\begin{equation}
  I(\su{x}) = I\ua{0} \cdot e^{-\alpha \su{x}} .
  \label{eqn:Intensiät}
\end{equation}

Da Luft einen sehr großen Absorptionskoeffizienten besitzt, wird bei dem folgenden
Versuch immer ein Kontaktmittel verwendet. Bei dem Kontaktmittel handelt es sich
um bidestilliertes Wasser oder Hydrogel.

Während des Experimentes werden zwei verschiedenen Messmethoden verwendet, wovon
eine auf der reflektierten Welle und ein auf der transmittierten Welle basiert.
Der Reflexionskoeffizient R beschreibt ein Verhältnis der einfallenden und
reflektierten Intensität und ist von den akustischen Impedanzen der beiden
Medien ab, die die Grenzfälche bilden.

\begin{equation}
  R = \left( \frac{Z\ua{1} - Z\ua{2}}{Z\ua{1} + Z\ua{2}}).
  \label{eqn:Reflexionskoeffizient}
\end{equation}

Der Transmittierende Anteil kann dann entsprechend aus $T = 1-R$ berechnet werden.

Für die Erzeugung von Ultraschall wird in diesem Versuch der piezo-elektrische
Effekt genutzt. Piezo-elektrische Kristalle können durch entsprechende Anregung
durch ein äußeres elektrisches Feld in Schwingung versetzt werden. Die Amplitude
der entstehenden Wellen und die daraus relutierende Energiedichte können dabei
maximiert werden, wenn zwischen Anregerfrequenz und Eigenfrequenz des Kristalls
Resonanz entsteht.  
