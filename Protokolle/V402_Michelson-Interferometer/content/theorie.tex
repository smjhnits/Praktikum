\section{Zielsetzung}

Der Versuch V$\num{401}$ setzt sich in erster Linie mit dem Michelson--Interferrometer
auseinander. Die Wellenlänge der verwendeten Lichtquelle, wowie der Brechungsindex
verschiedener Gase sollen im Laufe des Versuches bestimmt werden.

\section{Theorie}

Das Michelson-Interferometer ist ein Messgerät, welches auf dem Interferenzprinzip
beruht. Das Ausbreitungsverhalten von Licht kann sehr gut durch das
Verhalten von elektromagnetischen Wellen beschrieben werden.
Da die Maxwellschen-Gleichungen durch lineare Differentialgleichungen gegeben sind,
können ihre Lösungen, welche auch elektromagnetische Wellen sind, superponiert werden.
Der $\vec{E}$-Feld Vektor kann jedoch nicht direkt bestimmt werden, sondern
lediglich seine Intensität, die gegeben ist durch

\begin{equation}
  
\end{equation}
