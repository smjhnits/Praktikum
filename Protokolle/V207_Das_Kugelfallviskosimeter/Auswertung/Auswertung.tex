% bei Standalone in documentclass noch:
% \RequirePackage{luatex85}

\documentclass[captions=tableheading, titlepage= firstiscover, parskip = half , bibliography=totoc]{scrartcl}
%paper = a5 für andere optinen
% titlepage= firstiscover
% bibliography=totoc für bibdateien
% parskip=half  Veränderung um Absätze zu verbessern

\usepackage{scrhack} % nach \documentclass
\usepackage[aux]{rerunfilecheck}
\usepackage{polyglossia}
\usepackage[style=numeric, backend=biber]{biblatex} % mit [style = alphabetic oder numeric] nach polyglossia
\addbibresource{lit.bib}
\setmainlanguage{german}

\usepackage[autostyle]{csquotes}
\usepackage{amsmath} % unverzichtbare Mathe-Befehle
\usepackage{amssymb} % viele Mathe-Symbole
\usepackage{mathtools} % Erweiterungen für amsmath
\usepackage{fontspec} % nach amssymb
% muss ins document: \usefonttheme{professionalfonts} % für Beamer Präsentationen
\usepackage{longtable}

\usepackage[
math-style=ISO,    % \
bold-style=ISO,    % |
sans-style=italic, % | ISO-Standard folgen
nabla=upright,     % |
partial=upright,   % /
]{unicode-math} % "Does exactly what it says on the tin."
\setmathfont{Latin Modern Math}
% \setmathfont{Tex Gyre Pagella Math} % alternativ

\usepackage[
% die folgenden 3 nur einschalten bei documenten
locale=DE,
separate-uncertainty=true, % Immer Fehler mit ±
per-mode=symbol-or-fraction, % m/s im Text, sonst \frac
]{siunitx}

% alternativ:
% per-mode=reciprocal, % m s^{-1}
% output-decimal-marker=., % . statt , für Dezimalzahlen

\usepackage[
version=4,
math-greek=default,
text-greek=default,
]{mhchem}

\usepackage[section, below]{placeins}
\usepackage{caption} % Captions schöner machen
\usepackage{graphicx}
\usepackage{grffile}
\usepackage{subcaption}

% \usepackage{showframe} Wenn man die Ramen sehen will

\usepackage{float}
\floatplacement{figure}{htbp}
\floatplacement{table}{htbp}

\usepackage{mhchem} %chemische Symbole Beispiel: \ce{^{227}_{90}Th+}


\usepackage{booktabs}

 \usepackage{microtype}
 \usepackage{xfrac}

 \usepackage{expl3}
 \usepackage{xparse}

 % \ExplSyntaxOn
 % \NewDocumentComman \I {}  %Befehl\I definieren, keine Argumente
 % {
 %    \symup{i}              %Ergebnis von \I
 % }
 % \ExplSyntaxOff

 \usepackage{pdflscape}
 \usepackage{mleftright}

 % Mit dem mathtools-Befehl \DeclarePairedDelimiter können Befehle erzeugen werden,
 % die Symbole um Ausdrücke setzen.
 % \DeclarePairedDelimiter{\abs}{\lvert}{\rvert}
 % \DeclarePairedDelimiter{\norm}{\lVert}{\rVert}
 % in Mathe:
 %\abs{x} \abs*{\frac{1}{x}}
 %\norm{\symbf{y}}

 % Für Physik IV und Quantenmechanik
 \DeclarePairedDelimiter{\bra}{\langle}{\rvert}
 \DeclarePairedDelimiter{\ket}{\lvert}{\rangle}
 % <name> <#arguments> <left> <right> <body>
 \DeclarePairedDelimiterX{\braket}[2]{\langle}{\rangle}{
 #1 \delimsize| #2
 }

\setlength{\delimitershortfall}{-1sp}

 \usepackage{tikz}
 \usepackage{tikz-feynman}

 \usepackage{csvsimple}
 % Tabellen mit \csvautobooktabular{"file"}
 % muss in table umgebung gesetzt werden


% \multicolumn{#Spalten}{Ausrichtung}{Inhalt}

\usepackage{hyperref}
\usepackage{bookmark}
\usepackage[shortcuts]{extdash} %nach hyperref, bookmark

\newcommand{\ua}[1]{_\symup{#1}}
\newcommand{\su}[1]{\symup{#1}}


\begin{document}

\section{Messwerte}

\begin{table}
  \centering
  \caption{gemessene Werte}
  \label{tab:Messdaten}
  \begin{tabular}{c c c c c }
    \toprule $Fallzeit \,\, Kugel \, 2 \,\, [s]$ & $Fallzeit\,\, Kugel\, 1 \,\, [s]$ & $Temperatur \,\, [°C]$ & $Messung \, 1 \,\, [s]$ & $Messung \, 2 \,\, [s]$ \\
    \midrule
    11.80 & 68.47 & 31.0 & 68.86 & 68.86 \\
    11.80 & 68.95 & 36.0 & 68.33 & 68.21 \\
    12.15 & 68.69 & 39.0 & 65.75 & 65.76 \\
    11.73 & 68.53 & 45.0 & 60.32 & 60.52 \\
    12.21 & 68.50 & 49.5 & 59.27 & 59.27 \\
    11.47 & 67.69 & 51.5 & 58.52 & 58.66 \\
    12.10 & 68.83 & 56.0 & 58.30 & 58.29 \\
    11.96 & 68.41 & 60.0 & 56.60 & 56.72 \\
    11.86 & 68.38 & 64.0 & 55.10 & 55.15 \\
    11.87 & 68.60 & 68.0 & 54.35 & 54.40 \\
    \bottomrule
  \end{tabular}
\end{table}

\begin{table}
  \centering
  \label{tab:FallzeitenGemittelt}
  \caption{Mittelwerte der Fallzeiten für Teil 1 des Versuches}
  \begin{tabular}{c c c c}
    \toprule $Fallzeit \,\, Kugel \,\, 1$ & $\increment_{FK1}$ & $Fallzeit \,\, Kugel \,\, 2$ & $\increment_{FK2}$ \\
    \midrule
    68.50 & 0.30 & 11.89 & 0.13 \\
    \bottomrule
  \end{tabular}
\end{table}

\begin{table}
  \label{tab:TemperaturGemittelt}
  \centering
  \caption{Mittelwerte der Messung bei verschiedenen Temperaturen für die große Kugel}
  \begin{tabular}{c | c c c c c c c c c c }
    \toprule
    $Temperatur \,\, [K]$          & 304.15 & 309.15 & 312.15 & 318.15 & 322.65 \\
    $Fallzeit \,\, [s]$            & 68.86 & 68.27 & 65.76 & 60.42 & 59.27 \\
    $\increment_{FZ} \,\, [s]$     & 0 & 0.035 & 0.002 & 0.058 & 0 \\
    \midrule
    $Temperatur \,\, [K]$          & 324.65 & 329.15 & 333.15 & 337.15 & 341.15 \\
    $Fallzeit \,\, [s]$            & 58.59 & 58.29 & 56.66 & 55.13 & 54.38 \\
    $\increment_{FZ} \,\, [s]$     & 0.040 & 0.003 & 0.035 & 0.014 & 0.014 \\
    \bottomrule
  \end{tabular}
\end{table}

\newpage

\section{Auswertung}

\subsection{Bestimmung der Apparatekonstante für die große Kugel}

In dem ersten Teil des Versuches soll die Apparatekonstante für die große Kugel (Kugel 1) bestimmt
werden. Dafür wird mithilfe der bekannten Apparatekonstante für die kleine Kugel (Kugel 2) die Viskosität
des Wassers bei Raumtemperatur bstimmt und in folgende Formel eingesetzt:

\begin{align}
  \label{eqn:KappaGroß}
  \Kappa_{kl} &= 0.007640 \, \symup{[mPa\, cm^3 / g]} \\
  \eta        &= \Kappa_{gr} \cdot \left( \rho_K - \rho_{Fl} \right) \cdot t
\end{align}

Bei $\rho_K$ und $\rho_{Fl}$ handelt es sich um die Dichten der Kugel und der betrachteten
Flüssigkeit. Mithilfe der gemessenen Radien und Gewichte der Kugeln kann die Dichte
bestimmt werden:

\begin{align}
  r_{gr}    &= (0.0078017 \pm 0.0000017) \, \symup{[m]}     & r_{kl}    &= (0.0077167 \pm 0.0000017) \, \symup{[m]} \\
  m_{gr}    &= 0.00496 \, \symup{[kg]}                        & m_{kl}    &= 0.00446 \, \symup{[kg]} \\
  \rho_{gr} &= (2493.6 \pm 1.6) \, \symup{\left[ kg/m^3\right]} & \rho_{gr} &= (2312.0 \pm 1.5) \, \symup{\left[ kg/m^3\right]}
\end{align}

Somit ergeben sich für die Viskosität des Wassers bei Raumtemperatur $\eta_{20}$ und $K_{gr}$ folgende Werte:

\begin{align}
  \eta_{20}   &= (0.001194 \pm 0.00013) \, \symup{[Pa \, s]} \\
  \Kappa_{gr} &= (0.001165 \pm, 0.00012) \, \symup{[mPa\, cm^3 / g]}
\end{align}


\end{document}
