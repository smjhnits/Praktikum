\input{"../../Praeambel_prak.tex"}

\begin{document}

\section{Messwerte}

\begin{table}
  \centering
  \caption{gemessene Werte}
  \label{tab:Messdaten}
  \begin{tabular}{c c c c c }
    \toprule $Fallzeit \,\, Kugel \, 2 \,\, [\symup{s}]$ & $Fallzeit\,\, Kugel\, 1 \,\, [\symup{s}]$ & $Temperatur \,\, [\symup{°C}]$ & $Messung \, 1 \,\, [\symup{s}]$ & $Messung \, 2 \,\, [\symup{s}]$ \\
    \midrule
    11.80 & 68.47 & 31.0 & 68.86 & 68.86 \\
    11.80 & 68.95 & 36.0 & 68.33 & 68.21 \\
    12.15 & 68.69 & 39.0 & 65.75 & 65.76 \\
    11.73 & 68.53 & 45.0 & 60.32 & 60.52 \\
    12.21 & 68.50 & 49.5 & 59.27 & 59.27 \\
    11.47 & 67.69 & 51.5 & 58.52 & 58.66 \\
    12.10 & 68.83 & 56.0 & 58.30 & 58.29 \\
    11.96 & 68.41 & 60.0 & 56.60 & 56.72 \\
    11.86 & 68.38 & 64.0 & 55.10 & 55.15 \\
    11.87 & 68.60 & 68.0 & 54.35 & 54.40 \\
    \bottomrule
  \end{tabular}
\end{table}

\begin{table}
  \centering
  \caption{Mittelwerte der Fallzeiten \texorpdfstring{$[\symup{s}]$}{b} für Teil 1 des Versuches}
  \begin{tabular}{c c c c}
    \toprule $Fallzeit \,\, Kugel \,\, 1$ & $\increment_{FK1}$ & $Fallzeit \,\, Kugel \,\, 2$ & $\increment_{FK2}$ \\
    \midrule
    68.50 & 0.30 & 11.89 & 0.13 \\
    \bottomrule
  \end{tabular}
  \label{tab:FallzeitenGemittelt}
\end{table}

\begin{table}
  \centering
  \caption{Mittelwerte der Messung bei verschiedenen Temperaturen für die große Kugel}
  \label{tab:TemperaturGemittelt}
  \begin{tabular}{c | c c c c c c c c c c }
    \toprule
    $Temperatur \,\, [\symup{K}]$          & 304.15 & 309.15 & 312.15 & 318.15 & 322.65 \\
    $Fallzeit \,\, [\symup{s}]$            & 68.86 & 68.27 & 65.76 & 60.42 & 59.27 \\
    $\increment_{FZ} \,\, [\symup{s}]$     & 0 & 0.035 & 0.002 & 0.058 & 0 \\
    \midrule
    $Temperatur \,\, [\symup{K}]$          & 324.65 & 329.15 & 333.15 & 337.15 & 341.15 \\
    $Fallzeit \,\, [\symup{s}]$            & 58.59 & 58.29 & 56.66 & 55.13 & 54.38 \\
    $\increment_{FZ} \,\, [\symup{s}]$     & 0.040 & 0.003 & 0.035 & 0.014 & 0.014 \\
    \bottomrule
  \end{tabular}
\end{table}

\newpage

\section{Auswertung}

\subsection{Bestimmung der Apparatekonstante für die große Kugel}

In dem ersten Teil des Versuches soll die Apparatekonstante für die große Kugel (Kugel 1) bestimmt
werden. Dafür wird mithilfe der bekannten Apparatekonstante für die kleine Kugel (Kugel 2) die Viskosität
des Wassers bei Raumtemperatur bstimmt und in folgende Formel eingesetzt:

\begin{align}
  \label{eqn:KappaGroß}
  \Kappa_{kl} &= 0.007640 \, \symup{[mPa\, cm^3 / g]} \\
  \eta        &= \Kappa_{gr} \cdot \left( \rho_K - \rho_{Fl} \right) \cdot t
\end{align}

Bei $\rho_K$ und $\rho_{Fl}$ handelt es sich um die Dichten der Kugel und der betrachteten
Flüssigkeit. Mithilfe der gemessenen Radien und Gewichte der Kugeln kann die Dichte
bestimmt werden:

\begin{align}
  r_{gr}    &= (0.0078017 \pm 0.0000017) \, \symup{[m]}     & r_{kl}    &= (0.0077167 \pm 0.0000017) \, \symup{[m]} \\
  m_{gr}    &= 0.00496 \, \symup{[kg]}                        & m_{kl}    &= 0.00446 \, \symup{[kg]} \\
  \rho_{gr} &= (2493.6 \pm 1.6) \, \symup{\left[ kg/m^3\right]} & \rho_{gr} &= (2312.0 \pm 1.5) \, \symup{\left[ kg/m^3\right]}
\end{align}

In Tabelle \ref{tab:FallzeitenGemittelt} sind die Mittelwerte und Fehler der gemessenen Fallzeiten für die
kleine und Große Kugel bei Raumtemperatur eingetragen.
Somit ergeben sich für die Viskosität des Wassers bei Raumtemperatur $\eta_{20}$ und $K_{gr}$ folgende Werte:

\begin{align}
  \eta_{20}   &= (0.001194 \pm 0.00013) \, \symup{[Pa \, s]} \\
  \Kappa_{gr} &= \frac{\eta_{20}}{\left( \rho_{gr} - \rho_{w} \right) \cdot t_{gr}} \\
              &= (0.001165 \pm, 0.00012) \, \symup{[mPa\, cm^3 / g]}
\end{align}

Die Fehler für $\eta_{20}$ und $\Kappa_{gr}$ ergeben sich mit der Gaußschen Fehlerfortpflanzung:

\begin{align}
\increment \eta    &= \sqrt{ \left( \partial_{\rho_{gr}} \eta \cdot \increment \rho_{gr} \right)^2 + \left( \partial_{t_{gr}} \eta \cdot \increment t_{gr} \right)^2  }\\
\increment K_{gr} &= \sqrt{ \left( \partial_{\eta} K_{gr} \cdot \increment \eta \right)^2 + \left( \partial_{\rho_{gr}} K_{gr} \cdot \increment \rho_{gr} \right)^2 +
                      \left( \partial_{t_{gr}} K_{gr} \cdot \increment t_{gr} \right)^2  }
\end{align}

\begin{figure}
  \label{plt:Viskos}
  \includegraphics[height = 10 cm]{Plot_T.pdf}
  \includegraphics[height = 10 cm]{Plot_T_1.pdf}
\end{figure}

\subsection{Bestimmung der Konstanten A/B für die zeitabhängige Viskosität \texorpdfstring{$\eta(T)$}{z}}

In den beiden Graphen \ref{plt:Viskos} sieht man einmal die Viskosität gegen T und einmal gegen 1/T aufgetragen.
Die y - Skala ist dabei logarythmisch angepasst. Die Plots wurden mithilfe von Python erstellt und ergeben
für A und B folgende Parameter:

\begin{align}
  A &= 1.2651 \cdot 10^{-4} \\
  B &= 6.8567 \cdot 10^{2}
\end{align}




\end{document}
