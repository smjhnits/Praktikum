\input{"../../Praeambel_prak.tex"}

\title{Versuch 207}
\subtitle{Das Kugelfall-Viskosimeter nach Höppler}
\author{Sebastian Pape\\
        sepa@gmx.de \and
        Jonah Nitschke\\
        lejonah@web.de}
\date{Durchführung: 22.11.2016\\
      Abgabe: 29.11.2016}

\begin{document}

\maketitle
\tableofcontents
\newpage

\section{Einführung}

Der Versuch zielt darauf ab, die Temperaturabhängigkeit der dynamischen Viskosität von destilliertem
Wasser, mit Hilfe der Kugelfallmethode zu bestimmen. Zunächst werden die theoretischen Grundlagen
erklärt, danach wird die Durchführung und der Aufbau erläutert. Daraufhin folgt die Auswertung der Versuchsergerbnisse
mit abschließender Diskussion dieser.

\section{Theorie}

Sobald ein Körper durch ein Medium bewegt wird, entsteht innere Reibung. Der Körper tritt in Wechselwirkung mit
dem umgebenem Medium, wobei diese Wechselwirkung von der Geschwindigkeit des Körpers abhängt.
 
