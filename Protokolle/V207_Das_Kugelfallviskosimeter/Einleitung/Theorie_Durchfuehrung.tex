\input{"../../Praeambel_prak.tex"}

\title{Versuch 207}
\subtitle{Das Kugelfall-Viskosimeter nach Höppler}
\author{Sebastian Pape\\
        sepa@gmx.de \and
        Jonah Nitschke\\
        lejonah@web.de}
\date{Durchführung: 22.11.2016\\
      Abgabe: 29.11.2016}

\begin{document}

\maketitle
\tableofcontents
\newpage

\section{Einführung}

Der Versuch zielt darauf ab, die Temperaturabhängigkeit der dynamischen Viskosität von destilliertem
Wasser, mit Hilfe der Kugelfallmethode zu bestimmen. Zunächst werden die theoretischen Grundlagen
erklärt, danach wird die Durchführung und der Aufbau erläutert. Daraufhin folgt die Auswertung der Versuchsergerbnisse
mit abschließender Diskussion dieser.

\section{Theorie}

\subsection{Innere Reibung}

Sobald ein Körper durch ein Medium bewegt wird, entsteht innere Reibung. Die Stärke dieser Reibung hängt von der Beschaffenheit
des Umgebungsmediums ab und wird als dynamische Viskosität oder auch Zähigkeit bezeichnet.
In dem Versuch wurde eine Kugel, mit Radius $r$ durch destilliertes Wasser bewegt. Die Reibungskraft, die der Bewegung
entgegengerichtet ist, lässt sich über das \emph{Stokessche Gesetzt} bestimmen.
\begin{equation}
  \label{eqn:Stokes}
  F_R = 6\pi r \eta v
\end{equation}
Dabei ist $\eta$ die dynamische Viskosität und $v$ die Geschwindigkeit der Kugel. Die \emph{Stokes Gleichung} gilt nur für laminare
Strömungen, was bedeutet, dass die Ausdehnung der Flüssigkeit hinreichend groß sein muss, wodurch Wirbelbildungen verhindert werden.
Die Kugel wird in dem Versuch durch ein Rohr, mit leicht größerem Durchmesser als dem der Kugel fallen gelassen.
Die erreichten Fallgeschwindigkeit der Kugel, durch das Medium sind darüberhinaus hinreichend gering, sodass
alle Voraussetzungen für die Gültigkeit von Gleichung \eqref{enq:Stokes} gegeben sind.
Wenn die Kugel in einer viskosen Flüssigkeit fällt, wirken zum einem ihre Gewichtskrat $F_g$, die Reibungskraft der Flüssigkeit $F_R$
und der Auftrieb $F_A$. Die Auftriebskraft, ist wie die Reibungskraft der Bewegungsrichtung entgegengerichtet.
Die Reibungskraft nimmt mit zunehmender Geschwindigkeit der Kugel zu. Dies bedeutet, dass sich ein Kräftegleichgewicht zwischen
$F_g$ und $F_R$ einstellt, wodurch sich eine konstante Fallgeschwindigkeit ergibt.
Die Zähigkeit der Flüssigkeit lässt sich über das empirische Gesetz \eqref{eqn: dyn. Voskosität} bestimmen.
\begin{equation}
  \label{eqn: dyn. Viskosität}
  \eta = K(\rho_K - \rho_{Fl}) \cdot t
\end{equation}
Wobei $K$ eine Apparaturkonstante, $\rho_K$ die Dichte der Kugel, $\rho_{Fl} die Dichte der Umgebungsflüssigkeit$ und $t$ die Fallzeit ist.
Die Apparaturkonstante enthält sowohl Informationen über die Fallhöhe, als auch über die Kugelgeometrie.

\subsection{Temperaturabhängigkeit der Viskosität}

Die Temperaturabhängigkeit der dynamischen Viskosität lässt sich mit Hilfe der \emph{Andradeschen Gleichung} bestimmen, die da lautet
\begin{equation}
  \eta(T) = A \exp^{\frac{B}{T}}.
\end{equation}
$A$ und $B$ sind dabei Konstanten, die über eine Ausgleichsrechung bestimmt werden können.

\subsection{\emph{Reynoldsche Zahl}}

Die \emph{Stokes Gleichung} ist nur für laminare Strömungen gültig. Das trömungsverhlten einer

\section{Durchführung}

\subsection{Aufbau}

Zur Bestimmung der dynamischen Viskosität von destillierten Wasser wurde das Kugelfall-Viskosimeter nach Höppler verwendet.
Das Rohr, durch welches die Kugel fällt ist leicht angewinkelt, sodass die Kugel nicht unkontrolliert hindurchfällt, sondern an der Rohrinnenseite
entlang gleitet. Es ist daruach zu achten, dass keine Luftblasen an der Rohrinnenseite und der Kugel haften, da durch diese die
Fallzeit unvorhersehbar beeinflusst wird.

\end{document}
