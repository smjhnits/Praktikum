\usepackage{scrhack} % nach \documentclass
\usepackage[aux]{rerunfilecheck}
\usepackage{polyglossia}
\setmainlanguage{german}
\usepackage{fontspec}
\usepackage[unicode]{hyperref}
\usepackage{bookmark}
\usepackage[shortcuts]{extdash} %nach hyperref, bookmark
\usepackage[autostyle]{csquotes}
\setotherlanguages{english, french}
\usepackage{amsmath} % unverzichtbare Mathe-Befehle
\usepackage{amssymb} % viele Mathe-Symbole
\usepackage{mathtools} % Erweiterungen für amsmath
\usepackage{fontspec} % nach amssymb
\usepackage{booktabs}
\usepackage{longtable}
\usepackage{microtype}
\usepackage{xfrac}
\usepackage[
locale=DE,
separate-uncertainty=true, % Immer Fehler mit ±
per-mode=symbol-or-fraction,% m/s im Text , sonst \frac
% alternativ:
% per-mode=reciprocal, % m s^{-1}
% output-decimal-marker=.,  % . statt, für Dezimalzahlen
]{siunitx}
% Floats innerhalb einer Section halten
\usepackage[section, below]{placeins}
\usepackage{caption} % Captions schöner machen
\usepackage{subcaption}
\usepackage{graphicx}
\usepackage{grffile}
\usepackage{float}
\floatplacement{figure}{htbp}
\floatplacement{table}{htbp}
\usepackage[backend=biber]{biblatex}
\addbibresource{lit.bib}
\usepackage[
math-style=ISO, % \
bold-style=ISO, % |
sans-style=italic, % | ISO-Standard folgen
nabla=upright, % |
partial=upright, % /
]{unicode-math} % "Does exactly what it says on the tin."
\setmathfont{Latin Modern Math}
% \setmathfont{Tex Gyre Pagella Math} % alternativ
\usepackage[
version=4,
math-greek=default,
text-greek=default,
]{mhchem}
