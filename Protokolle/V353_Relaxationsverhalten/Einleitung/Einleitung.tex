% bei Standalone in documentclass noch:
% \RequirePackage{luatex85}

\documentclass[captions=tableheading, titlepage= firstiscover, parskip = half , bibliography=totoc]{scrartcl}
%paper = a5 für andere optinen
% titlepage= firstiscover
% bibliography=totoc für bibdateien
% parskip=half  Veränderung um Absätze zu verbessern

\usepackage{scrhack} % nach \documentclass
\usepackage[aux]{rerunfilecheck}
\usepackage{polyglossia}
\usepackage[style=numeric, backend=biber]{biblatex} % mit [style = alphabetic oder numeric] nach polyglossia
\addbibresource{lit.bib}
\setmainlanguage{german}

\usepackage[autostyle]{csquotes}
\usepackage{amsmath} % unverzichtbare Mathe-Befehle
\usepackage{amssymb} % viele Mathe-Symbole
\usepackage{mathtools} % Erweiterungen für amsmath
\usepackage{fontspec} % nach amssymb
% muss ins document: \usefonttheme{professionalfonts} % für Beamer Präsentationen
\usepackage{longtable}

\usepackage[
math-style=ISO,    % \
bold-style=ISO,    % |
sans-style=italic, % | ISO-Standard folgen
nabla=upright,     % |
partial=upright,   % /
]{unicode-math} % "Does exactly what it says on the tin."
\setmathfont{Latin Modern Math}
% \setmathfont{Tex Gyre Pagella Math} % alternativ

\usepackage[
% die folgenden 3 nur einschalten bei documenten
locale=DE,
separate-uncertainty=true, % Immer Fehler mit ±
per-mode=symbol-or-fraction, % m/s im Text, sonst \frac
]{siunitx}

% alternativ:
% per-mode=reciprocal, % m s^{-1}
% output-decimal-marker=., % . statt , für Dezimalzahlen

\usepackage[
version=4,
math-greek=default,
text-greek=default,
]{mhchem}

\usepackage[section, below]{placeins}
\usepackage{caption} % Captions schöner machen
\usepackage{graphicx}
\usepackage{grffile}
\usepackage{subcaption}

% \usepackage{showframe} Wenn man die Ramen sehen will

\usepackage{float}
\floatplacement{figure}{htbp}
\floatplacement{table}{htbp}

\usepackage{mhchem} %chemische Symbole Beispiel: \ce{^{227}_{90}Th+}


\usepackage{booktabs}

 \usepackage{microtype}
 \usepackage{xfrac}

 \usepackage{expl3}
 \usepackage{xparse}

 % \ExplSyntaxOn
 % \NewDocumentComman \I {}  %Befehl\I definieren, keine Argumente
 % {
 %    \symup{i}              %Ergebnis von \I
 % }
 % \ExplSyntaxOff

 \usepackage{pdflscape}
 \usepackage{mleftright}

 % Mit dem mathtools-Befehl \DeclarePairedDelimiter können Befehle erzeugen werden,
 % die Symbole um Ausdrücke setzen.
 % \DeclarePairedDelimiter{\abs}{\lvert}{\rvert}
 % \DeclarePairedDelimiter{\norm}{\lVert}{\rVert}
 % in Mathe:
 %\abs{x} \abs*{\frac{1}{x}}
 %\norm{\symbf{y}}

 % Für Physik IV und Quantenmechanik
 \DeclarePairedDelimiter{\bra}{\langle}{\rvert}
 \DeclarePairedDelimiter{\ket}{\lvert}{\rangle}
 % <name> <#arguments> <left> <right> <body>
 \DeclarePairedDelimiterX{\braket}[2]{\langle}{\rangle}{
 #1 \delimsize| #2
 }

\setlength{\delimitershortfall}{-1sp}

 \usepackage{tikz}
 \usepackage{tikz-feynman}

 \usepackage{csvsimple}
 % Tabellen mit \csvautobooktabular{"file"}
 % muss in table umgebung gesetzt werden


% \multicolumn{#Spalten}{Ausrichtung}{Inhalt}

\usepackage{hyperref}
\usepackage{bookmark}
\usepackage[shortcuts]{extdash} %nach hyperref, bookmark

\newcommand{\ua}[1]{_\symup{#1}}
\newcommand{\su}[1]{\symup{#1}}


\title{Versuch 353}
\subtitle{Das Relaxationsverhalten eines LC-Kreises}
\author{Sebastian Pape\\
        sepa@gmx.de \and
        Jonah Nitschke\\
        lejonah@web.de}
\date{Durchführung: 24.01.2017\\
      Abgabe: 31.01.2017}

\begin{document}
\maketitle
\setcounter{page}{1}

\section{Theorie}

In dem Versuch V353 wurde das Relaxationsverhalten eines $RC$-Kreises untersucht.
Relaxation bedeutet, dass ein System aus seinen Ruhezustand gebracht wird
und nicht-oszillatiorisch in diesen zurückkehrt.
Mit einem $RC$-Kreis ist solch ein Relaxationsverhalten zu erreichen.

\subsection{Ent- und Aufladevorgang}

Bei dem Ladevorgang eines Kondensators mit der Kapazität $C$ über einen Widerstand
$R$ lässt sich die Ladung auf dem Kondensator als Funktion der Zeit darstellen.
Es ergibt sich unter der Randbedingung $Q(\infty) = 0$ die folgende Formel.

\begin{equation}
  \label{eqn:Entladen}
  Q(t) = Q(0)\cdot \exp{\frac{-t}{RC}}
\end{equation}

Für den Aufladevorgang gelten die Randbedingungen $Q(0) = 0$ und $Q(\inty) = CU_0$. Dabei ist $U_0$ die Spannung der Spannungsquelle. Damit ergibt sich
die folgende Formel des Aufladevorgangs eines Kondensators über einen Widerstand.

\begin{equation}
  \label{eqn:aufladen}
  Q(t) = CU_0(1 - \exp{\frac{-t}{RC}})
\end{equation}

$RC$ wird auch spezifische Zeitkonstante genannt, da dieser Ausdruck ein Maß
der Geschwindigkeit des Entladevorgangs darstellt.
Nach $\tau = RC$ ändert sich die Ladung des Kondensators um den Faktor $0,368$. Nach $2,3\tau$ sind nur noch 10\% des Ausgangswertes derLadung auf dem
Kondensator vorhanden.

\subsection{Relaxationsverhalten bei periodischer Anregung}

Wird nun an den $RC$-Kreis eine periodische Spannung $U(t) = U_0\cos{\omega t}$
anstelle einer einzelnen Erregung angelegt ändert sich das Relaxationsverhalten
mit der Frequenz $\omega$. Ist $\omega \ll \frac{1}{RC}$ sind die
Spannungen des Generators und des Kondensators näherungsweise in Phase.
Je hochfrequenter die Spannung wird, desto deutlicher wird die Phase
zwischen den beiden Spannungen.
Für $\omega\rightarrow\infty$ nähert sich die Phasenverschiebung asymptotisch
den Wert $\frac{pi}{2}$.

\subsection{Der RC-Kreis als Integrator}

Mit Hilfe eines $RC$-Kreises lässt sich die angeschlossene Spannung $U(t)$
unter der Voraussetzun $\omega\ll\frac{1}{RC}$ integireren. Dies hängt damit zusammen, dass die Kondensatorspannung $U_C(t)$
proportional mit dem Integral von $U(t)$ zusammenhängt.
Unter der gestellten Bedingung gilt näherungsweise der Zusammenhang

\begin{equation}
  \label{eqn:Integrator}
  U_C(t) = \frac{1}{RC}\int_0^tU(t')\symup{d}t'.
\end{equation}

\section{Durchführung}

\end{document}
