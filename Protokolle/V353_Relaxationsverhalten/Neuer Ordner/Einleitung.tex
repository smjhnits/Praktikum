\input{"../../Praeambel_prak.tex"}

\title{Versuch 353}
\subtitle{Das Relaxationsverhalten eines LC-Kreises}
\author{Sebastian Pape\\
        sepa@gmx.de \and
        Jonah Nitschke\\
        lejonah@web.de}
\date{Durchführung: 24.01.2017\\
      Abgabe: 31.01.2017}

\begin{document}
\maketitle
\setcounter{page}{1}

\section{Theorie}

In dem Versuch V353 wurde das Relaxationsverhalten eines $RC$-Kreises untersucht.
Relaxation bedeutet, dass ein System aus seinen Ruhezustand gebracht wird
und nicht-oszillatiorisch in diesen zurückkehrt.
Mit einem $RC$-Kreis ist solch ein Relaxationsverhalten zu erreichen. Mit verschiedenen
Messmethoden soll die RC-Konstante bestimmt werden, sodass diese miteinander
verglichen werden können. Darüberhinaus soll die Eigenschaft des RC-Kreises als
Integrator nachgewiesen werden.

\subsection{Ent- und Aufladevorgang}

Bei dem Ladevorgang eines Kondensators mit der Kapazität $C$ über einen Widerstand
$R$ lässt sich die Ladung auf dem Kondensator als Funktion der Zeit darstellen.
Es ergibt sich unter der Randbedingung $Q(\infty) = 0$ die folgende Formel.

\begin{equation}
  \label{eqn:Entladen}
  Q(t) = Q(0)\cdot \exp{(-\frac{t}{RC})}
\end{equation}

Für den Aufladevorgang gelten die Randbedingungen $Q(0) = 0$ und $Q(\infty) = CU_0$. Dabei ist $U_0$ die Spannung der Spannungsquelle. Damit ergibt sich
die folgende Formel des Aufladevorgangs eines Kondensators über einen Widerstand.

\begin{equation}
  \label{eqn:aufladen}
  Q(t) = CU_0(1 - \exp{(-\frac{t}{RC})})
\end{equation}

$RC$ wird auch \emph{spezifische Zeitkonstante} genannt, da dieser Ausdruck ein Maß
der Geschwindigkeit des Entladevorgangs darstellt.
Nach $\tau = RC$ ändert sich die Ladung des Kondensators um den Faktor $0,368$. Nach $2,3\tau$ sind nur noch 10\% des Ausgangswertes der Ladung auf dem
Kondensator vorhanden.

\subsection{Relaxationsverhalten bei periodischer Anregung}

Wird nun an den $RC$-Kreis eine periodische Spannung $U(t) = U_0\cos{\omega t}$
anstelle einer einzelnen Erregung angelegt ändert sich das Relaxationsverhalten
mit der Frequenz $\omega$. Ist $\omega \ll \frac{1}{RC}$ sind die
Spannungen des Generators und des Kondensators näherungsweise in Phase.
Je hochfrequenter die Spannung wird, desto deutlicher wird die Phase
zwischen den beiden Spannungen.
Für $\omega\rightarrow\infty$ nähert sich die Phasenverschiebung asymptotisch
den Wert $\frac{\pi}{2}$.

\subsection{Der RC-Kreis als Integrator}

Mit Hilfe eines $RC$-Kreises lässt sich die angeschlossene Spannung $U(t)$
unter der Voraussetzun $\omega\ll\frac{1}{RC}$ integireren. Dies hängt damit zusammen, dass die Kondensatorspannung $U_C(t)$
proportional mit dem Integral von $U(t)$ zusammenhängt.
Unter der gestellten Bedingung gilt näherungsweise der Zusammenhang

\begin{equation}
  \label{eqn:Integrator}
  U_C(t) = \frac{1}{RC}\int_0^tU(t')\symup{d}t'.
\end{equation}

Daran ist erkenntlich, dass sich die Kondensatorspannung $U_C(t)$ als Integrator
der Generatorspannung $U(t)$ eignet.

\section{Durchführung}

Zu Beginn des Versuches wird der $RC$-Kreis aufgebaut. Eine Schaltplanskizze
ist in Abb. \ref{fig:Aufbau} zu sehen.
Mit dieser Schaltung kann der Auf- und Entladevorgangs des Kondensators
über das Oszilloskop beobachtet werden. Dazu wird der Generator auf den
Rechteckspannungsbetrieb eingestellt und das Oszilloskop so geregelt, dass
die Lage des Spannungsnullpunktes zu sehen ist. Während des Suchvorganges des
Spannungsnullpunktes wird die Periodendauer der Rechteckspannung möglichst groß gewählt.
Sobald eine geeignete Kurve gefunden wurde, kann diese
als Bild gespeichert werden.

Danach wird der Generator auf den Sinusspannunsbetrieb umgestellt, sodass
der $RC$-Kreis bei variirender Frequenz untersucht werden konnte.
Die Amplitude der Kondensatorspannung wird bei insgesamt 15 verschiedenen
Frequenzen beobachtet. Dabei werden fünf Frequenzen aus dem nieder
frequenten Bereich zwischen $10$ bis $\SI{100}{\hertz}$ betrachtet. Die
anderen zehn Frequenzen werden aus dem Frequenzbereich zwischen $200$ bis
$\SI{1100}{\hertz}$ in hunderter Schritten gewählt. Bei der Messung wird
neben der Frequenz und Kondensatorspannung auch die Generatorspannungen betrachtet.

Desweiteren wird der Phasenunterschied zwischen Generator- und Kondensatorspannung
in Abhängigkeit von der Frequenz gemessen. Dafür werden die beiden Spannungen
auf die Eingänge des Oszilloskops gelegt. Es werden der Abstand der beiden
Nulldurchgänge, sowie die Periodendauer der Generatorspannung mit Hilfe
des digitalen Periodendauermessgerät, welches in dem Oszilloskop integriert ist
gemessen.

Abschließend wird die Eigenschaft der Kondensatorspannung als Integrator
der Generatorspannung untersucht. Dafür wird die Frequenz maximal geregelt,
sodass sie der Bedingung $\omega$ >> $\frac{1}{RC}$ genügt. Bei dem verwendeten
Generator ist die Minimalfrequenz bei ca. $\SI{4}{\hertz}$ erreicht.
Nun wird die Generatorspannung parallel zu der Kondensatorspannung auf dem
Oszilloskop ausgegeben und bei geeigneten Einstellungen als Bild gespeichert.
Die Messung wird mit einer Sinusspannung, einer Rechteckspannung und einer
Dreieckspannung durchgeführt.\\

\FloatBarrier
\begin{figure}
  \includegraphics[width=\textwidth]{Aufbau_V353.PNG}
  \caption{Schaltplanskizze des verwendeten RC-Kreises}
  \label{fig:Aufbau}
\end{figure}

Der in Abbildung \ref{fig:Aufbau} dargestelle Schaltplan zeigt den verwendeten
RC-Kreis. Die Spannung $U\ua{G}$ beschreibt die Generatorspannung. Der Widerstand
R steht für den Innenwiderstand des Generators. Der verwendete Widerstand ist mit
dem Buchstaben R und der Kondensator mit C gekennzeichnet. Die Spannung $U\ua{C}$
symbolisiert die Kondensatorspannung. Die Eingänge des Oszillographen sind mit
$Y\ua{A}$ und $Y\ua{B}$ dargestellt. Die Generatorspannung wird auf den
$Y\ua{B}$-Eingang des Oszillographen gelegt und die Kondensatorspannung auf den
$Y\ua{A}$-Eingang.

\newpage

\input{"../../Praeambel_prak.tex"}

\title{Versuch 702}
\subtitle{Aktivierung mit Neutronen}
\author{Sebastian Pape\\
        sepa@gmx.de \and
        Jonah Nitschke\\
        lejonah@web.de}
\date{Durchführung: 18.07.2017\\
      Abgabe: 25.07.2017}

\begin{document}

\maketitle

\section{Auswertung}

Bei der Nullmessung wurde ein Zeitintervall von $\increment t = 900$ gewählt und
es wurden zwei Messungen durchgeführt, deren Mittelwert für weitere Berechnungen
verwendet wurde:

\begin{align*}
  N\ua{1} &= 218 \\
  N\ua{2} &= 224 \\
  \bar{N} &= 221 \\
  \sigma\ua{Nullmessung} = 14.87
\end{align*}

Bei allen Messungen wird eine lineare Regression in der folgenden Form verwendet,
um die Zerfallskonstante zu bestimmen:

\begin{equation}
  f(x) = A \cdot x + B
  \label{eqn:linRegress}
\end{equation}

\subsection{Halbwertzeit Indium}

Bei der Messung von Indium wurde ein Zeitintervall von $\increment t = 240 \,
\su{s}$ und ein Messzeitraum von $t\ua{ges} = 3600 \, \su{s}$ gewählt. Die
gemessenen Zerfälle sind in Tabelle \ref{tab:Indium} eingetragen und grafisch
in Abbildung \ref{fig:Indium} dargestellt.

\begin{table}
  \centering
  \caption{Gemessene Zerfälle bei Indium}
  \label{tab:Indium}
  \begin{tabular}{c c c c}
    \toprule $\increment t \, in \, \su{s}$ & $Anz. \, Zerfaelle$ & $\increment t \, in \, \su{s}$ & $Anz. \, Zerfaelle$ \\
    \midrule
    240 & 2995  & 480  & 2485 \\
    720 & 2465  & 960  & 2346 \\
    1200 & 2345 & 1440 & 2268 \\
    1680 & 2076 & 1920 & 1943 \\
    2160 & 1894 & 2400 & 1827 \\
    2640 & 1686 & 2880 & 1555 \\
    3120 & 1525 & 3360 & 1512 \\
    3600 & 1417 &      &      \\
    \bottomrule
  \end{tabular}
\end{table}

\begin{figure}
  \includegraphics[width = \textwidth]{Indium_log.pdf}
  \caption{logarythmische Darstellung der gemessenen Zerfälle bei Indium}
  \label{fig:Indium}
\end{figure}

Mithilfe einer linearen Regression der Form \eqref{eqn:linRegress} gemäß Formel
?? werden dabei die Zeitkonstante $\lambda$ und $N\ua{0,Indium}$ bestimmt:

\begin{align*}
A &= \lambda\ua{Indium} = (0.0002 \pm 9 \cdot 10^{-6}) \, \frac{1}{\su{s}}\\
B &= N\ua{0,Indium}     = (7.96 \pm 0.02)
\end{align*}

Mit der Formel ?? kann aus der bestimmten Zeitkonstante nun die Halbwertzeit von
Indium bestimmt werden, für die sich der folgende Wert ergibt:

\begin{equation*}
  T\ua{Indium} = (3278 \pm 141) \, \su{s}
\end{equation*}

\subsection{Halbwertzeit von Rhodium}

\begin{figure}
  \includegraphics[width = \textwidth]{Rhodium_normal_ohne.pdf}
  \caption{Gemessene Zerfälle bei Rhodium}
  \label{fig:RhodiumOhne}
\end{figure}

Um die Halbwertzeiten der zwei verschiedenen Isotope $Rh^{104}$ sowie $Rh^{104i}$
zu bestimmen, die bei dem Zerfall von $Rh_{45}^{103}$ entstehen, wurden für die
Unterteilung die Messzeiten $t^{*}=355 \, \su{s}$ und $t_i=80 \, \su{s}$ gewählt.





\end{document}


\end{document}
