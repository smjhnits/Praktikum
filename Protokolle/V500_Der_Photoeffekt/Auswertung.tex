\section{Auswertung}

In der folgenden Auswertung werden alle Berechnungen mit Python durchgeführt.
Bei der linearen Regression wird gemäß der Gleichung \eqref{} an eine
Funktion der folgenden Form gefittet:

\begin{equation}
  \label{eqn:Regression0}
  f(\su{x}) = \su{m} \cdot \su{x} + \su{b}
\end{equation}


\subsection{Bestimmung der Gegenspannung für die verschiedenen Spektralfarben}

Für die Bestimmung der Gegenspannung der gelben Spektrallinie werden aus der Tabelle
\ref{tab:Gelb_Komplett} lediglich die Werte für $U$ $\geq$ 0 $\su{V}$ betrachtet und in
Abb. \ref{fig:Gelb} grafisch dargestellt.

Mithilfe einer linearen Regression gemäß der Formel \eqref{eqn:Regression0} wird die
Gegenspannung bestimmt, indem der Schnittpunkt der Ausgleichsgeraden mit der
Spannungsachse nach folgender Formel bestimmt wird:

\begin{equation}
  \su{x}\ua{0} = - \frac{\su{b}}{\su{m}}.
\end{equation}

Damit ergeben sich für die Gegenspannungen der verschiedenen Spektralfarben
die folgenden Werte, wobei:

\begin{table}
  \centering
  \label{tab:Spannungen}
  \caption{Bestimmte Grenzspannungen für verschiedene Spektralfarben.}
  \begin{tabular}{c c c}
    \toprule
    $\lambda$ in $\si{nm}$ & $U\ua{g}$ in $\si{V}$ & $\sigma\ua{U}$ in $\si{V}$ \\
    \midrule
    578 & 0.57 & 0.04 \\
    546 & 0.69 & 0.04 \\
    492 & 0.88 & 0.03 \\
    435 & 1.15 & 0.07 \\
    405 & 1.19 & 0.05 \\
    365 & 1.25 & 0.04 \\
    \bottomrule
  \end{tabular}
\end{table}


Im folgenden sind noch einmal die gemessenen Messwerte sowie die dazugehörigen
Abbildungen für jede Spektralfarbe zu sehen.

\newpage

\begin{table}
  \centering
  \label{tab:Gelb_Komplett}
  \caption{Aufgenommene Werte bei der gelben Spektralfarbe.}
  \begin{tabular}{c c c}
    \toprule
    $U$ in $\su{V}$ & $I\cdot 10^{9}$ in $\su{A}$ & $\sqrt{I}\cdot10^{5}$ in $\su{A}^{\frac{1}{2}}$ \\
    \midrule
    -19.15 & 8.3  & 9.11 \\
    -17.05 & 8    & 8.94 \\
    -15.01 & 7.65 & 8.75 \\
    -13.02 & 7.3  & 8.54 \\
    -10.99 & 6.6  & 8.12 \\
    -9.99  & 6.4  & 8    \\
    -9.0   & 6.15 & 7.84 \\
    -8.05  & 5.9  & 7.68 \\
    -7.05  & 5.5  & 7.42 \\
    -6.02  & 5.1  & 7.14 \\
    -5.05  & 4.8  & 6.93 \\
    -4.04  & 4.3  & 6.56 \\
    -3.05  & 3.9  & 6.25 \\
    -2.02  & 2.7  & 5.2  \\
    -1.01  & 2    & 4.47 \\
    -0.5   & 1.4  & 3.74 \\
     0     & 0.7  & 2.65 \\
     0.1   & 0.45 & 2.12 \\
     0.2   & 0.32 & 1.79 \\
     0.31  & 0.14 & 1.18 \\
     0.4   & 0.05 & 0.7  \\
     0.55  & 0    & 0    \\
     \bottomrule
   \end{tabular}
 \end{table}


\begin{figure}
  \centering
  \includegraphics[width = 0.8\textwidth]{Pics/gelbe_Spektrallinie.pdf} \\[0cm]
  \caption{Gemessene Stromstärke in Abhängigkeit der angelegten Spannung für die
           gelbe Spektrallinie.}
  \label{fig:Gelb}
\end{figure}

\newpage


\begin{table}
  \centering
  \label{tab:Gruen}
  \caption{Aufgenommene Werte bei der gruenen Spektralfarbe.}
  \begin{tabular}{c c c}
    \toprule
    $U$ in $\su{V}$ & $I\cdot 10^{9}$ in $\su{A}$ & $\sqrt{I}\cdot10^{5}$ in $\su{A}^{\frac{1}{2}}$ \\
    \midrule
    0.01 & 1.2  & 3.46 \\
    0.1  & 1    & 3.16 \\
    0.2  & 0.8  & 2.83 \\
    0.3  & 0.5  & 2.24 \\
    0.4  & 0.22 & 1.48 \\
    0.5  & 0.09 & 0.95 \\
    0.6  & 0.02 & 0.45 \\
    0.64 & 0    & 0    \\
    \bottomrule
  \end{tabular}
\end{table}


\begin{figure}
  \centering
  \includegraphics[width = 0.7\textwidth]{Pics/gruene_Spektrallinie.pdf}\\[0cm]
  \caption{Gemessene Stromstärke in Abhängigkeit der angelegten Spannung für die
           grüne Spektralfarbe.}
  \label{fig:Gruen}
\end{figure}

\newpage

\begin{table}
  \centering
  \label{tab:Gelb_Komplett}
  \caption{Aufgenommene Werte bei der blau-grünen Spektralfarbe.}
  \begin{tabular}{c c c}
    \toprule
    $U$ in $\su{V}$ & $I\cdot 10^{9}$ in $\su{A}$ & $\sqrt{I}\cdot10^{5}$ in $\su{A}^{\frac{1}{2}}$ \\
    \midrule
    0.01  & 0.3  & 1.73 \\
    0.05  & 0.29 & 1.7  \\
    0.1   & 0.22 & 1.48 \\
    0.151 & 0.2  & 1.41 \\
    0.202 & 0.2  & 1.41 \\
    0.253 & 0.18 & 1.34 \\
    0.3   & 0.15 & 1.22 \\
    0.35  & 0.12 & 1.1  \\
    0.45  & 0.08 & 0.89 \\
    0.55  & 0.04 & 0.63 \\
    0.6   & 0.03 & 0.55 \\
    0.65  & 0.02 & 0.45 \\
    0.7   & 0.01 & 0.32 \\
    0.75  & 0.01 & 0.32 \\
    0.8   & 0    & 0    \\
    \bottomrule
  \end{tabular}
\end{table}


\begin{figure}
  \centering
  \includegraphics[width = 0.7\textwidth]{Pics/blau_gruene_Spektrallinie.pdf}\\[0cm]
  \caption{Gemessene Stromstärke in Abhängigkeit der angelegten Spannung für die
           blau-grüne Spektralfarbe.}
  \label{fig:BlauGruen}
\end{figure}

\newpage

\begin{table}
  \centering
  \label{tab:blau}
  \caption{Aufgenommene Werte bei der blauen Spektralfarbe.}
  \begin{tabular}{c c c | c c c}
    \toprule
    $U$ in $\su{V}$ & $I\cdot 10^{9}$ in $\su{A}$ & $\sqrt{I}\cdot10^{5}$ in $\su{A}^{\frac{1}{2}}$ &
    $U$ in $\su{V}$ & $I\cdot 10^{9}$ in $\su{A}$ & $\sqrt{I}\cdot10^{5}$ in $\su{A}^{\frac{1}{2}}$ \\
    \midrule
    0.01 & 7.4  & 8.6  & 0.6  & 2.3  & 4.8  \\
    0.1  & 6.6  & 8.12 & 0.7  & 1.4  & 3.74 \\
    0.21 & 6.4  & 8    & 0.81 & 0.6  & 2.45 \\
    0.31 & 5.6  & 7.48 & 0.9  & 0.33 & 1.82 \\
    0.4  & 4.0  & 6.32 & 1.0  & 0.1  & 1    \\
    0.5  & 3.2  & 5.66 & 1.07 & 0    & 0    \\
    \bottomrule
  \end{tabular}
\end{table}


\begin{figure}
  \centering
  \includegraphics[width = 0.8\textwidth]{Pics/blaue_Spektrallinie.pdf}\\[0cm]
  \caption{Gemessene Stromstärke in Abhängigkeit der angelegten Spannung für die
           blaue Spektralfarbe.}
  \label{fig:Blau}
\end{figure}

\newpage

\begin{table}
  \centering
  \label{tab:UV1}
  \caption{Aufgenommene Werte bei der 1. ultravioletten Spektralfarbe.}
  \begin{tabular}{c c c | c c c}
    \toprule
    $U$ in $\su{V}$ & $I\cdot 10^{9}$ in $\su{A}$ & $\sqrt{I}\cdot10^{5}$ in $\su{A}^{\frac{1}{2}}$ &
    $U$ in $\su{V}$ & $I\cdot 10^{9}$ in $\su{A}$ & $\sqrt{I}\cdot10^{5}$ in $\su{A}^{\frac{1}{2}}$ \\
    \midrule
    0.001 & 2.7  & 5.2  & 0.7   & 0.52 & 2.28 \\
    0.1   & 2.4  & 4.9  & 0.8   & 0.28 & 1.67 \\
    0.2   & 2.2  & 4.7  & 0.9   & 0.16 & 1.26 \\
    0.3   & 1.9  & 4.36 & 1     & 0.08 & 0.89 \\
    0.4   & 1.7  & 4.12 & 1.1   & 0.01 & 0.32 \\
    0.5   & 1.1  & 3.32 & 1.12  & 0    & 0    \\
    0.6   & 0.7  & 2.65 &      &      &      \\
    \bottomrule
  \end{tabular}
\end{table}


\begin{figure}
  \centering
  \includegraphics[width = 0.7\textwidth]{Pics/UV1_Spektrallinie.pdf}\\[0cm]
  \caption{Gemessene Stromstärke in Abhängigkeit der angelegten Spannung für die
           1. ultraviolette Spektralfarbe.}
  \label{fig:UV1}
\end{figure}

\newpage

\begin{table}
  \centering
  \label{tab:Gelb_Komplett}
  \caption{Aufgenommene Werte bei der 2. ultravioletten Spektralfarbe.}
  \begin{tabular}{c c c}
    \toprule
    $U$ in $\su{V}$ & $I\cdot 10^{9}$ in $\su{A}$ & $\sqrt{I}\cdot10^{5}$ in $\su{A}^{\frac{1}{2}}$ \\
    \midrule
    0.001 & 0.3  & 1.73 \\
    0.1   & 0.26 & 1.61 \\
    0.2   & 0.23 & 1.52 \\
    0.3   & 0.19 & 1.38 \\
    0.4   & 0.16 & 1.26 \\
    0.5   & 0.12 & 1.1  \\
    0.6   & 0.09 & 0.95 \\
    0.7   & 0.06 & 0.77 \\
    0.8   & 0.04 & 0.63 \\
    0.9   & 0.02 & 0.45 \\
    1     & 0    & 0    \\
    \bottomrule
  \end{tabular}
\end{table}


\begin{figure}
  \centering
  \includegraphics[width = 0.7\textwidth]{Pics/UV2_Spektrallinie.pdf}\\[0cm]
  \caption{Gemessene Stromstärke in Abhängigkeit der angelegten Spannung für die
           2.ultraviolette Spektralfarbe.}
  \label{fig:UV2}
\end{figure}

\newpage
\newpage

\subsection{Bestimmung der Austrittsarbeit $A\ua{k}$ und des Verhältnisses $\frac{h}{e\ua{0}}$}

\begin{align}
  U\ua{g}   &= \frac{h}{e\ua{0}} \cdot \nu + \frac{A\ua{k}}{e\ua{0}}
  \label{eqn:Regression1} \\
  f(\su{x}) &= \su{m} \cdot \su{x} + \su{b}
  \label{eqn:Regression2} \\
  \su{m}  &= \frac{h}{e\ua{0}}
  \label{eqn:Regression3} \\
  \su{b}  &= \frac{A\ua{k}}{e\ua{0}}.
  \label{eqn:Regression4}
\end{align}

\begin{figure}
  \centering
  \includegraphics[width = 0.8\textwidth]{Pics/U_g_gegen_Frequenz.pdf}
  \caption{Gegenspannungen der verschiedenen Lichtfrequenzen.}
  \label{figGegenspannung}
\end{figure}

Mithilfe der in Messung 3.1 erhaltenen Gegenspannungen für die verschiedenen
Spektralfarben (siehe Tab. \ref{tab:Spannungen}) lassen sich nun die gesuchten
Parameter
aus Gleichung \eqref{eqn:Regression1} bestimmen. Dafür wird eine lineare Regression
durchgeführt (siehe Abb. \ref{fig:GegenSpannung}), welche für die Parameter
folgende Werte ausgibt:

\begin{align}
  m = \frac{h}{e\ua{0}}                 &= \num{2.1(4)e-15} \, \si{eV}\\
  b = \frac{A\ua{k}}{e\ua{0}} = A\ua{k} &= (\num{0.48(27)}) \, \si{eV}
\end{align}

\newpage

\subsection{Betrachtung der gelben Spektralfarbe in einem Intervall von $U \, \in$ $[-20, 20] \, \su{V}$ }

\begin{table}
  \centering
  \label{tab:Gelb_Komplett}
  \caption{Aufgenommene Werte bei der gelben Spektralfarbe.}
  \begin{tabular}{c c c | c c c }
    \toprule
    $U$ in $\su{V}$ & $I\cdot 10^{9}$ in $\su{A}$ & $\sqrt{I}\cdot10^{5}$ in $\su{A}^{\frac{1}{2}}$ &
    $U$ in $\su{V}$ & $I\cdot 10^{9}$ in $\su{A}$ & $\sqrt{I}\cdot10^{5}$ in $\su{A}^{\frac{1}{2}}$ \\
    \midrule
    -19.15 & 8.3  & 9.11 & -4.04  & 4.3  & 6.56 \\
    -17.05 & 8    & 8.94 & -3.05  & 3.9  & 6.25 \\
    -15.01 & 7.65 & 8.75 & -2.02  & 2.7  & 5.2  \\
    -13.02 & 7.3  & 8.54 & -1.01  & 2    & 4.47 \\
    -10.99 & 6.6  & 8.12 &  0     & 0.7  & 2.65 \\
    -9.99  & 6.4  & 8    &  0.1   & 0.45 & 2.12 \\
    -9.0   & 6.15 & 7.84 &  0.2   & 0.32 & 1.79 \\
    -8.05  & 5.9  & 7.68 &  0.31  & 0.14 & 1.18 \\
    -7.05  & 5.5  & 7.42 &  0.4   & 0.05 & 0.7  \\
    -6.02  & 5.1  & 7.14 &  0.55  & 0    & 0    \\
    -5.05  & 4.8  & 6.93 &        &      &      \\
     \bottomrule
   \end{tabular}
 \end{table}


\begin{figure}
  \centering
  \includegraphics[width = 0.8\textwidth]{Pics/gelbe_Spektrallinie_komplett.pdf}\\[0cm]
  \caption{Gemessene Stromstärke in Abhängigkeit der angelegten Spannung für die
           gelbe Spektralfarbe.}
  \label{fig:GelbKomplett}
\end{figure}
