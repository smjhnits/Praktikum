\section{Theorie}

Wird ein Festkörper mit Licht bestrahlt, werden unter gewissen Vorraussetzungen
Elektronen aus diesem gelöst. Dieses Phänomen wird im Versuch V500: "Der Photoeffekt"
untersucht.

Eine widerspruchsfreie Erklärung von Licht, erlaubt nur die Quantenelektrodynamik.
Dieses Modell beinhaltet den Wellencharakter und den Teilchencharakter von Licht
als Grenzfälle. Das Wellenmodell zur Beschreibung von Licht ist immer dann sinnvoll,
wenn über eine große Anzahl von Photonen gemittelt werden kann. Hingegen 
