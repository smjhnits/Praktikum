\input{"../../Praeambel_prak.tex"}

\begin{document}

\section{Auswertung}

Im Folgendem werden die Messergebnisse ausgewertet und auf geeignete Weise
visualisiert.
Die verwendete Schaltung hatte die folgenden Daten.

\begin{align*}
  L &= \SI{3,53(3)}{\milli\henry} \\
  C &= \SI{5,015(15)}{\nano\farad} \\
  R_1 &= \SI{30,3(1)}{\ohm} \\
  R_2 &= \SI{271,6(3)}{\ohm}
\end{align*}

\subsection{Einhüllende der Schwingungskurve}

Die Wertepaare ($U_C(t_i), t_i$) müssen fürdie Ausgleichsrechnung bestimmt werden.
Die Werte $U_C(t_i)$ wurden mit dem Cursor des Oszilloskops gemessen.
Hingegen wurden die Zeiten $t_i$ aus dem Bild der Schwingungskurve mit
Hilfe eines Lineals abgelesen. Ein Abbild der Schwingungskurve ist in Abb.
\ref{fig:Schwingungskurve} dargestellt.

\begin{figure}
  \centering
  \includegraphics[width=\textwidth, angle=90, height=8cm]{F0001TEK.JPG}
  \caption{Gemessene Schwingungskurve.}
  \label{fig:Schwingungskurve}
\end{figure}

Die Schwingungskurve in Abb. \ref{fig:Schwingungskurve} wurde beim
Widerstand $R_1$ und einer Generatorfrequenz $\SI{5,82}{\hertz}$ erstellt.

Die diskreten Wertepaare ($U_C(t_i), t_i$) sind in der Tabelle \ref{tab:Schwingungskurve}
dargestellt. Dabei wurden für $U_C(t_i)$ jeweils die Maxima der Schwingungskurve
vermessen.

\floatplacement{table}{htbp}
\begin{table}
 \centering
 \sisetup{table-format=3.2}
 \begin{tabular}[width=\textwidth]{S S}
     \toprule
      {Zeit in $\si{\micro\second}$} & {Maxima in $\si{\volt}$} \\
     \midrule
      0 & 15,08 \\
      27,5 & 13,2 \\
      55 & 11,92 \\
      82,5 & 10,96 \\
      112,5 & 10,24 \\
      142,5 & 9,68 \\
      172,5 & 9,36 \\
      202,5 & 9,04 \\
      235 & 8,88 \\
      267,5 & 8,76 \\
      302,5 & 8,64 \\
      337,5 & 8,52 \\
      \bottomrule
  \end{tabular}
  \caption{Messdaten der Schwingungskurve.}
  \label{tab:Schwingungskurve}
\end{table}

Mit den Wertepaaren aus Tabelle \ref{tab:Schwingungskurve} wurde mittels
des \emph{Python}-Paketes \emph{curve\_fit} eine Ausgleichsrechnung an eine
exponential Funktion der Form

\begin{align}
  \label{eqn:exp}
  U\ua{c}(t) = a\cdot\exp^{-b\cdot t} + c
\end{align}

durchgeführt. Für die Parameter ergeben sich somit die Werte

\begin{align*}
  a & = \SI{6,62(3)}{\volt} \\
  b &= \num{1,17(1)e4}\frac{1}{\si{\per\second}} \\
  c &= \SI{8,44(2)}{\volt}
\end{align*}

Die Ausgleichfunktion ist mit den Daten aus Tabelle \ref{tab:Schwingungskurve} in Abb. \ref{fig:Ausgleichrechnung} dargestellt.

\floatplacement{table}{htbp}
\begin{figure}
  \centering
  \includegraphics[width=\textwidth]{ausgleichsrechnung.pdf}
  \caption{Darstellung der Ausgleichsfunktion.}
  \label{fig:Ausgleichsrechnung}
\end{figure}

Der Exponent der Ausgleichsfunktion liefert über die Formeln (??) den
effektiv Widerstand $R\ua{eff}$ und die Abklingzeit $T\ua{ex}$.
Damit ergeben sich die folgenden Werte.

\begin{align*}
  R\ua{eff} &= \SI{82,4(12)}{\ohm}\\
  T\ua{ex} &= \SI{8,56(1)e-5}{\second}
\end{align*}

Im Vergleich zu dem eigebauten, verwendeten Widerstand $R_1$ fällt auch, dass
$R\ua{eff}$ deutlich größer ist. Dies ist damit zu begründen, dass $R\ua{eff}$
den Innenwiderstand des Generators mit einbezieht, welcher in $R_1$ nicht erfasst
wird.

\subsection{Widerstand im aperiodischen Grenzfall}

Aus den Daten $L$ und $C$ der Apparatur lässt sich über Formel (??) der
Widerstand des aperiodischen Grenzfalles $R\ua{ap}$ errechnen.
Der Wert $R\ua{ap}$ wurde auch experimentell bestimmt. Die Messung ergeben
die folgenden Werte.

\begin{align*}
  R\ua{ap,theo} = \SI{1678(8)}{\ohm}
  R\ua{ap} = \SI{13500}{\ohm}
\end{align*}

Der Wert $R\ua{ap}$ ist der experimentell bestimmte Wert. Dieser wurde an
dem variablen Widerstand der Apparatur abgelesen und wird als fehlerfrei
angenommen. Die gemessene Wert ist ca. acht mal größer als der
theoretisch berechnete Wert. Dies hängt damit zusammen, dass der experimentelle
Wert an der Apparatur nur ungenau abzulesen war. Zudem war nicht sichergestellt,
dass der abgelesene Widerstand tatsächlich mit dem angelegten Widerstand
übereinstimmt. Damit ist die Diskrepantz auf einen systematischen
Fehler zurückzuführen.

\subsection{Resonanzfrequenz}

Die Resonanzfrequenz lässt sich mit den Apparaturdaten über die Formel (??)
errechnen. Als Widerstand wurde der gemessene effektiv Widerstand $R\ua{eff}$
verwendet.
Die berechnete Resonanzfrequenz beträgt:

\begin{align*}
  \nu\ua{res} = \SI{3.774(17)e4}{\hertz}.
\end{align*}

Der gemessene Wert wurde aus dem Diagramm \ref{fig:Resonanz} abgelesen.

\floatplacement{table}{htbp}
\begin{figure}
  \centering
  \includegraphics[width=\textwidth]{phase_gegen_nu.pdf}
  \caption{Kondensatorspannung gegenüber der Frequenz.}
  \label{fig:Resonanz}
\end{figure}

Der abgelesene Wert bei einer Phase von $\varphi = \frac{pi}{2}$ ist:

\begin{equation*}
  \nu\ua{res} = \SI{37000}{\hertz}.
\end{equation*}

\end{document}
