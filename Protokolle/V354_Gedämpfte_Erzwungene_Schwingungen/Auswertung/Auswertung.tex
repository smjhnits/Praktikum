\input{"../../Praeambel_prak.tex"}

\begin{document}

\section{Auswertung}

Im Folgendem werden die Messergebnisse ausgewertet und auf geeignete Weise
visualisiert.
Die verwendete Schaltung hatte die folgenden Daten.

\begin{align*}
  L &= \SI{3,53\pm 0,03}{\milli\henry} \\
  C &= \SI{5,015\pm 0,015}{\nano\farad} \\
  R_1 &= \SI{30,3\pm 0,1}{\ohm} \\
  R_2 &= \SI{271,6\pm 0,3}{\ohm}
\end{align*}

\subsection{Einhüllende der Schwingungskurve}

Die Wertepaare ($U_C(t_i), t_i$) müssen fürdie Ausgleichsrechnung bestimmt werden.
Die Werte $U_C(t_i)$ wurden mit dem Cursor des Oszilloskops gemessen.
Hingegen wurden die Zeiten $t_i$ aus dem Bild der Schwingungskurve mit
Hilfe eines Lineals abgelesen. Ein Abbild der Schwingungskurve ist in Abb.
\ref{fig:Schwingungskurve} dargestellt.

\begin{figure}
  \centering
  \includegraphics[width=\textwidth, angle=90, height=8cm]{F0001TEK.JPG}
  \caption{Gemessene Schwingungskurve.}
  \label{fig:Schwingungskurve}
\end{figure}

Die Schwingungskurve in Abb. \ref{fig:Schwingungskurve} wurde beim
Widerstand $R_1$ und einer Generatorfrequenz $\SI{5,82}{\hertz}$ erstellt.

Die diskreten Wertepaare ($U_C(t_i), t_i$) sind in der Tabelle \ref{tab:Messunga}
dargestellt. Dabei wurden für $U_C(t_i)$ jeweils die Maxima der Schwingungskurve
vermessen.

\floatplacement{table}{htbp}
\begin{table}
 \centering
 \sisetup{table-format=3.2}
 \begin{tabular}[width=\textwidth]{S S}
     \toprule
      {Zeit in $\si{\micro\second}$} & {Maxima in $\si{\volt}$} \\
     \midrule
      0 & 15,08 \\
      27,5 & 13,2 \\
      55 & 11,92 \\
      82,5 & 10,96 \\
      112,5 & 10,24 \\
      142,5 & 9,68 \\
      172,5 & 9,36 \\
      202,5 & 9,04 \\
      235 & 8,88 \\
      267,5 & 8,76 \\
      302,5 & 8,64 \\
      337,5 & 8,52 \\
      \bottomrule
  \end{tabular}
  \caption{Messdaten der Schwingungskurve.}
  \label{tab:Schwingungskurve}
\end{table}

Mit den Wertepaaren aus Tabelle \ref{tab:schwingungskurve} wurde mittels
des \emph{Python}-Paketes \emph{curve_fit} eine Ausgleichsrechnung an eine
exponential Funktion der Form

\begin{align}
  \label{eqn:exp}
  U_c(t) = a\cdot\exp^{-b\cdot t} + c
\end{align}

durchgeführt. Für die Parameter ergeben sich somit die Werte

\begin{align*}
  a & = \SI{6,62(3)}{\volt} \\
  b &= \num{1,17(1)e4}\frac{1}{\si{\per\second}} \\
  c &= \SI{8,44(2)}{\volt}
\end{align*}

Die Ausgleichfunktion ist mit den Daten aus Tabelle \ref{tab:Schwingungskurve} in Abb. \ref{fig:Ausgleichrechnung} dargestellt.

\begin{figure}
  \centering
  \includegraphics[width=\textwidth]{ausgleichsrechnung.pdf}
  \caption{Darstellung der Ausgleichsfunktion.}
  \label{fig:Ausgleichsrechnung}
\end{figure}

Der Exponent der Ausgleichsfunktion liefert über die Formeln (??) den
effektiv Widerstand $R_{eff}$ und die Abklingzeit $T_{ex}$.
Damit ergeben sich die folgenden Werte.

\begin{align*}
  R_{eff} &= \SI{82,4(12)}{\ohm}\\
  T_{ex} &= \SI{8,56(1)e-5}{\second}
\end{align*}

\end{document}
