\input{"../../Praeambel_prak.tex"}

\title{Versuch 356}
\subtitle{Kettenschaltung mit LC-Gliedern}
\author{Sebastian Pape\\
        sepa@gmx.de \and
        Jonah Nitschke\\
        lejonah@web.de}
\date{Durchführung: 17.01.2017\\
      Abgabe: 24.01.2017}

\begin{document}
\maketitle

\section{Einleitung}

In dem folgenden Versuch geht es um die Betrachtung von Wellen mithilfe eines
Schwingkreises als Analogon zum verketteten harmonischen Oszillators in der
Mechanik. Dabei werden verschiedene charakteristische Größen der entstehenden
stehenden Wellen betrachtet, wie zum Beispiel Dispersionsrelation sowie Phasenverschub
zwischen der 

\end{document}
