\input{"Praeambel_prak.tex"}

\title{Versuch 702}
\subtitle{Aktivierung mit Neutronen}
\author{Jonah Nitschke\\
        lejonah@web.de \and
        Sebastian Pape\\
        sepa@gmx.de}
\date{Durchführung: 18.04.2017\\
      Abgabe: 25.04.2017}



\begin{document}

\maketitle

\section{Theorie}

\subsection{Zielsetzung}

In dem Versuch 702 sollen die Halbwertszeiten von verschiedenen radioaktiven
Materialien ermittelt werden.

\subsection{Theoretische Grundlagen}

Damit instabile Kerne erzeugt werden, werden stabile Kene mit Neutronen beschossen.
Der Vorteil der Neutronensktivierung liegt darin, dass die Ladungsneutralen
Neutronen weniger Energie benötigen, um die Kerne zu aktivieren, da sie
nicht die Coulomb-Barriere des geladenen Kernes überwinden müssen.

Eine Allgemeine Kernreaktion eines Beispielkernes $\ce{^{m}_{z}A}$ sieht
folgendermaßen aus:

\begin{equation*}
  \ce{^{m}_{z}A + ^{1}_{0}n -> ^{m+1}_{z}A^{*} -> ^{m+1}_{z}A + \gamma} .
\end{equation*}

$\ce{^{m}_{z}A^{*}}$ ist dabei der sogenannte Zwischenkern oder auch Compoundkern.
Seine Energie ist im Vergleich zu dem Ausgangskern um die kinetische Energie
des Neutrons und der Bindungsenergie höher.
Bei geringer kinetischer Energie des Neutrons ist die eingebrachte Energie
zu gering um ein Nukleon oder ein Neutron wieder abzugeben. Deshalb wird
nach etwa $\si{10^{-16}}{\second}$ ein $\gamma$-Quant abgegeben, sodass der Kern
wieder in seinen Grundzustand zurückfällt. Dieser Kern ist immernoch instabil,
hat aber eine deutlich längere Lebensdauer als der Zwischenkern.
Die Zerfallsreihe dieses Kerns sieht folgendermaßen aus:

\begin{equation*}
  \ce{^{m+1}_{z}A -> ^{m+1}_{z+1}C + \beta^{-} + E_{kin} + \bar{\nu}}_{e}}.
\end{equation*}



\section{Durchführung}

Die zu untersuchenden Proben wurden im Vorhinein aktiviert.
Zuerst musst der Nullwert bestimmt werden. Dafür wird eine Messung über
$\si{900}{\second}$ ohne Probe gemacht. Danach wurden die zu untersuchenden
Proben in den Aufbau eingelegt. Es wurden das Isotope Indium-116 ($\ce{^{116}I}$)
und ein Rhodiumisomer ($\ce{^{104}Rh}$ & $\ce{^{104i}Rh}$) untersucht.
Für das Indiumisotop ist eine Messzeit von einer Stunde mit einem Messintervall von
$\delta t = \si{240}{\second}$ gewählt worden. Für das Rhodiumisomer ist eine
Messzeit von $\si{12}{\minute}$ angesetzt worden, mit einem Messintervall
$\delta t$ von $\si{20}{\second}$.
