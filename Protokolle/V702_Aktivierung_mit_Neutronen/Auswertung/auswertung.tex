\input{"../../Praeambel_prak.tex"}

\title{Versuch 702}
\subtitle{Aktivierung mit Neutronen}
\author{Sebastian Pape\\
        sepa@gmx.de \and
        Jonah Nitschke\\
        lejonah@web.de}
\date{Durchführung: 18.07.2017\\
      Abgabe: 25.07.2017}

\begin{document}

\maketitle

\section{Auswertung}

Bei der Nullmessung wurde ein Zeitintervall von $\increment t = 900$ gewählt und
es wurden zwei Messungen durchgeführt, deren Mittelwert für weitere Berechnungen
verwendet wurde:

\begin{align*}
  N\ua{1} &= 218 \\
  N\ua{2} &= 224 \\
  \bar{N} &= 221 \\
  \sigma\ua{Nullmessung} = 14.87
\end{align*}

Bei allen Messungen wird eine lineare Regression in der folgenden Form verwendet,
um die Zerfallskonstante zu bestimmen:

\begin{equation}
  f(x) = A \cdot x + B
  \label{eqn:linRegress}
\end{equation}

\subsection{Halbwertzeit Indium}

Bei der Messung von Indium wurde ein Zeitintervall von $\increment t = 240 \,
\su{s}$ und ein Messzeitraum von $t\ua{ges} = 3600 \, \su{s}$ gewählt. Die
gemessenen Zerfälle sind in Tabelle \ref{tab:Indium} eingetragen und grafisch
in Abbildung \ref{fig:Indium} dargestellt.

\begin{table}
  \centering
  \caption{Gemessene Zerfälle bei Indium}
  \label{tab:Indium}
  \begin{tabular}{c c c c}
    \toprule $\increment t \, in \, \su{s}$ & $Anz. \, Zerfaelle$ & $\increment t \, in \, \su{s}$ & $Anz. \, Zerfaelle$ \\
    \midrule
    240 & 2995  & 480  & 2485 \\
    720 & 2465  & 960  & 2346 \\
    1200 & 2345 & 1440 & 2268 \\
    1680 & 2076 & 1920 & 1943 \\
    2160 & 1894 & 2400 & 1827 \\
    2640 & 1686 & 2880 & 1555 \\
    3120 & 1525 & 3360 & 1512 \\
    3600 & 1417 &      &      \\
    \bottomrule
  \end{tabular}
\end{table}

\begin{figure}
  \includegraphics[width = \textwidth]{Indium_log.pdf}
  \caption{logarythmische Darstellung der gemessenen Zerfälle bei Indium}
  \label{fig:Indium}
\end{figure}

Mithilfe einer linearen Regression der Form \eqref{eqn:linRegress} gemäß Formel
?? werden dabei die Zeitkonstante $\lambda$ und $N\ua{0,Indium}$ bestimmt:

\begin{align*}
A &= \lambda\ua{Indium} = (0.0002 \pm 9 \cdot 10^{-6}) \, \frac{1}{\su{s}}\\
B &= N\ua{0,Indium}     = (7.96 \pm 0.02)
\end{align*}

Mit der Formel ?? kann aus der bestimmten Zeitkonstante nun die Halbwertzeit von
Indium bestimmt werden, für die sich der folgende Wert ergibt:

\begin{equation*}
  T\ua{Indium} = (3278 \pm 141) \, \su{s}
\end{equation*}

\subsection{Halbwertzeit von Rhodium}

\begin{figure}
  \includegraphics[width = \textwidth]{Rhodium_normal_ohne.pdf}
  \caption{Gemessene Zerfälle bei Rhodium}
  \label{fig:RhodiumOhne}
\end{figure}

Um die Halbwertzeiten der zwei verschiedenen Isotope $Rh^{104}$ sowie $Rh^{104i}$
zu bestimmen, die bei dem Zerfall von $Rh_{45}^{103}$ entstehen, wurden für die
Unterteilung die Messzeiten $t^{*}=355 \, \su{s}$ und $t_i=80 \, \su{s}$ gewählt.





\end{document}
