\input{"../../Praeambel_prak.tex"}

\title{Versuch 311}
\subtitle{Der Hall-Effekt}
\author{Sebastian Pape\\
        sepa@gmx.de \and
        Jonah Nitschke\\
        lejonah@web.de}
\date{Durchführung: 20.12.2017\\
      Abgabe: 10.01.2017}

\begin{document}
\maketitle

\section{Einleitung}

In dem folgenden Versuch geht es darum, mithilfe der Messungen der Hall-Spannung
und des Widerstandes die mikroskopischen Leitfähigkeitsparameter von verschiedenen
Metallen zu bestimmen. Für den folgenden Versuch wurden die Metalle Zink und Kupfer
verwendet.

\section{Theorie}
\subsection{Bandstruktur und elektrische Leitfähigkeit bei Kristallstrukuturen}

Grundlegend für den folgenden Versuch ist die Eigenschaft, dass sich in Metallatomen
die Valenzelektronen abspalten können und mit benachbarten Valenzelektronen ein
System bilden, dass dem Pauli-Prinzip unterliegt. Somit können die Energieniveaus
in der Atomhülle als Energiebänder aufgefasst werden. Diese können sich einerseits
überlappen, andererseits können jedoch auch Lücken auftreten, die als verbotene
Zone beschrieben werden. Es handelt sich hierbei um Energiewerte, die die Elektronen
nicht annehmen können.

Aus dem Pauli-Prinzip folgt des weiteren, dass Energiebänder nur eine begrenzte
Anzahl an Elektronen aufnehmen können. Gefüllte Energiebänder können somit keine
Energie mehr aufnehmen, lediglich teilweise gefüllte Bänder rufen die hohe elektrische
Leitfähigkeit von Metallen hervor. Diese Bänder nennt man Leitungsbänder und ihre
Elektronen werden als Leitungselektronen bezeichnet.

Die fehlende Leitfähigkeit bei Isolatoren lässt sich auf ein leeres oberes Band
zurückführen, durch das die verbotene Zone zu breit ist um den Elektronen zu
erlauben, die Lücke zu überspringen. Mithilfe der Quantentheorie kann nun gezeigt
werden, dass ein idealer Metallkristall eine unendlich hohe elektrische Leitfähigkeit
besitzen müsste. Die endliche Leitfähigkeit realer Proben beruht somit im weitesten
Sinne auf Kristallaufbaufehler.

\subsection{Bestimmung der elekrische Leitfähigkeit eines Metalles}

Um die elektrische Leitfähigkeit eines Metalles zu bestimmen müssen vorher noch
andere mikroskopische Größen bestimmt werden. Die mittlere Flugzeit $\bar{\tau}$
gibt zum Beispiel das gemittelte Zeitintervall zwischen zwei Zusammenstößen eines
Elektrons an.

Bei einem angelegten äußerem Feld $\vec{E}$ erfährt das Elektron eine Beschleunigung
$\vec{b}$ in Richtung des E-Feldes und erfährt somit folgende Geschwindigkeitsänderung:

\begin{align}
  \vec{b}            &= - \frac{e\ua{0}}{m\ua{0}} \vec{E} \\
  \increment \vec{\bar{v}} &= \vec{b} \cdot \bar{\tau} = - \frac{e\ua{0}}{m\ua{0}} \vec{E}
\end{align}

Da die Elektronen nach jedem Zusammenstoß zufällig in eine beliebige Richtung
gestreut werden, beträgt die Startgeschwindigkeit in Richtung von $\vec{E}$ im
Mittel null. Somit kann über $\increment \vec{\bar{v}}$ noch die Driftgeschwindigkeit
$\vec{\bar{v}}_d$ definiert werden:

\begin{equation}
  \vec{\bar{v}}_d = \frac{1}{2} \increment \vec{\bar{v}}
\end{equation}

\end{document}
