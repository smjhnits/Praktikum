\input{"../../Praeambel_prak.tex"}

\begin{document}

\section{Messergebnisse}

\subsection{Abmessungen der verwendeten Proben}

Bei der Probe Zink wurden folgende Maße genommen.
Für die Vermessung wurde eine Schieblehre verwendet.

\begin{description}
  \item[Höhe] $\SI{0,02603}{\meter}$
  \item[Breite] $\SI{0,04406}{\meter}$
  \item[Dicke] $\SI{0,00043}{\meter}$
\end{description}

Für die Probe Kupfer wurden folgenden Maße genommen.
Die Dicke der Probe wurde angegeben, die restlichen Maße wurden mit einer
Schieblehre genommen.

\begin{description}
  \item[Höhe] $\SI{0,0280}{\meter}$
  \item[Breite] $\SI{0,0253}{\meter}$
  \item[Dicke] $\SI{0,000018}{\meter}$
\end{description}

\subsection{Messung der Feldstärke bei variirendem Strom}

%\floatplacement{table}{htpb}
\begin{table}
 \centering
 \begin{tabular}[width=\textwidth]{S| S[table-format=1.1] S[table-format=3.1] S[table-format=3.0] S[table-format=3.1] S[table-format=3.0] S[table-format=3.1] S[table-format=3.0] S[table-format=3.1] S[table-format=4.0] S[table-format=4.1] S[table-format=4.0]}
     \midrule
      $I$ \text{\;in\;} \si{\ampere} & 0 & 0,5 & 1 & 1,5 & 2 & 2,5 & 3 & 3,5 & 4 & 4,5 & 5 \\
      $B$ \text{\;in\;} \si{\milli\tesla} & 7,7 & 142 & 272 & 420 & 556 & 700 & 840 & 975 & 1077 & 1158 & 1220 \\
      \bottomrule
\end{tabular}
  \caption{$B$-Feldstärke bei steigender Stromstärke}
  \label{tab:Messergebnisse_Feldstärke_Isteigt}
\end{table}


%\floatplacement{table}{htpb}
\begin{table}
 \centering
 \begin{tabular}[width=\textwidth]{S| S[table-format=4.0] S[table-format=4.1] S[table-format=4.0] S[table-format=3.1] S[table-format=3.0] S[table-format=3.1] S[table-format=3.0] S[table-format=3.1] S[table-format=3.0] S[table-format=3.1] S[table-format=3.1]}

     \midrule
      $I$ \text{\;in\;} \si{\ampere} & 5 & 4,5 & 4 & 3,5 & 3 & 2,5 & 2 & 1,5 & 1 & 0,5 & 1 \\
      $B$ \text{\;in\;} \si{\milli\tesla} & 1220 & 1169 & 1095 & 977 & 845 & 703 & 563 & 422 & 279 & 138 & 8,3 \\
      \bottomrule
\end{tabular}
  \caption{$B$-Feldstärke bei fallender Stromstärke}
  \label{tab:Messergebnisse_Feldstärke_Ifällt}
\end{table}
%\FloatBarrier

\newpage

\subsection{Messdaten für die Bestimmung der Widerstände der Proben}

%\floatplacement{table}{htpb}
\begin{table}
 \centering
 \begin{tabular}[width=\textwidth]{S| S[table-format=2.2] S[table-format=2.2] S[table-format=2.1] S[table-format=2.1] S[table-format=2.1] S[table-format=2.1] S[table-format=2.1] S[table-format=2.1] S[table-format=3.1] S[table-format=3.1] S[table-format=3.1]}

     \midrule
      $I$ \text{\;in\;} \si{\ampere} & 0 & 1 & 2 & 3 & 4 & 5 & 6 & 7 & 8 & 9 & 10 \\
      $U$ \text{\;in\;} \si{\milli\volt} & -0,02 & 14,13 & 27,7 & 41,1 & 55,5 & 68,3 & 81,5 & 94,7 & 107,1 & 120,3 & 133,7 \\
      \bottomrule
\end{tabular}
  \caption{Messdaten für die Probe Zink}
  \label{tab:Spannung_Zink}
\end{table}


%\floatplacement{table}{htpb}
\begin{table}
 \centering
 \begin{tabular}[width=\textwidth]{S| S[table-format=1.2] S[table-format=2.2] S[table-format=2.1] S[table-format=2.1] S[table-format=2.1] S[table-format=2.1] S[table-format=2.1] S[table-format=2.1] S[table-format=3.1] S[table-format=3.1] S[table-format=3.1]}

     \midrule
      $I$ \text{\;in\;} \si{\ampere} & 0 & 1 & 2 & 3 & 4 & 5 & 6 & 7 & 8 & 9 & 10 \\
      $U$ \text{\;in\;} \si{\milli\volt} & 0 & 7,83 & 15,54 & 23,3 & 30,9 & 38,6 & 46,3 & 53,9 & 61,5 & 68,8 & 76,5\\
      \bottomrule
\end{tabular}
  \caption{Messdaten für die Probe Kupfer}
  \label{tab:Spannung_Kupfer}
\end{table}
%\FloatBarrier

%\newpage

\subsection{Messdaten für die gemessene Hall--Spannung bei konstantem Probenstrom}

%\floatplacement{table}{htpb}
\begin{table}
 \centering
 \begin{tabular}[width=\textwidth]{S| S[table-format=1.3] S[table-format=1.3] S[table-format=1.3] S[table-format=1.3] S[table-format=1.3] S[table-format=1.3] S[table-format=1.3] S[table-format=1.3] S[table-format=1.3] S[table-format=1.3] S[table-format=1.3]}

     \midrule
      $I_{\text{\tiny{Spule}}}$ \text{\;in\;} \si{\ampere} & 0 & 0,5 & 1 & 1,5 & 2 & 2,5 & 3 & 3,5 & 4 & 4,5 & 5 \\
      $U$ \text{\;in\;} \si{\milli\volt} & 0,644 & 0,648 & 0,651 & 0,654 & 0,657 & 0,659 & 0,661 & 0,663 & 0,664 & 0,665 & 0,666\\
      \bottomrule
\end{tabular}
  \caption{Messdaten für Zink bei einem konstantem Probenstrom von $\SI{8}{\ampere}$}
  \label{tab:Zink_U_H}
\end{table}


%\floatplacement{table}{htpb}
\begin{table}
 \centering
 \begin{tabular}[width=\textwidth]{S| S[table-format=2.3] S[table-format=1.3] S[table-format=1.3] S[table-format=1.3] S[table-format=1.3] S[table-format=1.3] S[table-format=1.3] S[table-format=1.3]}

     \midrule
      $I_{\text{\tiny{Spule}}}$ \text{\;in\;} \si{\ampere} & 0 & 0,5 & 1 & 1,5 & 2 & 2,5 & 3 & 3,5 \\
      $U$ \text{\;in\;} \si{\milli\volt} & -0,342 & -0,340 & -0,338 & -0,336 & -0,334 & -0,332 & -0,330 & -0,328 \\
      \bottomrule
\end{tabular}
  \caption{Messdaten für Kupfer bei einem konstantem Probenstrom von $\SI{10}{\ampere}$}
  \label{tab:Kupfer_U_H}
\end{table}
%\FloatBarrier

\newpage

\subsubsection{Daten nach Umpolung}

%\floatplacement{table}{htpb}
\begin{table}
 \centering
 \begin{tabular}[width=\textwidth]{S| S[table-format=1.3] S[table-format=1.3] S[table-format=1.3] S[table-format=1.3] S[table-format=1.3] S[table-format=1.3] S[table-format=1.3] S[table-format=1.3] S[table-format=1.3] S[table-format=1.3] S[table-format=1.3]}

     \midrule
      $I_{\text{\tiny{Spule}}}$ \text{\;in\;} \si{\ampere} & 0 & 0,5 & 1 & 1,5 & 2 & 2,5 & 3 & 3,5 & 4 & 4,5 & 5 \\
      $U$ \text{\;in\;} \si{\milli\volt} & 0,647 & 0,646 & 0,645 & 0,644 & 0,642 & 0,641 & 0,639 & 0,638 & 0,636 & 0,635 & 0,634\\
      \bottomrule
\end{tabular}
  \caption{Messdaten für Zink bei einem konstantem Probenstrom von $\SI{8}{\ampere}$}
  \label{tab:Zink_U_H_umgepolt}
\end{table}


%\floatplacement{table}{htpb}
\begin{table}
 \centering
 \begin{tabular}[width=\textwidth]{S| S[table-format=2.3] S[table-format=1.3] S[table-format=1.3] S[table-format=1.3] S[table-format=1.3] S[table-format=1.3] S[table-format=1.3] S[table-format=1.3]}

     \midrule
      $I_{\text{\tiny{Spule}}}$ \text{\;in\;} \si{\ampere} & 0 & 0,5 & 1 & 1,5 & 2 & 2,5 & 3 & 3,5\\
      $U$ \text{\;in\;} \si{\milli\volt} & -0,340 & -0,342 & -0,343 & -0,345 & -0,347 & -0,349 & -0,351 & -0,353 \\
      \bottomrule
\end{tabular}
  \caption{Messdaten für Kupfer bei einem konstantem Probenstrom von $\SI{10}{\ampere}$}
  \label{tab:Kupfer_U_H_umgepolt}
\end{table}
%\FloatBarrier

\newpage

\subsection{Messdaten für die gemessene Hall-Spannung bei konstantem Spulenstrom}

%\floatplacement{table}{htpb}
\begin{table}
 \centering
 \begin{tabular}[width=\textwidth]{S| S[table-format=2.3] S[table-format=1.3] S[table-format=1.3] S[table-format=1.3] S[table-format=1.3] S[table-format=1.3] S[table-format=1.3] S[table-format=1.3] S[table-format=1.3] S[table-format=1.3]
 S[table-format=1.3]}

     \midrule
      $I_{\text{\tiny{Probe}}}$ \text{\;in\;} \si{\ampere} & 0 & 0,8 & 1,6 & 2,4 & 3,2 & 4 & 4,8 & 5,6 & 6,4 & 7,2 & 8 \\
      $U$ \text{\;in\;} \si{\milli\volt} & -0,020 & 0,045 & 0,109 & 0,174 & 0,234 & 0,304 & 0,365 & 0,431 & 0,495 & 0,560 & 0,626 \\
      \bottomrule
\end{tabular}
  \caption{Messdaten für Zink bei einem konstantem Spulenstrom von $\SI{5}{\ampere}$}
  \label{tab:Zink_U_H_2}
\end{table}


%\floatplacement{table}{htpb}
\begin{table}
 \centering
 \begin{tabular}[width=\textwidth]{S| S[table-format=2.3] S[table-format=1.3] S[table-format=1.3] S[table-format=1.3] S[table-format=1.3] S[table-format=1.3] S[table-format=1.3] S[table-format=1.3] S[table-format=1.3] S[table-format=1.3] S[table-format=1.3]}

     \midrule
      $I_{\text{\tiny{Probe}}}$ \text{\;in\;} \si{\ampere} & 0 & 1 & 2 & 3 & 4 & 5 & 6 & 7 & 8 & 9 & 10 \\
      $U$ \text{\;in\;} \si{\milli\volt} & - 0,336 & - 0,338 & - 0,340 & - 0,342 & - 0,343 & - 0,345 & - 0,347 & - 0,348 & -0,350 & -0,351 & -0,352 \\
      \bottomrule
 \end{tabular}
  \caption{Messdaten für Kupfer bei einem konstantem Probenstrom von $\SI{3}{\ampere}$}
  \label{tab:Kupfer_U_H_2}
\end{table}
%\FloatBarrier

\subsubsection{Daten nach Umpolung}

%\floatplacement{table}{htpb}
\begin{table}
 \centering
 \begin{tabular}[width=\textwidth]{S| S[table-format=2.2] S[table-format=1.3] S[table-format=1.3] S[table-format=1.3] S[table-format=1.3] S[table-format=1.3] S[table-format=1.3] S[table-format=1.3] S[table-format=1.3] S[table-format=1.3] S[table-format=1.3]}
     \midrule
      $I_{\text{\tiny{Probe}}}$ \text{\;in\;} \si{\ampere} & 0 & 0,8 & 1,6 & 2,4 & 3,2 & 4 & 4,8 & 5,6 & 6,4 & 7,2 & 8 \\
      $U$ \text{\;in\;} \si{\milli\volt} & -0,020 & 0,047 & 0,116 & 0,184 & 0,250 & 0,318 & 0,389 & 0,456 & 0,527 & 0,597 & 0,666 \\
      \bottomrule
\end{tabular}
  \caption{Messdaten für Zink bei einem konstantem Spulenstrom von $\SI{5}{\ampere}$}
  \label{tab:Zink_U_H_2_umgepolt}
\end{table}


%\floatplacement{table}{htpb}
\begin{table}
 \centering
 \begin{tabular}[width=\textwidth]{S| S[table-format=2.3] S[table-format=1.3] S[table-format=1.3] S[table-format=1.3] S[table-format=1.3] S[table-format=1.3] S[table-format=1.3] S[table-format=1.3] S[table-format=1.3] S[table-format=1.3] S[table-format=1.3]}
     \midrule
      $I_{\text{\tiny{Probe}}}$ \text{\;in\;} \si{\ampere} & 0 & 1 & 2 & 3 & 4 & 5 & 6 & 7 & 8 & 9 & 10 \\
      $U$ \text{\;in\;} \si{\milli\volt} & - 0,338  & - 0,337 & - 0,336 & - 0,335 & - 0,335 & - 0,334 & - 0,333 & - 0,332 & - 0,332 & - 0,332 & - 0,330 \\
      \bottomrule
\end{tabular}
  \caption{Messdaten für Kupfer bei einem konstantem Probenstrom von $\SI{3}{\ampere}$}
  \label{tab:Kupfer_U_H_2_umgepolt}
\end{table}


\newpage

\section{Auswertung}

\subsection{Hystereseeffekt}

In diesem Abschnitt wird der in dem Versuch auftretende Hystereseeffekt untersucht.
Dazu wird die gemessene B-Feldstärke gegenüber der Stromstärke aufgetragen. Dabei
wird einmal der Strom von \SI{0}{\ampere} bis auf \SI{5}{\ampere} aufgedreht und zum
anderen der Strom von \SI{5}{\ampere} auf \SI{0}{\ampere} runtergedeht.
Es wurden jeweils zehn Messungen erhoben. Die Messergebnisse sind in dem folgendem
Diagramm visualisiert.

\begin{figure}
  \includegraphics[width=\textwidth]{Hysterese.pdf}
  \caption{Der auftretende Hystereseeffekt}
  \label{fig:Hysterese}
\end{figure}

In dem Diagramm wird deutlich, dass sich die Verläufe der $B$-Feldstärke bei
unterschiedlich geregelter Stromstärke kaum unterscheiden. Daran ist ersichtlich,
dass der Hystereseeffekt bei der Auswertung der Messergebnisse nur einen vernachlässigbaren
Einfluss besitzt.

Bei den im Versuch angestellten Messungen wurde stets die Stromstärke hochgeregelt,
sodass die $B$-Feldstärke gegenüber des aufgedrehten Stroms verwendet wird, um
den Proportionalitätsfaktor zwischen der Stromstärke $I$ und $B$ zu ermitteln.
Der lineare Fit ist in dem folgendem Diagramm dargestellt.

\begin{figure}
  \includegraphics[width=\textwidth]{lineareRegression.pdf}
  \caption{'Lineare Regression an die $B$-Feldstärke bei aufsteigendem $I$'}
  \label{fig:lineareRegression}
\end{figure}

Als Proportionalitätsfaktor zwischen $I$ und $B$ ergibt sich somit $B =
253,35 * I$. Der Proportionalitätsfaktor wird als fehlerfrei angenommen.

\subsection{Messung der Widerstände}

Die Widerstände lassen sich über die Messergebnisse der Spannung bei variierender
Stromstärke errechnen. Das Ohmsche Gesetz besagt, dass der Widerstand der
Proportionalitätsfaktor zwischen der Spannung und der Stromstärke ist.
In den folgenden Diagrammen ist die Spannung gegenüber der Stromstärke aufgetragen.
Die Regressionsgerade wurde direkt in Diagramme integriert.

\begin{figure}
  \includegraphics[width=\textwidth]{Widerstandsmessung.pdf}
  \caption{Diagramme der Widerstandsmessung}
  \label{fig:Widerstände}
\end{figure}

Es ergibt sich für die Zinkprobe ein gemessener Widerstand von $R_Z =
\SI{13,32\pm0,067}{\milli\ohm}$. Für die Kupferprobe ergibt sich ein gemesserner
Widerstand von $R_K = \SI{7,64\pm0,016}{\milli\ohm}$.

\subsection{Bestimmen der Hall--Spannung $U_H$}

Die Hall--Spannung wurde nun unter variierenden Bedingungen gemessen.
Bei der ersten Messung wurde der Probenstrom konstant gelassen und der
Spulenstrom aufgedreht, wodurch die $B$-Feldstärke erhöht wird. Bei der zweiten
Messung wurde die vorgehensweise umgekehrt. Der Spulenstrom wurde konstant gelassen
und der Probenstrom wurde aufgedreht.
Die Messdaten der ersten Messung sind aus den Tabellen \ref{tab:Zink_U_H},
\ref{tab:Kupfer_U_H}, \ref{tab:Zink_U_H_umgepolt} und \ref{tab:Kupfer_U_H_umgepolt}
zu entnehmen.
Die Messdaten der zweiten Messung sind in den Tabellen \ref{tab:Zink_U_H_2},
\ref{tab:Kupfer_U_H_2}, \ref{tab:Zink_U_H_2_umgepolt} und \ref{tab:Kupfer_U_H_2_umgepolt}
dargestellt.
Die folgenden Diagramme visualisieren die gemessenen Daten.

\begin{figure}
  \includegraphics[width=\textwidth]{Hall_Spannung_gegenueber_I_s.pdf}
  \label{fig:U_H_const_Ip}
\end{figure}

\begin{figure}
  \includegraphics[width=\textwidth]{Hall_Spannung_gegenueber_I_p.pdf}
  \label{fig:U_H_const_Is}
\end{figure}
\FloatBarrier

Über den zuvor bestimmten Proportionalitätsfaktor kann die Hall-Spannung
der $B$-Feldstärke gegenüber aufgetragen werden. Es ergeben sich damit die
folgenden Diagramme.

\begin{figure}
  \includegraphics[width=\textwidth]{Hall_Spannung_gegenueber_B_s.pdf}
  \label{fig:U_H_const_Ip_B}
\end{figure}

\subsection{Bestimmen mikroskopischer Leitfähigkeitsparamter}

In diesem Abschnitt werden mit Hilfe der Messergebnisse für die Hall-Spannung und die
Widerstände der Proben mikroskopische Leitfähigkeitsparameter bestimmt.
Zuerst wurde die Ladungsträgeranzahl pro Volumen ermittelt.
Diese ergibt sich aus umstellen der Formel \eqref{eqn:Hall_Spannung} nach $n$ zu:

\begin{equation}
  n = -\frac{B\cdot I_q}{e_0 U_H d}.
\end{equation}

Für die Messungen bei konstantem Probenstrom ergeben sich die Werte

\begin{align*}
  n_{Zink} &= (\num{8,30\pm 0,34})\cdot 10^{27}\frac{1}{\si{\meter^3}}\\
  n_{Kupfer} &= (\num{2,278\pm 0,30})\cdot 10^{29}\frac{1}{\si{\meter^3}}.
\end{align*}

Für die Messungen bei konstantem Spulenstrom ergeben sich die Werte

\begin{align*}
  n_{{Zink}} &= (\num{7,14\pm 0,34}) \cdot 10^{27}\frac{1}{\si{\meter^3}}\\
  n_{{Kupfer}} &= (\num{2,25\pm 0,07})\cdot10^{29}\frac{1}{\si{\meter^3}}.
\end{align*}

Die Zahl der Ladungsträger pro Atom $z$ lässt sich über die folgende Formel bestimmen

\begin{equation}
  z = n \cdot V
\end{equation}
,wobei $n$ die Ladungsträgerdichte pro Volumen ist und $V$ das Molarevolumen ist.

Es ergeben sich für die Zahl der Ladungsträger pro Atom bei konstantem
Querstrom die folgenden Werte

\begin{align*}
  z_{Zink} &= \SI{0,0502\pm0,0021}{\mol\per\meter^3}\\
  z_{Kupfer} &= \SI{2,680\pm0,35}{\mol\per\meter^3}.
\end{align*}

Bei konstentem Spulenstrom ergeben sich

\begin{align*}
  z_{Zink} &= \SI{0,0432\pm0,0021}{\mol\per\meter^3}\\
  z_{Kupfer} &= \SI{2,64\pm0,08}{\mol\per\meter^3}.
\end{align*}

Nach Bestimmen der Ladungsträgerdichte kann die mittlere Flugzeit $\bar{\tau}$
über die Formel \eqref{eqn:mittlereZeit} ermittelt werden.
Aus den Messungen ergeben sich die folgenden Werte. Für die Ladungsträgerdichte
bei konstantem Querstrom ergeben sich

\begin{align*}
\bar{\tau}_{Zink} &= (\num{2,53\pm 0,10})\cdot 10^{-13}\si{\second}\\
\bar{\tau}_{Kupfer} &= (\num{2,046\pm 0.027})\cdot 10^{-13}\si{\second}.
\end{align*}

Für die Ladungsträgerdichte bei konstantem Spulenstrom ergeben sich

\begin{align*}
\bar{\tau}_{Zink} &= (\num{2,94\pm 0,14})\cdot 10^{-13}\si{\second}\\
\bar{\tau}_{Kupfer} &= (\num{2,07\pm 0,06})\cdot 10^{-13}\si{\second}.
\end{align*}

Die Driftgeschwindigkeit lässt sich durch den folgenden Zusammenhang
berechen.
\begin{equation}
  v_d = - \frac{j}{n * e_0}
\end{equation}
Dabei ist $j$ die Stromdichte und $n$ die Ladungsträgerdichte pro Volumen.
Für die Driftgeschwindigkeit bei konstatem Querstrom ergeben sich

\begin{align*}
  v_{d,Zink} &= (\num{7,52\pm0,31})\cdot 10^{-4}\si{\meter\per\second}\\
  v_{d,Kupfer} &= (\num{2,74\pm0,04})\cdot 10^{-5}\si{\meter\per\second}.
\end{align*}

Für die Driftgeschwindigkeit bei konstantem Spulenstrom ergeben sich

\begin{align*}
  v_{d,Zink} &= (\num{8,7\pm0,4})\cdot 10^{-4}\si{\meter\per\second}\\
  v_{d,Kupfer} &= (\num{2,78\pm0,08})\cdot 10^{-5}\si{\meter\per\second}.
\end{align*}

Damit die Totalegeschwindigkeit $|\bar{v}|$ und die mittlere freie Weglänge
$\bar{l}$ ermittelt werden kann muss die
Fermie-Energie der Proben bestimmt werden. Es gelten die folgenden Beziehungen.

\begin{align}
  \label{eqn:Fermi_E}
  E_F &= \frac{h^2}{2m_0}\sqrt[3]{\left(\frac{3}{8\pi}n\right)^2}\\
  \label{eqn:Totalgesch}
  |\bar{v}| &\approx \sqrt{\frac{2E_F}{m_0}}\\
  \label{eqn:l}
  \bar{l} &\approx \bar{\tau}\sqrt{\frac{2E_F}{m_0}}.
\end{align}

Somit lässt sich die Totalgeschwindigkeit allein durch das Bestimmen von $n$
approximieren.

Aus den Messdaten ergeben sich für bei konstantem Querstrom

\begin{align*}
  E_{F,Zink} &= (\num{2,39\pm0,07}) \si{\eV}\\
  E_{F,Kupfer} &= (\num{0,218\pm0,002})\si{\eV}.
\end{align*}

Für die Messungen mit konstantem Spulenstrom ergibt sich

\begin{align*}
  E_{F,Zink} &= (\num{2,17\pm0,07}) \si{\eV}\\
  E_{F,Kupfer} &= (\num{0,216\pm0,004})\si{\eV}.
\end{align*}

Nun kann über die Formel \eqref{eqn:Totalgesch} die Totalgeschwindigkeit errechnet werden. Die folgenden Werte sind bei konstantem Querstrom.

\begin{align*}
  |\bar{v}|_{Zink} &= (1,555\pm 0,021) \cdot 10^{6} \si{\meter\per\second}\\
  |\bar{v}|_{Kupfer} &= (4,692\pm0,020) \cdot 10^6 \si{\meter\per\second}
\end{align*}

Für die Messungen bei konstantem Spulenstrom ergeben sich die folgenden Werte.

\begin{align*}
  |\bar{v}|_{Zink} &= (1,48\pm 0,023) \cdot 10^{6} \si{\meter\per\second}\\
  |\bar{v}|_{Kupfer} &= (4,67\pm0,05) \cdot 10^6 \si{\meter\per\second}
\end{align*}

Desweiteren kann die mittlere freie Weglänge über \eqref{eqn:l} bestimmt werden.
Für die Messdaten bei konstantem Querstrom ergenen sich

\begin{align*}
  \bar{l}_{Zink} &= (\num{0,393\pm0,011})\si{\micro\meter}\\
  \bar{l}_{Kupfer} &= (\num{0,960\pm0,009})\si{\micro\meter}.
\end{align*}

Anhand der Messdaten bei kostantem Spulenstrom ergeben sich die folgenden Werte.

\begin{align*}
  \bar{l}_{Zink} &= (\num{0,435\pm0,014})\si{\micro\meter}\\
  \bar{l}_{Kupfer} &= (\num{0,969\pm0,020})\si{\micro\meter}
\end{align*}

Zwischen $\bar{\tau}$ und der Beweglichkeit $\mu$ besteht der folgende Zusammenhang.

\begin{equation}
  \label{eqn:mu}
  \mu = \frac{e_0}{2m_0}\bar{\tau}
\end{equation}

Darüber ergeben sich bei konstantem Probenstrom die beiliegenden Werte.

\begin{align*}
  \mu_{Zink} &= (\num{0,0222\pm0,0009})\si{\ampere\second^2\per\kilo\gram}\\
  \mu_{Kupfer} &= (\num{0,0180\pm0,00024})\si{\ampere\second^2\per\kilo\gram}
\end{align*}

Bei konstantem Spulenstrom entstehen die Widerstandsmessung
\begin{align*}
  \mu_{Zink} &= (\num{0,0258\pm0,0012})\si{\ampere\second^2\per\kilo\gram}\\
  \mu_{Kupfer} &= (\num{0,0182\pm0,0006})\si{\ampere\second^2\per\kilo\gram}.
\end{align*}

\section{Diskussion}

Abschließend werden die Ergebnisse der Auswertung diskutiert und hingehend ihrer
Aussagekrat bewertet. Zunächst werden mögliche Messungenauigkeiten erläutert.
Die Maße der Proben wurde mittels einer Schieblehre gemessen. Die Proben wiesen
dahingehend Mängel auf, dass ihre Maße nur ungenau bestimmt werden konnten, da sie
durch vorherige Messungen anderer Gruppen schon deformiert waren. Zudem konnte
die Dicke der Proben nur schwierig gemessen werden, da die Proben fixiert waren.
In der Auswertung wurden die Maße als fehlerfrei angenommen, weshalb die Aussagekraft
der ermittelten Größen eingeschränkt wird. Darüberhinaus machte das Voltmeter keinen
zuverlässigen Eindruck, weil es häufig zwischen willkürlich scheinenden Werten
herschwankte. Es musste ein Moment abgepasst werden, bei dem das Voltmeter zuverlässig
schien. Die Messungen konnten aus diesem Grund auch nicht widerholt werden, ohne
völlig verschiedene Resultate zu erhalten. Die Magnetfeldstärke wurde mit Hilfe
einer Hall-Sonde bestimmt werden und wurde ebenfalls als fehlerfrei angenommen.
Dadurch wird die Aussagekraft der erhaltenen Ergebnisse weiter beschnitten.
Anhand der aufgeführten Argumente wird deutlich, dass die Ergebnisse deutlich
durch Messfehler in ihrer Aussagekraft beschränkt sind.\\
Ausgehend von der Tatsache, dass Kupfer ein Elektronenleiter ist, ist die
Probe Zink als Löcherleiter zu bewerten.


\end{document}
