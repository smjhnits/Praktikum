\section{Messergebnisse}

\subsection{Abmessungen der verwendeten Proben}

Bei der Probe Zink wurden folgende Maße genommen.
Für die Vermessung wurde eine Schieblehre verwendet.

\begin{description}
  \item[Höhe] $\SI{0,02603}{\meter}$
  \item[Breite] $\SI{0,04406}{\meter}$
  \item[Dicke] $\SI{0,00043}{\meter}$
\end{description}

Für die Probe Kupfer wurden folgenden Maße genommen.
Die Dicke der Probe wurde angegeben, die restlichen Maße wurden mit einer
Schieblehre genommen.

\begin{description}
  \item[Höhe] $\SI{0,0280}{\meter}$
  \item[Breite] $\SI{0,0253}{\meter}$
  \item[Dicke] $\SI{0,000018}{\meter}$
\end{description}

\subsection{Messung der Feldstärke bei variirendem Strom}

\floatplacement{table}{htpb}
\begin{table}
 \centering
 \label{tab:Messergebnisse_Feldstärke_Isteigt}
 \begin{tabular}[width=\textwidth]{S| S[table-format=1.1] S[table-format=3.1] S[table-format=3.0] S[table-format=3.1] S[table-format=3.0] S[table-format=3.1] S[table-format=3.0] S[table-format=3.1] S[table-format=4.0] S[table-format=4.1] S[table-format=4.0]}
     \midrule
      $I$ \text{\;in\;} \si{\ampere} & 0 & 0,5 & 1 & 1,5 & 2 & 2,5 & 3 & 3,5 & 4 & 4,5 & 5 \\
      $B$ \text{\;in\;} \si{\milli\tesla} & 7,7 & 142 & 272 & 420 & 556 & 700 & 840 & 975 & 1077 & 1158 & 1220 \\
      \bottomrule
\end{tabular}
  \caption{$B$-Feldstärke bei steigender Stromstärke}
\end{table}


\floatplacement{table}{htpb}
\begin{table}
 \centering
 \label{tab:Messergebnisse_Feldstärke_Ifällt}
 \begin{tabular}[width=\textwidth]{S| S[table-format=4.0] S[table-format=4.1] S[table-format=4.0] S[table-format=3.1] S[table-format=3.0] S[table-format=3.1] S[table-format=3.0] S[table-format=3.1] S[table-format=3.0] S[table-format=3.1] S[table-format=3.1]}

     \midrule
      $I$ \text{\;in\;} \si{\ampere} & 5 & 4,5 & 4 & 3,5 & 3 & 2,5 & 2 & 1,5 & 1 & 0,5 & 1 \\
      $B$ \text{\;in\;} \si{\milli\tesla} & 1220 & 1169 & 1095 & 977 & 845 & 703 & 563 & 422 & 279 & 138 & 8,3 \\
      \bottomrule
\end{tabular}
  \caption{$B$-Feldstärke bei fallender Stromstärke}
\end{table}
\FloatBarrier

\subsection{Messdaten für die Bestimmung der Widerstände der Proben}

\floatplacement{table}{htpb}
\begin{table}
 \centering
 \label{tab:Spannung_Zink}
 \begin{tabular}[width=\textwidth]{S| S[table-format=2.2] S[table-format=2.2] S[table-format=2.1] S[table-format=2.1] S[table-format=2.1] S[table-format=2.1] S[table-format=2.1] S[table-format=2.1] S[table-format=3.1] S[table-format=3.1] S[table-format=3.1]}

     \midrule
      $I$ \text{\;in\;} \si{\ampere} & 0 & 1 & 2 & 3 & 4 & 5 & 6 & 7 & 8 & 9 & 10 \\
      $U$ \text{\;in\;} \si{\milli\volt} & -0,02 & 14,13 & 27,7 & 41,1 & 55,5 & 68,3 & 81,5 & 94,7 & 107,1 & 120,3 & 133,7 \\
      \bottomrule
\end{tabular}
  \caption{Messdaten für die Probe Zink}
\end{table}


\floatplacement{table}{htpb}
\begin{table}
 \centering
 \label{tab:Spannung_Kupfer}
 \begin{tabular}[width=\textwidth]{S| S[table-format=1.2] S[table-format=2.2] S[table-format=2.1] S[table-format=2.1] S[table-format=2.1] S[table-format=2.1] S[table-format=2.1] S[table-format=2.1] S[table-format=3.1] S[table-format=3.1] S[table-format=3.1]}

     \midrule
      $I$ \text{\;in\;} \si{\ampere} & 0 & 1 & 2 & 3 & 4 & 5 & 6 & 7 & 8 & 9 & 10 \\
      $U$ \text{\;in\;} \si{\milli\volt} & 0 & 7,83 & 15,54 & 23,3 & 30,9 & 38,6 & 46,3 & 53,9 & 61,5 & 68,8 & 76,5\\
      \bottomrule
\end{tabular}
  \caption{Messdaten für die Probe Kupfer}
\end{table}
\FloatBarrier

\subsection{Messdaten für die gemessene Hall--Spannung bei konstantem Probenstrom}

\floatplacement{table}{htpb}
\begin{table}
 \centering
 \label{tab:Zink_U_H}
 \begin{tabular}[width=\textwidth]{S| S[table-format=1.3] S[table-format=1.3] S[table-format=1.3] S[table-format=1.3] S[table-format=1.3] S[table-format=1.3] S[table-format=1.3] S[table-format=1.3] S[table-format=1.3] S[table-format=1.3] S[table-format=1.3]}

     \midrule
      $I_{\text{\tiny{Spule}}}$ \text{\;in\;} \si{\ampere} & 0 & 0,5 & 1 & 1,5 & 2 & 2,5 & 3 & 3,5 & 4 & 4,5 & 5 \\
      $U$ \text{\;in\;} \si{\milli\volt} & 0,644 & 0,648 & 0,651 & 0,654 & 0,657 & 0,659 & 0,661 & 0,663 & 0,664 & 0,665 & 0,666\\
      \bottomrule
\end{tabular}
  \caption{Messdaten für Zink bei einem konstantem Probenstrom von $\SI{8}{\ampere}$}
\end{table}


\floatplacement{table}{htpb}
\begin{table}
 \centering
 \label{tab:Kupfer_U_H}
 \begin{tabular}[width=\textwidth]{S| S[table-format=2.3] S[table-format=1.3] S[table-format=1.3] S[table-format=1.3] S[table-format=1.3] S[table-format=1.3] S[table-format=1.3] S[table-format=1.3]}

     \midrule
      $I_{\text{\tiny{Spule}}}$ \text{\;in\;} \si{\ampere} & 0 & 0,5 & 1 & 1,5 & 2 & 2,5 & 3 & 3,5 \\
      $U$ \text{\;in\;} \si{\milli\volt} & -0,342 & -0,340 & -0,338 & -0,336 & -0,334 & -0,332 & -0,330 & -0,328 \\
      \bottomrule
\end{tabular}
  \caption{Messdaten für Kupfer bei einem konstantem Probenstrom von $\SI{10}{\ampere}$}
\end{table}
\FloatBarrier

\subsubsection{Daten nach Umpolung}

\floatplacement{table}{htpb}
\begin{table}
 \centering
 \label{tab:Zink_U_H_umgepolt}
 \begin{tabular}[width=\textwidth]{S| S[table-format=1.3] S[table-format=1.3] S[table-format=1.3] S[table-format=1.3] S[table-format=1.3] S[table-format=1.3] S[table-format=1.3] S[table-format=1.3] S[table-format=1.3] S[table-format=1.3] S[table-format=1.3]}

     \midrule
      $I_{\text{\tiny{Spule}}}$ \text{\;in\;} \si{\ampere} & 0 & 0,5 & 1 & 1,5 & 2 & 2,5 & 3 & 3,5 & 4 & 4,5 & 5 \\
      $U$ \text{\;in\;} \si{\milli\volt} & 0,647 & 0,646 & 0,645 & 0,644 & 0,642 & 0,641 & 0,639 & 0,638 & 0,636 & 0,635 & 0,634\\
      \bottomrule
\end{tabular}
  \caption{Messdaten für Zink bei einem konstantem Probenstrom von $\SI{8}{\ampere}$}
\end{table}


\floatplacement{table}{htpb}
\begin{table}
 \centering
 \label{tab:Kupfer_U_H_umgepolt}
 \begin{tabular}[width=\textwidth]{S| S[table-format=2.3] S[table-format=1.3] S[table-format=1.3] S[table-format=1.3] S[table-format=1.3] S[table-format=1.3] S[table-format=1.3] S[table-format=1.3]}

     \midrule
      $I_{\text{\tiny{Spule}}}$ \text{\;in\;} \si{\ampere} & 0 & 0,5 & 1 & 1,5 & 2 & 2,5 & 3 & 3,5\\
      $U$ \text{\;in\;} \si{\milli\volt} & -0,340 & -0,342 & -0,343 & -0,345 & -0,347 & -0,349 & -0,351 & -0,353 \\
      \bottomrule
\end{tabular}
  \caption{Messdaten für Kupfer bei einem konstantem Probenstrom von $\SI{10}{\ampere}$}
\end{table}
\FloatBarrier

\subsection{Messdaten für die gemessene Hall-Spannung bei konstantem Spulenstrom}

\floatplacement{table}{htpb}
\begin{table}
 \centering
 \label{tab:Zink_U_H_2}
 \begin{tabular}[width=\textwidth]{S| S[table-format=2.3] S[table-format=1.3] S[table-format=1.3] S[table-format=1.3] S[table-format=1.3] S[table-format=1.3] S[table-format=1.3] S[table-format=1.3] S[table-format=1.3] S[table-format=1.3]
 S[table-format=1.3]}

     \midrule
      $I_{\text{\tiny{Probe}}}$ \text{\;in\;} \si{\ampere} & 0 & 0,8 & 1,6 & 2,4 & 3,2 & 4 & 4,8 & 5,6 & 6,4 & 7,2 & 8 \\
      $U$ \text{\;in\;} \si{\milli\volt} & -0,020 & 0,045 & 0,109 & 0,174 & 0,234 & 0,304 & 0,365 & 0,431 & 0,495 & 0,560 & 0,626 \\
      \bottomrule
\end{tabular}
  \caption{Messdaten für Zink bei einem konstantem Spulenstrom von $\SI{5}{\ampere}$}
\end{table}


\floatplacement{table}{htpb}
\begin{table}
 \centering
 \label{tab:Kupfer_U_H_2}
 \begin{tabular}[width=\textwidth]{S| S[table-format=2.3] S[table-format=1.3] S[table-format=1.3] S[table-format=1.3] S[table-format=1.3] S[table-format=1.3] S[table-format=1.3] S[table-format=1.3] S[table-format=1.3] S[table-format=1.3] S[table-format=1.3]}

     \midrule
      $I_{\text{\tiny{Probe}}}$ \text{\;in\;} \si{\ampere} & 0 & 1 & 2 & 3 & 4 & 5 & 6 & 7 & 8 & 9 & 10 \\
      $U$ \text{\;in\;} \si{\milli\volt} & - 0,336 & - 0,338 & - 0,340 & - 0,342 & - 0,343 & - 0,345 & - 0,347 & - 0,348 & -0,350 & -0,351 & -0,352 \\
      \bottomrule
 \end{tabular}
  \caption{Messdaten für Kupfer bei einem konstantem Probenstrom von $\SI{3}{\ampere}$}
\end{table}
\FloatBarrier

\subsubsection{Daten nach Umpolung}

\floatplacement{table}{htpb}
\begin{table}
 \centering
 \label{tab:Zink_U_H_2_umgepolt}
 \begin{tabular}[width=\textwidth]{S| S[table-format=2.2] S[table-format=1.3] S[table-format=1.3] S[table-format=1.3] S[table-format=1.3] S[table-format=1.3] S[table-format=1.3] S[table-format=1.3] S[table-format=1.3] S[table-format=1.3] S[table-format=1.3]}
     \midrule
      $I_{\text{\tiny{Probe}}}$ \text{\;in\;} \si{\ampere} & 0 & 0,8 & 1,6 & 2,4 & 3,2 & 4 & 4,8 & 5,6 & 6,4 & 7,2 & 8 \\
      $U$ \text{\;in\;} \si{\milli\volt} & -0,020 & 0,047 & 0,116 & 0,184 & 0,250 & 0,318 & 0,389 & 0,456 & 0,527 & 0,597 & 0,666 \\
      \bottomrule
\end{tabular}
  \caption{Messdaten für Zink bei einem konstantem Spulenstrom von $\SI{5}{\ampere}$}
\end{table}


\floatplacement{table}{htpb}
\begin{table}
 \centering
 \label{tab:Kupfer_U_H_2_umgepolt}
 \begin{tabular}[width=\textwidth]{S| S[table-format=2.3] S[table-format=1.3] S[table-format=1.3] S[table-format=1.3] S[table-format=1.3] S[table-format=1.3] S[table-format=1.3] S[table-format=1.3] S[table-format=1.3] S[table-format=1.3] S[table-format=1.3]}
     \midrule
      $I_{\text{\tiny{Probe}}}$ \text{\;in\;} \si{\ampere} & 0 & 1 & 2 & 3 & 4 & 5 & 6 & 7 & 8 & 9 & 10 \\
      $U$ \text{\;in\;} \si{\milli\volt} & - 0,338  & - 0,337 & - 0,336 & - 0,335 & - 0,335 & - 0,334 & - 0,333 & - 0,332 & - 0,332 & - 0,332 & - 0,330 \\
      \bottomrule
\end{tabular}
  \caption{Messdaten für Kupfer bei einem konstantem Probenstrom von $\SI{3}{\ampere}$}
\end{table}
