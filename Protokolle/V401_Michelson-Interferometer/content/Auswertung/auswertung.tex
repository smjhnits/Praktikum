\section{Auswertung}

Bei den folgenden Rechnungen werden die Fehler mit der Gaußschen Fehlerfortpflanzung
berechnet, wobei hierfür die automatische Fehlerrechnung mittels des Ufloat-Paketes
von Python genutzt wird.
Bei den Rechnungen werden die folgenden Konstanten für die Normalbedingungen, die
Umgebungstemperatur und die Länge des Gasbehälters ($\su{b}$), sowie die Übersetzung ($\su{\ddot{U}}$)
verwendet:

\begin{align*}
  \su{T}\ua{0} &= \SI{273.15}{K} \\
  \su{p}\ua{0} &= \SI{1.0132}{bar} \\
  \su{T}       &= \SI{293.15}{K} \\
  \su{b}       &= \SI{0.05}{m} \\
  \su{\ddot{U}}       &= 5.046
\end{align*}

\subsection{Bestimmung der Wellenlänge}

Die Wellenlängen ergeben sich dabei aus den gemessenen Werten mit der Formel \eqref{eqn:lambda},
wobei die gemessenen Werte als fehlerfrei angenommen werden. Die berechneten Werte
sind in der folgenden Tabelle eingetragen:

\begin{table}
  \centering
  \caption{Gemessene Intensitätsmaxima ($\su{z}$) und die daraus resultierende Wellenlänge.}
  \label{tab:Wellenlängen}
  \begin{tabular}{c c c}
    \toprule
    $\increment d$ in $\si{mm}$ & $\su{z}$ & $\lambda$ in $\si{nm}$ \\
    \midrule
    0.355 & 1051 & 675.05 \\
    0.339 & 1002 & 676.41 \\
    0.351 & 1044 & 671.98 \\
    0.349 & 1036 & 673.34 \\
    0.343 & 1013 & 676.89 \\
    0.343 & 1031 & 665.07 \\
    0.345 & 1027 & 671.52 \\
    0.347 & 1034 & 670.81 \\
    0.347 & 1029 & 674.07 \\
    0.343 & 1022 & 670.93 \\
    \bottomrule
  \end{tabular}
\end{table}

Durch das Mitteln der bestimmten Wellenlängen (Tabelle \ref{tab:Wellenlängen})
ergibt sich für die Wellenlänge des verwendeten Lasers ein Wert von $\lambda \,
= \, (672.6 \pm 1.1) \, \si{nm}$, wobei es sich bei dem Fehler um die
Standardabweichung handelt.

\subsection{Bestimmung des Brechungsindexes von Luft und Kohlenstoffdioxid}

Mit den anfangs angegebenen Werten und den Formeln \eqref{eqn:n_Druck} und \eqref{eqn:delta_n} lässt sich aus der
gemessenen Anzahl an Intensitätsmaxima ($\su{z}$) und der Druckdifferenz ($\increment
\su{p}$) der Brechungsindex von Luft sowie von Kohlenstoffdioxid bestimmen. Bei
$\increment \su{n}$ handelt es sich hier nicht um den Fehler sondern die Differenz des
Brechungsindexes, der Fehler wird durch $\sigma\ua{n}$ angegeben.

\begin{table}
  \centering
  \caption{Gemessene Intensitätsmaxima ($\su{z}$) und der daraus resultierende Brechungsindex für Luft.}
  \label{tab:IndexLuft}
  \begin{tabular}{c c c c c}
    \toprule
    $\increment \su{p}$ in $\si{bar}$ & $\su{z}$ & $\increment \su{n}$ & $\su{n}$ & $\sigma\ua{n} * 10^{6}$ \\
    \midrule
    0.76 & 35 & 0.000235 & 1.000337 & 0.543 \\
    0.8  & 34 & 0.000229 & 1.000311 & 0.501 \\
    0.8  & 34 & 0.000229 & 1.000311 & 0.501 \\
    0.8  & 35 & 0.000235 & 1.000320 & 0.516 \\
    0.82 & 35 & 0.000235 & 1.000312 & 0.503 \\
    0.83 & 35 & 0.000235 & 1.000308 & 0.497 \\
    0.8  & 34 & 0.000229 & 1.000311 & 0.501 \\
    0.81 & 35 & 0.000235 & 1.000316 & 0.509 \\
    0.82 & 35 & 0.000235 & 1.000312 & 0.503 \\
    0.8  & 33 & 0.000222 & 1.000302 & 0.486 \\
    \bottomrule
  \end{tabular}
\end{table}

Die Fehler ergeben sich mittels Gauß´scher Fehlerfortpflanzung durch folgende
Formel:

\begin{align}
  \label{eqn:FehlerDelN}
  \sigma\ua{\increment n} &= \frac{1}{2} \frac{z}{b} \cdot \sigma\ua{\lambda} \\
  \label{eqn:FehlerN}
  \sigma\ua{n}            &= \frac{T}{T\ua{0}} \frac{p\ua{0}}{\increment p} \cdot \sigma\ua{\increment n}
\end{align}

Mit den Werten aus Tabelle \ref{tab:IndexLuft} ergibt sich für den Brechungsindex
von Luft folgender Wert:

\begin{equation*}
  \su{n}\ua{Luft} = (1.000314 \pm 0.000001)
\end{equation*}

Äquivalent dazu lässt sich auch der Brechungsindex von Kohlenstoffdioxid berechnen.
Die gemessenen Werte sowie der jeweilige Brechungsindex sind in folgender Tabelle
eingetragen:

\begin{table}
  \centering
  \caption{Gemessene Intensitätsmaxima ($\su{z}$) und der daraus resultierende Brechungsindex für Kohlenstoffdioxid.}
  \label{tab:IndexLuft}
  \begin{tabular}{c c c c c}
    \toprule
    $\increment \su{p}$ in $\si{bar}$ & $\su{z}$ & $\increment \su{n}$ & $\su{n}$ & $\sigma\ua{n} * 10^{6}$ \\
    \midrule
    0.8  & 58 & 0.000390 & 1.000530 & 0.855 \\
    0.8  & 50 & 0.000336 & 1.000457 & 0.737 \\
    0.64 & 36 & 0.000242 & 1.000411 & 0.663 \\
    0.6  & 39 & 0.000262 & 1.000475 & 0.766 \\
    0.56 & 37 & 0.000249 & 1.000483 & 0.779 \\
    0.52 & 35 & 0.000235 & 1.000492 & 0.794 \\
    0.48 & 29 & 0.000195 & 1.000442 & 0.712 \\
    0.44 & 28 & 0.000188 & 1.000465 & 0.750 \\
    0.4  & 25 & 0.000168 & 1.000457 & 0.737 \\
    0.37 & 24 & 0.000161 & 1.000474 & 0.765 \\
    \bottomrule
  \end{tabular}
\end{table}

Die Fehler werden dabei wieder nach Formel \eqref{eqn:FehlerDelN} und \eqref{eqn:FehlerN}
berechnet. Für den Brechungsindex von Kohlenstoffdioxid ergibt sich damit der
folgende Wert:

\begin{equation*}
  \su{n}_{\ce{CO_2}} = (1.000469 \pm 0.000001)
\end{equation*}

\section{Diskussion}

Bei Vergleich der gemessenen Werte mit den Literaturwerten (Tabelle \ref{tab:Vergleich})
wird erkenntlich, dass die bestimmten Größen in der richtigen Größenordnung liegen
und auch nicht zu stark von den Literaturwerten abweichen. Jedoch liegt keiner
dieser Werte im Fehlerintervall des experimentell bestimmten Wertes.

\begin{table}
  \centering
  \caption{Vergleich der experimentell bestimmten Werte mit den Litearturwerten.}
  \label{tab:Vergleich}
  \begin{tabular}{c c c}
    \toprule
    & Experimenteller Wert  & Literaturwert \\
    \midrule
    $\lambda$ in in $\si{nm}$ & 672.6 \pm 1.1         & 632.8         \\
    $\su{n}\ua{Luft}$              & 1.000314 \pm 0.000001 & 1.000292      \\
    $\su{n}_{\ce{CO_2}}$          & 1.000469 \pm 0.000001 & 1.000449      \\
    \bottomrule
  \end{tabular}
\end{table}

Die auftretenden
Abweichungen lassen sich auf mehrer Ursachen zurückführen. Einerseits war die
verwendete Messdiode so empfindlich, dass schon bei kleinen Erschütterungen am
Messaufbau Interferenzmaxima registriert wurden. Ein weitere Fehlerquelle kann
auch das Ablesen der Wegstrecke der Spiegelverschiebung an der Mikrometerschraube
sein, da die nur in gewissem Maße genau ist.

Vor allem bei der Bestimmung des Brechungsindexes von Kohlenstoffdioxid ist die
Gaskammer eine Fehlerquelle. Es befand sich immer ein kleiner Restanteil an Luft
in der Kammer, so dass die gemessenen Werte nicht für reines Kohlenstoffdioxid
gelten. Um diesen Fehler so gering wie möglich zu halten, wurden jedoch mehrere
Messungen gemacht.

Da die bestimmten Werte jedoch alle den Literaturwerten ähneln, sind die auftretenden
Fehler vermutlich ausschließlich statistischer Natur.
