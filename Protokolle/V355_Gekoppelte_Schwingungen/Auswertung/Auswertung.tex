\input{"../../Praeambel_prak.tex"}

\begin{document}

\title{Versuch 355}
\subtitle{Gekoppelte Pendel}
\author{Sebastian Pape\\
        sepa@gmx.de \and
        Jonah Nitschke\\
        lejonah@web.de}
\date{Durchführung: 10.01.2017\\
      Abgabe: 17.01.2017}

\maketitle

\section{Auswertung}

Bei den Messungen wurden folgende Komponenten verwendet, die bei den anschließenden
Berechnungen als fehlerfrei betrachtet werden:

\begin{align}
  L        &= 0.03251              \, \su{H}       \\
  C        &= 8.01 \cdot 10^{-10}  \, \su{F}       \\
  C\ua{Sp} &= 3.7 \dot 10^{-10}    \, \su{F}       \\
  R        &= 48                   \, \su{\Omega}
\end{align}

\subsection{Justierung der Schwingkreise}

Vor den Messungen wurden die beiden Schwingkreise so justiert, dass sie die
selbe Resonanzfrequenz besitzen. Bei der Messung ergab sich als Resonanzfrequenz
für den linken Schwingkreis folgender Wert:

\begin{equation}
  \nu^+ = 31.10 \su{kHz}.
\end{equation}

Mithilfe der Formel ?? kann der Wert für $\nu_t^{+}$ ebenfalls bestimmt werden,
sodass sich für die beiden Frequenzen eine Abweichung von ca. 2 \% ergibt. Der
berechnete Wert für $\nu_t^{+}$ kann Tabelle \ref{tab:Messungb} entnommen werden.

\subsection{Bestimmung des Verhältnisses zwischen Schwingung und Schwebung}


Mithilfe der Formeln ?? und ?? werden ebenfalls die theoretischen Frequenzen
$\nu_t^{+}$ und $\nu_t^{-}$ aus den oben angegebenen Bauteilen bestimmt und mit
folgender Formel das Verhältniss bestimmt:

\begin{equation}
  n_t = \frac{\nu_t^{+} + \nu_t^{-}}{2(\nu_t^{-} - \nu_t^{+})}.
\end{equation}

In Tabelle \ref{tab:Messunga} sind die bestimmten Werte sowie die Abweichungen von
experimentell bestimmten Werten zu den theoretisch berechneten Werten zu sehen.
Die Abweichung berechnet sich dabei nach folgender Formel:

\begin{equation}
  a = \frac{\increment n}{n} = \frac{|n-n\ua{t}|}{n}.
  \label{eqn:Abweichungen}
\end{equation}

\begin{table}
  \centering
  \begin{tabular}{c | c | c | c | c | c | c | c}
  \toprule $C\ua{k}$ in $\su{nF}$ & $\sigma\ua{C\ua{k}}$ in $\su{nF}$ & $n$
           & $\nu_t^{+}$ in $\su{kHz}$ & $\nu_t^{-}$ in $\su{kHz}$
           & $\sigma\ua{\nu_t^{-}}$ in $\su{kHz}$ & $n\ua{t}$ & $a$ in $\su{\%}$ \\
  \midrule
  9.99 & 0.030 & 14 & 30.492 & 32.730 & 0.006 & 14.1 & 0.8 \\
  8.00 & 0.024 & 12 & 30.492 & 33.259 & 0.008 & 11.5 & 4.0 \\
  6.47 & 0.019 & 10 & 30.492 & 33.874 & 0.010 & 9.5  & 5.0 \\
  5.02 & 0.015 & 8  & 30.492 & 34.777 & 0.012 & 7.6  & 5.0 \\
  4.00 & 0.012 & 6  & 30.492 & 35.770 & 0.014 & 6.3  & 5.0 \\
  3.00 & 0.009 & 5  & 30.492 & 37.328 & 0.018 & 5    & 0.8 \\
  2.03 & 0.006 & 4  & 30.492 & 40.093 & 0.025 & 3.7  & 8.0 \\
  1.01 & 0.003 &    & 30.492 & 47.404 & 0.039 & 2.3  &     \\
  \bottomrule
  \end{tabular}
 \caption{Bestimmte Werte für die Fundamentalfrequenzen und Vergleich der
          gemessenen und berechneten Verhältnisse}
 \label{tab:Messunga}
\end{table}

Da bei dem Bestimmen der Anzahl an Maxima der Schwingung in einem Schwebungsbauch
vor allem an den äußeren Rändern die Maxima nicht klar erkennbar sind, wird bei den
gemessenen Werte ein Fehler von $\pm 1$ angegeben.

Für die letzte Kapazitätseinstellung konnte die Anzahl an Maxima nicht eindeutig
bestimmt werden, deswegen kann hier kein Vergleich mit den theoretischen Werten
erfolgen.

Die angegebenen Fehler berechnen sich dabei mit der Gaußschen Fehlerfortpflanzung
und berechnen sich wie folgt:

%hier müssen noch die Fehlerformeln eingefügt werden

\subsection{Bestimmung der Fundamentalfrequenzen mithilfe einer erzwungenen Schwingung}

Mithilfe einer erzwungenen Schwingung und verschiedenen Einstellungen des
Kopplungskondensators können die beiden Fundamentalfrequenzen erneut experimentell
bestimmt werden. In Tabelle \ref{tab:Messungb} sind die gemessenen Frequenzen sowie deren
Abweichungen zu den theoretisch bestimmten Werten zu sehen. Es handelt sich dabei
um die in Tabelle \ref{tab:Messunga} angegebenen Werte.

Die Abweichungen werden dabei analog zu Formel \eqref{eqn:Abweichungen} bestimmt.

\begin{table}
  \centering
  \begin{tabular}{ c | c | c | c | c }
    \toprule $C\ua{k}$ in $\su{nF}$
           & $\nu_t^{+}$ in $\su{kHz}$ & $a\ua{\nu_t^{+}}$ in $\su{\%}$
           & $\nu_t^{+}$ in $\su{kHz}$ & $a\ua{\nu_t^{-}}$ in $\su{\%}$ \\
    \midrule
    9.99 & 30.77 & 0.9 & 33.16 & 1.3 \\
    8.00 & 30.79 & 1.0 & 33.66 & 1.2 \\
    6.47 & 30.80 & 1.0 & 34.25 & 1.1 \\
    5.02 & 30.81 & 1.0 & 35.12 & 1.0 \\
    4.00 & 30.82 & 1.1 & 36.08 & 0.9 \\
    3.00 & 30.83 & 1.1 & 37.60 & 0.8 \\
    2.03 & 30.84 & 1.1 & 40.28 & 0.5 \\
    1.01 & 30.85 & 1.2 & 47.33 & 0.2 \\
    \bottomrule
  \end{tabular}
  \caption{Gemessene Fundamentalfrequenzen und die Abweichungen zu den theoretisch
           berechneten Werten}
  \label{tab:Messungb}
\end{table}











\end{document}
