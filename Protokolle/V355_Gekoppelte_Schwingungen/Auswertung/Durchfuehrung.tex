\section{Durchführung}

\subsection{Justierung der Schwingkreise}

Bevor die ersten Messreihen für den Versuch notiert werden können, müssen die beiden
Schwingkreise justiert werden. Für die Justierung wird die Schaltung aus Abbildung ??
verwendet, bei der der Kopplungskondensator überbrückt wird. Zuerst wird dabei
grob der Bereich für die Resonanzfrequenz eingestellt,
indem die Frequenz gesucht wird, bei der der Strom maximal wird. In einem weiteren
Schritt wird die Messung noch einmal etwas genauer durchgeführt, indem an das
Oszilloskop noch die Spannung des Sinus-Generators eingespeist wird. Auf dem Oszilloskop
wird dann mithilfe des XY-Betriebes und der Betrachtung der Lissajous-Figuren die
Resonanzfrequenz bestimmt, indem die Frequenz so lange verschoben wird, bis auf
dem Bildschirm ein Kreis zu sehen ist.

Danach werden sowohl Sinus-Generator als
auch Oszilloskop an den an den rechten Schiwngkreis angeschlossen. Hier wird die
bei dem linken Schwingkreis verwendete Resonanzfrezquenz an dem Generator eingestellt
und der veränderliche Kondensator so lange angepasst, bis auch hier auf dem Oszilloskop
ein Kreis zu sehen ist.

\subsection{Beobachtung des Energieaustausches}

Für den ersten Teil des Versuches wird die Abbildung ?? verwendet. Statt einem
Sinus-Generator wird nun eine Rechteckspannung an den Schwingkreis angelegt und
die Überbrückung des Kopplungskondensators entfernt. Als nächstes wird mit dem
Oszilloskop der Spannungsabfall an dem $\SI{48}{\ohm}$ Widerstand gemessen.
Die Frequenz wird dabei deutlich runter gedreht, um auf dem Oszilloskop
die Schwebungen und die Schwingungen sichtbar zu machen.
Dann wird für jede Kapazitätseinstellung die Anzahl an Maxima innerhalb einer
Schwebung gemessen, um hinterher das Verhältniss der Frequenzen zu bestimmen.

\subsection{Messung der Fundamentalfrequenzen}

\subsubsection{Methode über Phasenverschiebung}

Damit die Fundamentalfrequenzen $\nu_+$ und $\nu_-$ gemessen werden können, muss
ebenfalls der Aufbau aus Abb. ?? verwendet werden. Der Generator muss auf den
Sinusspannungsbetreib umgestellt werden. Die Fundamentalfrequenzen sind in Abhängigkeit
von dem Kopplungskondensator $C_K$ zubestimmen. Die Generatorspannung wird an das
Osilloskop angeschlossen, sodass die aus dem Aufbau abgeleitete Spannung ihr
gegenüber steht. Dadurch entstehen Lissajous-Figuren. $\nu_+$ ist erreicht, wenn
eine Phase von $\pi$ auf dem Oszilloskop dargestellt ist. $\nu_-$ ist dementsprechend
bei einer Phase von $0$ erreicht.

\subsubsection{Methode über Sweepen}

Eine weitere Möglichkeit für das Bestimmen der Fundamentalfrequenzen geht über die
Sweepmethode.
Sweepen kann an dem Generator eingestellt werden und bedeutet, dass ein voreingestelltes
Spannungsintervall zu einer einstellbaren Zeit von dem Generator durchgelaufen wird.
Es wird der selbe Aufbau wie bei der vorherigen Methode verwendet.
Damit die Fundamentalfrequenzen bestimmt werden können muss die Zeit zwischen den
auftretenden Peaks gemessen werden, sowie die Spannungsrate die pro Zeiteinheit
von dem Generator widergegeben wird.
