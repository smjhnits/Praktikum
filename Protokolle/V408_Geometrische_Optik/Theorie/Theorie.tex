% bei Standalone in documentclass noch:
% \RequirePackage{luatex85}

\documentclass[captions=tableheading, titlepage= firstiscover, parskip = half , bibliography=totoc]{scrartcl}
%paper = a5 für andere optinen
% titlepage= firstiscover
% bibliography=totoc für bibdateien
% parskip=half  Veränderung um Absätze zu verbessern

\usepackage{scrhack} % nach \documentclass
\usepackage[aux]{rerunfilecheck}
\usepackage{polyglossia}
\usepackage[style=numeric, backend=biber]{biblatex} % mit [style = alphabetic oder numeric] nach polyglossia
\addbibresource{lit.bib}
\setmainlanguage{german}

\usepackage[autostyle]{csquotes}
\usepackage{amsmath} % unverzichtbare Mathe-Befehle
\usepackage{amssymb} % viele Mathe-Symbole
\usepackage{mathtools} % Erweiterungen für amsmath
\usepackage{fontspec} % nach amssymb
% muss ins document: \usefonttheme{professionalfonts} % für Beamer Präsentationen
\usepackage{longtable}

\usepackage[
math-style=ISO,    % \
bold-style=ISO,    % |
sans-style=italic, % | ISO-Standard folgen
nabla=upright,     % |
partial=upright,   % /
]{unicode-math} % "Does exactly what it says on the tin."
\setmathfont{Latin Modern Math}
% \setmathfont{Tex Gyre Pagella Math} % alternativ

\usepackage[
% die folgenden 3 nur einschalten bei documenten
locale=DE,
separate-uncertainty=true, % Immer Fehler mit ±
per-mode=symbol-or-fraction, % m/s im Text, sonst \frac
]{siunitx}

% alternativ:
% per-mode=reciprocal, % m s^{-1}
% output-decimal-marker=., % . statt , für Dezimalzahlen

\usepackage[
version=4,
math-greek=default,
text-greek=default,
]{mhchem}

\usepackage[section, below]{placeins}
\usepackage{caption} % Captions schöner machen
\usepackage{graphicx}
\usepackage{grffile}
\usepackage{subcaption}

% \usepackage{showframe} Wenn man die Ramen sehen will

\usepackage{float}
\floatplacement{figure}{htbp}
\floatplacement{table}{htbp}

\usepackage{mhchem} %chemische Symbole Beispiel: \ce{^{227}_{90}Th+}


\usepackage{booktabs}

 \usepackage{microtype}
 \usepackage{xfrac}

 \usepackage{expl3}
 \usepackage{xparse}

 % \ExplSyntaxOn
 % \NewDocumentComman \I {}  %Befehl\I definieren, keine Argumente
 % {
 %    \symup{i}              %Ergebnis von \I
 % }
 % \ExplSyntaxOff

 \usepackage{pdflscape}
 \usepackage{mleftright}

 % Mit dem mathtools-Befehl \DeclarePairedDelimiter können Befehle erzeugen werden,
 % die Symbole um Ausdrücke setzen.
 % \DeclarePairedDelimiter{\abs}{\lvert}{\rvert}
 % \DeclarePairedDelimiter{\norm}{\lVert}{\rVert}
 % in Mathe:
 %\abs{x} \abs*{\frac{1}{x}}
 %\norm{\symbf{y}}

 % Für Physik IV und Quantenmechanik
 \DeclarePairedDelimiter{\bra}{\langle}{\rvert}
 \DeclarePairedDelimiter{\ket}{\lvert}{\rangle}
 % <name> <#arguments> <left> <right> <body>
 \DeclarePairedDelimiterX{\braket}[2]{\langle}{\rangle}{
 #1 \delimsize| #2
 }

\setlength{\delimitershortfall}{-1sp}

 \usepackage{tikz}
 \usepackage{tikz-feynman}

 \usepackage{csvsimple}
 % Tabellen mit \csvautobooktabular{"file"}
 % muss in table umgebung gesetzt werden


% \multicolumn{#Spalten}{Ausrichtung}{Inhalt}

\usepackage{hyperref}
\usepackage{bookmark}
\usepackage[shortcuts]{extdash} %nach hyperref, bookmark

\newcommand{\ua}[1]{_\symup{#1}}
\newcommand{\su}[1]{\symup{#1}}


\begin{document}

\section{Theorie}

\subsection{Zielsetzung}

In dem folgenden Versuch sollen mehrere Gesetzmäßigkeiten der geometrischen Optik,
wie zum Beispiel das Abbildungsgesetz und die Linsengleichung, mit verschiedenen
Versuchsaufbauten überprüft werden. Zudem sollen diese dann genutzt werden, um
für verschiedene Linsen(-Systeme) die Brennweite zu bestimmen, unter anderem auch
noch mit der Methode von Bessel sowie der Methode von Abbe.

\subsection{Grundlagen der geometrischen Optik}

Grundlegend werden in der Optik Linsen aus verschiedenen Materialien betrachtet,
die optisch dichter sind als Licht. Durch diese Eigenschaft werden Lichtstrahlen
bei Durchgang solch eines Mediums nach dem Brechungsgesetz gebrochen. Da die
auftretende Brechung auch Abhängig von der Geometrie des Körpers ist, werden
die Linsen meist in zwei Kategorien eingeteilt. Ein Typus ist die Sammellinse, bei
der parallel einfallende Lichtstrahlen zum Einfallslot hin gebrochen werden.
Dadurch entsteht an dem Punkt, an dem sich die Lichtstrahlen bündeln ein reeles
Bild in der Entfernung b (Bildweite) zur Hauptebene der Linse (siehe Abbildung
\ref{}).

Ein weiterer Typus ist die Streulinse, bei der die Lichtstrahlen vom Einfallslot
weg gebrochen werden. Dadurch entsteht lediglich ein virtuelles Bild, welches
eine negative Bildweite besitzt (siehe Abbildung \ref{}). Aufgrunf dieser
Eigenschaften wird bei Sammellinsen von einer positiven Brennweite und bei
Streulinsen von einer negativen Brennweite gesprochen.

Während bei einer dünnen Linse die Brechung auf lediglich eine Mttelenbene reduziert
werden kann, müssen bei dickeren Linsen zwei Hauptebenen betrachtet werden (siehe
Abbildung \ref{}). Um nun optische Bildrekonstruktionen durchzführen, werden
drei charakteristische Strahlen eingeführt.

\begin{itemize}
  \item \underline{Parallelstrahl}: Der Parallelstrahl läuft von der Gegenstand
  senkrecht auf die Mittel- bzw. Hauptebene zu, wird dort gebrochen und läuft
  dann durch den Brennpunkt.
  \item \underline{Mittelpunktsstrahl}: Der Mittelpunktsstrahl läuft direkt durch
  die Mitte der Linse und läuft durchgehen in die selbe Richtung.
  \item \underline{Brennpunktstrahl}: Der Brennpunktstrahl läuft durch den Brennpunkt
  auf die Mittel- bzw.  Haupteben zu, wird dort gebrochen und läuft dann senkrecht
  zur Mittel- bzw. Hauptebene bis zum Bild.
\end{itemize}

Der Schnittpunkt aller drei Strahlen markiert dann das Abbild. Zwischen den bei der
Bildrekonstruktion betrachteten Größen besteht nach dem Abbildungsgesetz folgender
Zusammenhang:

\begin{equation}
  V = \frac{B}{G} = \frac{b}{g},
  \label{eqn:Abbildungsgesetz}
\end{equation}

wobei V den Abbildungsmaßstab, B und G Bild- bzw. Gegenstandsgröße und b und g die
Bild- bzw. Gegenstandsweite beschreiben. Für dünne Linsen folgt zudem die Linsengleichung,
welche eine Zusammenhang zwischen der Brennweite f sowie b und g herstellt (Formel
\eqref{eqn:Linsengleichung}).

\begin{equation}
\frac{1}{f} = \frac{1}{b} + \frac{1}{g}
\label{eqn:Linsengleichung}
\end{equation}

Bei dickeren Linsen müssen wie vorher bereits erwähnt zwei Hauptebenen betrachtet
werden. Hier werden Brennweite und weitere Größen immer bezüglich einer Hauptebene
angegeben, sodass die Linsengleichung auch hier ihre Gültigkeit behält.

Bei durch Linsen erzeugte Abbildungen kann es aufgrund der Gültigkeit der Formeln
\eqref{eqn:Abbildungsgesetz} und \eqref{eqn:Linsengleichung} für achsennahe Strahlen
zu einem Abbildungsfehler bei der Betrachtung achsenferner Strahlen kommen.
Bei der sphärischen Abberration liege die Brennpunkte beider Strahlentypen nicht
aufeinander, sodass es zu einer Unschärfe kommt. Dieser Fehler kann allerdings
durch eine Irisblende behoben werden. Bei der chromatischen Abberation kommt es
aufgrund der unterschiedlichen Wellenlängen im Lichtspektrum zu einer Dispersion
bei der Brechung verschiedener Farben.

Eine weitere Größe zur Beschreibung von Linsen ist die Brechkraft D, welche als
reziproke Brennweite definiert ist. Bei einem System von mehreren dünnen Linsen
können die einzelnen Brechkräfte einfach summiert werden:

\begin{equation}
  D = \sum_{i}^{N} D\ua{i} = \sum_{i}^{N} \frac{1}{f\ua{i}} .
\end{equation}

Zur Bestimmung der Brennweite einer Linse werden im folgenden mehrere Methoden verwendet.
Erstens kann die Brennweite durch Messung von Bild- und Gegenstandsweite bestimmt
werden, indem die Linsengleichung (Formel \eqref{eqn:Linsengleichung}) genutzt wird.

Des weiteren wird die Methode von Bessel genutzt. Hier wird der Abstand zwischen
Gegenstand und Bild konstant gehalten und es werden zwei verschiedene Positionen
gesucht, bei denen das Bild scharf dargestellt wird (siehe Abbildung \ref{}).
Da es sich um eine symmetrische Linsenstellung handelt, gilt folgende Beziehung:

\begin{equation}
  b\ua{1} = g\ua{2} \,\, \su{und} \,\, b\ua{2} = g\ua{1} .
\end{equation}

Mit den Größen $e = g\ua{1} + b\ua{1} = g\ua{2} + b\ua{2}$ (Abstand zwischen Bild
und Gegenstand) sowie $d = g\ua{1} - b\ua{1} = g\ua{2} - b\ua{2}$ (Abstand zwischen
den beiden Linsenpositionen) lässt sich dann ein Formel für die Brennweite der
Linse herleiten (Formel \eqref{eqn:Bessel}).

\begin{equation}
  f = \frac{e^2 - d^2}{4e}
  \label{eqn:Bessel}
\end{equation}

Eine weitere Methode zur Bestimmung der Brennweite ist die Methode von Abbe. Bei
dieser können Brennweite und Lage der Hauptebenen aus dem Abbildungsmaßstab hergeleitet
werden. Dafür werden bei einer dicken Linse wieder Bild- und Gegenstandsweite in
Bezug zu einem beliebigen Bezugspunkt A gemessen. Hierbei gelten folgende Beziehungen:

\begin{align}
  g` &= g + h  = f \cdot \left( 1 + \frac{1}{V} \right) + h \\
  b` &= b + h` = f \cdot \left( 1 + V \right) + h` .
\end{align}

Bei der Messung muss der Bezugspunkt nicht notwendigerweise zwischen den beiden
Hauptebenen liegen.

\section{Versuchsaufbau und Durchführung}

\subsection{Bestimmung der Brennweite durch Messung der Bild- und Gegenstandsweite}

\subsection{Bestimmung der Brennweite mit der Methode von Bessel}

\subsection{Bestimmung der Brennweite mit der Methode von Abbe}



\end{document}
