\input{"../../Praeambel_doc.tex"}

\ihead{Tu Dortmund}
\chead{Kurzfragen}
\ohead{Tutorium Pape/Nitschke}

\begin{document}

\begin{enumerate}
  \item Wie lauten die drei Newtonschen Axiome?
  \item Wann ist ein Kraftfeld $\vec{F(\vec{r})}$ konservativ? Gib mindestens 3 Kriterien
  an. Ist das Kraftfeld $\vec{F(\vec{r})} = \vec{r}/|\vec{r}|^5$ konservativ?
  \item Welche Beziehung besteht zwischen einem konservativem Kraftfeld und seinem
  Potenzial? Berechne das Potenzial $V(\vec{r})$ für das Kraftfeld $\vec{F(\vec{r})}
  = y\vec{e_x} + x\vec{e_y}$ an.
  \item Welche Scheinkräfte können in einem gleichförmig beschleunigten Bezugssystem
  auftreten? Gib die Formeln mit an.
  \item Du kaufst eine neue Wäscheschleuder. Solltest du lieber die mit 1.5-fachem
  Radius oder die mit 1.5-facher Umdrehungszahl nehmen? Begründe deine Entscheidung.
  \item Wie ist die Standardabweichung einer Stichprobe definiert? Wie ist die
  Standardabweichung des Mittelwertes definiert?
  \item Bei welchem Stoß bleibt die Energie erhalten?
  \item In welche Richtung wirkt die Corioliskraft auf einen Zug, der sich auf
  dem Äquator bewegt?
  \item Wie lautet das Newtonsche Gesetz für Drehbewegungen?
  \item Wie lautem die verschiedenen Lösungen eines gedämpften Oszillators?
  Skizziere alle Lösungen.
  \item Wie lautet die DGL eines harmonischen Oszillators? Gib eine allgemeine
  Lösung mit an.
  \item Was sind Inertialsysteme. Welche Transformation überführt Inertialsysteme
  ineinander?
\end{enumerate}

\end{document}
