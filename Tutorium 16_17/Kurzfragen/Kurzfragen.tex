\input{"../Protokolle/Praeambel_prak.tex"}

\ihead{Tu Dortmund}
\chead{Kurzfragen}
\ohead{Tutorium Pape/Nitschke}

\begin{document}

\begin{enumerate}
  \item Wie lauten die drei Newtonschen Axiome?
  \item Was besagt das Superpositionsprinzip?
  \item Wie lautet die allgemeine Weg-Zeit Funktion für eine unbeschleunigte
  und eine beschleunigte Bewegung?
  \item Wann ist ein Kraftfeld $\vec{F(\vec{r})}$ konservativ? Gib mindestens 3 Kriterien
  an. Ist das Kraftfeld $\vec{F(\vec{r})} = \vec{r}/|\vec{r}|^5$ konservativ?
  \item Welche Beziehung besteht zwischen einem konservativem Kraftfeld und seinem
  Potenzial? Berechne das Potenzial $V(\vec{r})$ für das Kraftfeld $\vec{F(\vec{r})}
  = y\vec{e_x} + x\vec{e_y}$ an.
  \item Welche Scheinkräfte können in einem gleichförmig beschleunigten Bezugssystem
  auftreten? Gib die Formeln mit an.
  \item Du kaufst eine neue Wäscheschleuder. Solltest du lieber die mit 1.5-fachem
  Radius oder die mit 1.5-facher Umdrehungszahl nehmen? Begründe deine Entscheidung.
  \item Wie ist die Standardabweichung einer Stichprobe definiert? Wie ist die
  Standardabweichung des Mittelwertes definiert?
  \item Bei welchem Stoß bleibt die Energie erhalten und bei welchem nicht?
  \item Was bedeutet Energieerhaltung und wann gilt diese? Gib ein Beispiel für
  die beiden Möglichkeite an.
  \item In welche Richtung wirkt die Corioliskraft auf einen Zug, der sich auf
  dem Äquator bewegt?
  \item Du willst dein Raumschiff fotografieren. Nun schweben du aber mit der
  schweren Kamera in der Hand 50 Meter neben dem Raumschiff und weißt nicht, wie
  du zurückkommen sollst. Oder doch?
  \item Wie lautet das Newtonsche Gesetz ($\vec{\dot{p}} = \vec{F}$) für Drehbewegungen?
  \item Wie lauten die verschiedenen Lösungen eines gedämpften Oszillators?
  Skizziere alle Lösungen.
  \item Wie lautet die Kraftgleichung einer Feder mit Federkonstante k,
  und wie lautet die dazugehörige Formel der potentielle Energie?
  \item Wie lautet die DGL eines harmonischen Oszillators? Gib eine allgemeine
  Lösung mit an.
  \item Was sind Inertialsysteme. Welche Transformation überführt Inertialsysteme
  ineinander?
\end{enumerate}

\end{document}
