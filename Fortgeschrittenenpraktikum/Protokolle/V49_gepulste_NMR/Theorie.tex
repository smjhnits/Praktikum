\section{Theorie}

Das Ziel des Versuches ist es die longitudinale bzw. Spin-Gitter-Relaxationszeit, sowie die
transversale bzw. Spin-Spin-Relaxationszeit von bidestilierten Wasser
mit Hilfe der Kernspinresonanz (NMR) zu bestimmen.

\subsection{Magentisierung im thermischen Gleichgewicht}

Durch das anlegen eines Magnetfeldes werden entarteten Kenspinzustände
der Spinquantenzahl \textbf{I} in $2\text{\textbf{I}} + 1$ äquidistante
Unternivaus aufgespalten.
Die aufgespaltenen Niveaus werden durch die Orientierungsquantenzahl $m$
($-\text{\textbf{I}} \leq m \leq \text{\textbf{I}}$) unterschieden.
Benachbarte Energieniveaus haben einen Abstand von $\Delta E = \gamma B_0\hbar$,
wobei $\gamma$ das gyromagnetische Moment und somit das Verhältnis von
Drehimpuls oder Spin zu dem magnetischen Moment $\vec{\mu}$ darstellt.
Thermisch sind die Energieniveaus gemäß der Boltzmann-Verteilung
besetzt. Das Besetzungsverhältnis zweier benachbarter Energiezustände bei Temperatur $T$
hat folglich die Gestalt
\begin{equation}
  \label{eqn:boltzmann}
  \frac{N(m)}{N(m-1)} = \exp{\left(-\beta(T)\gamma B_0\hbar\right)},\qquad\text{mit }\beta(T) = \frac{1}{k\ua{B}T}.
\end{equation}

Der Erwartungswert der Kernspinpolarisation relativ zu dem angelegten Magnetfeld
$\braket{\text{\textbf{I}}\ua{z}}$ wird gemäß

\begin{equation}
  \label{eqn:erwartung_I}
  \braket{\text{\textbf{I}}\ua{z}} = \frac{\sum_{m = -\text{\textbf{I}}}^\text{\textbf{I}}
  \hbar m \exp{\left(-m\beta(T)\gamma B_0 \hbar\right)}}{\sum_{m = -\text{\textbf{I}}}^\text{\textbf{I}}
  \exp{\left(-m\beta(T)\gamma B_0 \hbar\right)}}
\end{equation}

beschrieben. Die folgenden Rechnungen beschränken sich auf die Vereinfachung
$\text{\textbf{I}} = \frac{1}{2}$, sowie große Magnetfelder $B_0 ~ \SI{1}{\tesla}$.
Dadurch gilt $m\beta(T)\gamma B_0\hbar \ll 1$, weshlab eine Linearisierung der
Exponentialfunktion in Formel~\ref{eqn:erwartung_I} gerechtfertigt ist.
Eine Darstellung der Energieniveaus ist in Abbildung~\ref{fig:proton} zu finden.

\begin{figure}
  \centering
  \includegraphics[width = 0.7\textwidth]{Pics/aufspaltungE}
  \caption{Energiezustände eines Protons ($\text{\textbf{I}} = \frac{1}{2}$) im Magnetfeld $B_0$\cite{anleitung}.}
  \label{fig:proton}
\end{figure}

Die Linearisierung von Formel~\ref{eqn:erwartung_I} ergibt
\begin{equation}
  \label{eqn:I_linear}
  \braket{\text{\textbf{I}}_{z} = \frac{1}{2}} = -\frac{\hbar^2}{4}\beta(T)\gamma B_0.
\end{equation}

Die Spinpolarisation führt eine Ausrichtung der mit dem Spin gekoppelten magnetischen
Momente $\vec{\mu}_\text{\textbf{I}}$ mit sich, woher eine makroskopische Magnetisierung
$\vec{M_0}$ rührt.
Liegen $N$ Momente $\vec{\mu}$ in der betrachteten Probe, kann der Betrag der
Magnetisierung $M_0$ geschrieben werden als
\begin{equation}
  \label{eqn:mag}
  M_0 = \frac{1}{4}\mu_0\gamma^2\frac{\hbar^2}N B_0\beta(T).
\end{equation}

\subsection{Lamor-Präzession}
Die magnetischen Momente führen um die Achse des angelegten Magnetfeldes
eine Präzessionsbewegung, welche als Lamor-Präzession bezeichnet wird aus.
Aus der Verbindung zwischen dem Gesamtdrehimpuls und der Magnetisierung
ergibt sich der folgende Zusammenhang.

\begin{equation}
  \label{eqn:mag_diff}
  \frac{\su{d}\vec{M}}{\su{d}t} = \gamma\vec{M}\times B_0\vec{e}_z
\end{equation}

Dabei ist $\vec{e}_z$ der Einheitsrichtungsvektor des Magnetfeldes.
Die Zerlegung von $\vec{M}$ in Einheitsvektoren eines karthesischen Systems
$\vec{e}_x, \vec{e}_y, \vec{e}_z$ führt auf drei Differentialgleichungen.

\begin{align}
  \label{eqn:M_x}
  \frac{\su{d}\vec{M}\cdot \vec{e}_x}{\su{d}t} &= \gamma B_0 \vec{M}\cdot \vec{e}_y, \\
  \label{eqn:M_y}
  \frac{\su{d}\vec{M}\cdot \vec{e}_y}{\su{d}t} &= -\gamma B_0 \vec{M}\cdot \vec{e}_x, \\
  \label{eqn:M_z}
  \frac{\su{d}\vec{M}\cdot \vec{e}_z}{\su{d}t} &= 0
\end{align}

Die Gleichungen~\ref{eqn:M_x}~bis~\ref{eqn:M_z} werden gelöst durch
\begin{align*}
  \vec{M}\cdot \vec{e}_x &= A\cos{(\gamma B_0 t)},\\
  \vec{M}\cdot \vec{e}_y &= -A\sin{(\gamma B_0 t)},\\
  \vec{M}\cdot \vec{e}_z &= \text{const},
\end{align*}
woran erkenntlich wird, dass die Magnetisierung $\vec{M}$ eine Präzession
um $\vec{e}_z$ mit der Lamor-Frequenz $\omega\ua{L} = \gamma B_0$ ausführt.

\subsection{Relaxationserscheinungen}

Relaxation tritt auf, wenn ein System aufgrund einer zeitlich begrenzten
Störung aus dem Gleichgewichtszustand gebracht wird.
In diesem Versuch wird solch eine Störung mit Hilfe hochfreqeunter
Strahlungsquanten realisiert.
Eine Relaxation lässt sich durch eine Abwandlung der Gleichungen~\ref{eqn:M_x}~bis~\ref{eqn:M_z}
mathematisch wie folgt beschreiben

\begin{align}
  \label{eqn:M_x_relax}
  \frac{\su{d}\vec{M}\cdot \vec{e}_x}{\su{d}t} &= \gamma B_0 \vec{M}\cdot \vec{e}_y-\frac{1}{T_2}\vec{M}\cdot \vec{e}_x, \\
  \label{eqn:M_y_relax}
  \frac{\su{d}\vec{M}\cdot \vec{e}_y}{\su{d}t} &= -\gamma B_0 \vec{M}\cdot \vec{e}_x-\frac{1}{T_2}\vec{M}\cdot \vec{e}_y, \\
  \label{eqn:M_z_relax}
  \frac{\su{d}\vec{M}\cdot \vec{e}_z}{\su{d}t} &= \frac{1}{T_1}\left(M_0 - \vec{M}\cdot \vec{e}_z\right).
\end{align}

Die in den Gleichungen auftretenden Zeitkonstanten $T_1$ und $T_2$ sind
verschiedene Relaxationszeiten.
\begin{description}
  \item[Spin-Gitter-Relaxationszeit/longitudinal:]Relaxationszeit parallel zur Magnetfeldrichtung $T_1$.
  Charakteristische Übergangszeit von Energie des Kernspins in Gitterschwingungen und umgekehrt.
  \item[Spin-Spin-Relaxationszeit/transversal:]Relaxationszeit senkrecht zur Magnetfeldrichtung $T_2$.
  Hauptsächlich durch Spinwechselwirkungen mit den nächsten Nachbarn hervorgerufen. Es können
  aber auch Spin-Gitter-Relaxationsprozesse zu $T_2$ beitragen.
\end{description}

\subsection{Einstrahlung von HF-Strahlungsquanten}

Die Funktion der Strahlungsquanten ist, wie bereits erwähnt das Auslenken der
Probenmagnetisierung aus der Gleichgewichtslage.
Das hochfrequente Strahlungsfeld ist so ausgerichtet, dass es senkrecht zu dem Magnetfeld
$\vec{B_0}$ und somit senkrecht zu $\vec{e}_z$ steht.
\begin{equation}
  \label{eqn:hochfrequent}
  \vec{B}\ua{HF} = 2\vec{B}_1 \cos{(\omega t)}
\end{equation}

Eine Transformation in ein mit $\omega$ um die $\vec{e}_z$-Achse
bewegtes Koordinatensystem führt dazu, die Zeitabhängigkeit
in $\vec{B}_1$ zu eliminieren. Die Zeitabhängigkeit wird
somit in den Einheitsvektoren eingebunden, sodass ein Übergang
von $\left\{\vec{e}_x, \vec{e}_y, \vec{e}_z\right\}$ in
$\left\{\vec{e}'_x, \vec{e}'_y, \vec{e}'_z\right\}$ statt findet.
Die zeitliche Änderung der Magnetisierung transformiert sich zu
\begin{equation}
  \label{eqn:mag_trafo}
  \frac{\su{d}\vec{M}}{\su{d}t} = \gamma\left\{\vec{M}\times\left(\vec{B} + \frac{\vec{\omega}}{\gamma}\right)\right\}.
\end{equation}
Das Einführen eines effektiven Magnetfeldes $\vec{B}\ua{eff} = \vec{B}_0 + \vec{B}_1 + \frac{\vec{\omega}}{\gamma}$
vereinfacht Gleichung~\ref{eqn:mag_trafo} wie folgt.
\begin{equation}
  \label{eqn:B_eff}
  \frac{\su{d}\vec{M}}{\su{d}t} =\gamma\left(\vec{M}\times\vec{B}\ua{eff}\right)
\end{equation}
Der Resonanzfall der Störung tritt auf, wenn die Einstrahlfrequenz $\omega$
gleich der Lamorfrequenz $\omega\ua{L}$ ist.
Da $\vec{\oemga}$ antiparallel zu $\vec{e}_z$ steht, ist
das effektive Magnetfeld $\vec{B}\ua{eff} = \vec{B}_1$.
Das bedeutet, die Magnetisierung präzediert um die $\vec{B}$- Achse.

Soll die Magnetisierung $\vec{M}$ um 90° aus der $\vec{e}_z$-Achse
herausgedreht werden muss die Einstrahlungszeit
\begin{equation}
  \label{eqn:90grad}
  \Delta t_{90} = \frac{\pi}{2\gamma B_1}
\end{equation}
betragen. Eine Drehung um 180° bedarf einer Einstrahlzeit von
$2\cdot\Delta t_{90}$.
Im Folgenden wird eine Drehung um 90° als 90°-Puls und eine
Drehung um 180° als 180°-Puls bezeichnet.

\subsection{Messmethoden für $T_2$}

Grundlegend wird für die Messung von $T_2$ ein Versuchsaufbau nach
Abbildung~\ref{fig:aufbau} benötigt. Die Polschuhe gehören zu einem
Permanentmagneten, der ein homogenes Magnetfeld $B_0\vec{e}_z$
bereitstellt. Eine Spule umschließt die Probe.
Diese Spule ist mit einem Sender und einem Empfänger gekoppelt.
Der Sender speißt Strom in die Spule ein um die
hochfrequenten Strahlungsquanten zu erzeugen.
Der Empfänger registriert die von den präzedierenden Kernspins
hervorgerufenen Magnetisierung.
\begin{figure}
  \centering
  \includegraphics[width = 0.7\textwidth]{Pics/aufbau.pdf}
  \caption{Grundlegender Versuchsaufbau zur Bestimmung von $T_2$\cite{anleitung}.}
  \label{fig:aufbau}
\end{figure}
Der Abbau der Magnetisierung wird als freier Induktionszerfall (FID)
bezeichnet. Dieser kann prinzipiell durch Aufzeichnen der
Empfängersignale dargestellt werden.
Aufgrund von Nachbar-Spinwechselwirkungen und Inhomogenität in dem Magnetfeld des
Permanentmagneten treten Dephasierungsprozesse der Kernspins auf.
Aus diesem Grund fächert die Präzessionsbewegung der Spins
auf. Für die messbare Relaxationszeit $T*_2$ gilt
\begin{equation}
  \label{eqn:T*}
  \frac{1}{T*_2} = \frac{1}{T_2} + \frac{1}{T_{\Delta B}}.
\end{equation}
Dabei ist $T_{\Delta B}$ eine apparative Zeitkonstante in der
Größenordnung von $\frac{1}{\gamma d G}$, wobei $d$ der Probendurchmesser
und $G$ der Gradient der Permanentmagnetfeldes.
Der freie Induktionszerfall zeigt bei dem Auftreten von mehreren
Resonanzstellen eine äußerst komplexe Struktur, die von der erwarteten
exponentiellen Form abweicht.
Ist das Permanentmagnetenfeld hinreichend homogen ist
$T_{\Delta B}$ vernachlässigbar. In Fällen wo $T_{\Delta B} < T_2$
ist, ist die Bestimmung von $T_2$ durch die FID Methode nicht mehr möglich.

\subsubsection{Spin-Echo-Verfahren}
Die apparativ bedingten Effekte auf $T_2$ können eliminiert werden,
wenn diese Effekte zeitlich konstant sind. Das Spin-Echo-Verfahren
arbeitet mit zwei Pulsen. Der erste Puls ist ein 90°-Puls, mit
dem die Spins aus der $\vec{e}_z$-Achse gedreht werden.
Werden zwei Spins betrachtet, die aufgrund von Feldinhomogenitäten verschiedene Lamorfrequenzen
$\omega\ua{L, i}$ besitzen. Dabei sei o.B.d.A. $\omega\ua{L, 1} > \omega$ und
$\omega\ua{L, 2} < \omega$. Der Spin zu $\omega\ua{L, 1}$
dreht somit im Uhrzeigersinn in der $\vec{e}'_x-\vec{e}'_y$-Ebene.
Der zweite Spin läuft in selben Ebene gegensinnig.
Das Spin-Echo-Verfahren ist zur Veranschaulichung in Abb.~\ref{eqn:spin_flip}
dargestellt.
Nach der Zeit $\tau$ kommt der zweite Puls. Dieser ist ein
180°-Puls mit dem der Spin um die $\vec{e}'_x$ gedreht wird.
Aus diesem Grund laufen die dephasierten Spins aufeinander zu, sodass
sie zum Zeitpunkt $2\tau$ in die selbe Richtung zeigen.
\begin{figure}
  \centering
  \includegraphics[width = 0.7\textwidth]{Pics/spin.pdf}
  \caption{Zeitliche Phasen des Spin-Echo-Verfahren~\cite{anleitung}.}
  \label{eqn:spin_flip}
\end{figure}
Das Verfahren produziert das folgende Signal.
\begin{figure}
  \centering
  \includegraphics[width = 0.7\textwidth]{Pics/signalverlauf.pdf}
  \caption{Signalverlauf des Spin-Echo-Verfahrens~\cite{anleitung}.}
  \label{eqn:signal}
\end{figure}
