\section{Auswertung}

In der folgenden Auswertung werden zuerst die beiden Relaxationszeiten $T\ua{1}$
sowie $T\ua{2}$ bestimmt. Mithilfe von kann dann die Diffusionskonstante $D$ sowie
der Molekülradius bestimmt werden. Der Molekülradius wird am Ende zudem mit
den Radien verglichen, die sich aus dem Molekulargewicht und dem Van-der-Waals-Kovolumen
ergeben.

\subsection{Bestimmung der longitudinalen Relaxationszeit $T\ua{1}$}

Die aufgenommenen Daten für die Bestimmung von $T\ua{1}$ sind Tabelle \ref{tab:T1}
eingetragen sowie in Abbildung \ref{fig:T1} grafisch dargestellt.

\begin{table}
  \centering
  \caption{Messdaten für die Spannungamplituden des ersten Echos bei verschiedenen
  Pulsabständen.}
  \label{tab:T1}
  \begin{tabular}{S  S | S  S}
    \toprule
    {$\tau / \si{ms}$} & {$U / \si{mV}$} & {$\tau /\si{ms}$} & {$U / \si{mV}$} \\
    \midrule
    1  & -785 & 100  & -633 \\
    2  & -780 & 200  & -565   \\
    3  & -765 & 500  & -395   \\
    5  & -745 & 1000 & -195   \\
    8  & -745 & 1500 &   35   \\
    9  & -735 & 2000 &  118   \\
    13 & -730 & 4000 &  612   \\
    20 & -715 & 7000 &  643 \\
    50 & -665 & 9000 &  700   \\
    75 & -648 & &           \\
    \bottomrule
  \end{tabular}
\end{table}


Um $T1$ zu bestimmen werden die experimentellen Daten an eine Exponentialfunktion
der folgenden Form gefittet:

\begin{equation}
  M(t) = M\ua{0}(1-2\exp{(-\frac{t}{T_1})})+M\ua{1}.
\end{equation}

Für die verschiedenen Parameter ergeben sich damit folgende Werte:
\begin{align}
  M\ua{0} &= (0.73\pm0.02)\,\si{V} \\
  M\ua{1} &= (0.04\pm0.03)\,\si{V} \\
  T\ua{1} &= (1.54\pm0.12)\,\si{s}
\end{align}

\begin{figure}
  \centering
  \includegraphics[width=\textwidth]{Plots2/T1.pdf}
  \caption{Gemessene Signalhöhe des Echos gegen den zeitlichen Abstand der beiden Pulse.}
  \label{fig:T1}
\end{figure}

\subsection{Bestimmung der transversalen Relaxationszeit $T\ua{2}$}

Um die transversale Relaxationszeit $T\ua{2}$ zu bestimmen wird das Meiboom-Gill Verfahren
verwendet. Zudem ist in Abbildung \ref{fig:T2CP} einmal die Burstsequenz mit dem
Carr-Purcell-Verfahren dargestellt.
\begin{figure}\centering
  \includegraphics[width=0.8\textwidth]{Plots2/T2CP.pdf}
  \caption{Signale der transversalen Magnetisierung mit der Carr-Purcell-Methode
  bei einem Pulsabstand von $\tau = \SI{2}{ms}$. }
  \label{fig:T2CP}
\end{figure}
In Abbildung \ref{fig:T2} ist die Burstsequenz mit der Meiboom-Gill Methode
dargestellt. Um $T\ua{2}$ zu bestimmen, werden alle Spannungsamplitudem bei
ganzzahligen Vielfachen von $2\tau$ verwendet. Die verwendeten Peaks sind als
rot gefärbte Datenpunkte markiert. Mit den in Tabelle \ref{tab:T2Ex} eingetragenen
extrahierten Daten wird ein Fit der Form
\begin{equation}
  M(t) = M\ua{0}\cdot\exp{(-\frac{t}{T\ua{2}})}+M\ua{1}
\end{equation}
durchgeführt.
Für die verschiedenen Parameter ergeben sich dabei folgende Werte:
\begin{align}
  M\ua{0} &= (0.59\pm0.02)\,\si{V}\\
  M\ua{1} &= (0.04\pm0.03)\,\si{V}\\
  T\ua{2} &= (1.54\pm0.12)\,\si{s}.
\end{align}
Die extrahierten Messwerte sowie der FIt sind noch einmal in Abbildung \ref{fig:T2Log} mit
logarithmierter Y-Achse dargestellt.
\begin{figure}
  \centering
  \includegraphics[width=0.8\textwidth]{Plots2/T2.pdf}
  \caption{Die aufgenommenen Spannungsamplituden unter Verwendung der
  Meiboom-Gill-Methode bei einem Pulsabstand von $\tau = \SI{2}{ms}$.}
  \label{fig:T2}
\end{figure}
\begin{figure}
  \centering
  \includegraphics[width=0.8\textwidth]{Plots2/T2Log.pdf}
  \caption{Darstellung der extrahierten Punkte mit logarithmierter Y-Achse.}
  \label{fig:T2Log}
\end{figure}
\begin{table}
  \centering
  \caption{Extrahierte Messdaten für die Spannungamplituden bei Verwendung der Meiboom-Gill-Methode}
  \label{tab:T2Ex}
  \begin{tabular}{S  S | S  S | S S}
    \toprule
    {$\tau / \si{ms}$} & {$U / \si{mV}$} & {$\tau /\si{ms}$} & {$U / \si{mV}$}
    & {$\tau /\si{ms}$} & {$U / \si{mV}$} \\
    \midrule
    0.00 & 0.754 & 0.63 & 0.418 & 1.22 & 0.321 \\
    0.02 & 0.642 & 0.64 & 0.433 & 1.24 & 0.297 \\
    0.04 & 0.627 & 0.66 & 0.425 & 1.26 & 0.297 \\
    0.05 & 0.618 & 0.68 & 0.410 & 1.28 & 0.297 \\
    0.08 & 0.594 & 0.70 & 0.418 & 1.30 & 0.297 \\
    0.10 & 0.586 & 0.72 & 0.401 & 1.32 & 0.289 \\
    0.12 & 0.586 & 0.74 & 0.409 & 1.34 & 0.297 \\
    0.14 & 0.578 & 0.76 & 0.409 & 1.36 & 0.281 \\
    0.16 & 0.570 & 0.78 & 0.385 & 1.38 & 0.273 \\
    0.18 & 0.554 & 0.80 & 0.385 & 1.40 & 0.265 \\
    0.20 & 0.546 & 0.82 & 0.385 & 1.42 & 0.265 \\
    0.22 & 0.554 & 0.84 & 0.401 & 1.44 & 0.257 \\
    0.24 & 0.538 & 0.86 & 0.369 & 1.46 & 0.257 \\
    0.26 & 0.514 & 0.88 & 0.385 & 1.48 & 0.265 \\
    0.28 & 0.530 & 0.90 & 0.369 & 1.50 & 0.257 \\
    0.30 & 0.514 & 0.92 & 0.369 & 1.52 & 0.257 \\
    0.32 & 0.522 & 0.94 & 0.353 & 1.54 & 0.241 \\
    0.34 & 0.506 & 0.96 & 0.337 & 1.56 & 0.257 \\
    0.36 & 0.505 & 0.98 & 0.353 & 1.58 & 0.257 \\
    0.38 & 0.489 & 1.00 & 0.353 & 1.60 & 0.257 \\
    0.40 & 0.497 & 1.02 & 0.345 & 1.62 & 0.241 \\
    0.42 & 0.498 & 1.04 & 0.353 & 1.64 & 0.241 \\
    0.44 & 0.474 & 1.06 & 0.337 & 1.66 & 0.257 \\
    0.46 & 0.473 & 1.08 & 0.329 & 1.68 & 0.241 \\
    0.48 & 0.450 & 1.10 & 0.321 & 1.70 & 0.233 \\
    0.50 & 0.458 & 1.12 & 0.313 & 1.72 & 0.233 \\
    0.52 & 0.465 & 1.14 & 0.329 & 1.74 & 0.233 \\
    0.54 & 0.458 & 1.16 & 0.321 & 1.76 & 0.233 \\
    0.56 & 0.450 & 1.18 & 0.313 & 1.78 & 0.209 \\
    0.58 & 0.434 & 1.20 & 0.313 & 1.80 & 0.217 \\
    0.60 & 0.441 & & & & \\
    \bottomrule
  \end{tabular}
\end{table}

\newpage
\subsection{Bestimmung der Diffusionskonstante}

Bevor die Diffusionskonstante bestimmt werden kann, muss vorher der Feldgradient
bestimmt werden. Dafür wurden bei der Messung der Echos für die Bestimmung
der Diffusionskonstante ebenfalls die Halbwertsbreiten bestimmt. Dei verwendeten
Daten sind in Tabelle \ref{tab:D} eingetragen. Für die gemittelte
Halbwertsbreite ergibt sich dabei $t\ua{1/2} = (0.9\pm0.3)\,\si{ms}$.
Die Gravitationskonstante lässt sich anschließend gemäß der folgenden Formel
bestimmen:
\begin{equation}
  G = \frac{8.8}{d\ua{P}\gamma t\ua{1/2}}.
\end{equation}
Dabei handelt es sich bei $d\ua{P} = \SI{4.4}{mm}$ um den Probendurchmesser
und bei $gamma = \SI{2.68E8}{\frac{rad}{sT}}$ um das gyromagnetische
Verhältnis von Protonen \cite{Gyro}.
