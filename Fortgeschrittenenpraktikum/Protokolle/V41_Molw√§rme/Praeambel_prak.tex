% bei Standalone in documentclass noch:
% \RequirePackage{luatex85}

\documentclass[captions=tableheading, titlepage= firstiscover, parskip = half , bibliography=totoc]{scrartcl}
%paper = a5 für andere optinen
% titlepage= firstiscover
% bibliography=totoc für bibdateien
% parskip=half  Veränderung um Absätze zu verbessern

\usepackage{scrhack} % nach \documentclass
\usepackage[aux]{rerunfilecheck}
\usepackage{polyglossia}
\usepackage[style=numeric, backend=biber]{biblatex} % mit [style = alphabetic oder numeric] nach polyglossia
\addbibresource{lit.bib}
\setmainlanguage{german}

\usepackage[autostyle]{csquotes}
\usepackage{amsmath} % unverzichtbare Mathe-Befehle
\usepackage{amssymb} % viele Mathe-Symbole
\usepackage{mathtools} % Erweiterungen für amsmath
\usepackage{fontspec} % nach amssymb
% muss ins document: \usefonttheme{professionalfonts} % für Beamer Präsentationen
\usepackage{longtable}

\usepackage[
math-style=ISO,    % \
bold-style=ISO,    % |
sans-style=italic, % | ISO-Standard folgen
nabla=upright,     % |
partial=upright,   % /
]{unicode-math} % "Does exactly what it says on the tin."
\setmathfont{Latin Modern Math}
% \setmathfont{Tex Gyre Pagella Math} % alternativ

\usepackage[
% die folgenden 3 nur einschalten bei documenten
locale=DE,
separate-uncertainty=true, % Immer Fehler mit ±
per-mode=symbol-or-fraction, % m/s im Text, sonst \frac
]{siunitx}

% alternativ:
% per-mode=reciprocal, % m s^{-1}
% output-decimal-marker=., % . statt , für Dezimalzahlen

\usepackage[
version=4,
math-greek=default,
text-greek=default,
]{mhchem}

\usepackage[section, below]{placeins}
\usepackage{caption} % Captions schöner machen
\usepackage{graphicx}
\usepackage{grffile}
\usepackage{subcaption}

% \usepackage{showframe} Wenn man die Ramen sehen will

\usepackage{float}
\floatplacement{figure}{htbp}
\floatplacement{table}{htbp}

%\usepackage[version=4]{mhchem} %chemische Symbole Beispiel: \ce{^{227}_{90}Th+}


\usepackage{booktabs}

 \usepackage{microtype}
 \usepackage{xfrac}

 \usepackage{expl3}
 \usepackage{xparse}

 % \ExplSyntaxOn
 % \NewDocumentComman \I {}  %Befehl\I definieren, keine Argumente
 % {
 %    \symup{i}              %Ergebnis von \I
 % }
 % \ExplSyntaxOff

 \usepackage{pdflscape}
 \usepackage{mleftright}

 % Mit dem mathtools-Befehl \DeclarePairedDelimiter können Befehle erzeugen werden,
 % die Symbole um Ausdrücke setzen.
 % \DeclarePairedDelimiter{\abs}{\lvert}{\rvert}
 % \DeclarePairedDelimiter{\norm}{\lVert}{\rVert}
 % in Mathe:
 %\abs{x} \abs*{\frac{1}{x}}
 %\norm{\symbf{y}}

 % Für Physik IV und Quantenmechanik
 \DeclarePairedDelimiter{\bra}{\langle}{\rvert}
 \DeclarePairedDelimiter{\ket}{\lvert}{\rangle}
 % <name> <#arguments> <left> <right> <body>
 \DeclarePairedDelimiterX{\braket}[2]{\langle}{\rangle}{
 #1 \delimsize| #2
 }

\setlength{\delimitershortfall}{-1sp}

 \usepackage{tikz}
 \usepackage{tikz-feynman}

 \usepackage{csvsimple}
 % Tabellen mit \csvautobooktabular{"file"}
 % muss in table umgebung gesetzt werden


% \multicolumn{#Spalten}{Ausrichtung}{Inhalt}

\usepackage{hyperref}
\usepackage{bookmark}
\usepackage[shortcuts]{extdash} %nach hyperref, bookmark

\newcommand{\ua}[1]{_{\symup{#1}}}
\newcommand{\su}[1]{\symup{#1}}
