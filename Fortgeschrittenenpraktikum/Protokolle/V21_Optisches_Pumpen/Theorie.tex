\section{Theorie}

In dem folgenden Versuch soll mithilfe eines präziesen Verfahrens die Zeeman-Aufspaltung
von \ce{^{85}Rb} und \ce{^{87}Rb} untersucht werden. Dafür wird das Verfahren des
optischen Pumpens verwendet.
Die Elektronenhülle eines isolierten Atoms besitzt diskrete Energieniveaus, wobei
nach dem die Nievaus der innersten Schalen vollständig mit Elektronen besetzt sind
(gemäß dem Pauli-Ausschlussprinzips). Für die äußeren Schalen ist bekannt, dass die
Niveaus mit einer höheren Energie schwächer besetzt sind, wobei die zwei Niveaus
mit Energien $W\ua{1}$ und $W\ua{2}$ ($W\ua{1}$ < $W\ua{2}$ ) und Besetzungszahlen
$N\ua{1}$ und $N\ua{2}$ gemäß der folgenden Relation miteinander verknüpft sind.

\begin{equation}
  \frac{N\ua{2}}{N\ua{1}} = \frac{g\ua{2}}{g\ua{1}} \frac{\exp{(-W\ua{2}/kt)}}{\exp{(-W\ua{1}/kt)}}
  \label{eqn:NRelation}
\end{equation}

Durch das Verfahren des optischen Pumpens kann nun eine Inversion erzeugt werden,
bei der $N\ua{1}$ < $N\ua{2}$ gilt. Durch hinterher induzierte Strahlungsübergänge
kann die Energie der emittierten Quanten präzise ausgemessen werden. In dem
folgenden Versuch sollen somit durch Untersuchung der Zeeman-Aufspaltung verschiedene
Größen der Rubidium-Isotope bestimmt werden.

\subsection{Magnetische Momente}

Sowohl der Bahndrehimpuls $\vec{L}$ als auch der Spin $\vec{S}$ besitzen beide
ein magnetisches Moment, welches Abhängig vom Betrag der Größe sowie der g-Faktor
und dem Bohrschen Magneton ist. Da die beiden Größen zu einem Gesamtdrehimpuls
$\vec{J}$ kann dementsprechend auch ein magnetisches Moment $\vec{\mu}\ua{J}$
definiert werden.

\begin{align}
  |\vec{\mu}\ua{L}| &= \mu\ua{B} \sqrt{L(L+1)} \\
  |\vec{\mu}\ua{S}| &= g\ua{S} \mu\ua{B} \sqrt{S(S+1)} \\
  \implies & \vec{\mu}\ua{J} = \vec{\mu}\ua{L} + \vec{\mu}\ua{S} \\
  & |\vec{\mu}\ua{J}| = g\ua{J} \mu\ua{B} \sqrt{J(J+1)}
\end{align}

$\vec{\mu}\ua{J}$ führt dabei eine Präzessionsbewegung um die $\vec{J}$-Richtung
aus, wodurch der senkrechte Anteil im zeitlichen Mittel verschwindet und als
magnetisches Moment nur der parallele Anteil eine Rolle spielt. Durch vektorielle
Betrachtung von $\vec{L}$, $\vec{S}$ und $\vec{J}$ lässt sich nun auch eine
Formel für $g\ua{J}$ bestimmen:

\begin{equation}
  g\ua{J} \approx \frac{3J(J+1)+S(S+1)-L(L+1)}{2J(J+1)}
  \label{eqn:LandeJ}
\end{equation}
