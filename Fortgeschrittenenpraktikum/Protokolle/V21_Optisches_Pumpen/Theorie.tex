\section{Theorie}

In dem folgenden Versuch soll mithilfe eines präziesen Verfahrens die Zeeman-Aufspaltung
von \ce{^{85}Rb} und \ce{^{87}Rb} untersucht werden. Dafür wird das Verfahren des
optischen Pumpens verwendet.
Die Elektronenhülle eines isolierten Atoms besitzt diskrete Energieniveaus, wobei
nach dem die Nievaus der innersten Schalen vollständig mit Elektronen besetzt sind
(gemäß dem Pauli-Ausschlussprinzips). Für die äußeren Schalen ist bekannt, dass die
Niveaus mit einer höheren Energie schwächer besetzt sind, wobei die zwei Niveaus
mit Energien $W\ua{1}$ und $W\ua{2}$ ($W\ua{1}$ < $W\ua{2}$ ) und Besetzungszahlen
$N\ua{1}$ und $N\ua{2}$ gemäß der folgenden Relation miteinander verknüpft sind.

\begin{equation}
  \frac{N\ua{2}}{N\ua{1}} = \frac{g\ua{2}}{g\ua{1}} \frac{\exp{(-W\ua{2}/kt)}}{\exp{(-W\ua{1}/kt)}}
  \label{eqn:NRelation}
\end{equation}

Durch das Verfahren des optischen Pumpens kann nun eine Inversion erzeugt werden,
bei der $N\ua{1}$ < $N\ua{2}$ gilt. Durch hinterher induzierte Strahlungsübergänge
kann die Energie der emittierten Quanten präzise ausgemessen werden. In dem
folgenden Versuch sollen somit durch Untersuchung der Zeeman-Aufspaltung verschiedene
Größen der Rubidium-Isotope bestimmt werden.

\subsection{Magnetische Momente}
\label{subsec:MagMo}

Sowohl der Bahndrehimpuls $\vec{L}$ als auch der Spin $\vec{S}$ besitzen beide
ein magnetisches Moment, welches Abhängig vom Betrag der Größe sowie der g-Faktor
und dem Bohrschen Magneton ist. Da die beiden Größen zu einem Gesamtdrehimpuls
$\vec{J}$ kann dementsprechend auch ein magnetisches Moment $\vec{\mu}\ua{J}$
definiert werden.

\begin{align}
  |\vec{\mu}\ua{L}| &= \mu\ua{B} \sqrt{L(L+1)} \\
  |\vec{\mu}\ua{S}| &= g\ua{S} \mu\ua{B} \sqrt{S(S+1)} \\
  \implies & \vec{\mu}\ua{J} = \vec{\mu}\ua{L} + \vec{\mu}\ua{S} \\
  & |\vec{\mu}\ua{J}| = g\ua{J} \mu\ua{B} \sqrt{J(J+1)}
\end{align}

$\vec{\mu}\ua{J}$ führt dabei eine Präzessionsbewegung um die $\vec{J}$-Richtung
aus, wodurch der senkrechte Anteil im zeitlichen Mittel verschwindet und als
magnetisches Moment nur der parallele Anteil eine Rolle spielt. Durch vektorielle
Betrachtung von $\vec{L}$, $\vec{S}$ und $\vec{J}$ lässt sich nun auch eine
Formel für $g\ua{J}$ bestimmen:

\begin{equation}
  g\ua{J} \approx \frac{3J(J+1)+S(S+1)-L(L+1)}{2J(J+1)}
  \label{eqn:LandeJ}
\end{equation}

Unter Einfluss eines äußeren Magnetfeldes ändern sich aufgrund der Wechselwirkung
mit dem magnetischen Moment die Energieniveaus. Da $\vec{mu}\ua{J}$ ebenfalls eine
Präzessionsbewegung um $\vec{B}$ ausführt, hat nur die zu $\vec{B}$ parallele
Komponente einen Einfluss auf die Energeiniveaus. Aufgrund der Richtungsquantelung
können die Energieniveaus nur ganzzahlige Vielfache von $g\ua{J}\mu\ua{B}B$
annehmen, verknüpft durch die die Orientierungsquantenzahl $M\ua{J}$.

\begin{equation}
  U\ua{mag} = M\ua{J}g\ua{J}\mu\ua{B}B, \qquad M\ua{J} \in [-J,...,J]
  \label{eqn:EdiffJ}
\end{equation}

Die Energieniveaus spalten somit in $2J+1$ Unternivaus auf.

\subsection{Einfluss des Kernspins}
\label{subsec:Kern}

Beide Rubidium-Isotope besitzen einen Kernspin ungleich Null, weshalb man eine
weitere Aufspaltung beachten muss, die Hyperfeinstruktur.
Sie entsteht durch die Kopplung von Gesamtdrehimpuls und Kernspin:

\begin{equation}
  \vec{F} = \vec{J} + \vec{I}
\end{equation}

Die Quantenzahl F läuft somit von $I+J$ bis $|I-J|$ und spaltet in $2J+1$ oder
$2J+1$, je nachdem ob $J<I$ oder $J>I$ ist. Legt man hier nun ein zusätzliches
äußeres Magnetfeldan, welches nicht zu hoch ist, dann spalten alle Hyperfeinstrukturniveaus
erneut in $2F+1$ Unterniveaus auf, charakterisiert durch die Quantenzahl $M\ua{F}$.
Die einzelnen Zeeman-Niveaus sind äquidistant mit folgender Energiedifferenz:

\begin{equation}
  U\ua{HF} = g\ua{F}\mu\ua{B}B
  \label{eqn:EdiffF}
\end{equation}

%\begin{figure}
%  \includegraphics{/path/to/figure}
%  \caption{}
%  \label{}
%\end{figure}

Durch eine ähnliche Vektorenbetrachtung wie in \ref{subsec:MagMo} lässt sich nun unter
Betrachtung von $|\mu\ua{F}| = \sqrt{F(F+1)}g\ua{F}\mu\ua{B}$ der entsprechende
Landé-Faktor bestimmen:

\begin{equation}
  g\ua{F} = g\ua{J} \frac{F(F+1)+J(J+1)-I(I+1)}{2F(F+1)}
\end{equation}

\subsection{Optisches Pumpen}
\label{subsec:OptP}

Wie in der Einleitung beschrieben, muss für die Vermessung der Zeeman-Niveaus erst
eine Inversion der Niveaus erzeugt werden. Im folgenden wird zur Erklärung des
Prinzips die vereinfachende Annahme getroffen, dass das zu betrachtende Alkali-Atom
keinen Kern mit Drehimpuls besitzt. Betrachtet man den Grundzustand \ce{^{2}S_{1/2}}
sowie die beiden ersten angeregten Zustände \ce{^{2}P_{1/2}} und \ce{^{2}P_{-1/2}},
so ergibt sich das D1-D2-Dublett (siehe Abbildung \ref{fig:D1D2}).

%\begin{figure}
%  \includegraphics{/path/to/figure}
%  \caption{}
%  \label{}
%\end{figure}

Bei allen Zuständen ist $J=\frac{1}{2}$, somit kann $M\ua{J}$ nur die Werte $+\frac{1}{2}$
oder $-\frac{1}{2}$ annehmen.
Jedes Niveau spaltet somit in zwei Zeeman-Unterniveaus auf (siehe Abbildung \ref{eqn:Zeemanohne}).

%\begin{figure}
%  \includegraphics{/path/to/figure}
%  \caption{}
%  \label{}
%\end{figure}

Gemäß der Auswahlregeln für Strahlungsübergänge lassen sich 3 Übergänge unterscheiden,
die durch ihre Energie und ihren Polarisationszustand charakterisiert werden.

\begin{enumerate}
  \item[a)] $\sigma^{+},\increment M\ua{J}=+1$: \\
    Das Licht ist rechtszirkular polarisiert, somit steht der Spin antiparallel
    zur Ausbreitungsrichtung.
  \item[b)] $\sigma^{-},\increment M\ua{J}=-1$: \\
    Das Licht ist linkszirkular polarisiert, somit steht der Spin parallel zur
    Ausbreitungsrichtung.
  \item[c)] $\pi, \increment M\ua{J}=0$: \\
    Das Licht ist linear polarisiert, parallel zu dem angelegten $\vec{B}$-Feld.
\end{enumerate}

Aufgrund des Dipolcharakters erscheinen die $\sigma$-Übergänge in allen zu $\vec{B}$
senkrechten Richtungen linear polarisiert.

Befindet sich in der Dampfkammer nun ein Gas aus den voher angesprochenen Alkali-Atome,
so kann mithilfe des optischen Pumpens eine Inversion der $M\ua{J}$ Niveaus des
$\ce{^{2}S_{1/2}}$ Zustandes erzeugt werden. Das $M\ua{J}=\frac{1}{2}$ Niveau ist
anfangs schwächer besetzt als das $M\ua{J}=-\frac{1}{2}$ Nivaeau. Durch Einstrahlen
von D1-Licht werden nun $\increment M\ua{J} = +1$ Übergänge angeregt. Somit ist
lediglich der Übergang von $\ce{^{2}S_{1/2}}$,$M\ua{J}=-\frac{1}{2}$ zu $\ce{^{2}P_{1/2}}$,
$M\ua{J}=\frac{1}{2}$ möglich. Durch spontane Emmsion geht der angeregte Zustand
nach ca. $10^{-8}$ s wieder in den Grundzustand über, wobei allerdings beide Niveaus
des Grundzustandes bevölkert werden.
Aufgrund der Auswahlregel kann jedoch kein Übergang von $\ce{^{2}S_{1/2}}$,
$M\ua{J}=\frac{1}{2}$ zu $\ce{^{2}P_{1/2}}$,$M\ua{J}=\frac{1}{2}$ stattfinden,
wodurch sich dieses Niveau anreichert, während das $M\ua{J}=-\frac{1}{2}$ Niveau
langsam entleert wird. Somit lässt sich zumindest in der Theorie eine komplette
Inversion der beiden Zustände erzeugen.
In der Realität ist es jedoch so, dass durch Zusammenstöße der Gasatome untereinander
die Elektronenspins umklappen können, so dass ein Übergang im Grundzustand vom
$M\ua{J}=\frac{1}{2}$ Niveau zum $M\ua{J}=-\frac{1}{2}$ entsteht. Dies lässt sich
durch ein zusätzliches Puffergas in der Dampfkammer verhindern, wobei meistens ein
Edelgas gewählt wird.

Der Vorgang kann durch Beobachten der Intensität des am Ende der Dampfkammer
austretenden D1-Lichtes überprüft werden. Dafür wird ein Lichtdetektor hinter der
Zelle montiert (siehe Kapitel \ref{sec:Aufbau}, Abbildung \ref{fig:Aufbau}).
 Durch die Entleerung des $M\ua{J}=-\frac{1}{2}$
Niveaus wird mit der Zeit immer weniger Licht von dem Gas absorbiert, weswegen
die Transparenz der Kammer immer weiter zunimmt und sich asymptotisch der vollkommenen
Transparenz annähert (siehe Abbildung \ref{fig:Transparenz1})

%\begin{figure}
%  \includegraphics{/path/to/figure}
%  \caption{}
%  \label{}
%\end{figure}
