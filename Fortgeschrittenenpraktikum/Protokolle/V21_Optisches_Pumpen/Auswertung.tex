\section{Auswertung}

Die Rubidium-Isotope $\ce{^{85}Rb}$ und $\ce{^{87}Rb}$ werden im Laufe der Auswertung
dem jeweiligen Datensatz zugeordnet. Bevor die Identifikation stattfindet werden die Messdaten, die zu
der ersten Resonanzstelle gehören mit einer eins indiziert und die Messdaten, die
zu der zweiten gehören mit einer zwei.

Ein typisches Signalbild des Versuches ist in Abbildung \ref{fig:typisch}
dargestellt.

\begin{figure}[h]
  \centering
  \includegraphics[angle = 90]{Pics/edit_typical.jpg}
  \caption{Typisches Signalbild. Die beiden Resonanzstellen sind in dem schraffierten
  Rechteck zu erkennen.}
  \label{fig:typisch}
\end{figure}

\subsection{Erdmagnetfeld}

Der Tisch mit dem darauf befindlichen Versuchsaufbau ist so ausgerichtet, dass nur die
Horizontalkomponente des Erdmagnetfeldes Einfluss auf den Versuch hat.
Die Spule kompensiert bei einen eingestellten Strom von $I\ua{vertikal} = \SI{0.23}{\ampere}$ das
Erdmagnetfeld.
Durch die Formel für Helmholtzspulen

\begin{equation}
  \label{eqn:helmholtz}
  B\ua{Spule} = \mu_0\cdot\frac{8\cdot I\cdot N}{\sqrt{125}\cdot R}
\end{equation}

berechnet sich die Vertikalkomponente des Erdmagnetfeldes zu $B\ua{Erd} = \num{3.49}\mu\si{\tesla}$.

\subsection{Bestimmung der Landé-Faktoren und des Kernspins der Rubidium--Isotope}

Die Landé-Faktoren $g_{F_i}$ lassen sich durch eine Lineareregression mit der Formel \ref{eqn:Bm}
ermitteln. Die Messdaten der Frequenzen und den zugehörigen gemessenen Stromstärken sind
in den Tabellen \ref{tab: Messdaten_Resonanz_1} und \ref{tab: Messdaten_Resonanz_2} dargestellt.
Eine Darstellung der linearen Regression ist
in \ref{fig:regression} abgebildet.

\begin{figure}[h]
  \centering
  \includegraphics[width = \textwidth]{Python/frequenz_B_feld.pdf}
  \caption{Lineare Regression der B-Feldstärken an den Resonanzstellen mit steigender Frequenz}
  \label{fig:regression}
\end{figure}

Die Landé-Faktoren der Isotope sind wie folgt.

\begin{align*}
  g\ua{F_1} &= \num{0.5011(19)} \\
  g\ua{F_2} &= \num{0.3339(11)}
\end{align*}

Die Rubidium-Isotope haben die folgenden Quantenzahlen.

\begin{align*}
  L = 0, \qquad
  S = \frac{1}{2}, \qquad
  J = \frac{1}{2} \\
\end{align*}

Mit den obigen Quantenzahlen ist $g\ua{J} = \num{2.0023}$.

Somit kann unter Berücksichtigung von $F = I + J$ Gleichung \eqref{eqn:g_F} nach $I$
umgestellt werden.

\begin{equation}
  \label{eqn:Kernspin}
  I = \frac{g\ua{J}}{4g\ua{F}} - 1 + \sqrt{\left(\frac{g\ua{J}}{4g\ua{F}} - 1\right)^2 + \frac{3}{4}\left(\frac{g\ua{J}}{g\ua{F}} - 1\right)}
\end{equation}

Die Kernspins der Rubidium-Isotope ergeben sich zu:

\begin{align*}
  I_1 &= \num{1.498(7)} \\
  I_2 &= \num{2.498(10)}.
\end{align*}

Der Kernspin des Rubidium-Isotopes $\ce{^{85}Rb}$ wird in der Literatur
mit einem Wert von $I\ua{Literatur} = \frac{5}{2}$ angegeben.
Der Kernspin des Rubidium-Isotopes $\ce{^{87}Rb}$wird in der Literatur
mit einem Wert von $I\ua{Literatur} = \frac{3}{2}$ angegeben.

Im Vergleich mit den Literaturwerten wird erkenntlich, dass die erste Resonanzstelle zu dem Isotop $\ce{^{87}Rb}$
und die zweite Resonanzstelle zu dem Isotop $\ce{^{85}Rb}$ gehört.

\subsection{Isotopenverhältnis}

Aus der Abbildung \ref{fig:Resonanzstellen} sind die Amplituden der beiden
Resonanzstellen zu entnehmen.
Das Isotopenverhätnis $R = \frac{p_1}{p_2}$ ergibt sich aus dem Amplitudenverhältnis.
In der Versuchkammer liegen $p_1$ Anteile von $\ce{^{87}Rb}$ und $p_2$ Anteile
von $\ce{^{85}Rb}$ vor. Insgesamt gilt $p_1 + p_2 = 1$.
Somit folgt, dass $p_2 = \frac{1}{1 + R}$ ist.
Für den ersten Peak wurde eine Größe von 118 Pixeln und für den zweiten
Peak eine Größe von 265 Pixeln. Damit ergibt sich $R \approx 0.436$.
Weiterhin ergeben sich die Anteile $p_1$ und $p_2$ zu:


\begin{align*}
  p_1 &= 30,81 \% \\
  p_2 &= 69,19 \%.
\end{align*}

Daraus ist das Isotopenverhätnis mit ca. 1:3 ($\ce{^{87}Rb}$:$\ce{^{85}Rb}$) zu entnehmen.

\begin{figure}[h]
  \centering
  \includegraphics[angle = 90]{Pics/TEK0002.JPG}
  \caption{Detailaufnahme der beiden Resonanzstellen}
  \label{fig:Resonanzstellen}
\end{figure}

\subsection{Quadratischer Zeeman-Effekt}

Mit Formel \eqref{eqn:Uneu} lässt sich der quadratische Zeeman-Effekt abschätzen und mit dem
linearen Term vergleichen.

Für die Rubidium-Isotope werden die folgenden Daten verwendet.
Dabei sind für die B-Felder die Maximalwerte an den Resonanten verwendet worden.

\begin{align*}
  \ce{^{87}Rb}: &\\
  B &= \SI{0.161}{\milli\tesla} \\
  \symup{M}\ua{F} &= 2 \\
  \Delta E\ua{Hy} &= \SI{4.53e-24}{\joule}\\
  \\
  \ce{^{85}Rb}: &\\
  B &= \SI{0.233}{\milli\tesla} \\
  \symup{M}\ua{F} &= 3 \\
  \Delta E\ua{Hy} &= \SI{2.01e-24}{\joule}
\end{align*}

\newpage

Damit ergeben sich die Terme der Hyperfeinstrukturaufspaltung zu:

\begin{align}
  \ce{^{87}Rb}: &\\
  E_{z, \symup{lin}} &= \SI{7.482(28)e-28}{\joule}\\
  E_{z, \symup{quad}} &= \SI{-3.707(28)e-31}{\joule}\\
  \\
  \ce{^{85}Rb}: &\\
  E_{z, \symup{lin}} &=\SI{7.482(23)e-28}{\joule}\\
  E_{z, \symup{quad}} &= \SI{-1.294(8)e-30}{\joule}.
\end{align}

\section{Diskussion}

Im Allgemeinen liegen die erhobenen und ausgewerteten Ergebnisse im Bereich der
Literaturwerte. Die Auswertung der Messdaten ergab
einen Kernspin von $I_{\ce{^{85}Rb}} = \num{2.498(10)}$. Der Literaturwert liegt
in der Fehlerschranke und konnte hier experimentell bestätigt werden.
Gleiches gilt für den Kernspin des Isotopes $\ce{^{87}Rb}$. Der zugehörige
Literaturwert wurde im Experiment mit $I_{\ce{^{87}Rb}} = \num{1.498(7)}$,
im Rahmen der Fehlerschranke wieder gefunden.
Die Kernspins stimmen somit mit den Literaturwerten nahezu überein, weshalb angenommen wird,
dass die berechneten Landé-Faktoren   $g\ua{F_1} = \num{0.5011(19)}$ und $g\ua{F_2} = \num{0.3339(11)}$
ebenfalls im Rahmen der Fehlerschranken mit den tatsächlichen Landé-Faktoren
übereinstimmen.

Das natürliche Isotopenverhältnis ist nach \cite{Isotopenverhältnis}
$p_{1,Literatur} = 27,83 \%$ und $p_{2,Literatur} = 72,17 \%$. Die entspricht
ungefähr dem gefundenen Verhältnis in dem Versuchsaufbau. In der
Versuchskammer ist ebenfalls der Großteil von dem Isotop $\ce{^{85}Rb}$ enthalten.

Das Experiment ergab, dass die verwendeten B-Felder noch zu klein sind, damit
der quadratische Zeeman-Effekt eine signifikante Rolle spielt.
Der lineare Term ist bei beiden Isotopen um zwei Größenordnungen größer als der
quadratische Term, weshalb der quadratische Term als vernachlässigbar anzusehen ist.

In der Auswertung sind keine Ablesefehler der Spulenströme berücksichtigt worden,
wordurch die geringe Diskrepanz der experimentell ermittelten Werte zu den
Theoriewerten erklärt werden kann.

\section{Messdaten}

In diesem Kapitel sind die aufgenommenen Messdaten, sowie die berechneten
Magnetischenflussdichten tabelarisch aufgeführt.

\begin{table}
\centering
\caption{Messdaten der ersten Resonanzstelle}
\label{tab: Messdaten_Resonanz_1}
\begin{tabular}{S S S S S S }
\toprule
{$\nu$ in $\si{\kilo\hertz}$} & {$I_{sweep}$ in $\si{\ampere}$} & {$B_{sweep}$ in $\si{\milli\tesla}$} & {$I_{horizontal}$ in  $\si{\milli\ampere}$} & {$B_{horizontal}$ in $\si{\milli\tesla}$} & {$B_{ges}$ in $\si{\milli\tesla}$}  \\
\midrule
101 & 0.54 & 0.033 & 0  & 0.000  & 0.033 \\
210 & 0.38 & 0.023 & 29  & 0.025  & 0.048 \\
300 & 0.55 & 0.033 & 31  & 0.027  & 0.061 \\
400 & 0.47 & 0.028 & 53  & 0.046  & 0.075 \\
502 & 0.11 & 0.007 & 93  & 0.082  & 0.088 \\
600 & 0.18 & 0.011 & 105  & 0.092  & 0.103 \\
700 & 0.09 & 0.006 & 128  & 0.112  & 0.118 \\
800 & 0.35 & 0.021 & 126  & 0.110  & 0.132 \\
904 & 0.40 & 0.024 & 140  & 0.123  & 0.147 \\
1001 & 0.52 & 0.031 & 148  & 0.130  & 0.161 \\
\bottomrule
\end{tabular}
\end{table}


\begin{table}
\centering
\caption{Messdaten der zweiten Resonanzstelle}
\label{tab: Messdaten_Resonanz_2}
\begin{tabular}{S S S S S S }
\toprule
{$\nu$ in $\si{\kilo\hertz}$} & {$I_{sweep}$ in  $\si{\ampere}$} & {$B_{sweep}$ in $\si{\milli\tesla}$} & {$I_{horizontal}$ in  $\si{\milli\ampere}$} & {$B_{horizontal}$ in $\si{\milli\tesla}$} & {$B_{ges}$ in $\si{\milli\tesla}$}  \\
\midrule
101 & 0.66 & 0.040 & 0  & 0.000  & 0.040 \\
210 & 0.63 & 0.038 & 29  & 0.025  & 0.064 \\
300 & 0.90 & 0.055 & 31  & 0.027  & 0.082 \\
400 & 0.94 & 0.057 & 53  & 0.046  & 0.103 \\
502 & 0.71 & 0.043 & 93  & 0.082  & 0.124 \\
600 & 0.89 & 0.053 & 105  & 0.092  & 0.146 \\
700 & 0.92 & 0.055 & 128  & 0.112  & 0.168 \\
800 & 0.75 & 0.045 & 164  & 0.144  & 0.189 \\
904 & 0.77 & 0.046 & 188  & 0.165  & 0.211 \\
1001 & 0.72 & 0.044 & 216  & 0.189  & 0.233 \\
\bottomrule
\end{tabular}
\end{table}


\newpage

\begin{figure}[h]
  \centering
  \includegraphics[width = \textwidth]{Pics/V21_Messdaten}
  \caption{Aufgeschriebene Messdaten.}
  \label{fig:aufgeschMessdaten}
\end{figure}
