\section{Zielsetzung}

Der vorliegende Versuch behandelt die Ermittelung der Lebensdauer
von kosmischen Myonen.

\seciton{Theorie}

Myonen $\mu$ sind Leptonen der zweiten Generation und besitzen eine Masse
von $\approx \SI{106}{\mega\eV}$.
Sie entstehen zum Großteil in der oberen Atmosphäre durch Pion-Zerfälle und
besitzen eine hohe kinetische Energie.

\begin{equation}
  \label{eqn:pion+ nach myon}
  \ce{\pi^+ -> \mu^+ + \nu\ua{\mu}}\\
  \label{eqn:pion- nach myon}
  \ce{\pi^- -> \mu^- + \bar{\nu}\ua{\mu}}
\end{equation}

\subsection{Lebensdauer}

Die Lebensdauer $\tau$ eines instabilen Teilchens beschreibt die Zeit,
nachder eine Teilchenpopulation auf $\frac{1}{\exp}$ ihrer ursprünglichen
Anzahl abgefallen ist.
Der Zerfall eines Teilchens ist ein stochastischer Prozess, der durch
ein Exponentialgesetz der Form:

\begin{equation}
  \label{eqn:Lebensdauer}
  N(t) = N_0\cdot\exp{-\frac{t}{\tau}}
\end{equation}

beschrieben wird.





\section{Versuchsaufbau}

Damit die Lebensdauer von Myonen bestimmt werden kann, dürfen nur
Myonen gemessen werden, von denen die Einlaufzeit in den Versuchsaufbau bekannt ist
und weiterhin deren Zerfallszeit nach eindringen gemessen wurde.
Somit sind nur Myonen, die auch in dem Versuchsaufbau zerfallen relevant
für die Messung. Im Folgenden wird der verwendete Versuchsaufbau beschrieben,
der das herausfiltern dieser relevanten Ereignisse ermöglicht.

Grundlegend wird für die dedektion von Teilchen ist ein Szintillator.
Geladene Teilchen, die den Szintillator durchqueren wechselwirken mit den Atomen
im Szintillatormaterial. Durch diese Wechselwirkung können
die Atome im Szintillator in einen angeregten Zustand überführen werden.
Myonen haben eine hohe kinetische Energie und können mehrere MeV ihrer kinetischen Energie
an das Szintillatormaterial abgeben. Die angeregten Szintillatoratome geben
bei dem Übergang von dem angeregten Zustand in den Grundzustand einen Lichtquant
ab, der von einer optisch an den Szintillator gekoppelten Photokathode
dedektiert werden kann.
An der Photokathodeliegt ein sekundärer Elektronenvervielfältiger (SEV) an.
Der SEV verstärkt die Signale aus der Photokathode, damit sie dedektiert werden
können.

Das Filtern der Ereignisse von Myonen, die auch in dem Szintillator zerfallen
geschieht durch eine logische Schaltung. Diese Schaltung ist in Abb. \ref{fig:Aufbau}
durch die beiden AND-Gatter und den Monoflop/Univibrator realisiert.
Anliegende Spannungen werden im Folgenden als up und das fehlen von Spannungen
als down bezeichnet.
Wenn ein up aus der Koinzidenz (vgl. Abb. \ref{fig:Aufbau}), deren Funktion im
Verlauf der Durchfürhung noch erklärt wird, in die untere Schaltung einläuft
wird zunächst der rechte Eingang des 1. und 2. UND-Gatter auf up gesetzt.
Das aus der Koinzidenz stammende
up wird mit einer Verzögerung von $\SI{30}{\nano\second}$ an den Univibrator
abgegeben. An dem einen Ausgang des Univibrators, der mit dem 1. UND verbunden
ist liegt im down Zustand ein up an. Der zweite Ausgang ist mit dem 2. UND verbunden
und gibt im down Zusant des Univibrators ein down weiter.
Somit sind an dem 1. UND beide Eingänge mit einem up belegt und das Startsignal wird
an einen Zeit-Amplituden-Konverter und Impulszähler gegeben. Der Impulszähler
zählt lediglich die Anzahl der eintreffenden Startsignale.
Der Zeit-Amplituden-Konverter wandelt die vergehende Zeit zwischen einem
up an dem Starteingang und einem up an dem Stopeingang in eine
dazu proportionalen Impuls. Die Zeit wird in der Höhe des Impulses
kodiert.
Ein einlaufendes up in den Univibrator kehrt die up und down Signale an
den Ausgängen um.
Nach einer einegstellten Zeit $T\ua{s}$ geht der Monoflop in seinen Grundzustand
zurück. $T\ua{s}$ stellt die Suchzeit des Intervalls einer Myonenlebensdauermessung
dar. In dem Versuch wurde $T\ua{s} = \SI{10}{\micro\second}$ gewählt.
Der linke Eingang des 2. UND-Gatters nimmt empfängt nachdem
der Monoflop ein up erhalten hat ebenfalls ein up, welches für
die eingestellte Zeit $T\ua{s}$ anliegt. Wird in diesem
Zeitraum ein weiteres Signal von der Koinzidenz an den rechten Eingang des
2. UND-Gatters wird ein up an den Stop-Eingang des Zeit-Impuls-Konverters gegeben
und die Zeitmessung ist abgeschlossen. Die in einem Impuls kodierte Zerfallszeit
wird in ein Vielkanalanalysator gegeben, welcher mit einem Computer
verbunden ist, der das Signal mit der geeigneten Software verarbeiten kann.

Durch thermische Prozesse werden in dem Szintillatormaterial statischtisch Verteilt
Elektronen gelöst, welche von den SEV als Myonenereignisse interpretiert werden.
Um dieses thermische Rauschen zu unterdrücken gibt es zwei parallel angewendete
Mechanismen. Die thermischen Störungen lösen in der Regel eine niederige
Spannung als die Myonenereignisse aus. Durch einen Diskriminator werden nur
Ereignisse durchgelassen, die einen einstellbaren Schwellenwert $U_0$ übersteigen
durchgelassen. Damit können thermische Ereignisse signifikant unterdrückt werden.
Zudem werden die Ereignisse die den Schwellenwert des Diskriminatros übersteigen
mit einheitlicher Spannungsamplitude weitergeleitet.
Vor den Diskriminatoren sind Verzögerungen angebracht, damit Materialeigenschaften
der Diskriminatoren kompensiert werden können.

Der zweite rauschunterdrückende Mechanismus ist die Koinzidenz.
Diese lässt nur nahzu gleichzeitig einlaufende Signale passieren.
Die einlaufenden Signale dürfen einen Zeitversatz von $\Delta t\ua{k}\approx
\SI{4}{\nano\second}$ besitzen.
Die thermisch gelösten Elektronen entstehen örtlich Verteilt in dem
Szintillator und sind deutlich langsamer als durch Myonen entstehende Ereignisse und
erreichen die Koinzidenz mit nur einer geringen Wahrscheinlichkeit in dem
Zeitintervall $\Delta t\ua{k}$.




\begin{figure}[h]
  \centering
  \includegraphics[width=9cm, angle=90]{Pics/Aufbau.png}
  \caption{Schematischer Versuchsaufbau \cite{anleitung01}}
  \label{fig:Aufbau}
\end{figure}

Der Versuch wird mit dem Aufbau aus Abb. \ref{fig:Aufbau} durchgeführt. In
dem realen Aufbau ist ein weiterer Verzögerer vor dem rechten Diskriminator angebracht,
sodass eine Verzögerung der beiden Messkanäle relativ zu einander eingestellt werden kann.

Der in Abb. \ref{fig:Aufbau} dargestellte Doppelimpulsgenerator wird
für die Zeiteichung des Vielkanalanalysators verwendet, worauf in der
Durchführung näherdrauf eingegangen wird.

In dem Versuch wird ein organischer Szintillator verwendet, da dieser
im Vergleich zu einem anorganischen Szintillator eine kürzere
Totzeit besitzt. Insgesamt fasst der Szintillator 50l und ist mit Toluol befüllt.

\subsection{Durchführung}
