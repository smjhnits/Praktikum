\section{Theorie}
\label{sec:Theorie}

In dem folgenden Versuch soll die effektive Masse von Elektronen in GaAs bestimmt
werden. Eine mögliche Methode basiert dabei auf dem Faraday-Effekt. Als Faraday-Effekt
wird die Drehung der Polarisationsebene von linear polarisiertem
Licht bei der Transmission in Materie bezeichnet. Die Grundlagen dieses Effektes und die
daraus folgende Bestimmung der effektiven Masse sollen im folgenden bestimmt werden.

\subsection{Effektive Masse}
\label{subsec:Masse}

% Für die Dicke D einstzen
Die Badstruktur eines Kristalls besitzt im Allgemeinen eine sehr komplizierte
Struktur. Für die Beschreibung einiger Eigenschaften reicht es jedoch, lediglich
den Verlauf der Energiedispersionsrelation $\varepsilon(\vec{k})$ an der unteren
Bandkante des Leitungsbandes zu kennen. Wird die Energiedispersionsrelation
um das Leitungsbandminimum bei $k=0$ entwickelt, ergibt sich unter Vergleich mit der
Energie eines freien Elektrons die Definition der effektiven Masse:
\begin{equation}
  \frac{1}{m^*} := \frac{1}{\hbar^2}\left\{\frac{\partial^2\varepsilon}{\partial k_i^2}\right\}_{k=0}, \qquad\text{mit }i = x, y, z.
  \label{eqn:effMass}
\end{equation}
Die Energiedispersionsrelation enthält somit die folgenden Terme:
\begin{equation}
  \varepsilon(\vec{k}) = \varepsilon(0) + \frac{\hbar^2k_x^2}{2m^*_1}  + \frac{\hbar^2k_y^2}{2m^*_2} + \frac{\hbar^2k_z^2}{2m^*_3}.
  \label{eqn:Energie}
\end{equation}
Für unterschiedliche $m^*_i$ ergibt sich im k-Raum somit ein Ellipsoid für Flächen
konstanter Energie und im Fall gleich großer Massen eine kugelförmige Fläche.
Es handelt sich dabei um die Energieeigenwerte der Schrödinger-Gleichung für
freie Elektronen, weshalb bei den folgenden Rechnungen die Quantenmechanik
freier Teilchen verwendet werden kann. Der Einfluss des periodischen Potentials
wird somit in der Schrödinger-Gleichung komplett durch die effektive Masse
berücksichtigt. Beim Anlegen äußerer Felder lässt sich das Verhalten gemäß
des Newtonschen Kraftgesetztes
\begin{equation}
  \vec{F} = m\cdot\vec{a}
  \label{eqn:Newton2}
\end{equation}
beschreiben, weshalb die Berechnung der Faraday-Rotation nach den Gesetzen der klassichen
Mechanik berechnet werden kann.

\subsection{Zirkulare Doppelbrechung}
\label{subsec:Doppelbrechung}

Die zirkulare Doppelbrechung beschreibt die Fähigkeit eines Materials, die Polarisationsebene
von linear polarisiertem Licht bei der Transmission zu drehen. Zur Erklärung dieses
Effektes wird im Folgenden eine linear polarisierte Lichtwelle betrachtet, welche
in einen linkszirkularen und einen rechtszirkularen Anteil zerlegt wird:
\begin{equation}
  E(z) = \frac{1}{2}(E\ua{R}(z)+E\ua{L}(z)) \; \text{mit} \; k\ua{R}\neq k\ua{L}.
  \label{eqn:Zirkular}
\end{equation}
Wird nun die Transmission der Welle durch den Kristall betrachtet, ergibt sich
für die Welle in Abhängigkeit von der durchlaufenen Länge mithilfe der Eulerschen
Formel folgende Gestalt:
\begin{equation}
  E(L) = E\ua{0}e^{i\Psi}(\cos{(\Theta)}\vec{x}\ua{0}+\sin{(\Theta)}\vec{y}\ua{0})
  \label{eqn:Trans}
\end{equation}
Die Abkürzungen $\Psi$ und $\Theta$ sind dabei wie folgt definiert:
\begin{align}
  \Psi &:= \frac{L}{2}(k\ua{R}+k\ua{L}),
  \label{eqn:psi} \\
  \Theta&:= \frac{L}{2}(k\ua{R}-k\ua{L}).
  \label{eqn:theta}
\end{align}
Mithilfe der Beziehungen $v\ua{Ph} = \omega/k$ sowie $n = c/v\ua{Ph}$ kann der
Winkel $\theta$ zudem über die Phasengeschwindigkeit und den Brechungsindex
ausgedrückt werden:
\begin{equation}
  \theta = \frac{L\omega}{2} \left\{ \frac{1}{v\ua{Ph,R}}-\frac{1}{v\ua{Ph,L}} \right\}
  = \frac{L\omega}{2c}(n\ua{R}-n\ua{L}).
  \label{eqn:thetavn}
\end{equation}
Die zirkulare Doppelbrechung lässt sich auf die induzierten Dipole innerhalb eines
Kristalls zurückführen. Die Dipole werden dabei einerseits durch die an den
Gitterplätzen sitzenden Atome erzeugt sowie durch die Wechselwirkung der Elektronen
mit den Atomrüpfen. Die Dipole erzeugen dann eine makroskopische Polarisation
des Kristalls, welche unter anderem über die dielektrische Suszeptibilität $\chi$
mit dem angelegten Feld verbunden ist:
\begin{equation}
  \vec{P} = \varepsilon\ua{0}\chi\vec{E}.
\end{equation}
$\chi$ ist im Falle eines isotropen Materials ohne äußeres Magnetfeld lediglich
ein Skalar, während es für anisotrope Materialien als Tensor definiert ist:
\begin{equation}
  (\chi) = \left(
  \begin{array}{rrr}
    \chi\ua{xx} & \chi\ua{xy} & \chi\ua{xz}  \\
    \chi\ua{yx} & \chi\ua{yy} & \chi\ua{yz}  \\
    \chi\ua{zx} & \chi\ua{zy} & \chi\ua{zz}  \\
  \end{array}\right)
  \label{chi:Allg}
\end{equation}
Der Tensor kann jedoch meistens durch eine Hauptachsentransformation in eine
diagonale Form gebracht werden. Unter der Bedingung, dass auf der nicht-diagonalen
zwei zueinander komplex konjugierte Koeffizienten stehen, wird Materie doppelbrechend.
Der ($\chi$)-Tensor hat dabei im einfachsten Fall die Form
\begin{equation}
  (\chi) = \left(
  \begin{array}{rrr}
    \chi\ua{xx} & i\chi\ua{xy} & 0  \\
    -i\chi\ua{yx} & \chi\ua{xx} & 0  \\
    0 & 0 & \chi\ua{zz}  \\
  \end{array}\right).
  \label{eqn:chiIso}
\end{equation}
Unter Verwendung der Wellengleichung in Kombination mit der dielektrischen Verschiebung
\begin{equation}
  \vec{D} = \varepsilon\ua{0}\vec{E}+\vec{P}
  \label{eqn:dielektrVersch}
\end{equation}
können nun die möglichen Wellenzahlen einer ebenen Welle bestimmt werden. Dabei
zeigt sich, dass bei $\omega\neq 0$ und $\chi\ua{zz}\neq 0$ auch $E\ua{z}=0$
gelten muss. Es handelt sich also um eine transversale Welle. Aus den beiden
möglichen Wellenzahlen
\begin{equation}
  k\ua{\pm} = \frac{\omega}{c}\sqrt{(1+\chi\ua{xx})\pm\chi\ua{xy}}
  \label{eqn:kpm}
\end{equation}
ergeben sich somit auch zwei verschiedene Phasengeschwindigkeiten:
\begin{equation}
  v\ua{Ph,R} = \frac{c}{\sqrt{1+\chi\ua{xx}+\chi\ua{xy}}} \;\; \text{und} \;\;
  v\ua{Ph,L} = \frac{c}{\sqrt{1+\chi\ua{xx}-\chi\ua{xy}}}.
  \label{eqn:PhasenV}
\end{equation}
Unter der im Allgemeinen zutreffenden Annahme, dass $\chi\ua{xy} << 1 + \chi\ua{xx}$
gilt, ergibt sich gemäß Formel \eqref{eqn:theta} der Zusammenhang
\begin{equation}
  \theta \approx \frac{L\omega}{2c}\left\{\sqrt{1+\chi\ua{xx}}\right\}^{-1}\chi\ua{xy}
  = \frac{L\omega}{2c^2}v\ua{Ph}\chi\ua{xy} = \frac{l\omega}{2cn}\chi\ua{xy},
  \label{eqn:ThetaXxy}
\end{equation}
wobei es sich bei $v\ua{Ph}$ um die Phasengeschwindigkeit mit $\chi\ua{xy}=0$
handelt.

\subsection{Bestimmung des Rotationswinkels \theta}
\label{subsec:Theta}

In dem folgenden Kapitel soll gezeigt werden, dass optisch inaktive Materie
unter Einwirkung eines äußeren Magnetfeldes die Polarisationsebene von parallel
zur Feldrichtung verlaufenden Licht dreht. Hierbei werden gebundene Elektronen
betrachtet, welche der Bewegungsgleichung
\begin{equation}
  m\frac{\su{d^2}\vec{r}}{\su{d}t^2} + K\vec{r} = -e\ua{0}\vec{E}(r)- e\ua{0}\frac{\su{d}\vec{r}}{\su{d}t}
  \label{eqn:BewGl}
\end{equation}
gehorchen. $K$ ist hier eine Konstante, welche die Bindung des um $\vec{r}$ ausgelenkten
Elektrons an die Umgebung beschreibt. Dämpfungseffekte sowie der Einfluss des
Magnetfeldes der elektromagnetischen Welle werden dabei aufgrund des geringen
Einflusses vernachlässigt. Aufgrund der exponentiellen Zeitabhängigkeit des
elektrischen Feldes lässt sich unter mit Hinzunahme der Polarisation die
Formel \eqref{eqn:BewGl} in drei Komponenten zerlegen:
\begin{align}
  (-m\omega^2+K)P\ua{x} &= N e\ua{0}^2E\ua{x} + i e\ua{0}\omega P\ua{y}B
  \label{eqn:Kompx}\\
  (-m\omega^2+K)P\ua{y} &= N e\ua{0}^2E\ua{y} + i e\ua{0}\omega P\ua{x}B
  \label{eqn:Kompy}\\
  (-m\omega^2+K)P\ua{z} &= N e\ua{0}^2 E\ua{z}.
  \label{eqn:Kompz}
\end{align}
Damit eine nicht-triviale Lösung des Gleichungssystemes existiert und die Einträge
von $\chi$ unabhängig von den Feldstärkekomponenten sind, müssen auch nicht-diagonale,
imaginäre Einträge existieren. Nach Zerlegung von \eqref{eqn:Kompx} in Real-
und Imaginärteil ergibt sich zwischen den beiden nicht-diagonalen Komponenten
der Zusammenhang
\begin{equation}
  \label{eqn:RelationChiNonDiag}
  \chi\ua{xy}=-\chi\ua{yx}.
\end{equation}
Die Tensorkomponente $\chi\ua{xy}$ lässt sich aus den voherigen Gleichungen
bestimmen (vgl. \cite{Anleitung}), sodass sich für den Rotationswinkel nach
Formel \eqref{eqn:ThetaXxy} der folgende Zusammenhang ergibt:
\begin{equation}
  \theta = \frac{e\ua{0}^3}{2\varepsilon\ua{0}c}\frac{\omega^2}{(-m\omega^2+K^2)^2 - (e\ua{0}\omega B)^2}\frac{NBL}{n}.
  \label{eqn:thetafinal}
\end{equation}
Der Rotationswinkel \theta ist somit proportional zu Flussdichte B, der Probenlänge L
und der Anzahl Ladungsträger N pro Volumeneinheit.
Um die komplizierte Frequenzabhängigkeit zu betrachten werden die Frequenzen
$\omega\ua{0} = \sqrt{K/m}$ und $\omega\ua{c} = B e\ua{0}/m$ eingeführt. Bei
$\omega\ua{0}$ handelt es sich um die Resonanzfrequenz und bei $\omega\ua{c}$ um
die Zyklotronfrequenz. Für $B\approx 1T$ liegt $\omega\ua{c}$ im Bereich von
$\SI{e11}{Hertz}$. $\omega\ua{0}$ und die Messfrequenz liegen im nahen Infrarotbereich
($\omega = \num{e14}-\SI{e15}{Hertz}$), weshalb im Allgemeinen
\begin{equation}
  (\omega\ua{0}^2-\omega^2)^2 >> \omega^2\omega\ua{c}^2
  \label{eqn:wRelation}
\end{equation}
gilt. Unter der Bedingung $\omega << \omega\ua{0}$ sowie unter Verwendung
von $\omega = 2\pi / \lambda$ ergibt sich für $\theta$
\begin{equation}
  \theta(\lambda) \approx \frac{2\pi e\ua{0}^3c}{\varepsilon\ua{0}}\frac{1}{m^2\lambda^2\omega\ua{0}^4}\frac{NBL}{n}.
\end{equation}
Mithilfe der Näherung $\omega\ua{0}\rightarrow 0$ kann auch der Rotationswinkel
für freie Ladungsträger approximiert werden:
\begin{equation}
  \label{eqn:m_eff}
  \theta\ua{frei} \approx \frac{e\ua{0}^3}{8\pi^2\varepsilon\ua{0}c^3}\frac{1}{m^2}\lambda^2\frac{NBL}{n}.
\end{equation}
Da die bestimmten Formeln für $\theta$ unter bestimmten Voraussetzungen auch für
Kristallelektronen gültig bleibt (ersetze $m$ durch $m^*$), kann nun aus dem
Rotationswinkel $\theta$ die effektive Masse bestimmt werden.

\subsection{Experimentelle Hinweise}

\subsubsection{Interferenzfilter}

Da die im Aufbau verwendete Halogen-Lampe mehrere Wellenlängen emittiert, muss
zur Selektion einzelner Wellenlängen ein Interferenzfilter verwendet werden. Dieser
besteht grundlegend aus einem optissch transparenten Dielektrikum mit Brechungsindex
n sowie zwei semitransparenten Beschichtungen. Einfallende Lichtstrahlen werden
aufgrund der reflektierenden Beschichtungen im Inneren des Filters mehrfach
reflektiert, sodass es zu Interferenzen kommt. Dabei ensteht konstruktive
Interferenz ausschließlich für Wellenlängen, die der Bedingung
\begin{equation}
  j \lambda\ua{j} = 2nd + \frac{\lambda}{2} + \frac{\lambda}{2}^+
\end{equation}
gehorchen. Der 3. Summand muss nur hinzugenommen werden, wenn an den Beschichtungen
ein Phasensprung von $\pi$ erfolgt. Der Reflexionkoeffizient R ist < 1, weshalb
in dem Filter nur eine begrenzte Anzahl an Reflexionen auftritt. Deshalb werden
dicht neben $\lambda\ua{j}$ liegende Wellenlängen nicht komplett ausgelöscht. Die
Durchlasskurve hat somit eine endliche Breit, welche stark vom Reflexionskoeffizient
abhängt. Die Transmission in Abhänhigkeit der Wellenlänge wird dabei durch
die Airysche Funktion beschrieben:
\begin{equation}
  T(\lambda) = \frac{1}{1+\frac{4R}{(1-R)^2}\sin^2(\frac{2\pi nd}{\lambda})}.
\end{equation}
Der Verlauf der Kurve ist für drei verschiedene Reflexionskoeffizienten in
Abbildung \ref{fig:Airy} dargestellt.
\begin{figure}
  \centering
  \includegraphics[width=\textwidth]{Pics/Airy.pdf}
  \caption{Verlauf der Transmissionskurve eines Interferenzfilter für verschiedene
  Reflexionskoeffizienten. \cite{Anleitung}}
  \label{fig:Airy}
\end{figure}

\subsubsection{Glan-Thompson-Prisma}

Als Polarisatoren werden in diesem Experiment Glan-Thompson-Prismen (GTP's) verwendet.
Es handelt sich dabei um doppelbrechende Kristalle, bei denen die Diagonalelemente
der dielektrischen Suszeptibilität unterschiedlich sind. Analog zu Kapitel
\ref{subsec:Doppelbrechung} können zur Bestimmung der Phasengschwindigkeit
drei Komponentegleichungen formuliert werden:
\begin{align}
  k^2E\ua{x} &= \frac{\omega^2}{c^2}(1+\chi\ua{xx})E\ua{x}
  \label{eqn:GTPx}\\
  k^2E\ua{y} &= \frac{\omega^2}{c^2}(1+\chi\ua{yy})E\ua{y}
  \label{eqn:GTPy}\\
  0 &= \frac{\omega^2}{c^2}(1+\chi\ua{zz})E\ua{z}
  \label{eqn:GTPz}
\end{align}
Aus Formel \eqref{eqn:GTPz} folgt erneut $E\ua{z} = 0$. Für $E\ua{x}\neq 0$ ergibt
sich
\begin{equation}
  v\ua{Ph,1} = \frac{c}{\sqrt{1+\chi\ua{xx}}}.
  \label{eqn:vPhx}
\end{equation}
Unter dieser Annahme lässt sich Gleichung \eqref{eqn:GTPy} wegen $\chi\ua{xx}\neq\chi\ua{yy}$
nur für $E\ua{y}=0$ lösen. Somit handelt es sich hier um eine linear polarisierte
Welle. Analog ergibt sich $E\ua{x}=0$, wenn zuerst $E\ua{y}\neq 0$ gewählt wird.
Es handelt sich also um eine Aufspaltung in zwei Strahlen, die aufgrund unterschiedlicher
Phasengeschwindigkeiten senkrecht zueinander polarisiert sind.
Bei der allgemeinen Lösung der Wellengleichung ergibt sich für den außerordentlichen
Strahl eine Richtungsabhängigkeit. Die Achse, entlang derer
der Brechungsindex für beide Strahlen gleich ist und somit keine Doppelbrechung
auftritt, wird auch optische Achse genannt.

Zur Trennung der beiden Strahlen
wird ein doppelbrechender Kristall parallel zu seiner optischen Achse zerschnitten
(siehe Abbildung \ref{fig:GTP}).
Trifft der Lichtstrahl im Brewsterwinkel auf die Grenzfläche zur Luftschicht,
wird aufgrund der unterschiedlichen Brechungsindizes lediglich der ordentliche
Strahl totalreflektiert, während der außerordentliche Strahl transmittiert wird.
Im Falle von $n\ua{ao}<n\ua{o}$ gilt für den Winkel $\alpha$ dann folgende
Ungleichung:
\begin{equation}
  \frac{1}{n\ua{o}} < \sin{(\alpha)} < \frac{1}{n\ua{ao}}
  \label{eqn:Alpha}
\end{equation}
\begin{figure}
  \centering
  \includegraphics{Pics/GTP.pdf}
  \caption{Schematischer Aufbau eines Glan-Thompson-Prismas (GTP)\cite{Anleitung}.}
  \label{fig:GTP}
\end{figure}

\newpage
\section{Aufbau und Durchführung}

\begin{figure}
  \centering
  \includegraphics[width = \textwidth]{Pics/Aufbau.pdf}
  \caption{Schematischer Aufbau des Experimentes\cite{Anleitung}.}
  \label{fig:Aufbau}
\end{figure}

Der grundlegende Aufbau des Experiments ist in Abbildung \ref{fig:Aufbau}
dargestellt. Als Lichtquelle wird eine Halogen-Lampe verwendet, deren Lichtstrahlen
mittels einer Kondensorlinse parallelisiert werden. Darauf folgt ein Lichtzerhacker,
der das Licht mit einer einstellbaren Frequeuz modelliert. Als nächstes durchläuft
der Strahl das erste GTP, welches mittels eines Goniometers gedreht werden kann.
Mit dem GTP wird das für die Messung notwendige linear-polarisierte Licht erzeugt.
Anschließend folgt ein Elektromagnet, in dessen Inneren sich die betrachtete
Probe befindet. Aufgrund des relativ geringen Homogenitätbereiches wird eine
scheibenförmige Probe verwendet, die über einen seitlichen Einlass in das
Magnetfeld gebracht werden kann. Um eine bestimmte Wellenlänge des Lichtstrahls
zu filtern, befindet sich hinter dem Magneten ein Interferenzfilter.

Bei diesem Experiment wird zur genaueren Bestimmung des Rotationswinkels das
sogenannte Zweistrahlverfahren verwendet. Dafür wird der resultierende Strahl
nach dem Interferenzfilter in zwei Strahlen aufgeteilt, die von zwei verschiedenen
Photodioden detektiert werden. Mittels eines Differenzverstärkers wird dann die
Differenz der beiden Signale bestimmt und anschließend über ein Oszillographen
angezeigt. Der Oszillograph hat bei geeigneter Orientierung der GTP die Funktion
eines Nulldetektors. Zur Minimierung von Hintegrundrauschen wird zusätzlich ein Selektivverstärker
zwischengeschaltet. Dieser wird auf dieselbe Frequenz wie der Lichtzerhacker
eingestellt, so dass nur das mit dieser Frequenz modulierte Signale gemessen werden.

Vor dem Experiment muss die Apparatur justiert werden. Dafür wird zuerst die Funktion
der beiden Glan-Thompson-Prisma überprüft, indem sie so zueinder verdreht werden,
dass hinter dem 2. GTP kein Licht mehr zu sehen ist. Anschließend wird kurz überprüft
ob die Lichtstrahlen auf die Sensoren der Photowiderstände fallen. Zur Überprüfung
des Selektivverstärkers wird der Lichtzerhacker eingeschaltet und die Frequenz
am Selektivverstärker variiert, bis das Signal auf dem Oszilloskop maximal ist.
Zum Abschluss der Justierung wird mit eingesetzter Probe und Interferenzfilter
überprüft, ob wirklich das Nullsignal erreicht wird. Dafür werden alternierend
das erste GTP und die Zeitkonstante des Photowiderstandes variiert.

Zu Beginn der eigentlichen Messreihe wird mittels einer Hallsonde die Magnetfeldstärke
vermessen. Dafür wird das Magnetfeld auf den maximalen Strom eingestellt und in einem
Bereich von einigen Zentimetern um die Probenstelle, das Feld vermessen. Anschließend
wird für eine hochreine GaAs-Probe die Faraday-Rotation vermessen. Dafür wird
bei maximalem Feld das erste GTP so gedreht, dass das Signal am Oszilloksop
minimiert ist. Danach wird das Feld einmal umgepolt und das GTP erneut so gedreht,
dass das Signal am Oszilloskop minimal wird. Die beiden Winkel werden notiert und
die selbe Messung wird für insgesamt 8 verschiedene Interferenzfilter
durchgeführt. Bei den beiden Filtern mit der höchsten Wellenlänge wird zudem
die angelegte Spannung an der Halogen-Lampe etwas reduziert. Die Messreihe wird
anschließend noch mit zwei verschiedenen dotierten GaAs-Proben wiederholt.
