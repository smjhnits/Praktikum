\section{Theorie}

In dem folgenden Versuch soll die effektive Masse von Elektronen in GaAs bestimmt
werden. Eine mögliche Methode basiert dabei auf dem Faraday-Effekt. Als Faraday-Effekt
bezeichnet man die Drehung der Polarisationsebene von linear polarisiertem
Licht bei der Transmission in Materie. Die Grundlagen dieses Effektes und die
daraus folgende Bestimmung der effektiven Masse sollen im folgenden bestimmt werden.

\subsection{Effektive Masse}

% Für die Dicke D einstzen
Die Badstruktur eines Kristalles besitzt im Allgemeinen eine sehr komplizierte
Struktur. Für die Beschreibung einiger Eigenschaften reicht es jedoch, lediglich
den Verlauf der Energiedispersionsrelation $\varepsilon(\vec{k})$ an der unteren
Bandkante des Leitungsbandes zu kennen. Entwickelt man die Energiedispersionsrelation
um das Leitungsbandminimum bei $k=0$, ergibt sich unter Vergleich mit der
Energie eines freien Elektrons die Definition der effektiven Masse:
\begin{equation}
  \frac{1}{m^*} := \frac{1}{\hbar^2}\left\{\frac{\partial^2\varepsilon}{\partial\varepsilon^2}\right\}_{k=0}.
  \label{eqn:effMass}
\end{equation}
Die Energiedispersionsrelation enthält somit die folgenden Terme:
\begin{equation}
  \varepsilon(\vec{k}) = \varepsilon(0) + \frac{\hbar^2k_x^2}{2m^*_1}  + \frac{\hbar^2k_y^2}{2m^*_2} + \frac{\hbar^2k_z^2}{2m^*_3}.
  \label{eqn:Energie}
\end{equation}
Für unterschiedliche $m^*_i$ ergibt sich im k-Raum somit ein Ellipsoid für Flächen
konstanter Energie und im Fall gleich großer Massen eine kugelförmige Fläche.
Es handelt sich dabei um die Energieeigenwerte der Schrödinger-Gleichung für
freie Elektronen, weshalb bei den folgenden Rechnungen die Quantenmechanik
freier Teilchen verwendet werden kann. Der Einfluss des periodischen Potentials
wird somit in der Schrödinger-Gleichung komplett durch die effektive Masse
berücksichtigt. Beim Anlegen äußerer Felder lässt sich das Verhalten gemäß
des Newtonschen Kraftgesetztes
\begin{equation}
  \vec{F} = m\cdot\vec{a},
  \label{eqn:Newton2}
\end{equation}
weshalb die Berechnung der Faraday-Rotation nach den Gesetzen der klassichen
Mechanik berechnet werden kann.

\subsection{Zirkulare Doppelbrechung}

Die zirkulare Doppelbrechung beschreibt die Fähigkeit eines Materials, die Polarisationsebene
von linear polarisiertem Licht bei der Transmission zu drehen. Zur Erklärung dieses
Effektes wird im folgenden eine linear polarisierte Lichtwelle betrachtet, welche
in einen linkszirkularen und einen rechtszirkularen Anteil zerlegt wird:
\begin{equation}
  E(z) = \frac{1}{2}(E\ua{R}(z)+E\ua{L}(z)) \; \text{mit} \; k\ua{R}\neq k\ua{L}.
  \label{eqn:Zirkular}
\end{equation}
