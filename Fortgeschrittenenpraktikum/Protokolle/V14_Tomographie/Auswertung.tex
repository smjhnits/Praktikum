\section{Auswertung}

Für die Auswertung wird zur Berechnung der verschiedenen Werte das Programm
\emph{Python} mit dem Packet \emph{numpy} verwendet. Am Ende wird zur Berechnung
des Mittelwertes der Absorptionskoeffizienten die automatische Fehlerrechnung mit
dem Packet \emph{uncertainties} durchgeführt. In den Abschnitten, in denen die
Fehler per Hand mithilfe der Gauß´schen Fehlerfortpflanzung (\ref{eqn:Gauß})
berechnet werden, ist die verwendete Fehlerformel mit angegeben.

\begin{equation}
  \sigma y = \sqrt{ \sum_{i} \left(\frac{\partial y}{\partial x_i} \cdot \sigma x_i\right)^2 }
  \label{eqn:Gauß}
\end{equation}

Für die Auswertung werden bei den einzelnen Messungen nur die Counts in einem
bestimmten Intervall verwendet. Die verwendeten Intervalle sowie die Anzahl
an Counts, deren Fehler und die Messzeit sind in Kapitel \label{subsec:Absorption}
in den Tabellen \ref{tab:Luft} bis \ref{tab:Unbekannt} dargestellt. Als Fehler
wird dabei immer die Wurzel der gemessenen Counts verwendet.

\subsection{Spektrum von \ce{^{137}Cs}}

In Abbildung \ref{fig:Spektrum} ist das aufgenommene Spektrum bei der Messung
der Projektion $I_6$ des mit Luft gefüllten Würfels dargestellt. Es ist deutlich
ein Plateau zwischen den Kanälen ?? und ?? zu erkennen.

% Hier fehlt noch das Spektrum

\subsection{Bestimmung der Absorptionskoeffizienten}
\label{subsec:Absorption}

Um die Intensität der einzelnen Projektionen zu bestimmen, werden die gemessenen
Counts bei jeder Projektion mittels der Messzeit normiert. Der Fehler der
Rate ergibt sich dann über Gauß'sche Fehlerfortpflanzung. Die Messzeit wird
hierbei als Fehlerfrei angenommen.

\begin{align}
   I\ua{j} &= \frac{N\ua{j}}{T\ua{j}} \\
   \sigma\ua{I\ua{j}} &= \frac{1}{T\ua{j}} \cdot \sigma\ua{N\ua{j}} .
\end{align}

Über die Rate lässt sich dann auch die wie folgt definierte Größe $y$ bestimmen:

\begin{align}
  y\ua{j} &= \su{ln}\left(\frac{I\ua{0}}{I\ua{j}}\right)  \\
  \sigma\ua{y\ua{j}} &= \sqrt{ \left( \frac{\sigma\ua{I}}{I}\right)^2 + \left(\frac{\sigma\ua{I\ua{0}}}{I\ua{0}}\right)^2} .
\end{align}

Der Fehler ergibt sich dabei ebenfalls aus der Gauß'schen Fehlerfortpflanzung. Mit
$I\ua{0}$ ist hierbei die gemessene Rate bei einem leeren Würfel gemeint. Diese
wurde für jede Art von Projektion einmal gemessen ($I\ua{2}$, $I\ua{3}$ und $I\ua{6}$).

\begin{table}
  \centering
  \caption{$I\ua{0}$ für die verschiedenen Projektionen des leeren Würfels.}
  \label{tab:Luft}
  \begin{tabular}{c | c c c c c c}
    \toprule
    Projektion & Kanäle & $T$ in s & $N$ & $\sigma\ua{N}$ & $I$ in $\su{cm}^{-1}$ & $\sigma\ua{I}$ in $\su{cm}^{-1}$ \\
    \midrule
    $I\ua{2}$ & 55-62 & 87 & 16042 & 127 & 184,4 & 1,5 \\
    $I\ua{3}$ & 55-62 & 87 & 16115 & 127 & 185,2 & 1,5 \\
    $I\ua{6}$ & 55-62 & 83 & 15341 & 124 & 184,8 & 1,5 \\
    \bottomrule
  \end{tabular}
\end{table}

\begin{table}
  \centering
  \caption{$I$ für die verschiedenen Projektionen des Aluminiumwürfels.}
  \label{tab:Alu}
  \begin{tabular}{c | c c c c c c c c}
    \toprule
    Projektion & Kanäle & $T$ in s & $N$ & $\sigma\ua{N}$ & $I$ in $\su{cm}^{-1}$
    & $\sigma\ua{I}$ in $\su{cm}^{-1}$ & $y$ & $\sigma\ua{y}$ \\
    \midrule
    $I\ua{1}$  & 63-69 & 84 & 8206 & 91 & 97,7  & 1,1 & 0,64 & 0,01 \\
    $I\ua{2}$  & 62-68 & 30 & 2268 & 48 & 75,6  & 1,6 & 0,89 & 0,02 \\
    $I\ua{3}$  & 61-66 & 26 & 2135 & 46 & 82,1  & 1,8 & 0,81 & 0,02 \\
    $I\ua{4}$  & 59-65 & 26 & 2336 & 48 & 89,9  & 1,9 & 0,72 & 0,02 \\
    $I\ua{5}$  & 59-65 & 30 & 2837 & 53 & 94,6  & 1,8 & 0,67 & 0,02 \\
    $I\ua{6}$  & 59-65 & 24 & 2186 & 47 & 91,1  & 2,0 & 0,71 & 0,02 \\
    $I\ua{7}$  & 59-65 & 24 & 2138 & 47 & 89,1  & 1,9 & 0,73 & 0,02 \\
    $I\ua{8}$  & 59-65 & 29 & 2204 & 47 & 76,0  & 1,6 & 0,89 & 0,02 \\
    $I\ua{9}$  & 58-65 & 34 & 3057 & 55 & 89,9  & 1,6 & 0,72 & 0,02 \\
    $I\ua{10}$ & 58-64 & 26 & 2504 & 50 & 96,3  & 1,9 & 0,65 & 0,02 \\
    $I\ua{11}$ & 58-64 & 24 & 2228 & 47 & 92,8  & 2,0 & 0,69 & 0,02 \\
    $I\ua{12}$ & 57-64 & 23 & 2446 & 50 & 106,3 & 2,2 & 0,55 & 0,02 \\
    \bottomrule
  \end{tabular}
\end{table}

\begin{table}
  \centering
  \caption{$I$ für die verschiedenen Projektionen des Bleiwürfels.}
  \label{tab:Blei}
  \begin{tabular}{c | c c c c c c c c}
    \toprule
    Projektion & Kanäle & $T$ in s & $N$ & $\sigma\ua{N}$ & $I$ in $\su{cm}^{-1}$
    & $\sigma\ua{I}$ in $\su{cm}^{-1}$ & $y$ & $\sigma\ua{y}$ \\
    \midrule
    $I\ua{1}$  & 57-64 & 110 & 1218 & 35 & 11,1 & 0,3 & 2,81 & 0,05 \\
    $I\ua{2}$  & 57-64 & 375 & 1197 & 35 & 3,2  & 0,1 & 4,06 & 0,10 \\
    $I\ua{3}$  & 57-64 & 203 & 1203 & 35 & 5,9  & 0,2 & 3,44 & 0,07 \\
    $I\ua{4}$  & 57-64 & 161 & 1198 & 35 & 7.4  & 0,2 & 3,21 & 0,06 \\
    $I\ua{5}$  & 57-64 & 183 & 1205 & 35 & 6,6  & 0,2 & 3,33 & 0,07 \\
    $I\ua{6}$  & 56-62 & 187 & 1206 & 35 & 6,4  & 0,2 & 3,36 & 0,07 \\
    $I\ua{7}$  & 56-62 & 99  & 1212 & 35 & 12,2 & 0,4 & 2,72 & 0,05 \\
    $I\ua{8}$  & 56-62 & 424 & 1197 & 35 & 2,8  & 0,1 & 4,18 & 0,10 \\
    $I\ua{9}$  & 56-62 & 182 & 1240 & 35 & 6,8  & 0,2 & 3,30 & 0,07 \\
    $I\ua{10}$ & 56-62 & 124 & 1255 & 35 & 10,1 & 0,3 & 2,90 & 0,05 \\
    $I\ua{11}$ & 56-62 & 192 & 1207 & 35 & 6,3  & 0,2 & 3,38 & 0,07 \\
    $I\ua{12}$ & 56-61 & 195 & 1207 & 35 & 6,2  & 0,2 & 3,40 & 0,07 \\
    \bottomrule
  \end{tabular}
\end{table}

\begin{table}
  \centering
  \caption{$I$ für die verschiedenen Projektionen des unbekannten Würfels.}
  \label{tab:Unbekannt}
  \begin{tabular}{c | c c c c c c c c}
    \toprule
    Projektion & Kanäle & $T$ in s & $N$ & $\sigma\ua{N}$ & $I$ in $\su{cm}^{-1}$
    & $\sigma\ua{I}$ in $\su{cm}^{-1}$ & $y$ & $\sigma\ua{y}$ \\
    \midrule
    $I\ua{1}$  & 51-65 & 822  & 12575 & 112 & 15,3  & 0,1 & 2,82 & 0,03 \\
    $I\ua{2}$  & 51-65 & 1503 & 12758 & 113 & 8,5   & 0,1 & 4,06 & 0,03 \\
    $I\ua{3}$  & 51-65 & 690  & 12622 & 112 & 18,3  & 0,2 & 3,44 & 0,02 \\
    $I\ua{4}$  & 52-64 & 648  & 12916 & 114 & 19,9  & 0,2 & 3,21 & 0,02 \\
    $I\ua{5}$  & 52-64 & 700  & 12965 & 114 & 18,5  & 0,2 & 3,33 & 0,02 \\
    $I\ua{6}$  & 52-64 & 683  & 13018 & 114 & 19,1  & 0,2 & 3,36 & 0,02 \\
    $I\ua{7}$  & 52-64 & 341  & 12905 & 114 & 37,8  & 0,3 & 2,72 & 0,02 \\
    $I\ua{8}$  & 52-64 & 1345 & 12840 & 113 & 9,5   & 0,1 & 4,18 & 0,03 \\
    $I\ua{9}$  & 52-64 & 1193 & 12898 & 114 & 10,8  & 0,1 & 3,30 & 0,03 \\
    $I\ua{10}$ & 51-63 & 125  & 12888 & 114 & 103,1 & 0,9 & 2,90 & 0,01 \\
    $I\ua{11}$ & 49-63 & 1407 & 12510 & 112 & 8,9   & 0,1 & 3,38 & 0,03 \\
    $I\ua{12}$ & 49-63 & 799  & 12536 & 112 & 15,7  & 0,1 & 3,40 & 0,03 \\
    \bottomrule
  \end{tabular}
\end{table}


\newpage

Gemäß der Formel \eqref{eqn:kovarianz_I} lässt sich nun die Varianzmatrix $V[\vec{y}]$
bestimmen. Mit ihr wird gemäß Formel \eqref{eqn:kovarianz_mu} die Varianzmatrix
der Absoprtionskoeffizienten bestimmt. Mit den Einträgen auf der Diagonale
lassen sich die Fehler der einzelnen Absorptionskoeffizienten bestimmen.
Zudem werden über Formel \eqref{eqn:mu} die Absorptionskoeffizienten bestimmt. Die
Fehler der einzelnen Koeffizienten lassen sich aus den Diagonalelementen von
$V[\vec{\mu}]$ entnehmen.
Bei Aluminum und Blei wird mithilfe der automatischen Fehlerrechnung von Python
der Mittelwert der Absorptionskoeffizienten bestimmt. In Tabelle \ref{tab:AluUndBlei}
sind die bestimmten Koeffizienten, deren Fehler sowie die Abweichung zu dem
Literaturwert zu sehen.

\begin{table}
  \centering
  \caption{Bestimmte Absorptionskoeffizienten von Aluminium und Blei und
  die Abweichung zu den Literaturwerten. \cite{Koeffs}}
  \label{tab:AluUndBlei}
  \begin{tabular}{c | c c c c}
    \toprule
    Material & $\mu\ua{exp}$ in $\su{cm}^{-}$ & $\sigma\ua{mu\ua{exp}}$ in
    $\su{cm}^{-}$ & $\mu\ua{lit}$ in $\su{cm}^{-}$ & $\increment \mu$
    in $\su{cm}^{-}$ \\
    \midrule
    Aluminium & 0,227 & 0,003 & 0,202 & 0,0248 $\pm$ 0,003 \\
    Blei      & 1,059 & 0,013 & 1,25  & 0,191 $\pm$ 0,013 \\
    \bottomrule
  \end{tabular}
\end{table}


Für den Würfel unbekannter Zusammensetzung ergeben sich so ebenfalls die einzelnen
Absorptionskoeffizienten. Um die einzelnen Würfel einem bestimmten Material zuzuordnen
sind in Tabelle \ref{tab:UnbekanntErgebnis} die Absorptionskoeffizienten sowie deren
absolute Abweichung
zu den Experimentell bestimmten Werten und den Literaturwerten eingetragen. Die Abweichung
wird dabei durch $\increment M$ bezeichnet, wobei $M$ immer der zum Abgleich gezogenen
Koeffizient ist. Die Fehler der Abweichung sind hier nicht von interesse und werden
deswegen nicht mit angegeben.

\begin{table}
  \centering
  \caption{Bestimmte Koeffizienten für den unbekannten Würfel sowie die Abweichungen
  zu den anderen Materialien ($|\increment M|$ in $\su{cm}^{-1}$).}
  \label{tab:UnbekanntErgebnis}
  \begin{tabular}{c | c c c c c c c}
    \toprule
    $\mu\ua{i}$ & $\mu$ in $\su{cm}^{-}$ & $\sigma\ua{\mu}$ in $\su{cm}^{-}$ &
    $|\increment \mu\ua{Alu,lit}|$ &
    $|\increment \mu\ua{Alu,exp}|$ &
    $|\increment \mu\ua{Blei,lit}|$ &
    $|\increment \mu\ua{Blei,exp}|$ &
    $\increment\ua{min}$ \\
    \midrule
    $\mu{1}$ & 0,05 & 0,01 & 0,15 & 0,18 & 1,20 & 1,01 & $\mu\ua{Alu}$  \\
    $\mu{2}$ & 1,29 & 0,01 & 1,09 & 1,07 & 0,04 & 0,23 & $\mu\ua{Blei}$ \\
    $\mu{3}$ & 0,82 & 0,02 & 0,61 & 0,59 & 0,43 & 0,24 & $\mu\ua{Blei}$ \\
    $\mu{4}$ & 0,28 & 0,01 & 0,08 & 0,06 & 0,97 & 0,76 & $\mu\ua{Alu}$  \\
    $\mu{5}$ & 1,08 & 0,01 & 0,88 & 0,86 & 0,17 & 0,02 & $\mu\ua{Blei}$ \\
    $\mu{6}$ & 0,92 & 0,01 & 0,62 & 0,59 & 0,43 & 0,24 & $\mu\ua{Blei}$ \\
    $\mu{7}$ & 0,28 & 0,01 & 0,08 & 0,05 & 0,97 & 0,78 & $\mu\ua{Alu}$  \\
    $\mu{8}$ & 0,87 & 0,01 & 0,67 & 0,65 & 0,38 & 0,19 & $\mu\ua{Blei}$ \\
    $\mu{9}$ & 0,96 & 0,02 & 0,76 & 0,74 & 0,29 & 0,10 & $\mu\ua{Blei}$ \\
    \bottomrule
  \end{tabular}
\end{table}


\section{Diskussion}
