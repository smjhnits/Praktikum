\section{Background}

In the following experiment we will use a diode laser to measure the absorption
spectrum of rubidium. Therefore, in the following chapters, the basic principle
of a general laser system will be considered before discussing the specific
properties of a diode laser.

\subsection{General lasing system}

In this chapter we will look at the main interaction between light and a
quantum mechanical system like an atom. In atoms the electrons posess different
dicreete energy levels. Nontheless, to discuss the main principle of a laser
we will use a system with only two energy levels shown in Figure \ref{fig:twolevelsystem}.
By absorbing a photon, an electron can be excited from the lower state to the
higher state. The relaxation back into the ground state can be radiative or
nonradiative.

In nonradiative relaxation processes the remaining energy will be converted into
heat by phonon scattering. In a radiative relaxation process the energy will
be displaced by emitting a photon. This can happen spontanous (Figure \ref{fig;twolevelsystem} b))
or stimulated by an incoming photon (Figure \ref{fig;twolevelsystem} b)).
Therefore the energy of the incoming photon must be the difference between the
two states. As a result there will be two photons emitted with the same phase,
direction and energy. This process called 'stimulated emission' is necessary
for constructing a lasing system. To reach a higher propability for 'stimulated emission'
than 'spontanous emission' the population of the second state has to be higher
than he the ground state. This situation is called 'population inversion'.

Although the principle of a lasing sytem was explained by using a two state system,
for constructing a real laser at least three energy levels are necessary. The possesion
of a single energy level is given by the maxwell-boltzmann-distribution. The only
parameter which can be changed is the temperature. It´s easy to see, that even
by heating up the system to an infinite temperature, only an equal possesion
of both states in a two level sytsem is possible. In the three level system
shown in figure \ref{fig:threestatesystem},
electrons will be excited into the highest state by using a external pump source.
The lifetime of the second state has to be much lower than the lifetime of the
first state to create a population inversion. The relaxation from the first state
into the ground state is called 'lasing transition'.
