\section{Auswertung}
\label{sec:auswertung}

In dem Versuch wird die Molwärme $C\ua{p}$ nach Formel~\eqref{eqn:c_p}
bestimmt. Die in Kapitel~\ref{sec:tabellen} enthaltene Tabelle~\ref{tab: c_p}
zeigt die Ergebnisse der Wärmekapazität $C\ua{p}$ für verschiedene Temperaturen.

\subsection{Verlauf der Temperaturen}

Der Zylinder und die Probe werden durch zwei seperate Heizungen erwärmt.
Im Idealfall sind die beiden Temperaturen gleich, sodass die Wärmestrahlung der
Probe durch die Wärmestrahlung des Zylinders kompensiert wird. Experimentell ergeben
sich jedoch Abweichungen der Temperaturen.
Der Verlauf der beiden Temperaturen ist in Abb.~\ref{fig:temp}
dargestellt.

\begin{figure}
  \centering
  \includegraphics[width = 0.8\textwidth]{Plots/temp.pdf}
  \caption{Temperaturverläufe des Zylinders und der Probe gegenüber der Zeit.
  Häufig Überlappen die Probentemperatur $\T_{Probe}$ und die Zylindertemperature $\T_{Zylinder}$,
  sodass nur $\T_{Zylinder}$ abgebildet wird.}
  \label{fig:temp}
\end{figure}


\subsection{Wärmekapazität $C\ua{V}$}

Aus der Wärmekapazität bei konstantem Druck $C\ua{p}$ kann
durch den Zusammenhang~\eqref{eqn:c_p-c_v} die Wärmekapazität bei
konstantem Volumen $C\ua{V}$ berechnet werden.
Für das Kompressionsmodul $\kappa$ wird der Wert $\kappa = \SI{140}{\giga\pascal}$
angenommen~\cite{kompression}.
Für den Ausdehnungskoeffizienten $\alpha(T)$ sind Werte in Quelle~cite{anleitung}
gegeben. Diese werden durch eine Ausgeichsrechnung an ein Polynom dritten Grades
angenähert. Der Verlauf der Ausgeichsfunktion mit den Daten aus Quelle~\cite{anleitung}
ist in Abb.~\ref{fig:alpha} dargestellt.
Der Verlauf der aus den Messdaten errechneten Wärmekapazität $C\ua{V}$
für verschiedene Temperaturen $T$ ist in Abb.~\ref{fig:c_v} erkenntlich.

\begin{figure}[h]
  \centering
  \includegraphics[width = 0.8\textwidth]{Plots/alpha_T.pdf}
  \caption{Ausgeichsfunktion des Ausdehnungskoeffizienten für steigende Temperaturen.}
  \label{fig:alpha}
\end{figure}

\begin{figure}
  \centering
  \includegraphics[width = 0.8\textwidth]{Plots/C_V.pdf}
  \caption{Daten der Wärmekapazität für steigende Temperaturen.}
  \label{fig:c_v}
\end{figure}
\FloatBarrier
Es wird ersichtlich, dass in Abb.~\ref{fig:c_v} Werte auftauchen, die sehr
stark erwarteten Trend abweichen.
Diese Abweichung wird auf zwei vermutlich fehlerhafte Messdaten in der Zeitmessung zurückgeführt.
Eine Darstellung von $C\ua{V}$ ohne diese beiden Messdaten ist in Abb.~\ref{fig:c_v_korrektur}
einzusehen.

\begin{figure}
  \centering
  \includegraphics[width = 0.8\textwidth]{Plots/C_V_korrektur.pdf}
  \caption{Angepasste Daten der Wärmekapazität für steigende Temperaturen.}
  \label{fig:c_v_korrektur}
\end{figure}

\subsection{Debye-Funktion}

In Quelle~\cite{anleitung} sind die Werte der Funktion $C\ua{V} = f\left(\frac{\Theta\ua{D}}{T}\right)$
angegeben. Die ermittelten Wärmekapazitäten $C\ua{V}(T)$ können
mit den Daten der Wertetabelle verglichen werden, sodass
der zugehörige $\left(\frac{\Theta\ua{D}}{T}\right)$--Wert nach Multiplikation
mit der Temperatur $T$ die Debye-Temperatur ergibt.
Die gefundenen Debye-Temperaturen sind in Tab.~\ref{tab:debye} angegeben.
Es werden nur Wärmekapazitäten bis zu $T = \SI{170}{\kelvin}$ berücksichtigt.\\
Der experimentelle Wert der Debye-Temperatur $\Theta\ua{D, exp}$ ergibt sich aus dem Mittelwert
der Daten aus Tabelle~\ref{tab:debye}.

\begin{equation}
  \label{eqn:debye}
  \Theta\ua{D, exp} = \SI{279.21(6811)}{\kelvin}
\end{equation}

\subsection{Berechnung von $\omega\ua{D}$ und $\Theta\ua{D}$}

Die Debye-Frequenz $\omega\ua{D}$ wird wie in dem Abschnitt zu Formel~\eqref{eqn:omega_D}
beschrieben errechnet.
Für die einzelnen Größen ergeben sich die folgenden Werte.

\begin{align*}
  m\ua{Probe} &= \SI{342}{\gram} \text{~\cite{anleitung}}\\
  M\ua{Cu} &= \SI{63.55}{\dalton}\cdot N\ua{A} \text{~\cite{cu_mass}}\\
  v\ua{l} &= \SI{4.7}{\meter\per\second}\\
  v\ua{t} &= \SI{2.26}{\meter\per\second}\\
  N_L &= \num{3.24e24} \\
  V &= L^3 = V_0\cdot\frac{m}{M}\approx \SI{0.382}{\centi\meter^3}\\
  \omega\ua{D} &\approx \SI{43.5e12}{\per\second}\\
  \Theta\ua{D} &\approx \SI{322.21}{\kelvin}
\end{align*}

Der Mittelwert der experimentell gefundenen Debye-Temperaturen $\Theta\ua{D, exp}$ ist ca. $\num{13.34}\,\%$ kleiner als der Theoriewert $\Theta\ua{D}$.

\section{Diskussion}
\label{sec:diskussion}



\section{Tabellen}
\label{sec:tabellen}
\pagestyle{empty}
\begin{table}[b!]
\vspace{-10pt}
\centering
\caption{Messdaten zu der Wärmekapazität $C\ua{p}$}
\label{tab: c_p}
\begin{tabular}{S S S S S }
\toprule
{$C\ua{p} / \si{\joule \per \kelvin \per \mol}$} & {$U / \si{\volt}$} & {$ I / \si{\milli\ampere}$} & {$ \increment t / \si{\s}$} & {$ \increment T / \si{\kelvin}$}  \\
\midrule
 15.79  & 15.79  & 150.90  & 210  & 5.89\\
14.68  & 15.87  & 151.55  & 163  & 4.96\\
15.99  & 15.85  & 151.25  & 187  & 5.21\\
16.12  & 15.80  & 150.65  & 173  & 4.75\\
16.36  & 15.82  & 150.70  & 202  & 5.47\\
16.27  & 15.86  & 150.90  & 192  & 5.25\\
18.54  & 15.88  & 151.10  & 179  & 4.30\\
17.93  & 15.89  & 151.25  & 202  & 5.03\\
17.85  & 15.88  & 151.00  & 202  & 5.04\\
19.39  & 15.87  & 150.75  & 210  & 4.81\\
19.78  & 15.89  & 150.85  & 225  & 5.07\\
19.77  & 15.91  & 150.95  & 225  & 5.08\\
19.30  & 15.91  & 151.00  & 220  & 5.09\\
21.34  & 15.93  & 151.05  & 232  & 4.86\\
20.83  & 15.89  & 150.70  & 228  & 4.87\\
21.10  & 15.86  & 150.35  & 244  & 5.12\\
21.23  & 15.88  & 150.45  & 234  & 4.89\\
6.18  & 15.88  & 150.50  & 75  & 5.39\\
40.64  & 15.89  & 150.50  & 427  & 4.67\\
21.59  & 15.89  & 150.55  & 251  & 5.17\\
22.28  & 15.91  & 150.60  & 247  & 4.94\\
22.24  & 15.91  & 150.65  & 247  & 4.95\\
23.28  & 15.91  & 150.70  & 272  & 5.20\\
22.85  & 15.91  & 150.70  & 293  & 5.71\\
23.45  & 15.92  & 150.70  & 262  & 4.98\\
21.34  & 15.92  & 150.70  & 203  & 4.24\\
23.27  & 15.92  & 150.75  & 287  & 5.50\\
23.45  & 15.92  & 150.80  & 303  & 5.76\\
22.99  & 15.93  & 150.80  & 207  & 4.02\\
23.74  & 15.93  & 150.80  & 241  & 4.53\\
27.25  & 15.92  & 150.80  & 308  & 5.04\\
24.28  & 15.92  & 150.85  & 275  & 5.05\\
25.18  & 15.92  & 150.90  & 300  & 5.32\\
24.24  & 15.92  & 150.90  & 248  & 4.57\\
24.32  & 15.92  & 150.90  & 277  & 5.08\\
24.35  & 15.92  & 150.90  & 264  & 4.84\\
23.43  & 15.92  & 150.90  & 268  & 5.11\\
24.34  & 15.92  & 150.90  & 265  & 4.86\\
25.42  & 15.91  & 150.90  & 292  & 5.13\\
25.10  & 15.91  & 150.95  & 289  & 5.14\\
23.82  & 15.91  & 151.00  & 261  & 4.89\\
24.78  & 15.91  & 151.00  & 315  & 5.67\\
24.79  & 15.91  & 151.00  & 244  & 4.39\\
23.68  & 15.91  & 151.00  & 261  & 4.92\\
\bottomrule
\end{tabular}
\end{table}

\begin{table}[H]
\centering
\caption{Ergebnisse der Wärmekapazität $C_V$.}
\label{tab:c_v}
\begin{tabular}{S S }
\toprule
{$C_V / \si{\joule \per \kelvin \per \mol}$} & {$ \alpha / 10^6\si{\per\kelvin}$}  \\
\midrule
 15.72  & 8.40\\
14.61  & 9.05\\
15.90  & 9.57\\
16.02  & 10.07\\
16.24  & 10.50\\
16.14  & 10.97\\
18.40  & 11.38\\
17.78  & 11.70\\
17.68  & 12.05\\
19.21  & 12.37\\
19.58  & 12.66\\
19.55  & 12.94\\
19.07  & 13.20\\
21.09  & 13.44\\
20.57  & 13.64\\
20.82  & 13.84\\
20.93  & 14.02\\
5.86  & 14.19\\
40.31  & 14.35\\
21.24  & 14.48\\
21.92  & 14.61\\
21.87  & 14.73\\
22.89  & 14.84\\
22.44  & 14.94\\
23.03  & 15.05\\
20.90  & 15.13\\
22.82  & 15.20\\
22.98  & 15.29\\
22.50  & 15.38\\
23.25  & 15.44\\
26.74  & 15.50\\
23.76  & 15.58\\
24.64  & 15.66\\
23.68  & 15.75\\
23.74  & 15.83\\
23.75  & 15.92\\
22.81  & 16.02\\
23.70  & 16.13\\
24.76  & 16.24\\
24.42  & 16.37\\
23.12  & 16.52\\
24.04  & 16.67\\
24.02  & 16.86\\
22.89  & 17.02\\
\bottomrule
\end{tabular}
\end{table}

\begin{table} 
\centering 
\caption{Gefundene Werte von $\theta_{\symup{D}}$} 
\label{tab:debye} 
\begin{tabular}{S S S } 
\toprule  
{$\theta_{\symup{D}} / \si{\kelvin}$} & {$C_V / \si{\joule \per \kelvin \per \mol}$} & {$T / \si{\kelvin}$}  \\ 
\midrule  
272.20  & 15.72  & 85.06\\ 
310.58  & 14.61  & 90.02\\ 
299.98  & 15.90  & 95.23\\ 
129.97  & 16.02  & 99.98\\ 
137.09  & 16.24  & 105.45\\ 
143.91  & 16.14  & 110.70\\ 
293.25  & 18.40  & 115.00\\ 
324.09  & 17.78  & 120.03\\ 
337.70  & 17.68  & 125.08\\ 
305.24  & 19.21  & 129.89\\ 
310.40  & 19.58  & 134.96\\ 
322.08  & 19.55  & 140.03\\ 
348.30  & 19.07  & 145.12\\ 
284.97  & 21.09  & 149.98\\ 
309.70  & 20.57  & 154.85\\ 
303.95  & 20.82  & 159.97\\ 
313.25  & 20.93  & 164.87\\ 
\bottomrule 
\end{tabular} 
\end{table}
