\section{Auswertung}

Im Folgenden werden die erhobenen Messdaten ausgewertet, mit dem Ziel
die Lande-Faktoren der aufgespalteten roten und blauen Spektrallinie bei verschiedener
Polarisation zu bestimmen.

\subsection{Hysterese}

Der aufsteigenden Hysteresekurve wurde eine Polynomgleichung dritten Gerades zugesprochen,
welche die folgenden Einträge hat.
\begin{equation}
  \label{eqn:Hysterese_fit}
  A_1 \cdot x^3 + A_2 \cdot x^2 + A_3 \cdot x + A_4
\end{equation}
\begin{description}
  \item[$A_1$] $\SI{-0.082(7)}{\milli\tesla\per\ampere^3}$
  \item[$A_2$] $\SI{1.757(236)}{\milli\tesla\per\ampere^2}$
  \item[$A_3$] $\SI{49.121(2104)}{\milli\tesla\per\ampere}$
  \item[$A_4$] $\SI{14.152(4984)}{\milli\tesla}$
\end{description}

Für die absteigende Hysteresekurve wurde ebenfalls eine Augleichsrechnung an die
Gleichung \eqref{eqn:Hysterese_fit} betrieben.
\begin{description}
  \item[$A_1$] $\SI{-0.067(6)}{\milli\tesla\per\ampere^3}$
  \item[$A_2$] $\SI{1.217(191)}{\milli\tesla\per\ampere^2}$
  \item[$A_3$] $\SI{54.665(1703)}{\milli\tesla\per\ampere}$
  \item[$A_4$] $\SI{6.666(4033)}{\milli\tesla}$
\end{description}

Die Ausgleichsrechnungen wurden mit Hilfe des \emph{Python}-Paketes
\emph{SciPy-$curve_fit$} durchgeführt.

Die Ausgleichsfunktionen sind im Folgenden graphisch dargestellt.

\begin{figure}
  \centering
  \includegraphics[width=\textwidth]{Python/Hysterese.pdf}
  \caption{Messdaten der Hysteresekurve mit eingetragenen Ausgleichsfunktionen.
  Dabei wurde die gemessene B-Feldstärke in mT gegenüber dem Strom I in A aufgetragen}
  \label{fig:Hysterese}
\end{figure}

\begin{figure}
  \centering
  \includegraphics[width=\textwidth]{Python/Hysterese_Messdaten.pdf}
  \caption{Messdaten der gemessenen B-Feldstärke in Abhängigkeit von der Stromstärke}
  \label{fig:Hysterese_Messdaten}
\end{figure}

Die in den Grafiken verwendeten Daten sind in Tabelle \ref{tab:Hysterese} dargestellt.

\subsection{Auswertung der roten Spektrallinie}



\section{Messdaten}

Die gemessenen Daten zur Erstellung der Hysteresekurve sind in der folgenden Tabelle
dargestellt.

\begin{table}[h]
\centering
\caption{Messdaten der Hysterese}
\label{tab:Hysterese}
\begin{tabular}{S S S }
\toprule
{Stromstärke in  $\si{\ampere}$} & {B-Feldstärke aufsteigend in  $\si{\milli\tesla}$} & {B-Feldstärke aufsteigend in  $\si{\milli\tesla}$}  \\
\midrule
 0  & 4  & 7\\
1  & 87  & 57\\
2  & 112  & 120\\
3  & 174  & 180\\
4  & 230  & 251\\
5  & 290  & 306\\
6  & 352  & 361\\
7  & 419  & 428\\
8  & 476  & 480\\
9  & 540  & 550\\
10  & 600  & 612\\
11  & 662  & 654\\
12  & 714  & 715\\
13  & 775  & 780\\
14  & 823  & 830\\
15  & 872  & 878\\
16  & 916  & 924\\
17  & 959  & 962\\
18  & 987  & 993\\
19  & 1015  & 1020\\
20  & 1046  & 1050\\
21  & 1072  & 1072\\
\bottomrule
\end{tabular}
\end{table}

