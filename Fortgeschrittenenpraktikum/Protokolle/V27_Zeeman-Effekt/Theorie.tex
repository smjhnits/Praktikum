\section{Theorie}

In dem Folgenden Versuch sollen die bei Einwirken eines Magnetfeldes auf ein Atom
entstehende Aufspaltung und Polarisation von emittierten Spektrallinien untersucht
werden. Dies geschieht anhand der blauen und roten Spektrallinien einer
Cadmium Lampe.

\subsection{Das magnetische Moment}

Aufgrund der Quantenmechanik ist bekannt, dass jedes Hüllenelelektron eines Atoms
zwei verschiedene Drehimpulse besitzt, den Bahndrehimpuls $\vec{l}$ sowie den
Spin $\vec{s}$, welcher auch als Eigendrehimpuls bezeichnet wird. Beiden Impulsen
wird eine für das Atom charakterisierende Quantenzahl zugewiesen mit der sich
die Beträge errechnen lassen:

\begin{align}
  \lvert \vec{l} \rvert  &= \sqrt{l(l+1)} \hbar \,\, (l=0,1,2...,n-1) \\
  \lvert \vec{s} \rvert  &= \sqrt{s(s+1)} \hbar \,\, (s=\frac{1}{2}).
\end{align}

Aufgrund der Ladung des Elektrons lässt sich mit dem Bahndrehimpuls ein magnetisches
Moment $\vec{\mu\ua{l}}$ verknüpfen. Mithilfe des Stern-Gerlach-Experimentes wurde
zudem nachgewiesen, dass aufgrund des Spins ebenfalls ein magnetisches Moment
$\vec{\mu\ua{s}}$ definiert werden kann. Beide Momente sind neben dem Betrag der
Impulse auch Abhängig von $\hbar$ sowie dem Bohrschen Magneton $\mu\ua{b}$ abhängig.

\begin{align}
  \mu\ua{B} &:= - \frac{1}{2} e\ua{0} \frac{\hbar}{m\ua{0}} \\
  \vec{\mu\ua{l}} &= - \mu\ua{B} \frac{\vec{l}}{\hbar} = - \mu\ua{B}\sqrt{l(l+1)}\vec{l\ua{e}} \\
  \vec{\mu\ua{s}} &= - g\ua{s}\mu\ua{B} \frac{\vec{s}}{\hbar} = - g\ua{s}\mu\ua{B}\sqrt{s(s+1)}\vec{s\ua{e}}
\end{align}

Beide Momente besitzen zudem den Landé-Faktor g, welcher beim Bahndrehipmuls 1 und
beim Spin 2 ist. Aufgrund dessen ist bei s=l $\vec{\mu\ua{s}}$ doppelt so groß wie
$\vec{\mu\ua{l}}$, was auch als magnetomechanische Anomalie bezeichnet wird.

\subsection{Wechselwirkungen im Atom}

In einem Mehrelektronenatom treten die vorher eingeführten verschiedenen Größen
aufgrund mehrerer Elektronen
in Wechselwirkung zueinander, weswegen im folgenden die zwei in der Natur am häufigsten
realisierten Grenzfälle betrachtet werden. Charakterisiert werden diese dabei
mittel der Kernladungszahl $Z$.

Bei Atomen mit einer relativ geringen Kernladungszahl ist die Wechselwirkung der
einzelnen $\vec{l\ua{i}}$ untereinander so groß, dass sich
ein Gesamtbahndrehimpuls $\vec{L}$ definieren lässt:

\begin{equation}
  \vec{L} = \sum \vec{l\ua{i}} \;\; \su{mit} \;\; \lvert\vec{L}\rvert=\sqrt{L(L+1)} \hbar.\\
\end{equation}

Da bei abgeschlossenen Schalen der Bahndrehimpuls stets Null ist, müssen hier nur
Elektronen der äußeren Schalen betrachtet werden. Für die Werte 0,1,2,3 der
Quantenzahl L werden die Bezeichnungen S,P,D und F zugeordnet.

Besitzt das Atom ebenfalls eine nicht allzu hohe Ordnungszahl, dann lässt sich
zudem ein Gesamptspin $\vec{S}$ definieren.

\begin{equation}
  \vec{S} = \sum \vec{s\ua{i}} \;\; \su{mit} \;\; \lvert\vec{S}\rvert=\sqrt{S(S+1)} \hbar\\
\end{equation}

Somit ändert sich für beide auch das magnetische Moment:

\begin{align}
  \lvert\vec{\mu\ua{B}}\rvert &= \mu\ua{L}\sqrt{L(L+1)} \\
  \lvert\vec{\mu\ua{B}}\rvert &= g\ua{S}\mu\ua{S}\sqrt{S(S+1)}.
\end{align}

Aufgrund des in der relativistischen Betrachtung vom Proton erzeugtem Kreiststromes
kommt es zu einer Spin-Bahn-Kopplung (\textbf{LS-Kopplung}), welche bei nicht
allzu hohen Feldstärken
auch bestehen bleibt. Somit kann ein neuer Drehimpuls als Summe von Spin- und
Bahndrehimpuls definiert werden:

\begin{align}
  \vec{J} &= \vec{L} + \vec{S} \\
  \lvert \vec{J} \rvert &= \sqrt{J(J+1)}\hbar.
\end{align}

Um die verschiedenen Größen zu notieren verwendet man folgende Schreibweise:

\begin{equation}
  \ce{^{2s+1}L}\ua{J}.
\end{equation}


Bei dem zweiten Grenzfall beobachtet man schwere Atome. Hier dominieren die Wechselwirkungen
zwischen den $\vec{l\ua{i}}$ und den $\vec{s\ua{i}}$, sodass sich kein Gesamptspin
und Gesamtdrehimpuls mehr definieren lassen. Deshalb bezeichnet man diese Fall auch
als \textbf{j-j-Kopplung}. Für den Gesamtdrehimpuls ergibt sich daher folgendes:

\begin{equation}
  \vec{J} = \sum \vec{j\ua{i}} = \sum \vec{l\ua{i}} + \vec{s\ua{i}}.
\end{equation}

Zwischen beiden Fällen besteht ein fließender Übergang bei mittleren Kernladungszahlen.

\subsection{Aufspaltung im homogenen Magnetfeld}

Das zu J gehörende magnetische Moment ist die Summe der magnetischen Momente vo
Spin- und Bahndrehimpuls. Da $\vec{\mu\ua{J}}$ und $\vec{J}$ im Allgemeinen nicht
aufeinander fallen, lassen sich eine zu $\vec{J}$ senkrechte sowie eine parallele
Komponente von $\vec{\mu}$ definieren. Im klassischen Sinne führt $\mu$ dabei eine
Präzissionsbewegung um die Gesamtdrehimpulsachse aus, weswegen der Erwartungswert
von $\vec{\mu\ua{\perp}}$ verschwindet. Für $\lvert \vec{\mu\ua{J}} \rvert$ ergibt sich
somit folgende Formel:

\begin{align}
  \lvert \vec{\mu\ua{J}} \rvert &= \mu\ua{B}g\ua{J}\sqrt{J(J+1)} \\
  g\ua{J} &= \frac{3J(J+1)+S(S+1)-L(L+1)}{2J(J+1)}.
\end{align}

$g\ua{J}$ ist dabei der Landé-Faktor des entsprechenden Atoms. Aufgrund der
Quantenmechanik muss im Folgenden die sogenannte Richtungsquantelung beachtet werden.
Sie besagt, dass die in Feldrichtung zeigende Komponente von $\lvert \vec{\mu\ua{J}} \rvert$
ein ganzzahliges Vielfaches von $g\ua{J}\mu\ua{B}$ darstellt. Verknüpft wird dies
mit dem Faktor m, welche auch Orientierungsquantenzahl genannt wird:

\begin{equation}
  \mu\ua{J\ua{z}} = - mg\ua{J}\mu\ua{B} \;\; \su{mit}  \;\; m \in [-J,-J+1,...,0,1,...,J].
\end{equation}

Somit existieren 2J+1 Einstellmöglichkeiten für $\vec{\mu}$ relativ zu $\vec{B}$.
Die Zusatzenergie aufgrund des äußeren Feldes ergibt sich somit zu folgendem Wert:

\begin{equation}
  E\ua{mag} = m g\ua{J}\mu\ua{B}B .
\end{equation}

Da diese Aufspaltung auch bei angeregten Zuständen auftritt, wird bei dem Einschalten
des Magnetfeldes auch eine Aufspaltung der emittierten Spektrallinien erwartet.
Die Anzahl der aufgespalteten Linien wird dabei durch Auswahlregeln festgelegt,
da Übergänge nicht zwischen allen Energieniveaus möglich sind.

\subsection{Auswahlregeln für Energieübergänge}

Die grundliegende Gleichung bei der Betrachtung mittels der Quantenmechanik ist
die Schrödingergleichung. Aufgrund ihrer Linearität können mehrer Lösungen superponiert
werden. Um die Auswahregeln zu bestimmen, geht man von einer Lösung mit zwei
verschiedenen Wellenfunktionen aus:

\begin{equation}
  \Psi\ua{ges}(\vec{r},t) = C\ua{\alpha}\psi\ua{\alpha}(\vec{r})\exp{-\frac{i}{\hbar}E\ua{\alpha}t}
  +  C\ua{\beta}\psi\ua{\beta}(\vec{r})\exp{-\frac{i}{\hbar}E\ua{\beta}t}.
\end{equation}

Im folgenden sollen nun die Energieübergänge zwischen diesen beiden Zuständen
untersucht werden. Die Dichteverteilung dieser Wellenfunktion ist eine zeitabhängige
Größe, welche ein Schwingung des Elektrons mit einer durch die Energiedifferenz
festgelegten Frequenz $\nu\ua{\alpha,\beta}$ beschreibt.

\begin{equation}
  \nu\ua{\alpha,\beta} := \frac{E\ua{\alpha} - E\ua{\beta}}{2}
\end{equation}

Um die Intensität der emittierten Strahlung zu berechnen, muss man das durch die
Schwingung hervorgerufenen Dipolmoment des Elektrons betrachten. Der Beitrag des
Volumenelemts dV zu der X-Komponente des Dipols beträgt dabei

\begin{equation}
  - e\ua{0}\su{x}\psi^*\psi \su{dV}
  \label{eqn:Dipol_X}
\end{equation}

wobei der Beitrag zu Y- und Z-Komponente analog ist. Um das gesamte Dipolmoment
in diese Richtung zu berechnen, muss $\eqref{eqn:Dipol_X}$ über den gesamten
Raum integriert werden. Durch verschiedene Rechenschritte lässt sich der Beitrag
in X-Richtung zu folgender Form vereinfachen:

\begin{equation}
  D\ua{x} = - e\ua{0} \, \su{const} \, 2 \, \symbffrak{Re} \left( \int \su{x} \psi\ua{\beta}^*
  \psi\ua{\alpha} \su{dV} \exp{2 \pi i \nu\ua{\alpha,\beta}t} \right).
\end{equation}

Der Term beschreibt einen mit der Frequenz $\nu\ua{\alpha,\beta}$ schwingenden
Dipol der Quanten entsprechender Energie abstrahlt.
Die Komponenten in die beiden anderen Raumrichtungen lassen sich analog berechnen.
Um die Strahlungsemission zu berechnen wird der Poynting-Vektor mit den Matrixelementen
$x\ua{\alpha,\beta}$, $y\ua{\alpha,\beta}$ und $z\ua{\alpha,\beta}$ bestimmt, wobei
$\gamma$ den Winkel zwischen Ausbreitungsrichtung der Strahlung und Dipolmoment
beschreibt.

\begin{align}
  \su{x}\ua{\alpha,\beta} &:= \int \su{x} \psi\ua{\alpha}^*\psi\ua{\beta} \su{dV} \\
  \su{y}\ua{\alpha,\beta} &:= \int \su{y} \psi\ua{\alpha}^*\psi\ua{\beta} \su{dV} \\
  \su{z}\ua{\alpha,\beta} &:= \int \su{z} \psi\ua{\alpha}^*\psi\ua{\beta} \su{dV} \\
  \lvert \vec{S\ua{\alpha,\beta}} \rvert &= \left( \lvert x\ua{\alpha,\beta} \rvert^2
  + \lvert y\ua{\alpha,\beta} \rvert^2 + \lvert z\ua{\alpha,\beta} \rvert^2 \right)
  \sin^2{\gamma}
  \label{eqn:Poynting}
\end{align}

Als Wellenfunktion für ein Atom im Magnetfeld wird der Ansatz

\begin{equation}
  \Psi = \frac{1}{\sqrt{2\pi}}\su{R(r)\Theta(\vartheta)} e^{im\varphi}
\end{equation}

gewählt. Das Magnetfeld ist in Richtung der z-Achse ausgerichtet. Untersucht man
nun die einzelnen Komponenten des Dipols, ist erkenntlich, dass die z-Komponente
nur dann nicht verschwindet, wenn $m\ua{\alpha}$ = $m\ua{\beta}$ gilt. Somit ist
die erste Auswahlregel, dass für eine nicht verschwindende Komponente in z-Richtung
die beiden Orientierungsquantenzahlen der Zustände $E\ua{\alpha}$ und $E\ua{\beta}$
gleich sein müssen.

Für die x- und y-Komponente erhält man analog zwei Auswahregeln, bei denen die
Dipolkomponenten nicht verschwinden. Es muss entweder $m\ua{\beta}$ = $m\ua{\alpha}$
+ 1 ($\increment m = -1$) oder $m\ua{\beta}$ = $m\ua{\alpha}$ - 1 ($\increment m = +1$)
gelten. Zusammengefasst tritt die Emission von Spektrallinien beim Zeeman-Effekt
nur dann auf, wenn sich die Orientierungsquantenzahlen der beiden Zustände
entweder gar nicht oder um ein Differenz von $\pm$1 unterscheiden.

Gilt nun $\increment m = 0$, dann ist lediglich die z-Komponente des Dipols von
Null verschieden. Der Dipol schwingt also parallel zur Magnetfeldrichtung und
emittiert linear polarisiertes Licht. Zudem strahlt der Dipol aufgrund der Winkelabängigkeit
in $\eqref{eqn:Poynting}$ nicht in Feldrichtung und besitzt seine stärkste Emission
senkrecht zu dieser.

In den Fällen $\increment m = \pm 1$ folgt beides mal $\su{z}\ua{\alpha,\beta}=0$,
wobei die x- und y-Komponenten folgenden Zusammenhang besitzen:

\begin{align}
  \su{x}\ua{\alpha,\beta} = -i\su{y}\ua{\alpha,\beta} \;\; (\increment m = -1) \\
  \su{x}\ua{\alpha,\beta} = i\su{y}\ua{\alpha,\beta} \;\; (\increment m = +1).
\end{align}

In beiden Fällen ergibt sich somit ein Phasenverschub von $\frac{\pi}{2}$ zwischen
x- und y-Komponente. Bei der emittierten Strahlung handelt es sich also um zirkular-
polarisiertes Licht wobei die beiden Fälle eine entgegensätzliche Drehrichtung besitzen.

\subsection{Der normale Zeeman-Effekt}

In den vorangegangenen Rechnungen wurde der Elektronenspin nicht berücksichtigt,
so dass lediglich der Fall behandelt wird, bei dem für das
Atom $S=0$ und somit $g\ua{J}=0$ gilt. Dieser Spezialfall wird historisch bedingt
Zeeman-Effekt genannt. Hier ist die Verschiebung der Energieniveaus somit
unabhängig von den Quantenzahlen L und J, so dass die Verschiebung stehts den
Wert

\begin{equation}
  \increment E = m \mu\ua{B}B \;\; \su{für} \;\; -J \leq m \leq J
  \label{eqn:EdiffZeemanNormal}
\end{equation}

besitzt. In Abbildung \ref{fig:normalerZeeman} ist diese Aufspaltung für die
Fälle $J=1$ und $J=2$ beispielhaft dargestellt.

\begin{figure}
  %\includegraphics{/path/to/figure}
  \caption{Aufspaltung und Polarisation der emittierten Spektrallinien beim
  normalen Zeeman-Effekt}
  \label{fig:normalerZeeman}
\end{figure}

Alle Übergänge mit gleichem $\increment m$ werden in eine Liniengruppe zusammegefasst,
wobei innerhalb dieser Gruppen die Energiedifferenz konstant ist. Diese Aufspaltung
in 3 Komponenten wird auch \textbf{"Zeeman-Triplett"} genannt. Da die aufgrund des
Übergangs mit $\increment m = 0$ hervorgerufene Strahlung parallel zur Magnetfeldrichtung
linear polarisiert ist, lässt sie sich in voller Intensität nur senkrecht zur
Feldrichtung beobachten, während sie parallel zur Feldrichtung nicht beobachtbar
ist. Gemäß Formel \eqref{eqn:EdiffZeemanNormal} ist ihre
Energie gegenüber dem feldfreien Fall nicht verändert. Sie wird als $\pi$-Komponente
bezeichnet.

Die beiden zirkular poloarisierten Linien mit $\increment m = \pm 1$ erscheinen
bei der transversalen Betrachtung linear polarisiert und können deswegen nicht
unterschieden werden. Die Linien besitzen immer ein Energiedifferenz von
$\mu\ua{B}B$ bezogen auf die unverschobenen Linien und werden als $\sigma$-Komponente
bezeichnet. Die beobachtbare Aufspaltung bei den zwei verschiedenen Betrachtungswinkeln
ist in Abbildung \ref{fig:Betrachtungswinkel} noch einmal grob dargestellt.

\begin{figure}
  %\includegraphics{/path/to/figure}
  \caption{Sichtbare Aufspaltung bei transversaler und longitudinaler Betrachtung
  der emittierten Spektrallinien}
  \label{fig:Betrachtungswinkel}
\end{figure}

\subsection{Der anomale Zeeman-Effekt}
