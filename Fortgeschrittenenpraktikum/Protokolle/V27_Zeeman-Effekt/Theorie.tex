\section{Theorie}

In dem Folgenden Versuch sollen die bei Einwirken eines Magnetfeldes auf ein Atom
entstehende Aufspaltung und Polarisation von emittierten Spektrallinien untersucht
werden. Dies geschieht anhand der blauen und roten Spektrallinien einer
Cadmium Lampe.

\subsection{Das magnetische Moment}

Aufgrund der Quantenmechanik ist bekannt, dass jedes Hüllenelelektron eines Atoms
zwei verschiedene Drehimpulse besitzt, den Bahndrehimpuls $\vec{l}$ sowie den
Spin $\vec{s}$, welcher auch als Eigendrehimpuls bezeichnet wird. Beiden Impulsen
wird eine für das Atom charakterisierende Quantenzahl zugewiesen mit der sich
die Beträge errechnen lassen:

\begin{align}
  \lvert \vec{l} \rvert  &= \sqrt{l(l+1)} \hbar \,\, (l=0,1,2...,n-1) \\
  \lvert \vec{s} \rvert  &= \sqrt{s(s+1)} \hbar \,\, (s=\frac{1}{2}).
\end{align}

Aufgrund der Ladung des Elektrons lässt sich mit dem Bahndrehimpuls ein magnetisches
Moment $\vec{\mu\ua{l}}$ verknüpfen. Mithilfe des Stern-Gerlach-Experimentes wurde
zudem nachgewiesen, dass aufgrund des Spins ebenfalls ein magnetisches Moment
$\vec{\mu\ua{s}}$ definiert werden kann. Beide Momente sind neben dem Betrag der
Impulse auch Abhängig von $\hbar$ sowie dem Bohrschen Magneton $\mu\ua{b}$ abhängig.

\begin{align}
  \mu\ua{B} &:= - \frac{1}{2} e\ua{0} \frac{\hbar}{m\ua{0}} \\
  \vec{\mu\ua{l}} &= - \mu\ua{B} \frac{\vec{l}}{\hbar} = - \mu\ua{B}\sqrt{l(l+1)}\vec{l\ua{e}} \\
  \vec{\mu\ua{s}} &= - g\ua{s}\mu\ua{B} \frac{\vec{s}}{\hbar} = - g\ua{s}\mu\ua{B}\sqrt{s(s+1)}\vec{s\ua{e}}
\end{align}

Beide Momente besitzen zudem den Landé-Faktor g, welcher beim Bahndrehipmuls 1 und
beim Spin 2 ist. Aufgrund dessen ist bei s=l $\vec{\mu\ua{s}}$ doppelt so groß wie
$\vec{\mu\ua{l}}$, was auch als magnetomechanische Anomalie bezeichnet wird.

\subsection{Wechselwirkungen im Atom}

In einem Mehrelektronenatom treten die vorher eingeführten verschiedenen Größen
aufgrund mehrerer Elektronen
in Wechselwirkung zueinander, weswegen im folgenden die zwei in der Natur am häufigsten
realisierten Grenzfälle betrachtet werden. Charakterisiert werden diese dabei
mittel der Kernladungszahl $Z$.

Bei Atomen mit einer relativ geringen Kernladungszahl ist die Wechselwirkung der
einzelnen $\vec{l\ua{i}}$ untereinander so groß, dass sich
ein Gesamtbahndrehimpuls $\vec{L}$ definieren lässt:

\begin{equation}
  \vec{L} = \sum \vec{l\ua{i}} \,\, mit \,\, \lvert\vec{L}\rvert=\sqrt{L(L+1)} \hbar.\\
\end{equation}

Da bei abgeschlossenen Schalen der Bahndrehimpuls stets Null ist, müssen hier nur
Elektronen der äußeren Schalen betrachtet werden. Für die Werte 0,1,2,3 der
Quantenzahl L werden die Bezeichnungen S,P,D und F zugeordnet.

Besitzt das Atom ebenfalls eine nicht allzu hohe Ordnungszahl, dann lässt sich
zudem ein Gesamptspin $\vec{S}$ definieren.

\begin{equation}
  \vec{S} = \sum \vec{s\ua{i}} \,\, mit \,\, \lvert\vec{S}\rvert=\sqrt{S(S+1)} \hbar\\
\end{equation}

Somit ändert sich für beide auch das magnetische Moment:

\begin{align}
  \lvert\vec{\mu\ua{B}}\rvert &= \mu\ua{L}\sqrt{L(L+1)} \\
  \lvert\vec{\mu\ua{B}}\rvert &= g\ua{S}\mu\ua{S}\sqrt{S(S+1)}.
\end{align}

Aufgrund des in der relativistischen Betrachtung vom Proton erzeugtem Kreiststromes
kommt es zu einer Spin-Bahn-Kopplung ($\textbf{LS-Kopplung}$), welche bei nicht
allzu hohen Feldstärken
auch bestehen bleibt. Somit kann ein neuer Drehimpuls als Summe von Spin- und
Bahndrehimpuls definiert werden:

\begin{align}
  \vec{J} &= \vec{L} + \vec{S} \\
  \lvert \vec{J} \rvert &= \sqrt{J(J+1)}\hbar.
\end{align}

Um die verschiedenen Größen zu notieren verwendet man folgende Schreibweise:

\begin{equation}
  \ce{^{2s+1}L}\ua{J}.
\end{equation}


Bei dem zweiten Grenzfall beobachtet man schwere Atome. Hier dominieren die Wechselwirkungen
zwischen den $\vec{l\ua{i}}$ und den $\vec{s\ua{i}}$, sodass sich kein Gesamptspin
und Gesamtdrehimpuls mehr definieren lassen. Deshalb bezeichnet man diese Fall auch
als $\textbf{j-j-Kopplung}$. Für den Gesamtdrehimpuls ergibt sich daher folgendes:

\begin{equation}
  \vec{J} = \sum \vec{j\ua{i}} = \sum \vec{l\ua{i}} + \vec{s\ua{i}}.
\end{equation}

Zwischen beiden Fällen besteht ein fließender Übergang bei mittleren Kernladungszahlen.

\subsection{Aufspaltung im homogenen Magnetfeld}
