\input{"../Praeambel_prak.tex"}

\subject{104}
\title{Der Doppler-Effekt}
\author{Sebastian Pape \and Jonah Nitschke}
\date{Durchführung: 18.10.2016}

\begin{document}

\maketitle

\input{"Theorie_Durchfuehrung.tex"}

\section{Auswertung}
\subsection{Einführung}
In der folgenden Auswertung werden an verschiedenen Stellen statistische Größen berechnet,
die meist mit einem zusätzlichen Fehler angegeben werden. Die verwendeten Formeln für die
Berechnung dieser Größen soll in der folgenden Einleitung erläutet werden.\\
Wird ein Messreihe mit mehreren Messwerten $(x_1, x_2, \ldots , x_N)$ für eine Messgröße angelegt,
so wird ihr Mittelwert mit der Formel

\begin{equation}
  \overline{x} = \frac{1}{n} \sum_{i=1}^N x_i
\end{equation}

definiert. Des weiteren wird die Standardabweichung einer Messreihe durch

\begin{equation}
  \sigma_x = \sqrt{\frac{1}{n - 1} \sum_{i=1}^N (x_i - \overline{x})^2}
\end{equation}

errechnet. Mithilfe der Standardabweichung kann folgendermaßen der Fehler des Mittelwertes
angegeben werden:

\begin{equation}
  \increment \overline{x} = \frac{\sigma_x}{\sqrt{n}}
\end{equation}

Da in vielen Formeln zur Berechnung von weiteren Größen ebenfalls die Fehler der eingehenden
Messgrößen beachtet werden soll, nutz man zur Berechnung dieser Fehler die Gauß´sche
Fehlerfortpflanzung, welche durch folgende Formel definiert ist:

\begin{equation}
  \increment f = \sqrt{ \sum_{i=1}^N (\frac{\partial f}{\partial y_i})^2 (\increment y_i)^2}
\end{equation}

Bei $\sfrac{\partial f}{\partial y_i}$ handelt es sich hierbei um die Ableitung der Formel nach
der Messgröße $y_i$ und bei $\increment y_i$ handelt es sich um den Fehler von der jeweiligen Messgröße.

\subsection{Bestimmung der Geschwindigkeit}
Der für den Versuch bereitstehende Wagen mit dem montierten Lautsprecher lässt sich mit 10 unterschiedlichen
Gangeinstellungen bewegen. Für die Bestimmung der jeweiligen Geschwindigkeiten wurden mithilfe des oben
beschriebenen Aufbaus von dem Zeitbasisgenerator im Abstand von einer $\mu s$ Impulse abgesetzt. Durch
die Einstellung des Untersetzers auf $10^3$ bzw. $10^2$ wurde die Impulsrate auf 1/$10^-3$ bzw. 1/$10^-4$
reduziert. Die Geschwindigkeit ergibt sich dann aus der eingehenden Impulszahl und der vorher gemessenen
Strecke zwischen den Lichtschranken folgendermaßen:

\begin{equation}
  v = \frac{s}{Impulse \cdot 10^-4 s}
  \label{eqn:Ges}
\end{equation}

Die Länge der Strecke wurde vorher gemessen und beträgt $s = (0,202 \pm 0,001)m$. Pro Gangeinstellung
wurden dafür 5 Messungen vorgenommen, aus denen jeweils zuerst der Mittelwert und dessen Fehler bestimmt
wird. Mit der Formel \eqref{eqn:Ges} ergibt dich dann folgende Tabelle:

\begin{table}
  \centering
  \caption{Geschwindigkeit in Abhänigkeit von der Gangeinstellung.}
  \csvautobooktabular{Speed.csv}
\end{table}

Der Fehler der Geschwindigkeit folgt dabei mithilfe der Gauß´schen Fehlerfortpflanzung aus den
Fehlern der Streckenmessung $\increment s$ und dem Fehler der Zeit $\increment t$:

\begin{equation}
  \increment v = \sqrt{\Bigl(\frac{\increment s}{t}\Bigr)^2 + \Bigl(\frac{s \cdot \increment t}{t^2}\Bigr)^2}
\end{equation}

\subsection{Bestimmung der Ruhefrequenz}
Die Messung der Ruhefrequenz funktioniert ebenfalls mit der Schaltung der Messung c), jedoch
wird hier das Low-Signal für den Flip-Flop manuell geschaltet. Der Untersetzter wurde auf $10^6$
eingestellt, so dass die gemessene Zeitspanne genau einer Sekunde entspricht. Die gemessene Impulszahl
entspricht somit genau der Ruhefrequenz. Die Ergebnisse der Messung sind in Tabelle 2 aufgelistet:

\begin{table}
  \centering
  \caption{Werte der Ruhefrequenz.}
  \begin{tabular}{ c | c }
    \toprule Messung & $\nu_0$ [Hz] \\
    \midrule
    1 & 20742 \\
    2 & 20742 \\
    3 & 20742 \\
    4 & 20742 \\
    5 & 20742 \\
    \bottomrule
  \end{tabular}
\end{table}

Die Ruhefrequenz entspricht bei allen fünf Messungen dem selben Wert und wird dehalb in der folgenden Auswertung
als Fehlerfrei angenommen. Eine Fehlerbehaftung ist zwar auch hier nicht auszuschließen,
allerdings dürfte der Fehler der vorliegenden Messung gegenüber den Fehlern anderer Messwerte vernachlässigbar
gering sein.

\subsection{Bestimmung der Wellenlänge}

Zur Bestimmung der Wellenlänge wurden wie oben beschrieben Lissajous-Figuren auf dem Oszilloskop
erzeugt. In der folgenden Tabelle ist die gemessene Strecke x [cm] in Abhängigkeit der vorhandenen
Phasenverschiebung aufgetragen:

\begin{table}
 \centering
 \caption{Gemessenen Strecken bei denen die Schwingungen in Phase sind.}
  \begin{tabular}{c c}
    \toprule
    {Strecke[cm]} & {Phasenverschiebung} \\
    \midrule
    0.0  & 0     \\
    0.9  & $\symup{\pi}$ \\
    1.8  & 0     \\
    2.7  & $\symup{\pi}$ \\
    3.5  & 0     \\
    4.4  & $\symup{\pi}$ \\
    \bottomrule
  \end{tabular}
\end{table}

Die Differenz von zwei aufeinanderfolgenden Positionen entspricht hierbei immer der halben
Wellenlänge (berechnete Werte Tabelle 4):

\begin{table}
 \centering
 \caption{Berechnete Werte der Wellenlänge.}
  \begin{tabular}{c c c}
    \toprule
    {Differenz der Positionen} & {$\frac{\lambda}{s}$ [cm]} & {$\lambda$ [cm]} \\
    \midrule
    2-1  & 0.09 & 0.18  \\
    3-2  & 0.09 & 0.18  \\
    4-3  & 0.09 & 0.18  \\
    5-4  & 0.08 & 0.16  \\
    6-5  & 0.09 & 0.18  \\
    \bottomrule
  \end{tabular}
\end{table}

Der Mittelwert und dessen Fehler ergeben sich wie in der Einführung beschrieben:

\begin{equation}
  \overline{\lambda} = (0,0176 \pm 0,0018) m
\end{equation}

Aus der vorhandenen Wellenlänge lässt nun der Faktor $\sfrac{\nu_0}{c}$, mittels einer Umformung
zu $\sfrac{1}{\lambda}$ über die Definition der Schallgeschwindkeit, ermitteln:

\begin{align}
  c                        &=  \lambda \cdot \nu_o \\
  \implies \frac{\nu_0}{c} &=  \frac{1}{\lambda} = (56.818 \pm 5.811) \frac{1}{m}
\end{align}

Der Wert des Fehlers folgt hierbei mithilfe der Gauß´schen Fehlerfortpflanzung aus dem
Fehler der Wellenlänge:

\begin{equation}
  \increment \frac{\nu_0}{c} = \frac{1}{\lambda^2} \increment \lambda
\end{equation}

\subsection{Bestimmung der Schallgeschwindigkeit $c$}

Die Schallgeschwindigkeit kann nun mithilfe der gemessenen Ruhefrequenz und der oberhalb
berechneten Wellenlänge ermittelt werden:

\begin{equation}
  c = \lambda \cdot \nu_0
\end{equation}

Der Fehler der Schallgeschwindigkeit wird somit durch

\begin{equation}
  \increment c = \increment \lambda \cdot \nu_0
\end{equation}

bestimmt. Wie oben erwähnt wird der Fehler der Ruhefrequenz dabei als verschwindend gering
angesehen, sodass sich folgender Wert für die Schallgeschwindigkeit ergibt:

\begin{equation}
  c = (365.06 \pm 37.34) \frac{m}{s}
\end{equation}

\subsection{Abschätzung der Differenz von Formel 2 und 5}

Mithilfe der Ruhefrequenz und den der Messung entnommenen Daten über die
Geschwindigkeit des Wagens soll im Folgendem abgeschätzt werden, ob mit diesem Versuchsaufbau
eine Geschwindigkeit erreicht wird, die eine Unterscheidung zwischen bewegter Quelle und
bewegtem Empfänger sichtbar macht.

Die Höchstgeschwindigkeit des Wagens in der Gangstufe 60 beträgt $v_{max} = 0,503 \frac{m}{s}$,
die Schallgeschwindigkeit $c = 365.06 \frac{m}{s}$ und die Ruhefrequenz hat den Wert $\nu_0 = 20742 \, [Hz]$.
Werden diese Werte nun in die Formeln für den bewegten Empfänger $\nu_E$ und die bewegte Quelle
$\nu_Q$ eingesetzt, zeigt sich ein Unterschied beider Ergebnisse erst in der zweiten Nachkommastelle:

\begin{align}
  \nu_E &= 20770,579\\
  \nu_Q &= 20770,619
\end{align}

Mit dem Ergebniss dieser Abschätzung kann der Unterschied zwischen den Formeln für $\nu_E$ und
$\nu_Q$ als geringfügig und unbedeutend angenommen werden, da auch bei weiteren Messungen der
Frequenzen eine höhere Genauigkeit nicht erreicht wird. Ersichtlich ist jedoch, dass für alle
$v>0$ weiterhin $\nu_Q > \nu_E > \nu_0$ gilt.

\subsection{Bestimmung von \texorpdfstring{$\frac{\nu_0}{c}$} aus der Messung der Frequenzänderung}
\subsubsection{Bestimmung aus der indirekten Messung der Frequenzveränderung}

Analog zu der oben beschriebenen Messung der Ruhefrequenz wurden die Frequenzen in Abhängigkeit
der Geschwindigkeit des Wagens bestimmt. Indem man die Differenz beider Frequenzen nimmt, lässt
sich die Frequenzänderung errechnen:

\begin{equation}
  \increment \nu = \nu_v - \nu_0
\end{equation}

In den folgenden Tabelle ist für jede Geschwindigkeit die jeweilig gemessene Frequenz und die
Differenz zur Ruhefrequenz eingetragen.

\begin{table}
  \centering
  \caption{Frequenzänderung bei "positiver Geschwindigkeit."}
  \csvautobooktabular{Delta1.csv}
\end{table}

\begin{table}
  \centering
  \caption{Frequenzänderung bei "negativer Geschwindigkeit."}
  \csvautobooktabular{Delta2.csv}
\end{table}

\newpage

Wird nun $\increment \nu$ gegen die Geschwindigkeit $v$ aufgetragen, so lässt sich mittels linearer
Regressionsrechnung eine Gerade ermitteln, die als Steigung den Faktor $\frac{\nu_0}{c}$ besitzt.
Somit ergibt sich für die Messung folgender Faktor:

\begin{align}
  \frac{\nu_0}{c} &= (56.818 \pm 5.812) \frac{1}{m} \\
  a               &= (58.326 \pm 0.764) \frac{1}{m} \\
  b               &= (0.418 \pm 0.220) Hz
\end{align}

Im Graph sichtbar wird hier, dass die Werte für die größten negativen Geschwindigkeiten etwas
von der linearen Regression abweichen. Dies liegt daran, dass die Messstrecke bei den höheren
Geschwindigkeiten nicht ausreichte, um alle Impulse zu zählen. Somit kamen ähnliche Werte für
die Gänge 42 und 48 heraus und alle höheren Gänge wurden bei der Messung weggelassen, das sie
das Ergebniss stark verfälschen würden.

%\includegraphics[width=\textwidth]{plotC.pdf}

\subsubsection{Berechnung aus der Messung mittel Schwebemethode}

Wie im Aufbau schon beschrieben, gibt es für die Messung der Frequenzveränderung eine Alternative
mit der Schwebungsmethode. Mit dieser wird im Gegensatz zu 1.7.1 die Frequenzveränderung direkt gemessen.

Auffällig in Tabelle 7 ist, dass alle Werte für die Differenz zur Ruhefrequenz ca. doppelt so groß sind,
wie bei der indirekten Messung der Frequenzveränderung. Die Ursache hierfür ist die Verwendung eines
reflektors für die Messung. Das Signal muss die Strecke von Lautsprecher bis zum Reflektor zweimal durchlaufen
bis das Signal beim Mikrophon ankommt. Die Strecke und somit auch die Frequenzveränderung ist
dadurch doppelt so groß wie bei der Messung 1.7.1 . Für die Bestimmung des Faktors $\frac{\nu_0}{c}$ muss
also der Faktor 2 noch aus der Frequenzdifferenz herausgezogen werden.

\begin{table}
  \centering
  \caption{Frequenzänderung bei Messung mit der Schwebemethode.}
  \csvautobooktabular{Schwebung.csv}
\end{table}

Mit der Schebungsmethode kann zudem immer nur der Betrag der Frequenzänderung ermittelt werden,
die Ergebnisse für die Differenz müssen also im nachhinein mit verschiedenen Vorzeichen versehen werden,
je nachdem auf welche Geschwindigkeit sie sich beziehen. Da die Messung bei der Bewegung des Wagens
in Ausbreitungsrichtung keine Ergebnisse lieferte, sind in der obigen Tabelle nur die Werte für eine
Bewegung entgegen der Ausbreitungsrichtung angegeben.

Außerdem sind lediglich die Geschwindigkeiten bis zur Gangstufe 36 angegeben, da die
Strecke nicht ausreichte, um bei weiteren Gangeinstellungen bis zum Ende Impulse empfangen zu können.
Für alle Gangeinstellungen über 36 ergaben sich deshalb die gleichen Frequenzdifferenzen wie bei
der Gangeinstellung 36.

Analog zur Messung in 1.7.1 wird die Frequenzdifferenz gegen die Geschwindigkeit aufgetragen. Für die
Steigung $a$ und den Y-Achsenabschnitt $b$ erhält man mittels einem linearen Fit mit Python
folgende Werte:

\begin{align}
  a &= (109.911 \pm 10.356) \frac{1}{m} \\
  b &= (0.466 \pm 2.040) Hz
\end{align}

Da wie oben beschrieben in der Steigung a der Faktor 2 enthalten ist, muss für $\frac{\nu_0}{c}$
noch aus a der Faktor 2 gezogen werden:

\begin{equation}
  \frac{\nu_0}{c} = \frac{a}{2} = (54.955 \pm 5.178) \frac{1}{m}
\end{equation}

%\includegraphics[width=\textwidth]{plotCS.pdf}

\subsection{Vergleich der Ergebnisse aus 1.4 und 1.7 mithilfe des Studentschen T-Tests}

Der "Student´sche T-Test" ist ein Hypothesentest. Er dient zum Vergleich der Messmethoden
bzw. gibt an, ob zwischen den Messmethoden A und B eine systematische Abweichung vorliegt.
Dafür muss zuerst die Prüfgröße $t$ bestimmt werden, welche durch folgenden Formeln definiert
wird:

\begin{equation}
  t := \frac{\overline{x_a} - \overline{x_b}}{S_D}
\end{equation}

$\overline{x_a}$ und $\overline{x_b}$ sind die Mittelwerte der Stichproben bzw. der Messreihen.
$S_D$ ist die Standardabweichung und durch:

\begin{equation}
  S_D = \sqrt{ \frac{s_a^2(n_a-1) + s_b^2(n_b-1)}{n_a + nb - 2} \cdot \frac{n_a + n_b}{n_a \cdot n_b}}
\end{equation}

$s_a$ und $s_b$ sind dabei die Fehler der Stichproben und n ist die Anzahl an Messungen der jeweiligen
Messreihen.\\
Anschließend wird ein Signifikanzniveau $\alpha$ festgelegt, welches angibt, mit welcher Wahrscheinlichkeit
die Hypothese bestätigt bzw. widerlegt werden soll. Zudem wird die Anzahl der Freiheitsgerade $f$ durch
folgende Formel bestimmt:

\begin{equation}
  f = n_a + n_b -2
\end{equation}

Mit der Anzahl an Freiheitsgeraden und den Werten für $t$ kann in einer Vergleichstabelle nach Werten für
 $\alpha$ gesucht werden, so dass für den ermittelten Wert $t$ gilt:

\begin{equation}
  \tilde{t} < \lvert t \rvert .
\end{equation}

Mit dieser Bedingung beträgt die Wahrscheinlichkeit, dass ein systematischer Fehler vorliegt, $1 - \alpha$.

\begin{table}
  \centering
  \caption{Benötigte Werte für den T-Test}
  \begin{tabular}{c | c c c}
    \toprule
    {} & {Mittelwert $\overline{x}$} & {Fehler $s$} & {Anzahl der Messungen $n$} \\
    \midrule
    Wellenlänge-Messung A & 56.818 & 5.811 & 5  \\
    indirekte Messung B   & 58.326 & 0.764 & 18 \\
    Schwebung C           & 54.955 & 5.178 & 6  \\
    \bottomrule
  \end{tabular}
\end{table}

\begin{table}
  \centering
  \caption{Ergebnisse des T-Test}
  \begin{tabular}{c | c c c c c c}
    \toprule
    {Vergleich von} & {$S_D$}& {$t$} & {$\alpha$} & {$f$} & {$\tilde{t}$} & {Wahrscheinlichkeit} \\
    \midrule
    A und B & 1.361 & 1.108 & 0.15 & 21 & 1.063 & 85.0 \% \\
    B und C & 1.206 & 2.762 & 0.01 & 22 & 2.508 & 99.9 \% \\
    A und C & 3.311 & 0.563 & 0.30 & 9  & 0.543 & 70.0 \% \\
    \bottomrule
  \end{tabular}
\end{table}

Die Wert für $\tilde{t}$ können aus Quelle [1] entnommen werden. Der "Student´sche T-Test" zeigt, dass mit hoher
Wahrscheinlichkeit systematische Fehler zwischen denen jeder Messung bestehen. Zu sehen sind aber auch sehr
hohe statistische Fehler besonders bei dem Messungsvergleich A und B sowie dem Vergleich A und C.

\end{document}
