\input{../Praeambel_prak.tex}

\subject{106}
\title{Das gekoppelte Pendel}
\author{Sebastian Pape \and Jonah Nitschke}
\date{Durchführung: 25.10.2016}

\begin{document}

\maketitle

\section{Theorie}

In dem folgenden Versuch werden verschiedene Schwingungsarten von gekoppelten Pendeln betrachtet.
Bei den verwendeten Pendeln handelt es sich um Fadenpendel mit der Fadenlänge $l$ und der Masse $m$.
Bei einer Auslenkung des Pendels um den Winkel $\varphi$ wirkt auf das Massenstück ein durch die
Gravitation verursachtes Drehmoment:

\begin{equation}
  M = D_p \cdot \varphi
\end{equation}

Für kleine Auslenkungen aus der Ruhelage ($\sin(\varphi) = \varphi$) kann mit dieser Kraft eine
Bewegungsgleichung aufgestellt werden:

\begin{equation}
  J \cdot  \ddot{\varphi} + D_p \cdot \varphi = 0
\end{equation}

Bei $D_p$ handelt es sich dabei um die Winkelrichtgröße und bei $J$ um das Trägheitsmoment des Pendels.
Bei der Lösung der Bewegungsgleichung handelt es sich um eine harmonische Schwingung mit folgender
Frequenz:

\begin{equation}
  \omega = \sqrt{\frac{D_p}{J}} = \sqrt{\frac{g}{l}}
\end{equation}

Werden zwei identische Pendel nun mit einer Feder gekoppelt, schwingen sie nicht mehr unabhängig
voneinander, da auf jedes Pendel eine zusätzliche Kraft wirkt. Somit ergeben sich zwei neue
Bewegungsgleichung:

\begin{align}
  J \cdot  \ddot{\varphi_1} + D \cdot \varphi_1 &= D_F(\varphi_2 - \varphi_1) \\
  J \cdot  \ddot{\varphi_2} + D \cdot \varphi_2 &= D_F(\varphi_1 - \varphi_2)
\end{align}

Diese beiden Differentialgleichungen können entkoppelt werden, so dass sich die Bewegung des Systems
als Überlagerung von Eigenschwingungen darstellen lässt. Bei den lösungen handelt es sich dabei
ebensfalls um harmonische Schwingungen. Für verschiedene Anfangsbedigungen ergeben
sich daraus verschiedene Schwingungsarten.

\subsection{Gleichphasige Schwingung: \texorpdfstring{$\alpha_1 = \alpha_2$}{}}

Bei dieser Schwingung werden die beiden Pendel zu jedem Zeitpunkt um den gleichen Winkel
$\alpha_1 = \alpha_2$ ausgelenkt. Resultierend daraus
wird von der Kopplungsfeder keine Kraft auf die beiden Pendel ausgeübt und im idealisierten Fall
entsteht somit keine Auslenkung der Feder. Da die rücktreibende Kraft der beiden Pendel alleine
durch die Gewichtskraft ausgeübt wird, kann die Feder auch entfernt werden, ohne das sich die System-
Schwingung verändern würde.

Die Freqguenz des gekoppelten Pendels stimmt somit auch mit der Frequenz der ursprünglichen Pendel
und kann mit

\begin{equation}
  \omega_+ = \sqrt{\frac{g}{l}}
\end{equation}

beschrieben werden. Die Schwingungsdauer ergibt sich dann anhand der Frequenz mit folgender Formel:

\begin{equation}
  T_+ = \frac{2\pi}{\omega_+} = 2\pi \cdot \sqrt{\frac{l}{g}}
\end{equation}

\subsection{Gegenphasige Schwingung: \texorpdfstring{$\alpha_1 = -\alpha_2$}{}}

Bei der gegenphasigen Schwingung werden die beiden Pendel um den entgegengesetzten Winkel $\alpha_1 = -\alpha_2$
ausgelenkt. Dabei wird von der Kopplungsfeder auf jede Masse die gleiche aber entgegen-
gesetzte Kraft ausgeübt, so dass ein symmetrisches Schwingungssystem entsteht. Die resultierenden
Formeln für die Schwingungsfrequenz und Schwingungsdauer lautet wie folgt:

\begin{align}
  \omega_-  &= \sqrt{\frac{g}{l} + \frac{2 K}{l}} \\
  T_-       &= 2\pi \cdot \sqrt{\frac{l}{g + 2K}}
\end{align}

Bei $K$ handelt es sich hierbei um die Kopplungskonstante der Feder.

\subsection{Gekoppelte Schwingung: \texorpdfstring{$\alpha_1 = 0, \alpha_2 \neq 0$}{}}

Wird am Anfang nur ein Pendel $\alpha_1 \neq 0$ ausgelenkt, während das andere in der unausgelenkten
Position verharrt $\alpha_2 = 0$, kommt es zu einer gekoppelten Schwingung. Wird nun das erste Pendel
losgelassen, überträgt es seine Energie auf das zweite Pendel, so dass dieses anfängt zu schwingen.
Sobald das erste Pendel stillsteht, kehrt sich dieser Prozeß wieder um. Die maximale Amplitude
wird dabei immer erreicht, wenn das andere Pendel stillsteht. Die vollständige Energieübetragung
wiederholt sich immer wieder.
Den zeitlichen Abstand zwischen zwei Stillständen nennt man dabei Schwebung. Die Schwebefrquenz
und Schwebungsdauer sind durch folgende Formeln definiert:

\begin{align}
   \omega_S &= \omega_+ - \omega_- \\
   T_S      &= \frac{T_+ \cdot T_-}{T_+ - T_-}
\end{align}

Dabei entspricht $T_+$ der gleichphasigen und $T_-$ der gegenphasigen Schwingung. Die in den Formeln
vorkommende Kopplungskonstante K kann entweder mit den Frequenzen oder mit
den Schwingungsdauern der gleich- und gegenphasigen Schwingung bestimmt werden:

\begin{equation}
  K = \frac{\omega_-^2 - \omega_+^2}{\omega_-^2 + \omega_+^2}
    = \frac{T_+^2 - T_-^2}{T_+^2 + T_-^2}
\end{equation}

\section{Durchführung}

Als erstes werden die Gewichte am Pendel so verschoben, dass beide Pendel dieselbe Pendellänge
besitzen. Dann wird mithilfe einer Stoppuhr erst die Schwingungdauer des ersten Pendels und danach
die des zweiten Pendels gemessen. Dabei enthält eine Messreihe gemessene Werte von jeweils 5
Schwingungen.

Nachdem die einzelnen Schwingungsdauern bestimmt sind, werden die beiden Pendel mithilfe der Feder
gekoppelt. Anschließend wird zuerst die Schwingungsdauer der gleichphasigen Schwingung gemessen, indem
die Pendel so ausgelenkt werden, dass die Feder nicht gespannt wird. Für die zweite Messreihe werden die
beiden Pendel dann exakt gegenphasig ausgelenkt. Eine Messung entspricht erneut jeweils 5 Schwingungen.

Bei der letzten Messung wird nur ein Pendel ausgelenkt, während das andere in seiner Ruhelage
verharrt. Zuerst werden 10 Werte der Dauer einer einzelnen Schwebung aufgenommen, indem die
Zeit zwischen dem beschleunigen und dem erneuten Stillstand eines Pendels gemessen wird.
Anschließend werden noch einmal die Zeiten von 5 Schwebungen gemessen.

Alle Messungen werden anschließend für eine beliebige weitere Länge durchgeführt.







\end{document}
