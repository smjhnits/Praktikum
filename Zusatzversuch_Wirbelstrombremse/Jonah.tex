\input{../Praeambel_prak.tex}

\begin{document}

\section{Einleitung und Idee}

Als es darum ging ein Thema für unseren Zusatzversuch rauszusuchen, haben wir uns
eigentlich sofort für die Wirbelstrombremse entschieden. Einerseits gehört sie zu
dem Bereich des Magnetismus und wurde in der Physik II schon ausführlich behandelt,
so dass wir mit dem Thema schon vertraut waren. Andererseits ist es ein Prinzip,
dass auch im Alltag sehr viel Anwendung findet. Somit ist es ein eigentlich sehr
allgemeines Thema, dass zu dem auch sehr viele Möglichkeiten für ein Experiment
bietet.

\section{Aufbau}

Wir hatten für unseren Versuch eigentlich einen ziemlich klaren Aufbau im Kopf,
den wir bei der tatsächlichen Versuchsdurchführung jedoch etwas abändern mussten.
Theoretisch wollten wir uns bei unserem Versuch stark an dem Waltenhofeschen Pendel
orientieren. Bis auf ein paar kleine Umänderungen kam unser Aufbau auch ziemlich
nahe an dieses Vorbild heran. Grundsätzlich bestand der Aufbau dabei aus nicht
allzu vielen Bestandteilen:

\begin{itemize}
  \item 2 Magnetspulen
  \item 1 großer U-Eisenkern, 2 kleine Eisenkerne
  \item 1 Stativ mit Pendel
  \item 2 große Hebewagen, 1 kleiner Hebewagen
  \item 1 Stromgenerator
  \item 3 Aluminiumplatten
  \item mehrere Stromkabel
\end{itemize}

Zuerst wurden die beiden Spulen mit einem U-förmigen Eisenkern verbunden und in
Reihe an den Generator angeschlossen, um diesen nicht zu stark zu belasten. Um
eine möglichst große Durchflussfläche des Magnetfeldes zu erhalten, wurden auf
beide Spulen noch zwei weitere Eisenkerne gelegt. Danach wurde das Stativ mit dem
Pendel mithilfe der Hebewagen so in Position gebracht, dass das am Pendel befestigte
Metallstück frei durch das Magnetfeld bzw. die Lücke zwischen den beiden Eisenkernen
schwingen konnte.

\begin{figure}
  \centering
  \begin{subfigure}{0.48\textwidth}
    \includegraphics[height=5cm]{20170306_101228.jpg}
  \end{subfigure}
\begin{subfigure}{0.48\textwidth}
    \includegraphics[height=5cm]{20170306_101230.jpg}
  \end{subfigure}
\end{figure}

\section{Durchführung}

Zuerst wurde an dem Pendel die Aluminium-Platte ohne Schlitze angebracht. Dann
wurde das Pendel zuerst ohne Dämpfung ausgelenkt und die Zeit gemessen, bis das
Pendel stehen bleibt. Bei den weiteren Messungen wurde der Strom Stückweise um
0.5 V erhöht bis zu einem Maximalwert von 5 V. Danach wurden die beiden anderen
Platten mit einem groben und einem feineren Schlitmuster angebracht und die Messungen
erneut ausgeführt.
Anschließend haben wir noch versucht, eine Platte aus einem anderen Material an
dem Pendel anzubringen. Allerdings ließen sich keine vernünftigen Messungen
durchführen, da der Aufbau für die vorhandenen Platten scheinbar nicht geeignet
war.



\section{Auswertung}

\begin{figure}
  \includegraphics[width=0.7\textwidth]{Messung1.pdf}
  \caption{Aluminiumplatte ohne Schlitze}
  \label{fig:Messung1}
\end{figure}

\begin{figure}
  \includegraphics[width=0.7\textwidth]{Messung3.pdf}
  \caption{Aluminiumplatte mit mittlerer Schlitzahl}
  \label{fig:Messung1}
\end{figure}

\begin{figure}
  \includegraphics[width=0.7\textwidth]{Messung3.pdf}
  \caption{Aluminiumplatte mit höchster Schlitzahl}
  \label{fig:Messung1}
\end{figure}

\begin{figure}
  \includegraphics[width=0.7\textwidth]{Kombiniert.pdf}
  \caption{Alle Messungne im vergleich}
  \label{fig:Kombiniert}
\end{figure}

Die Graphen zeigen alle, dass mit zunehmender Strom bzw. Feldstärke die Schwingungsdauer
bis zum absoluten Stillstand abnimmt. Dabei ergeben sich bei Vergleich der Dauern
für den minimalen und maximalen Stromwert verschiedene Dämpfungswerte.
Bei der Metallplatte ohne Schlitze ergibt sich eine Dämpfung auf ca (0.84 $\pm$ 0.21)
$\%$ des Ursprungswertes ohne eingeschaltetes Magnetfeld.
Bei der Metallplatte mit mittlerer Schlitzzahl ergibt sich eine Dämpfung auf
ca. (5.47 $\pm$ 0.20) $\%$ des Ursprungwertes und bei der Platte mit der maximalen
Schlitzzahl ergibt sich eine Dämpfung auf ca. (16.18 $\pm$ 0.19) $\%$ des
Ursprungswertes.

\section{Anwendungsbeispiele}

Ein Beispiel dafür ist der Einsatz bei Schienenfahrzeugen. Hier wird in zwei
Kategorien bzw. Techniken unterschieden, die lineare und die rotierende Wirbelstrombremse.
Bei der linearen Wirbelstrombremse wird dabei ein zu den Schienen paralleles
Magnetfeld mithilfe eine Reihe von Magneten, die über einen Integralträger und
Tragärme am Radsatzlager befestigt sind. Diese Magneten werden bei Aktivierung
auf ca. 7 mm Entfernung über die Schienen gesenkt, so dass ein längs zu der Schiene
verlaufendes Magnetfeld erzeugt wird.

\begin{figure}
  \includegraphics[width=\textwidth]{Wirbelstrombremse_Aufbau.jpg}
  \caption{Aufbau einer linearen Wirbelstrombremse}
  \label{fig:linWAufbau}
\end{figure}

Diese Variante
wird allerdings nur bei sehr hohen Geschwindigkeiten sowie auf extra eingerichteten
Schienen verwendet.
Bei der rotierenden Wirbelstrombremse hingegen wird die Schiene als Elektromagnet
verwendet um Wirbelströme in den Rädern des Zuges zu erzeugen. Diese Variante wird
zurzeit allerdings nur bei Versuchsfahrzeugen eingesetzt bzw. getestet.


\end{document}
