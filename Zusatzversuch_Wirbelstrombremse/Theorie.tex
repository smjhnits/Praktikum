% bei Standalone in documentclass noch:
% \RequirePackage{luatex85}

\documentclass[captions=tableheading, titlepage= firstiscover, parskip = half , bibliography=totoc]{scrartcl}
%paper = a5 für andere optinen
% titlepage= firstiscover
% bibliography=totoc für bibdateien
% parskip=half  Veränderung um Absätze zu verbessern

\usepackage{scrhack} % nach \documentclass
\usepackage[aux]{rerunfilecheck}
\usepackage{polyglossia}
\usepackage[style=numeric, backend=biber]{biblatex} % mit [style = alphabetic oder numeric] nach polyglossia
\addbibresource{lit.bib}
\setmainlanguage{german}

\usepackage[autostyle]{csquotes}
\usepackage{amsmath} % unverzichtbare Mathe-Befehle
\usepackage{amssymb} % viele Mathe-Symbole
\usepackage{mathtools} % Erweiterungen für amsmath
\usepackage{fontspec} % nach amssymb
% muss ins document: \usefonttheme{professionalfonts} % für Beamer Präsentationen
\usepackage{longtable}

\usepackage[
math-style=ISO,    % \
bold-style=ISO,    % |
sans-style=italic, % | ISO-Standard folgen
nabla=upright,     % |
partial=upright,   % /
]{unicode-math} % "Does exactly what it says on the tin."
\setmathfont{Latin Modern Math}
% \setmathfont{Tex Gyre Pagella Math} % alternativ

\usepackage[
% die folgenden 3 nur einschalten bei documenten
locale=DE,
separate-uncertainty=true, % Immer Fehler mit ±
per-mode=symbol-or-fraction, % m/s im Text, sonst \frac
]{siunitx}

% alternativ:
% per-mode=reciprocal, % m s^{-1}
% output-decimal-marker=., % . statt , für Dezimalzahlen

\usepackage[
version=4,
math-greek=default,
text-greek=default,
]{mhchem}

\usepackage[section, below]{placeins}
\usepackage{caption} % Captions schöner machen
\usepackage{graphicx}
\usepackage{grffile}
\usepackage{subcaption}

% \usepackage{showframe} Wenn man die Ramen sehen will

\usepackage{float}
\floatplacement{figure}{htbp}
\floatplacement{table}{htbp}

\usepackage{mhchem} %chemische Symbole Beispiel: \ce{^{227}_{90}Th+}


\usepackage{booktabs}

 \usepackage{microtype}
 \usepackage{xfrac}

 \usepackage{expl3}
 \usepackage{xparse}

 % \ExplSyntaxOn
 % \NewDocumentComman \I {}  %Befehl\I definieren, keine Argumente
 % {
 %    \symup{i}              %Ergebnis von \I
 % }
 % \ExplSyntaxOff

 \usepackage{pdflscape}
 \usepackage{mleftright}

 % Mit dem mathtools-Befehl \DeclarePairedDelimiter können Befehle erzeugen werden,
 % die Symbole um Ausdrücke setzen.
 % \DeclarePairedDelimiter{\abs}{\lvert}{\rvert}
 % \DeclarePairedDelimiter{\norm}{\lVert}{\rVert}
 % in Mathe:
 %\abs{x} \abs*{\frac{1}{x}}
 %\norm{\symbf{y}}

 % Für Physik IV und Quantenmechanik
 \DeclarePairedDelimiter{\bra}{\langle}{\rvert}
 \DeclarePairedDelimiter{\ket}{\lvert}{\rangle}
 % <name> <#arguments> <left> <right> <body>
 \DeclarePairedDelimiterX{\braket}[2]{\langle}{\rangle}{
 #1 \delimsize| #2
 }

\setlength{\delimitershortfall}{-1sp}

 \usepackage{tikz}
 \usepackage{tikz-feynman}

 \usepackage{csvsimple}
 % Tabellen mit \csvautobooktabular{"file"}
 % muss in table umgebung gesetzt werden


% \multicolumn{#Spalten}{Ausrichtung}{Inhalt}

\usepackage{hyperref}
\usepackage{bookmark}
\usepackage[shortcuts]{extdash} %nach hyperref, bookmark

\newcommand{\ua}[1]{_\symup{#1}}
\newcommand{\su}[1]{\symup{#1}}


\begin{document}

\section{Die Wirbelstrombremse}

Wenn ein elektrischer Leiter durch ein Magnetfeld bewegt wird,
wird aufgrund der Lorentzkraft ein Strom in ihm induziert.
Der Leiter besitzt einen elektrischen Widerstand der von der
bewegten Ladung überwunden werden muss. Strom der durch einen
ohmschen Widerstand fließt erzeugt Energie in Form von Wärme,
welche aus der Bewegungsenergie umgewandelt wurde.
(Der induzierte Strom bildet widerum gemäß der Lenschen Regel ein dem
äußerem Feld entgegengesetztes Magnetfeld aus.) Die Bremsleistung
ist auf diesen Vorgang zurückzuführen.
Die Bremsleistung der Wirbelstrombremse hängt somit nur von wenigen
Faktoren ab. Diese Faktoren sind im Folgendem aufgezählt.

\begin{enumerate}
  \item Magnetfeldstärke
  \item elektische Leitfähigkeit
  \item Geschwindigkeit der Elektronen
\end{enumerate}

Die Magnetfeldstärke lässt sich bei einem Elektromagneten durch
Veränderung der Spulen, sowie ihrer Kerne beeinflussen. Zudem hängt
die Magnetfeldstärke von dem Spulenstrom ab, der ebenfalls reguliert
werden kann.

Die elektrische Leitfähigkeit ist eine materialkonstante.
Es gilt der Zusammenhang je größer die Leitfähigkeit des Leiters ist,
desto größer ist die Bremsleistung.
Dies lässt sich daran einsehen, dass bei einer größeren
elektrischen Leitfähigkeit der induzierte Strom größer ist, weshalb
die produzierte Abwärme gesteigert wird.

Die Geschwindigkeit der Elektronen bezieht sich auf die
Relativbewegung zwischen Magnetfeld und Leiter. Der Einfluss dieser
Geschwindigkeit lässt sich mit der Lorentzkraft direkt einsehen.

\subsection{Vorteile gegenüber mechanischen Bremsen}

Die Wirbelstrombremse ist im Vergleich zu mechanischen Bremsen
verschleißfrei, da die Bremsleistung nicht durch Kontaktreibung
gewährleistet wird. Zudem kann die Bremsleistung präzise eingestellt
werden.

\section{Erwartungswerte}

Anhand der theoretischen Erkentnisse waren die Erwartungswerte eindeutig.
Wir haben wie beobachtet erwartet, dass die Bremsleistung der
ungezackten Aluminiumplatte am größten war, verglichen mit den
beschnittenen Platten.
Zudem haben wir erwartet, dass die Bremsleistung mit der Anzahl der
Zacken absinkt.

\section{Diskussion}

Bei der Diskussion dieses Versuches ist zu sagen, dass unsere Erwartungswerte
mit den beobachteten Werten übereinstimmen.

Doch es ist anzumerken, dass nicht alle unsere Hypothesen überprüft
werden konnten.
Der Versuch wurde lediglich mit einem Material, also auch nur einer
Leitfähigkeit durchgeführt. Damit konnten wir den Zusammenhang
zwischen der Leitfähigkeit und der Bremsleistung nicht überprüfen.
Die Variationsparameter in unserem Versuch bezogen sich
ausschließlich auf die Stromstärke der Elektromagenten, sowie
die durchflutet Leiterfläche, die durch Schnitte beeinflusst wurde.
Möglichkeiten um den Versuch zu optimieren fallen im Nachhinein auf.
Es hätten mehrere Materialien überprüt werden können.

\subsection{Feedback}

Wir können jedem der überlegt einen Zusatzversuch durchzuführen nur empfehlen
die zu machen. Wir haben es als sehr spannend empfunden  uns auf
diesen Versuch vorzubereiten und fanden es sehr gut, dass
wir uns Versuche aus jedem Bereich der Physik aussuchen konnten.
Außerdem hat das Konstruieren des Aufbaues, sowie das Durchführen dieses
sehr viel Spaß gemacht.

Wir hatten vor Beginn des Versuches schon eine
konkrete Vorstellung, wie der Aufbau aussehen sollte und es war
eine Herausforderung diesen Aufbau auch tatsächlich in die
Realität umzusetzen.

\end{document}
